\ifx\type\undefined
  \documentclass[10pt, t]{beamer}
  \setbeamertemplate{footline}[page number]
\else
  \documentclass[10pt]{article}
  \usepackage[margin=1in]{geometry}
\fi

\usepackage{amsmath}
\usepackage{amssymb}
\usepackage{amsthm}
\usepackage{bbm}
\usepackage{cancel}
\usepackage{listings}
\usepackage{mathrsfs}
\usepackage{multirow}
\usepackage{soul}
\usepackage{stmaryrd}
\usepackage{tikz}
\usepackage{tikz-cd}
\usepackage{wrapfig}

\newtheorem*{algorithm}{Algorithm}
\newtheorem*{assumptions}{Assumptions}
\newtheorem*{conjecture}{Conjecture}
\newtheorem*{consequences}{Consequences}
\newtheorem*{exercise}{Exercise}
\newtheorem*{formalisation}{Formalisation}
\newtheorem*{proposition}{Proposition}
\newtheorem*{question}{Question}
\newtheorem*{remark}{Remark}

\ifx\type\undefined\else
  \newtheorem*{definition}{Definition}
  \newtheorem*{example}{Example}
  \newtheorem*{lemma}{Lemma}
  \newtheorem*{theorem}{Theorem}
\fi

\definecolor{keywordcolor}{rgb}{0.7, 0.1, 0.1}
\definecolor{tacticcolor}{rgb}{0.0, 0.1, 0.6}
\definecolor{commentcolor}{rgb}{0.4, 0.4, 0.4}
\definecolor{symbolcolor}{rgb}{0.0, 0.1, 0.6}
\definecolor{sortcolor}{rgb}{0.1, 0.5, 0.1}
\definecolor{attributecolor}{rgb}{0.7, 0.1, 0.1}
\def\lstlanguagefiles{lstlean.tex}
\lstset{language=lean}

\newcommand\A{\mathbb{A}}
\newcommand\C{\mathbb{C}}
\newcommand\F{\mathbb{F}}
\newcommand\G{\mathbb{G}}
\renewcommand\H{\mathbb{H}}
\newcommand\I{\mathbb{I}}
\newcommand\N{\mathbb{N}}
\renewcommand\P{\mathbb{P}}
\newcommand\Q{\mathbb{Q}}
\newcommand\R{\mathbb{R}}
\newcommand\Z{\mathbb{Z}}

\renewcommand\AA{\mathcal{A}}
\newcommand\BB{\mathcal{B}}
\newcommand\CC{\mathcal{C}}
\newcommand\DD{\mathcal{D}}
\newcommand\EE{\mathcal{E}}
\newcommand\FF{\mathcal{F}}
\newcommand\GG{\mathcal{G}}
\newcommand\HH{\mathcal{H}}
\newcommand\II{\mathcal{I}}
\newcommand\LL{\mathcal{L}}
\newcommand\MM{\mathcal{M}}
\newcommand\NN{\mathcal{N}}
\newcommand\OO{\mathcal{O}}
\newcommand\PP{\mathcal{P}}
\newcommand\RR{\mathcal{R}}
\renewcommand\SS{\mathcal{S}}
\newcommand\TT{\mathcal{T}}
\newcommand\XX{\mathcal{X}}

\renewcommand\aa{\mathfrak{a}}
\newcommand\cc{\mathfrak{c}}
\newcommand\dd{\mathfrak{d}}
\newcommand\ff{\mathfrak{f}}
\renewcommand\gg{\mathfrak{g}}
\newcommand\mm{\mathfrak{m}}
\newcommand\pp{\mathfrak{p}}
\newcommand\qq{\mathfrak{q}}
\renewcommand\ss{\mathfrak{s}}

\newcommand\LLL{\mathscr{L}}

\newcommand\ab{\mathrm{ab}}
\newcommand\Ab{\mathbf{Ab}}
\newcommand\Alg{\mathbf{Alg}}
\newcommand\Aff{\mathbf{Aff}}
\newcommand\Aut{\operatorname{Aut}}
\newcommand\Az{\mathrm{Az}}
\newcommand\Br{\operatorname{Br}}
\newcommand\BSD{\operatorname{BSD}}
\newcommand\ch{\operatorname{char}}
\newcommand\Cl{\operatorname{Cl}}
\newcommand\coker{\operatorname{coker}}
\newcommand\cris{\mathrm{cris}}
\renewcommand\d{\mathrm{d}}
\newcommand\Div{\operatorname{Div}}
\newcommand\dR{\mathrm{dR}}
\newcommand\EN{\operatorname{EN}}
\newcommand\End{\operatorname{End}}
\newcommand\ES{\operatorname{ES}}
\newcommand\et{\mathrm{\acute{e}t}}
\newcommand\Et{\mathbf{\acute{E}t}}
\newcommand\Ext{\operatorname{Ext}}
\newcommand\Fr{\operatorname{Fr}}
\newcommand\Frac{\operatorname{Frac}}
\newcommand\Gal{\operatorname{Gal}}
\newcommand\GL{\operatorname{GL}}
\newcommand\Gr{\mathrm{Gr}}
\newcommand\Hom{\operatorname{Hom}}
\newcommand\HT{\mathrm{HT}}
\newcommand\id{\operatorname{id}}
\newcommand\im{\operatorname{im}}
\newcommand\Ind{\operatorname{Ind}}
\renewcommand\inf{\operatorname{inf}}
\newcommand\inv{\operatorname{inv}}
\newcommand\Irr{\operatorname{Irr}}
\newcommand\Jac{\operatorname{Jac}}
\newcommand\lcm{\operatorname{lcm}}
\newcommand\Mat{\operatorname{Mat}}
\newcommand\Mod{\mathbf{Mod}}
\newcommand\Nm{\operatorname{Nm}}
\newcommand\nr{\mathrm{nr}}
\newcommand\NS{\operatorname{NS}}
\newcommand\Ob{\operatorname{Ob}}
\newcommand\ord{\operatorname{ord}}
\newcommand\op{\mathrm{op}}
\newcommand\PGL{\operatorname{PGL}}
\newcommand\Pic{\operatorname{Pic}}
\newcommand\Prob{\operatorname{Prob}}
\newcommand\Proj{\operatorname{Proj}}
\newcommand\PSh{\mathbf{PSh}}
\newcommand\Reg{\operatorname{Reg}}
\newcommand\res{\operatorname{res}}
\newcommand\rk{\operatorname{rk}}
\newcommand\Sch{\mathbf{Sch}}
\newcommand\Sel{\operatorname{Sel}}
\newcommand\Set{\mathbf{Set}}
\newcommand\sgn{\operatorname{sgn}}
\newcommand\Sh{\mathbf{Sh}}
\newcommand\SL{\operatorname{SL}}
\newcommand\Spec{\operatorname{Spec}}
\newcommand\supp{\operatorname{supp}}
\newcommand\Tam{\operatorname{Tam}}
\newcommand\Top{\mathbf{Top}}
\newcommand\tor{\operatorname{tor}}
\newcommand\tr{\operatorname{tr}}
\newcommand\tra{\operatorname{tra}}
\newcommand\WC{\operatorname{WC}}

\DeclareFontFamily{U}{wncyr}{}
\DeclareFontShape{U}{wncyr}{m}{n}{<->wncyr10}{}
\DeclareSymbolFont{cyr}{U}{wncyr}{m}{n}
\DeclareMathSymbol{\Sha}{\mathord}{cyr}{"58}

\newcommand{\function}[5][]{
  \if &#1&
    \begin{array}{rcl}
      #2 & \longrightarrow & #3 \\
      #4 & \longmapsto     & #5
    \end{array}
  \else
    \begin{array}{rcrcl}
      #1 & : & #2 & \longrightarrow & #3 \\
         &   & #4 & \longmapsto     & #5
    \end{array}
  \fi
}

\newcommand{\functions}[7][]{
  \if &#1&
    \begin{array}{rcl}
      #2 & \longrightarrow & #3 \\
      #4 & \longmapsto     & #5 \\
      #6 & \longmapsto     & #7 \\
    \end{array}
  \else
    \begin{array}{rcrcl}
      #1 & : & #2 & \longrightarrow & #3 \\
         &   & #4 & \longmapsto     & #5 \\
         &   & #6 & \longmapsto     & #7
    \end{array}
  \fi
}
\title{Kolyvagin's theorem}
\subtitle{The conjecture of Birch and Swinnerton-Dyer}
\author{David Kurniadi Angdinata}
\institute{University College London}
\date{Wednesday, 13 March 2024}

\begin{document}

\frame\maketitle

\begin{frame}{Some recapitulation}

Let $ E $ be a rational elliptic curve of conductor $ N $, and let $ K = \Q(\sqrt{-D}) $ be an imaginary quadratic field satisfying the \textbf{Heegner hypothesis}
$$ \ell \mid N \qquad \implies \qquad \ell \ \text{is split in} \ K. $$
For any $ n $ coprime to $ N $, define a \textbf{Heegner point of conductor $ n $}
$$ P_n := \Phi_N(\C / \OO_n, \C / \NN_n) \in E(H_n). $$
For any $ \ell $ coprime to $ nN $ that is inert in $ K $, there are norm compatibilities
$$ \tr_{H_n}^{H_{n\ell}} P_{n\ell} = a_\ell P_n. $$
These form a \textbf{Heegner system for $ (E, K) $}.

\bigskip Furthermore, define the \textbf{basic Heegner point}
$$ P_K := \tr_K^{H_1}(P_1) \in E(K). $$

\end{frame}

\begin{frame}{Application to BSD}

We will do the following next week:

\begin{theorem}[Gross--Zagier '86]
There is an explicit constant $ \alpha \ne 0 $ such that $ L'(E / K, 1) = \alpha \cdot \widehat{h}(P_K) $.
\end{theorem}

\bigskip We will do the following this week:

\begin{theorem}[Kolyvagin '90]
If $ \widehat{h}(P_K) \ne 0 $, then $ \rk_\Z E(K) = 1 $ and $ \#\Sha(E / K) < \infty $.
\end{theorem}

In particular, $ E(K)_{/ \tor} = \Z \cdot \tfrac{1}{n}P_K $.

\bigskip This almost proves the following:

\begin{corollary}[of Gross--Zagier '86, Kolyvagin '90, etc]
If $ \ord_{s = 1} L(E, s) \le 1 $, then $ \rk_\Z E(\Q) = \ord_{s = 1} L(E, s) $ and $ \#\Sha(E) < \infty $.
\end{corollary}

The missing ingredient is the existence of $ K $.

\end{frame}

\begin{frame}{Existence of Heegner fields}

Let $ -\epsilon $ be the sign in the functional equation
$$ \Lambda(E, s) = -\epsilon \cdot \Lambda(E, 2 - s). $$

\begin{theorem}[Waldspurger '85, Murty--Murty '97]
If $ \epsilon = + $, there are many imaginary quadratic fields $ K = \Q(\sqrt{-D}) $ satisfying the Heegner hypothesis such that $ \ord_{s = 1} L(E_D, s) = 0 $.
\end{theorem}

In particular,
$$ \ord_{s = 1} L(E, s) = \ord_{s = 1} L(E / K, s). $$

\begin{theorem}[Bump--Friedberg--Hoffstein '90, Murty--Murty '91]
If $ \epsilon = - $, there are many imaginary quadratic fields $ K = \Q(\sqrt{-D}) $ satisfying the Heegner hypothesis such that $ \ord_{s = 1} L(E_D, s) = 1 $.
\end{theorem}

In particular,
$$ \ord_{s = 1} L(E, s) = \ord_{s = 1} L(E / K, s) - 1. $$

\end{frame}

\begin{frame}{Complex conjugation on Heegner points}

\begin{lemma}[$ \tau $]
Complex conjugation $ \tau $ maps $ P_n \in E(H_n)_{/ \tor} $ to
$$ \tau(P_n) = \epsilon \cdot \sigma(P_n), $$
for some $ \sigma \in \Gal(H_n / K) $.
\end{lemma}

\begin{proof}
Note that $ \epsilon $ is precisely the eigenvalue of the Fricke involution $ w_N $ on the eigenform $ f_E $ associated to $ E $. On the other hand,
$$ w_N(\C / \OO_n, \C / \NN_n) = (\C / \NN_n^{-1}, \C / \overline{\NN_n}), $$
which differs from $ \tau(\C / \OO_n, \C / \NN_n) $ by some $ \sigma \in \Gal(H_n / K) \cong \Cl(\OO_n) $. Now apply $ \Phi_N $ and the Manin--Drinfeld theorem.
\end{proof}

\bigskip In particular, $ P_K \in E(\Q)_{/ \tor} $ precisely if $ \epsilon = + $.

\end{frame}

\begin{frame}{Proof of Gross--Zagier--Kolyvagin}

\begin{proof}[Proof of BSD for $ \ord_{s = 1} L(E, s) \le 1 $]
The functional equation says that $ L(E, 1) = -\epsilon \cdot L(E, 1) $ and $ L'(E, 1) = \epsilon \cdot L'(E, 1) $. Since $ \ord_{s = 1} L(E, s) \le 1 $,
$$ \ord_{s = 1} L(E, s) =
\begin{cases}
1 & \text{if} \ \epsilon = +, \\
0 & \text{if} \ \epsilon = -.
\end{cases}
$$
Choose any imaginary quadratic field $ K $ satisfying the Heegner hypothesis such that $ \ord_{s = 1} L(E / K, s) = 1 $, which exists by W/MM and BFH/MM. By Gross--Zagier and Kolyvagin, $ E(K)_{/ \tor} = \Z \cdot \tfrac{1}{n}P_K $. By Lemma $ (\tau) $,
$$ \rk_\Z E(\Q) =
\begin{cases}
1 & \text{if} \ \epsilon = +, \\
0 & \text{if} \ \epsilon = -.
\end{cases}
$$
Finally, $ \#\Sha(E) < \infty $ follows from $ \#\Sha(E / K) < \infty $ by Kolyvagin.
\end{proof}

\end{frame}

\begin{frame}{A weaker version of Kolyvagin}

\begin{theorem}[Kolyvagin '90]
If $ \widehat{h}(P_K) \ne 0 $, then $ \rk_\Z E(K) = 1 $ and $ \#\Sha(E / K) < \infty $.
\end{theorem}

For any prime $ \ell $,
$$ 0 \to E(K) / \ell E(K) \xrightarrow{\delta} \Sel_\ell(E / K) \to \Sha(E / K)[\ell] \to 0. $$
Choose any prime $ \ell \nmid 6ND $ such that $ \overline{\rho_{E, \ell}} $ is surjective and $ P_K \notin \ell E(K) $. Then $ E(K)[\ell] = 0 $, so $ \rk_\Z E(K) = \dim_{\F_\ell} E(K) / \ell E(K) $.

\begin{theorem}[weak Kolyvagin '90]
$ \Sel_\ell(E / K) = \F_\ell \cdot \delta(P_K) $, so $ \rk_\Z E(K) \le 1 $ and $ \#\Sha(E / K)[\ell] < \infty $.
\end{theorem}

\bigskip When $ E $ has no complex multiplication, this excludes finitely many primes by Serre's theorem, so this proves that $ \widehat{h}(P_K) \ne 0 $ implies $ \rk_\Z E(K) = 1 $. Kolyvagin proves $ \#\Sha(E / K) < \infty $ by refining the argument for these primes and bounding the $ \ell $-primary components using Iwasawa theory.

\end{frame}

\begin{frame}{Selmer structures}

Let $ M $ be a discrete finite irreducible self-dual $ \F_\ell[G_K] $-module.

\bigskip The inflation-restriction exact sequence says
$$ 0 \to H^1(G_p^\nr, M^{I_p}) \to H^1(K_p, M) \to H^1(I_p, M)^{G_p^\nr} \to 0. $$
For $ M = E[\ell] $ and good $ p \nmid \ell $, this can be identified with
$$ 0 \to E(K_p) / \ell E(K_p) \to H^1(K_p, E[\ell]) \to H^1(K_p, E)[\ell] \to 0. $$
More generally, a \textbf{Selmer structure} for $ (K, M) $ is an assignment
$$ p \longmapsto H_f^1(K_p, M) \subseteq H^1(K_p, M), $$
such that $ H_f^1(K_p, M) = H^1(G_p^\nr, M^{I_p}) $ for all but finitely many places $ p $ of $ K $. Its associated \textbf{singular quotient} $ H_s^1(K_p, M) $ sits in
$$ 0 \to H_f^1(K_p, M) \to H^1(K_p, M) \xrightarrow{(\cdot)^s} H_s^1(K_p, M) \to 0. $$

\end{frame}

\begin{frame}{Selmer groups}

The \textbf{Selmer group} $ \Sel := \Sel(K, M) $ sits in
$$ 0 \to \Sel(K, M) \to H^1(K, M) \xrightarrow{\prod_p (\cdot)_p^s} \prod_p H_s^1(K_p, M). $$
For $ M = E[\ell] $ and $ H_f^1(K_p, M) = E(K_p) / \ell E(K_p) $, this is just $ \Sel_\ell(E / K) $.

\bigskip Let $ S $ be a finite set of places of $ K $.
\begin{itemize}
\item The \textbf{relaxed Selmer group} $ \Sel^S := \Sel^S(K, M) $ sits in
$$ 0 \to \Sel(K, M) \to \Sel^S(K, M) \xrightarrow{\bigoplus_{p \in S} (\cdot)_p^s} \bigoplus_{p \in S} H_s^1(K_p, M). $$
\item The \textbf{restricted Selmer group} $ \Sel_S := \Sel_S(K, M) $ sits in
$$ 0 \to \Sel_S(K, M) \to \Sel(K, M) \xrightarrow{\bigoplus_{p \in S} (\cdot)_p} \bigoplus_{p \in S} H_f^1(K_p, M). $$
\end{itemize}

\end{frame}

\begin{frame}{Duality of Selmer groups}

\begin{corollary}[of Tate duality]
\vspace{-0.5cm}
$$
\begin{tikzcd}[ampersand replacement=\&, column sep=small, row sep=0.1in]
0 \arrow{r} \& \Sel \arrow{r} \& \Sel^S \arrow{r} \& \displaystyle\bigoplus_{p \in S} H_s^1(K_p, M) \arrow[dash]{d}{\mid} \& \& \& \\
\& \& \& \displaystyle\bigoplus_{p \in S} H_f^1(K_p, M)^\vee \arrow{r} \& \Sel^\vee \arrow{r} \& \Sel_S^\vee \arrow{r} \& 0.
\end{tikzcd}
$$
\vspace{-0.5cm}
\end{corollary}

\begin{proof}
Local Tate duality gives a perfect pairing $ H_s^1(K_p, M) \times H_f^1(K_p, M) \to \F_\ell $. The Poitou--Tate exact sequence gives exactness at
$$ \Sel^S \to \bigoplus_{p \in S} H^1(K_p, M) \to \Sel^{S\vee}. $$
Now apply the snake lemma and diagram chase.
\end{proof}

\end{frame}

\begin{frame}{Complex conjugation on Selmer groups}

To compute $ \Sel $, it suffices to consider the last three terms
$$ 0 \to \coker\left(\Sel^S \to \bigoplus_{p \in S} H_s^1(K_p, M)\right) \to \Sel^\vee \to \Sel_S^\vee \to 0, $$
for some appropriate finite set of places $ S $ of $ K $.

\bigskip If $ \tau \in G_\Q $ is an involution with non-zero eigenspaces $ M^+ $ and $ M^- $, then
$$ 0 \to \coker\left(\Sel^{S_1+} \to \bigoplus_{p \in S_1} H_s^1(K_p, M)^+\right) \to \Sel^{+\vee} \to \Sel_{S_1}^{+\vee} \to 0, $$
$$ 0 \to \coker\left(\Sel^{S_2-} \to \bigoplus_{p \in S_2} H_s^1(K_p, M)^-\right) \to \Sel^{-\vee} \to \Sel_{S_2}^{-\vee} \to 0, $$
for some appropriate finite sets of places $ S_1 $ and $ S_2 $ of $ K $.

\end{frame}

\begin{frame}{Computing Selmer groups}

Now consider $ M = E[\ell] $.

\begin{corollary}[of Chebotarev density]
There is a finite set $ S $ of primes of $ \Q $ inert in $ K $ such that
$$ \coker\left(\underbrace{\Sel^{S-\epsilon}}_{\bigoplus_{p \in S} \F_\ell \cdot c(p)_p^s} \to \bigoplus_{p \in S} \underbrace{H^1(K_p, E)[\ell]^{-\epsilon}}_{\F_\ell \cdot c(p)_p^s}\right) \to \Sel^{-\epsilon\vee} \to \underbrace{\Sel_S^{-\epsilon\vee}}_0. $$
For any $ p \in S $, there is a finite set $ S_p $ of primes of $ \Q $ inert in $ K $ such that
$$ \coker\left(\underbrace{\Sel^{S_p\epsilon}}_{\bigoplus_{q \in S_p} \F_\ell \cdot c(pq)_q^s} \to \bigoplus_{q \in S_p} \underbrace{H^1(K_p, E)[\ell]^\epsilon}_{\F_\ell \cdot c(pq)_q^s}\right) \to \Sel^{\epsilon\vee} \to \underbrace{\Sel_{S_p}^{\epsilon\vee}}_{\F_\ell \cdot \delta(P_K)} \to 0. $$
\vspace{-0.5cm}
\end{corollary}

\begin{proof}
Chebotarev density and a lot of Galois cohomology.
\end{proof}

\end{frame}

\begin{frame}{Derivative operators}

The classes $ c(n) \in H^1(K, E[\ell]) $ are derived from $ P_n \in E(H_n) $.

\bigskip It suffices to let $ n $ be a product of primes $ p \nmid ND\ell $ inert in $ K $, so
$$ \Gal(H_n / H_1) \cong \prod_{p \mid n} \Gal(H_p / H_1) \cong \prod_{p \mid n} \Z / (p + 1)\Z \cdot \sigma_p. $$
Define the \textbf{derivative operator} $ D_n \in \Z[\Gal(H_n / H_1)] $ by
$$ D_n := \prod_{p \mid n} D_p, $$
where $ D_p $ is any solution to $ (\sigma_p - 1)D_p = p + 1 - \tr_{H_1}^{H_p} $, and define
$$ \PP_n := \sum_{\tau \in T_n} \tau(D_nP_n), $$
where $ T_n $ is a set of coset representatives for $ \Gal(H_n / H_1) $ in $ \Gal(H_n / K) $.

\end{frame}

\begin{frame}{Derived classes}

\begin{lemma}
The class of $ \PP_n $ in $ E(H_n) / \ell E(H_n) $ is invariant under the action of $ G_n := \Gal(H_n / K) $ and lies in the $ \epsilon_n := \epsilon \cdot (-1)^{\sigma_0(n)} $ eigenspace.
\end{lemma}

\begin{proof}
Norm compatibilities and Lemma $ (\tau) $.
\end{proof}

\bigskip Define the \textbf{derived class} $ c(n) \in H^1(K, E[\ell])^{\epsilon_n} $ by $ \res_n(c(n)) = \delta_n(\PP_n) $ in
$$
\begin{tikzcd}[ampersand replacement=\&, column sep=small, row sep=small]
\& H^1(G_n, E(H_n)[\ell])^{\epsilon_n} = 0 \arrow{d}{\inf_n} \& \\
H_f^1(K, E[\ell])^{\epsilon_n} \arrow{r}{\delta} \arrow{d} \& H^1(K, E[\ell])^{\epsilon_n} \arrow{r} \arrow{d}{\res_n} \& H_s^1(K, E[\ell])^{\epsilon_n} \arrow{d} \\
H_f^1(H_n, E[\ell])^{G_n\epsilon_n} \arrow{r}[swap]{\delta_n} \& H^1(H_n, E[\ell])^{G_n\epsilon_n} \arrow{r} \arrow{d}{\tra_n} \& H_s^1(H_n, E[\ell])^{G_n\epsilon_n} \\
\& H^2(G_n, E(H_n)[\ell])^{\epsilon_n} = 0. \&
\end{tikzcd}
$$

\end{frame}

\begin{frame}{Ramification of derived classes}

\begin{lemma}
\begin{enumerate}
\item If $ p \nmid n $, then $ c(n)_p^s = 0 $, so $ c(n) \in \Sel^{\{p \mid n\}\epsilon_n} $.
\item If $ p \mid n $, then $ c(n)_p^s = 0 $ if and only if $ \PP_{n / p} \in \ell E(K_p) $.
\end{enumerate}
\end{lemma}

\begin{proof}[Proof of 1 for good $ p \nmid \ell $]
Note that $ H_s^1(I_p, E[\ell]) = \Hom(I_p, E[\ell])^{G_p^\nr} $. Since $ (H_n)_p / K $ is unramified at $ p $, the inertia subgroups of $ K_p $ and $ (H_n)_p $ are both $ I_p $, so
$$
\begin{tikzcd}[ampersand replacement=\&, row sep=0.1in]
H_f^1(K_p, E[\ell]) \arrow{r} \arrow{d} \& H^1(K_p, E[\ell]) \arrow{r}{(\cdot)^s} \arrow{d}{\res_n} \& \Hom(I_p, E[\ell]) \arrow[dash]{d}{\mid} \\
H_f^1((H_n)_p, E[\ell]) \arrow{r}[swap]{\delta_n} \& H^1((H_n)_p, E[\ell]) \arrow{r}[swap]{(\cdot)^s} \& \Hom(I_p, E[\ell]).
\end{tikzcd}
$$
Thus $ c(n)_p^s = (\res_n(c(n)_p))^s = 0 $ by exactness.
\end{proof}

\bigskip Note that 2 is precisely the reason for the assumption $ P_K \notin \ell E(K) $.

\end{frame}

\begin{frame}[c]{References}

Different accounts of Kolyvagin's paper on Euler systems:
\begin{itemize}
\item K Rubin (1989) The work of Kolyvagin on the arithmetic of elliptic curves
\item B Gross (1991) Kolyvagin's work on modular elliptic curves
\item T Weston (2001) The Euler system of Heegner points
\item H Darmon (2004) Rational points on modular elliptic curves
\end{itemize}
Relevant papers on non-vanishing of L-functions:
\begin{itemize}
\item J-L Waldspurger (1985) Sur les valeurs de certaines fonctions L automorphes en leur centre de symetrie
\item D Bump, S Friedberg, and J Hoffstein (1990) Nonvanishing theorems for L-functions of modular forms
and their derivatives
\item M R Murty and V K Murty (1991) Mean values of derivatives of modular L-series
\item M R Murty and V K Murty (1997) Non-vanishing of L-functions and applications
\end{itemize}

\end{frame}

\end{document}