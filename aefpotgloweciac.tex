\ifx\type\undefined
  \documentclass[10pt, t]{beamer}
  \setbeamertemplate{footline}[page number]
\else
  \documentclass[10pt]{article}
  \usepackage[margin=1in]{geometry}
\fi

\usepackage{amsmath}
\usepackage{amssymb}
\usepackage{amsthm}
\usepackage{bbm}
\usepackage{cancel}
\usepackage{listings}
\usepackage{mathrsfs}
\usepackage{multirow}
\usepackage{soul}
\usepackage{stmaryrd}
\usepackage{tikz}
\usepackage{tikz-cd}
\usepackage{wrapfig}

\newtheorem*{algorithm}{Algorithm}
\newtheorem*{assumptions}{Assumptions}
\newtheorem*{conjecture}{Conjecture}
\newtheorem*{consequences}{Consequences}
\newtheorem*{exercise}{Exercise}
\newtheorem*{formalisation}{Formalisation}
\newtheorem*{proposition}{Proposition}
\newtheorem*{question}{Question}
\newtheorem*{remark}{Remark}

\ifx\type\undefined\else
  \newtheorem*{definition}{Definition}
  \newtheorem*{example}{Example}
  \newtheorem*{lemma}{Lemma}
  \newtheorem*{theorem}{Theorem}
\fi

\definecolor{keywordcolor}{rgb}{0.7, 0.1, 0.1}
\definecolor{tacticcolor}{rgb}{0.0, 0.1, 0.6}
\definecolor{commentcolor}{rgb}{0.4, 0.4, 0.4}
\definecolor{symbolcolor}{rgb}{0.0, 0.1, 0.6}
\definecolor{sortcolor}{rgb}{0.1, 0.5, 0.1}
\definecolor{attributecolor}{rgb}{0.7, 0.1, 0.1}
\def\lstlanguagefiles{lstlean.tex}
\lstset{language=lean}

\newcommand\A{\mathbb{A}}
\newcommand\C{\mathbb{C}}
\newcommand\F{\mathbb{F}}
\newcommand\G{\mathbb{G}}
\renewcommand\H{\mathbb{H}}
\newcommand\I{\mathbb{I}}
\newcommand\N{\mathbb{N}}
\renewcommand\P{\mathbb{P}}
\newcommand\Q{\mathbb{Q}}
\newcommand\R{\mathbb{R}}
\newcommand\Z{\mathbb{Z}}

\renewcommand\AA{\mathcal{A}}
\newcommand\BB{\mathcal{B}}
\newcommand\CC{\mathcal{C}}
\newcommand\DD{\mathcal{D}}
\newcommand\EE{\mathcal{E}}
\newcommand\FF{\mathcal{F}}
\newcommand\GG{\mathcal{G}}
\newcommand\HH{\mathcal{H}}
\newcommand\II{\mathcal{I}}
\newcommand\LL{\mathcal{L}}
\newcommand\MM{\mathcal{M}}
\newcommand\NN{\mathcal{N}}
\newcommand\OO{\mathcal{O}}
\newcommand\PP{\mathcal{P}}
\newcommand\RR{\mathcal{R}}
\renewcommand\SS{\mathcal{S}}
\newcommand\TT{\mathcal{T}}
\newcommand\XX{\mathcal{X}}

\renewcommand\aa{\mathfrak{a}}
\newcommand\cc{\mathfrak{c}}
\newcommand\dd{\mathfrak{d}}
\newcommand\ff{\mathfrak{f}}
\renewcommand\gg{\mathfrak{g}}
\newcommand\mm{\mathfrak{m}}
\newcommand\pp{\mathfrak{p}}
\newcommand\qq{\mathfrak{q}}
\renewcommand\ss{\mathfrak{s}}

\newcommand\LLL{\mathscr{L}}

\newcommand\ab{\mathrm{ab}}
\newcommand\Ab{\mathbf{Ab}}
\newcommand\Alg{\mathbf{Alg}}
\newcommand\Aff{\mathbf{Aff}}
\newcommand\Aut{\operatorname{Aut}}
\newcommand\Az{\mathrm{Az}}
\newcommand\Br{\operatorname{Br}}
\newcommand\BSD{\operatorname{BSD}}
\newcommand\ch{\operatorname{char}}
\newcommand\Cl{\operatorname{Cl}}
\newcommand\coker{\operatorname{coker}}
\newcommand\cris{\mathrm{cris}}
\renewcommand\d{\mathrm{d}}
\newcommand\Div{\operatorname{Div}}
\newcommand\dR{\mathrm{dR}}
\newcommand\EN{\operatorname{EN}}
\newcommand\End{\operatorname{End}}
\newcommand\ES{\operatorname{ES}}
\newcommand\et{\mathrm{\acute{e}t}}
\newcommand\Et{\mathbf{\acute{E}t}}
\newcommand\Ext{\operatorname{Ext}}
\newcommand\Fr{\operatorname{Fr}}
\newcommand\Frac{\operatorname{Frac}}
\newcommand\Gal{\operatorname{Gal}}
\newcommand\GL{\operatorname{GL}}
\newcommand\Gr{\mathrm{Gr}}
\newcommand\Hom{\operatorname{Hom}}
\newcommand\HT{\mathrm{HT}}
\newcommand\id{\operatorname{id}}
\newcommand\im{\operatorname{im}}
\newcommand\Ind{\operatorname{Ind}}
\renewcommand\inf{\operatorname{inf}}
\newcommand\inv{\operatorname{inv}}
\newcommand\Irr{\operatorname{Irr}}
\newcommand\Jac{\operatorname{Jac}}
\newcommand\lcm{\operatorname{lcm}}
\newcommand\Mat{\operatorname{Mat}}
\newcommand\Mod{\mathbf{Mod}}
\newcommand\Nm{\operatorname{Nm}}
\newcommand\nr{\mathrm{nr}}
\newcommand\NS{\operatorname{NS}}
\newcommand\Ob{\operatorname{Ob}}
\newcommand\ord{\operatorname{ord}}
\newcommand\op{\mathrm{op}}
\newcommand\PGL{\operatorname{PGL}}
\newcommand\Pic{\operatorname{Pic}}
\newcommand\Prob{\operatorname{Prob}}
\newcommand\Proj{\operatorname{Proj}}
\newcommand\PSh{\mathbf{PSh}}
\newcommand\Reg{\operatorname{Reg}}
\newcommand\res{\operatorname{res}}
\newcommand\rk{\operatorname{rk}}
\newcommand\Sch{\mathbf{Sch}}
\newcommand\Sel{\operatorname{Sel}}
\newcommand\Set{\mathbf{Set}}
\newcommand\sgn{\operatorname{sgn}}
\newcommand\Sh{\mathbf{Sh}}
\newcommand\SL{\operatorname{SL}}
\newcommand\Spec{\operatorname{Spec}}
\newcommand\supp{\operatorname{supp}}
\newcommand\Tam{\operatorname{Tam}}
\newcommand\Top{\mathbf{Top}}
\newcommand\tor{\operatorname{tor}}
\newcommand\tr{\operatorname{tr}}
\newcommand\tra{\operatorname{tra}}
\newcommand\WC{\operatorname{WC}}

\DeclareFontFamily{U}{wncyr}{}
\DeclareFontShape{U}{wncyr}{m}{n}{<->wncyr10}{}
\DeclareSymbolFont{cyr}{U}{wncyr}{m}{n}
\DeclareMathSymbol{\Sha}{\mathord}{cyr}{"58}

\newcommand{\function}[5][]{
  \if &#1&
    \begin{array}{rcl}
      #2 & \longrightarrow & #3 \\
      #4 & \longmapsto     & #5
    \end{array}
  \else
    \begin{array}{rcrcl}
      #1 & : & #2 & \longrightarrow & #3 \\
         &   & #4 & \longmapsto     & #5
    \end{array}
  \fi
}

\newcommand{\functions}[7][]{
  \if &#1&
    \begin{array}{rcl}
      #2 & \longrightarrow & #3 \\
      #4 & \longmapsto     & #5 \\
      #6 & \longmapsto     & #7 \\
    \end{array}
  \else
    \begin{array}{rcrcl}
      #1 & : & #2 & \longrightarrow & #3 \\
         &   & #4 & \longmapsto     & #5 \\
         &   & #6 & \longmapsto     & #7
    \end{array}
  \fi
}
\usetheme{warsaw}
\title{The Group Law on Weierstrass Elliptic Curves}
\subtitle{An Elementary Formal Proof in Any Characteristic}
\author[David Ang]{\textbf{David Kurniadi Angdinata} \inst{1} \and Junyan Xu \inst{2}}
\institute{\inst{1} \textbf{London School of Geometry and Number Theory, UK} \and \inst{2} Cancer Data Science Laboratory, National Cancer Institute, Bethesda, MD, USA}
\date{Fourteenth International Conference on Interactive Theorem Proving \\ \bigskip Wednesday, 2 August 2023}

\begin{document}

\frame\maketitle

\begin{frame}{Elliptic curves}

An \textbf{elliptic curve} over a field $ F $ is a pair $ (E, \OO) $:
\begin{itemize}
\item $ E $ is a \emph{smooth projective curve} of \emph{genus one} defined over $ F $
\item $ \OO $ is a distinguished point on $ E $ defined over $ F $
\end{itemize}

\begin{center}
\includegraphics[width=0.3\textwidth]{img/ellipticcurve.png}
\end{center}

Applications:
\begin{itemize}
\item computational mathematics
\begin{itemize}
\item primality testing, integer factorisation, public-key cryptography
\end{itemize}
\item algebraic geometry and number theory
\begin{itemize}
\item Fermat's last theorem, the Birch and Swinnerton-Dyer conjecture
\end{itemize}
\end{itemize}

\end{frame}

\begin{frame}{Weierstrass equations}

\begin{theorem}[corollary of \emph{Riemann--Roch}]
Any elliptic curve $ E $ over $ F $ can be given by $ E(X, Y) = 0 $, where
$$ E(X, Y) := Y^2 + a_1XY + a_3Y - (X^3 + a_2X^2 + a_4X + a_6), $$
for some $ a_i \in F $ such that $ \Delta(a_i) \ne 0 $, with $ \OO $ the point at infinity.
\end{theorem}

This is the \textbf{Weierstrass model} for $ E $, but $ E $ has other models.
\begin{itemize}
\item If $ \ch(F) \ne 2, 3 $, then $ E $ has a \textbf{short Weierstrass model}
$$ E(X, Y) := Y^2 - (X^3 + aX + b), \qquad a, b \in F, $$
where $ \Delta(a, b) = -16(4a^3 + 27b^2) $.
\item If $ \ch(F) \ne 2 $, then $ E $ has an \textbf{Edwards model}
$$ E(X, Y) := X^2 + Y^2 - (1 + dX^2Y^2), \qquad d \in F \setminus \{0, 1\}, $$
with $ \OO := (1, 0) $.
\end{itemize}

\end{frame}

\begin{frame}[fragile]{Weierstrass equations}

\begin{theorem}[corollary of \emph{Riemann--Roch}]
Any elliptic curve $ E $ over $ F $ can be given by $ E(X, Y) = 0 $, where
$$ E(X, Y) := Y^2 + a_1XY + a_3Y - (X^3 + a_2X^2 + a_4X + a_6), $$
for some $ a_i \in F $ such that $ \Delta(a_i) \ne 0 $, with $ \OO $ the point at infinity.
\end{theorem}

In the Weierstrass model, an \textbf{elliptic curve} over $ F $ is the data of:
\begin{itemize}
\item five coefficients $ a_1, a_2, a_3, a_4, a_6 \in F $, and
\item a proof that $ \Delta(a_1, a_2, a_3, a_4, a_6) \ne 0 $.
\end{itemize}

\begin{lstlisting}[backgroundcolor=\color{lime}, basicstyle=\scriptsize, frame=single]
structure weierstrass_curve (F : Type) := (a₁ a₂ a₃ a₄ a₆ : F)

def weierstrass_curve.Δ {F : Type} [comm_ring F] (W : weierstrass_curve F) : F :=
  -(E.a₁^2 + 4*E.a₂)*(E.a₁^2*E.a₆ + 4*E.a₂*E.a₆ - E.a₁*E.a₃*E.a₄ + E.a₂*E.a₃^2 - E.a₄^2)
    - 8*(2*E.a₄ + E.a₁*E.a₃)^3 - 27*(E.a₃^2 + 4*E.a₆)^2
    + 9*(E.a₁^2 + 4*E.a₂)*(2*E.a₄ + E.a₁*E.a₃)*(E.a₃^2 + 4*E.a₆)

structure elliptic_curve (F : Type) [comm_ring F] extends weierstrass_curve F :=
  (Δ' : units F) (coe_Δ' : ↑Δ' = to_weierstrass_curve.Δ)
\end{lstlisting}

\end{frame}

\begin{frame}[fragile]{Weierstrass equations}

\begin{theorem}[corollary of \emph{Riemann--Roch}]
Any elliptic curve $ E $ over $ F $ can be given by $ E(X, Y) = 0 $, where
$$ E(X, Y) := Y^2 + a_1XY + a_3Y - (X^3 + a_2X^2 + a_4X + a_6), $$
for some $ a_i \in F $ such that $ \Delta(a_i) \ne 0 $, with $ \OO $ the point at infinity.
\end{theorem}

In the Weierstrass model, a \textbf{point} on $ E $ is either:
\begin{itemize}
\item the point at infinity $ \OO $, or
\item two affine coordinates $ x, y \in F $ and a proof that $ (x, y) \in E $.
\end{itemize}

\begin{lstlisting}[backgroundcolor=\color{lime}, basicstyle=\scriptsize, frame=single]
variables {F : Type} [field F] (E : elliptic_curve F)

def polynomial : F[X][Y] :=
  Y^2 + C (C E.a₁*X + C E.a₃)*Y - C (X^3 + C E.a₂*X^2 + C E.a₄*X + C E.a₆)

def equation (x y : F) : Prop := (E.polynomial.eval (C y)).eval x = 0

inductive point
  | zero
  | some {x y : F} (h : E.equation x y)
\end{lstlisting}

\end{frame}

\begin{frame}[fragile]{Group law}

\begin{theorem}[the group law]
The points of $ E $ form an abelian group under a geometric addition law.
\end{theorem}

Identity is given by $ \OO $.

\begin{lstlisting}[backgroundcolor=\color{lime}, basicstyle=\scriptsize, frame=single]
instance : has_zero E.point := ⟨zero⟩
\end{lstlisting}

Negation and addition are characterised by
$$ P + Q + R = 0 \qquad \iff \qquad P, Q, R \ \text{are collinear}. $$

\begin{center}
\includegraphics[width=0.7\textwidth]{img/grouplaw.png}
\end{center}

\end{frame}

\begin{frame}[fragile]{Group law}

\begin{theorem}[the group law]
The points of $ E $ form an abelian group under a geometric addition law.
\end{theorem}

Negation is given by $ -(x, y) := (x, \sigma(y)) $, where
$$ \sigma(Y) := -Y - a_1X - a_3. $$

\begin{lstlisting}[backgroundcolor=\color{lime}, basicstyle=\scriptsize, frame=single]
def neg_Y (x y : F) : F := -y - E.a₁ * x + E.a₃

lemma equation_neg {x y : F} : E.equation x y → E.equation x (E.neg_Y x y) := ...

def neg : E.point → E.point
  | zero := zero
  | (some h) := some (equation_neg h)

instance : has_neg E.point := ⟨neg⟩
\end{lstlisting}

\underline{\textbf{Note:}} in the \textbf{coordinate ring} $ F[E] := F[X, Y] / (E(X, Y)) $,
$$ -(Y \cdot \sigma(Y)) = Y^2 + a_1XY + a_3Y \equiv X^3 + a_2X^2 + a_4X + a_6. $$

\end{frame}

\begin{frame}[fragile]{Group law}

\begin{theorem}[the group law]
The points of $ E $ form an abelian group under a geometric addition law.
\end{theorem}

Addition is given by $ (x_1, y_1) + (x_2, y_2) := -(x_3, y_3) $, where
$$ x_3 := \lambda^2 + a_1\lambda - a_2 - x_1 - x_2, \qquad y_3 := \lambda(x_3 - x_1) + y_1. $$

\begin{lstlisting}[backgroundcolor=\color{lime}, basicstyle=\scriptsize, frame=single]
def add : E.point → E.point → E.point
  | zero P := P
  | P zero := P
  | (some h₁) (some h₂) := some (equation_add h₁ h₂)

instance : has_add E.point := ⟨add⟩
\end{lstlisting}

Here,
$$ \lambda :=
\begin{cases}
\dfrac{y_1 - y_2}{x_1 - x_2} & \text{if} \ x_1 \ne x_2, \\[0.1cm]
\dfrac{3x_1^2 + 2a_2x_1 + a_4 - a_1y_1}{y_1 - \sigma(y_1)} & \text{if} \ y_1 \ne \sigma(y_1), \\
\infty & \text{otherwise}.
\end{cases}
$$

\end{frame}

\begin{frame}[fragile]{Attempts at proof}

One may attempt to prove the axioms directly.

\begin{lstlisting}[backgroundcolor=\color{lime}, basicstyle=\scriptsize, frame=single]
instance : add_group E.point :=
  { zero            := zero,
    neg              := neg,
    add              := add,
    zero_add       := rfl,     -- by definition
    add_zero       := rfl,     -- by definition
    add_left_neg := ...,      -- by cases
    add_comm       := ...,      -- by cases
    add_assoc     := sorry } -- seems impossible?
\end{lstlisting}

Associativity is a proof that
$$ (P + Q) + R = P + (Q + R), $$ where each $ + $ has five cases!

\bigskip In the generic case, this is an equality of polynomials with 26,082 terms.

\bigskip In contrast, the \texttt{ring} tactic in Lean can handle at most 1,000 terms.

\end{frame}

\begin{frame}{Attempts at proof}

Associativity is known to be mathematically difficult with many proofs.

\bigskip Proof 1: just do it.
\begin{itemize}
\item elementary but slow
\item several known formalisations
\begin{itemize}
\item Th\'ery (Coq, 2007): short Weierstrass model $ Y^2 = X^3 + aX + b $
\item Hales, Raya (Isabelle, 2020): Edwards model $ X^2 + Y^2 = 1 + dX^2Y^2 $
\item Fox, Gordon, Hurd (HOL4, 2006): long Weierstrass model $ Y^2 + a_1XY + a_3Y = X^3 + a_2X^2 + a_4X + a_6 $ but no associativity
\end{itemize}
\end{itemize}

\bigskip Proof 2: ad-hoc argument with projective geometry.
\begin{itemize}
\item only works generically via \emph{Cayley--Bacharach}
\item no known formalisations
\begin{itemize}
\item our original attempt
\end{itemize}
\end{itemize}

\end{frame}

\begin{frame}{Attempts at proof}

One may instead identify the set of points $ E(F) $ with a known group.

\bigskip Proof 3: identify with a quotient of $ \C $ by the \emph{fundamental lattice} $ \Lambda_E $.
\begin{itemize}
\item only works in characteristic zero via \emph{uniformisation}
\item no known formalisations
\begin{itemize}
\item needs a lot of theory
\end{itemize}
\end{itemize}

\bigskip Proof 4: identify with the \emph{degree zero Weil divisor class group} $ \Pic_F^0(E) $.
\begin{itemize}
\item algebro-geometric and usually uses \emph{Riemann--Roch}
\item one known formalisation
\begin{itemize}
\item Bartzia, Strub (10,000 lines of Coq, 2014): short Weierstrass model
\end{itemize}
\end{itemize}

\bigskip Proof 5: identify with the \emph{ideal class group} $ \Cl(F[E]) $.
\begin{itemize}
\item purely algebraic and uses commutative algebra
\item one known formalisation
\begin{itemize}
\item our final proof (1,000 lines of Lean, 2023): long Weierstrass model
\end{itemize}
\end{itemize}

\end{frame}

\begin{frame}{Sketch of proof}

\begin{proof}[Proof of the group law]
\begin{enumerate}
\item Construct a function $ E(F) \to \Cl(F[E]) $.
\item Prove that $ E(F) \to \Cl(F[E]) $ respects addition.
\item Prove that $ E(F) \to \Cl(F[E]) $ is injective.
\end{enumerate}
\vspace{-0.5cm}
\end{proof}

Here, the \textbf{ideal class group} $ \Cl(R) $ of an integral domain $ R $ is the quotient group of \emph{invertible fractional ideals} by \emph{principal fractional ideals}.

\begin{example}
Any nonzero ideal $ I \trianglelefteq R $ such that $ I \cdot J $ is principal for some ideal $ J \trianglelefteq R $ is an invertible fractional ideal of $ R $.
\end{example}

Ideal class groups were formalised in Lean's mathematical library \texttt{mathlib} by Baanen, Dahmen, Narayanan, Nuccio (2021).

\bigskip \underline{\textbf{Key:}} the coordinate ring $ F[E] $ is an integral domain.

\end{frame}

\begin{frame}{Sketch of proof}

\begin{proof}[Proof of the group law]
\begin{enumerate}
\item Construct a function $ E(F) \to \Cl(F[E]) $. $ \checkmark $
\item Prove that $ E(F) \to \Cl(F[E]) $ respects addition. $ \checkmark $
\item Prove that $ E(F) \to \Cl(F[E]) $ is injective.
\end{enumerate}
\vspace{-0.5cm}
\end{proof}

Consider the function \texttt{point.to\_class} given by
$$ \functions{E(F)}{\Cl(F[E])}{\OO}{[(1)]}{(x, y)}{[(X - x, Y - y)]}. $$
\underline{\textbf{Note:}} $ (X - x, Y - y) $ is invertible, since
$$ (X - x, Y - y) \cdot (X - x, Y - \sigma(y)) = (X - x). $$
The function \texttt{point.to\_class} respects addition, since
$$ (X - x_1, Y - y_1) \cdot (X - x_2, Y - y_2) \cdot (X - x_3, Y - \sigma(y_3)) = (Y - \lambda(X - x_3) - y_3). $$

\end{frame}

\begin{frame}{Proof of injectivity}

\begin{theorem}[Xu, 2022]
The function \texttt{point.to\_class} is injective.
\end{theorem}

\underline{\textbf{Key:}} $ F[E] = F[X, Y] / (E(X, Y)) $ is free over $ F[X] $ with basis $ \{1, Y\} $, so it has a norm $ \Nm : F[E] \to F[X] $ given by $ \Nm(f) := \det([\cdot f]) $.

\begin{lemma}[A]
If $ f \in F[E] $, then $ \deg(\Nm(f)) \ne 1 $.
\end{lemma}

\begin{proof}[Proof of Lemma (A)]
Let $ f = p + qY $ for $ p, q \in F[X] $. Then
\begin{align*}
\Nm(f) & \equiv \det\begin{pmatrix} p & q \\ q(X^3 + a_2X^2 + a_4X + a_6) & p - q(a_1X + a_3) \end{pmatrix} \\
& = p^2 - pq(a_1X + a_3) - q^2(X^3 + a_2X^2 + a_4X + a_6).
\end{align*}
Then $ \deg(\Nm(f)) = \max(2\deg(p), 2\deg(q) + 3) $.
\end{proof}

\end{frame}

\begin{frame}{Proof of injectivity}

\begin{theorem}[Xu, 2022]
The function \texttt{point.to\_class} is injective.
\end{theorem}

\underline{\textbf{Key:}} $ F[E] = F[X, Y] / (E(X, Y)) $ is free over $ F[X] $ with basis $ \{1, Y\} $, so it has a norm $ \Nm : F[E] \to F[X] $ given by $ \Nm(f) := \det([\cdot f]) $.

\begin{lemma}[B]
If $ f \in F[E] $, then $ \deg(\Nm(f)) = \dim_F(F[E] / (f)) $.
\end{lemma}

\begin{proof}[Proof of Lemma (B)]
Multiplication by $ f $ has Smith normal form
$$ [\cdot f] \sim \begin{pmatrix} p & 0 \\ 0 & q \end{pmatrix}, \qquad p, q \in F[X]. $$
\vspace{-0.5cm}
\begin{itemize}
\item Taking determinants gives $ \Nm(f) = pq $.
\item Taking quotients gives $ F[E] / (f) \cong F[X] / (p) \oplus F[X] / (q) $.
\end{itemize}
\vspace{-0.5cm}
\end{proof}

\end{frame}

\begin{frame}{Proof of injectivity}

\begin{theorem}[Xu, 2022]
The function \texttt{point.to\_class} is injective.
\end{theorem}

\underline{\textbf{Key:}} $ F[E] = F[X, Y] / (E(X, Y)) $ is free over $ F[X] $ with basis $ \{1, Y\} $, so it has a norm $ \Nm : F[E] \to F[X] $ given by $ \Nm(f) := \det([\cdot f]) $.

\begin{proof}[Proof of Theorem]
Suffices to show if $ (x, y) \in E(F) $, then $ (X - x, Y - y) $ is not principal.

\bigskip Suppose otherwise that $ (X - x, Y - y) = (f) $ for some $ f \in F[E] $. Then
\vspace{-0.3cm}
$$ F \overset{1^{\text{st}} \text{iso}}{\cong} F[X, Y] / (X - x, Y - y) \overset{3^{\text{rd}} \text{iso}}{\cong} F[E] / (X - x, Y - y) = F[E] / (f). $$
Taking dimensions gives
\vspace{-0.3cm}
$$ 1 = \dim_F(F) = \dim_F(F[E] / (f)) \overset{(B)}{=} \deg(\Nm(f)) \overset{(A)}{\ne} 1. $$
Contradiction!
\end{proof}

\end{frame}

\begin{frame}[c]{Concluding retrospectives}

Some thoughts:
\begin{itemize}
\item proof works for nonsingular points of Weierstrass curves
\item formalisation encouraged proof accessible to undergraduates
\item heavy use of linear algebra and ring theory in \texttt{mathlib}
\item fully integrated to \texttt{mathlib} and even ported to \texttt{mathlib4}
\end{itemize}

\bigskip Some projects:
\begin{itemize}
\item division polynomials, torsion subgroups, and Tate modules
\item elliptic curves over discrete valuation rings and the reduction map
\item verification of computational algorithms and cryptographic protocols
\item equivalence with scheme-theoretic definitions via Riemann--Roch
\item elliptic curves over specific fields: finite fields, local fields, number fields, global function fields, complete fields
\end{itemize}

\end{frame}

\end{document}