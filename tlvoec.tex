\ifx\type\undefined
  \documentclass[10pt, t]{beamer}
  \setbeamertemplate{footline}[page number]
\else
  \documentclass[10pt]{article}
  \usepackage[margin=1in]{geometry}
\fi

\usepackage{amsmath}
\usepackage{amssymb}
\usepackage{amsthm}
\usepackage{bbm}
\usepackage{cancel}
\usepackage{listings}
\usepackage{mathrsfs}
\usepackage{multirow}
\usepackage{soul}
\usepackage{stmaryrd}
\usepackage{tikz}
\usepackage{tikz-cd}
\usepackage{wrapfig}

\newtheorem*{algorithm}{Algorithm}
\newtheorem*{assumptions}{Assumptions}
\newtheorem*{conjecture}{Conjecture}
\newtheorem*{consequences}{Consequences}
\newtheorem*{exercise}{Exercise}
\newtheorem*{formalisation}{Formalisation}
\newtheorem*{proposition}{Proposition}
\newtheorem*{question}{Question}
\newtheorem*{remark}{Remark}

\ifx\type\undefined\else
  \newtheorem*{definition}{Definition}
  \newtheorem*{example}{Example}
  \newtheorem*{lemma}{Lemma}
  \newtheorem*{theorem}{Theorem}
\fi

\definecolor{keywordcolor}{rgb}{0.7, 0.1, 0.1}
\definecolor{tacticcolor}{rgb}{0.0, 0.1, 0.6}
\definecolor{commentcolor}{rgb}{0.4, 0.4, 0.4}
\definecolor{symbolcolor}{rgb}{0.0, 0.1, 0.6}
\definecolor{sortcolor}{rgb}{0.1, 0.5, 0.1}
\definecolor{attributecolor}{rgb}{0.7, 0.1, 0.1}
\def\lstlanguagefiles{lstlean.tex}
\lstset{language=lean}

\newcommand\A{\mathbb{A}}
\newcommand\C{\mathbb{C}}
\newcommand\F{\mathbb{F}}
\newcommand\G{\mathbb{G}}
\renewcommand\H{\mathbb{H}}
\newcommand\I{\mathbb{I}}
\newcommand\N{\mathbb{N}}
\renewcommand\P{\mathbb{P}}
\newcommand\Q{\mathbb{Q}}
\newcommand\R{\mathbb{R}}
\newcommand\Z{\mathbb{Z}}

\renewcommand\AA{\mathcal{A}}
\newcommand\BB{\mathcal{B}}
\newcommand\CC{\mathcal{C}}
\newcommand\DD{\mathcal{D}}
\newcommand\EE{\mathcal{E}}
\newcommand\FF{\mathcal{F}}
\newcommand\GG{\mathcal{G}}
\newcommand\HH{\mathcal{H}}
\newcommand\II{\mathcal{I}}
\newcommand\LL{\mathcal{L}}
\newcommand\MM{\mathcal{M}}
\newcommand\NN{\mathcal{N}}
\newcommand\OO{\mathcal{O}}
\newcommand\PP{\mathcal{P}}
\newcommand\RR{\mathcal{R}}
\renewcommand\SS{\mathcal{S}}
\newcommand\TT{\mathcal{T}}
\newcommand\XX{\mathcal{X}}

\renewcommand\aa{\mathfrak{a}}
\newcommand\cc{\mathfrak{c}}
\newcommand\dd{\mathfrak{d}}
\newcommand\ff{\mathfrak{f}}
\renewcommand\gg{\mathfrak{g}}
\newcommand\mm{\mathfrak{m}}
\newcommand\pp{\mathfrak{p}}
\newcommand\qq{\mathfrak{q}}
\renewcommand\ss{\mathfrak{s}}

\newcommand\LLL{\mathscr{L}}

\newcommand\ab{\mathrm{ab}}
\newcommand\Ab{\mathbf{Ab}}
\newcommand\Alg{\mathbf{Alg}}
\newcommand\Aff{\mathbf{Aff}}
\newcommand\Aut{\operatorname{Aut}}
\newcommand\Az{\mathrm{Az}}
\newcommand\Br{\operatorname{Br}}
\newcommand\BSD{\operatorname{BSD}}
\newcommand\ch{\operatorname{char}}
\newcommand\Cl{\operatorname{Cl}}
\newcommand\coker{\operatorname{coker}}
\newcommand\cris{\mathrm{cris}}
\renewcommand\d{\mathrm{d}}
\newcommand\Div{\operatorname{Div}}
\newcommand\dR{\mathrm{dR}}
\newcommand\EN{\operatorname{EN}}
\newcommand\End{\operatorname{End}}
\newcommand\ES{\operatorname{ES}}
\newcommand\et{\mathrm{\acute{e}t}}
\newcommand\Et{\mathbf{\acute{E}t}}
\newcommand\Ext{\operatorname{Ext}}
\newcommand\Fr{\operatorname{Fr}}
\newcommand\Frac{\operatorname{Frac}}
\newcommand\Gal{\operatorname{Gal}}
\newcommand\GL{\operatorname{GL}}
\newcommand\Gr{\mathrm{Gr}}
\newcommand\Hom{\operatorname{Hom}}
\newcommand\HT{\mathrm{HT}}
\newcommand\id{\operatorname{id}}
\newcommand\im{\operatorname{im}}
\newcommand\Ind{\operatorname{Ind}}
\renewcommand\inf{\operatorname{inf}}
\newcommand\inv{\operatorname{inv}}
\newcommand\Irr{\operatorname{Irr}}
\newcommand\Jac{\operatorname{Jac}}
\newcommand\lcm{\operatorname{lcm}}
\newcommand\Mat{\operatorname{Mat}}
\newcommand\Mod{\mathbf{Mod}}
\newcommand\Nm{\operatorname{Nm}}
\newcommand\nr{\mathrm{nr}}
\newcommand\NS{\operatorname{NS}}
\newcommand\Ob{\operatorname{Ob}}
\newcommand\ord{\operatorname{ord}}
\newcommand\op{\mathrm{op}}
\newcommand\PGL{\operatorname{PGL}}
\newcommand\Pic{\operatorname{Pic}}
\newcommand\Prob{\operatorname{Prob}}
\newcommand\Proj{\operatorname{Proj}}
\newcommand\PSh{\mathbf{PSh}}
\newcommand\Reg{\operatorname{Reg}}
\newcommand\res{\operatorname{res}}
\newcommand\rk{\operatorname{rk}}
\newcommand\Sch{\mathbf{Sch}}
\newcommand\Sel{\operatorname{Sel}}
\newcommand\Set{\mathbf{Set}}
\newcommand\sgn{\operatorname{sgn}}
\newcommand\Sh{\mathbf{Sh}}
\newcommand\SL{\operatorname{SL}}
\newcommand\Spec{\operatorname{Spec}}
\newcommand\supp{\operatorname{supp}}
\newcommand\Tam{\operatorname{Tam}}
\newcommand\Top{\mathbf{Top}}
\newcommand\tor{\operatorname{tor}}
\newcommand\tr{\operatorname{tr}}
\newcommand\tra{\operatorname{tra}}
\newcommand\WC{\operatorname{WC}}

\DeclareFontFamily{U}{wncyr}{}
\DeclareFontShape{U}{wncyr}{m}{n}{<->wncyr10}{}
\DeclareSymbolFont{cyr}{U}{wncyr}{m}{n}
\DeclareMathSymbol{\Sha}{\mathord}{cyr}{"58}

\newcommand{\function}[5][]{
  \if &#1&
    \begin{array}{rcl}
      #2 & \longrightarrow & #3 \\
      #4 & \longmapsto     & #5
    \end{array}
  \else
    \begin{array}{rcrcl}
      #1 & : & #2 & \longrightarrow & #3 \\
         &   & #4 & \longmapsto     & #5
    \end{array}
  \fi
}

\newcommand{\functions}[7][]{
  \if &#1&
    \begin{array}{rcl}
      #2 & \longrightarrow & #3 \\
      #4 & \longmapsto     & #5 \\
      #6 & \longmapsto     & #7 \\
    \end{array}
  \else
    \begin{array}{rcrcl}
      #1 & : & #2 & \longrightarrow & #3 \\
         &   & #4 & \longmapsto     & #5 \\
         &   & #6 & \longmapsto     & #7
    \end{array}
  \fi
}
\title{Twisted L-values of elliptic curves}
\subtitle{75th British Mathematical Colloquium}
\author{David Kurniadi Angdinata}
\institute{London School of Geometry and Number Theory}
\date{Wednesday, 19 June 2024}

\begin{document}

\frame\maketitle

\begin{frame}{L-functions}

Let $ E $ be an elliptic curve over $ \Q $. Let $ K $ be finite Galois over $ \Q $.

\bigskip Recall that the L-function of $ E / K $ is
$$ L(E / K, s) := \prod_\pp \dfrac{1}{\det(1 - \Nm(\pp)^{-s} \cdot \Fr_\pp^{-1} \mid \rho_{E, q}^{\vee I_\pp})}. $$

\begin{conjecture}[Birch--Swinnerton-Dyer]
\begin{itemize}
\item The order of vanishing $ r $ of $ L(E / K, s) $ at $ s = 1 $ is $ \rk(E / K) $.
\item The leading term of $ L(E / K, s) $ at $ s = 1 $ is
$$ \underbrace{\lim_{s \to 1} \dfrac{L(E / K, s)}{(s - 1)^r} \cdot \dfrac{\sqrt{\Delta(K)}}{\Omega(E / K)}}_{\LLL(E / K)} = \underbrace{\dfrac{\Reg(E / K) \cdot \Tam(E / K) \cdot \#\Sha(E / K)}{\#\tor(E / K)^2}.}_{\BSD(E / K)} $$
\end{itemize}
\end{conjecture}

\end{frame}

\begin{frame}{Twisted L-functions}

Artin's formalism for L-functions gives
$$ L(E / K, s) = \prod_{\rho : \Gal(K / \Q) \to \C^\times} L(E, \rho, s)^{\dim\rho}. $$
Here the L-function of $ E $ twisted by an Artin representation $ \rho $ is
$$ L(E, \rho, s) := \prod_p \dfrac{1}{\det(1 - p^{-s} \cdot \Fr_p^{-1} \mid (\rho_{E, q}^\vee \otimes \rho^\vee)^{I_p})}. $$
If $ K $ is abelian, then $ \rho $ corresponds to a Dirichlet character $ \chi $, and
$$ L(E, s) = \sum_{n \in \N} \dfrac{a_n}{n^s} \quad \overset{\chi}{\rightsquigarrow} \quad L(E, \chi, s) = \sum_{n \in \N} \dfrac{a_n\chi(n)}{n^s}. $$

\bigskip What can be said about $ L(E, \rho, s) $ algebraically and analytically?

\end{frame}

\begin{frame}{Algebraic result: twisted conjectures}

\begin{conjecture}[Deligne--Gross]
The order of vanishing of $ L(E, \rho, s) $ at $ s = 1 $ is $ \langle\rho, E(K) \otimes_\Z \C\rangle $.
\end{conjecture}

\bigskip What is the conjectural leading term? Assuming $ L(E, 1) \ne 0 $, define
$$ \LLL(E, \chi) := L(E, \chi, 1) \cdot \dfrac{p}{\tau(\chi) \cdot \Omega(E)}, $$
for any primitive Dirichlet character $ \chi $ of conductor $ p $.

\begin{example}[Dokchitser--Evans--Wiersema 2021]
Let $ E_1 $ and $ E_2 $ be the elliptic curves given by 1356d1 and 1356f1, and let $ \chi $ be the cubic character of conductor $ 7 $ given by $ \chi(3) = \zeta_3^2 $. Then
$$ \Reg(E_i / K) = \Tam(E_i / K) = \Sha(E_i / K) = \tor(E_i / K) = 1, $$
for $ K = \Q $ and $ K = \Q(\zeta_7)^+ $, but $ \LLL(E_1, \chi) = \zeta_3^2 $ and $ \LLL(E_2, \chi) = -\zeta_3^2 $.
\end{example}

\end{frame}

\begin{frame}{Algebraic result: determining units}

Assume $ E $ has conductor $ N $ and satisfies $ c_1(E) = 1 $, and assume $ \chi $ has odd prime conductor $ p \nmid N $ and odd prime order $ q \nmid \#E(\F_p) \cdot \BSD(E) $.

\begin{theorem}[Dokchitser--Evans--Wiersema 2021]
Let $ \zeta := \chi(N)^{(q - 1) / 2} $. Then $ \LLL(E, \chi) \cdot \zeta \in \Z[\zeta_q]^+ \setminus \{0\} $, and has norm
$$ \Nm_\Q^{\Q(\zeta_q)^+}(\LLL(E, \chi) \cdot \zeta) = \pm\underbrace{\sqrt{\dfrac{\BSD(E / K)}{\BSD(E)}}}_B, $$
where $ K $ is the degree $ q $ subfield of $ \Q(\zeta_p) $ cut out by $ \chi $.
\end{theorem}

\begin{theorem}[A. 2024]
If $ q = 3 $, then
$$ \LLL(E, \chi) \cdot \zeta =
\begin{cases}
B & \text{if} \ \#E(\F_p) \cdot \BSD(E) \cdot B^{-1} \equiv 2 \mod 3, \\
-B & \text{if} \ \#E(\F_p) \cdot \BSD(E) \cdot B^{-1} \equiv 1 \mod 3.
\end{cases}
$$
\end{theorem}

\end{frame}

\begin{frame}{Analytic result: numerical evidence}

Assume $ E $ as before, and let $ q $ be an odd prime. As $ p $ varies over odd primes $ p \equiv 1 \mod q $, how does $ \LLL(E, \chi) $ vary, for some uniform choice of primitive Dirichlet characters $ \chi $ of conductor $ p $ and order $ q $?

\begin{example}[$ E = 20a1, q = 3 $]
\vspace{-0.5cm}
$$
\begin{array}{c|cccccccccc}
p & 7 & 13 & 19 & 31 & 37 & 43 & 61 & 67 & 73 & 79 \\
\hline
\LLL(E, \chi) & 2 & -2\zeta_3 & -4 & -6\zeta_3 & -6\zeta_3 & 6\zeta_3 & 2 & -2\zeta_3 & 0 & -6\zeta_3 \\
\mod 3 & 2 & 1 & 2 & 0 & 0 & 0 & 2 & 1 & 0 & 0
\end{array}
$$
\vspace{-0.3cm}
$$
\begin{array}{c|ccccccccc}
p & 97 & 103 & 109 & 127 & 139 & 151 & 157 & 163 & 181 \\
\hline
\LLL(E, \chi) & -4 & -6\zeta_3 & 6\zeta_3 & 6 & 18\zeta_3 & -4 & 30\zeta_3 & 4\zeta_3 & -2\zeta_3 \\
\mod 3 & 2 & 0 & 0 & 0 & 0 & 2 & 0 & 1 & 1
\end{array}
$$
\vspace{-0.2cm}
$$
\begin{array}{c|ccccccccc}
p & 193 & 199 & 211 & 223 & 229 & 241 & 271 & 277 & 283 \\
\hline
\LLL(E, \chi) & -4 & 4\zeta_3 & 10\zeta_3 & -24\zeta_3 & 0 & -14\zeta_3 & -6\zeta_3 & 0 & 6\zeta_3 \\
\mod 3 & 2 & 1 & 1 & 0 & 0 & 1 & 0 & 0 & 0
\end{array}
$$
\end{example}

Kisilevsky--Nam 2022 gave heuristic predictions on the asymptotic distribution of $ \LLL(E, \chi) $, and computed data for the six elliptic curves given by 11a1, 14a1, 15a1, 17a1, 19a1, and 37b1.

\end{frame}

\begin{frame}{Analytic result: residual densities}

Let $ X_{E, q}^{< n} $ be the set of order $ q $ primitive Dirichlet characters $ \chi $ of conductor $ p_\chi < n $ such that $ \chi_1 \equiv \chi_2 $ whenever $ p_{\chi_1} = p_{\chi_2} $. Define
$$ \delta_{E, q}(\lambda) := \lim_{n \to \infty} \dfrac{\#\{\chi \in X_{E, q}^{< n} : \LLL(E, \chi) \equiv \lambda \mod (1 - \zeta_q)\}}{\#X_{E, q}^{< n}}. $$

\begin{theorem}[A. 2024]
Let $ m = 1 - \ord_q(\BSD(E)) $. Then $ \delta_{E, q} $ counts certain matrices in
$$ G_{E, q^m} := \{M \in \im\overline{\rho_{E, q^m}} : \det(M) \equiv 1 \mod q\}. $$
If $ \overline{\rho_{E, q}} $ is surjective, then
$$ \delta_{E, q}(\lambda) =
\begin{cases}
\tfrac{1}{q - 1} & \text{if} \ L_0(q)L_4(q) = 1, \\
\tfrac{q}{q^2 - 1} & \text{if} \ L_0(q)L_4(q) = 0, \\
\tfrac{1}{q + 1} & \text{if} \ L_0(q)L_4(q) = -1,
\end{cases}
\qquad L_n(q) := \left(\dfrac{\tfrac{\lambda}{\BSD(E)} + n}{q}\right). $$
\end{theorem}

\end{frame}

\begin{frame}{Analytic result: explicit algorithm}

\begin{theorem}[A. 2024]
If $ q = 3 $, then $ \delta_{E, 3} $ only depends on $ \im\overline{\rho_{E, 9}} $ and $ b := 3\BSD(E) \bmod 9 $.
\end{theorem}

{\scriptsize $$
\renewcommand{\arraystretch}{1.5}
\begin{array}{|c|c|c|c|c|c|}
\hline
\im\overline{\rho_{E, 3}} \ \text{or} \ \im\overline{\rho_{E, 9}} & b & \delta_{E, 3}(0) & \delta_{E, 3}(1) & \delta_{E, 3}(2) & example \\
\hline
\multirow{2}{*}{$ \GL_2(\F_3) $} & 3 & 3 / 8 & 1 / 4 & 3 / 8 & 11a2 \\
\cline{2-6}
& 6 & 3 / 8 & 3 / 8 & 1 / 4 & \text{11a1} \\
\hline
\multirow{2}{*}{3B, 3Cs} & 3 & 1 / 2 & 0 & 1 / 2 & \text{50b3} \\
\cline{2-6}
& 6 & 1 / 2 & 1 / 2 & 0 & \text{50b1} \\
\hline
\multirow{2}{*}{3Nn} & 3 & 1 / 8 & 3 / 4 & 1 / 8 & \text{704e1} \\
\cline{2-6}
& 6 & 1 / 8 & 1 / 8 & 3 / 4 & \text{245b1} \\
\hline
\multirow{2}{*}{3Ns} & 3 & 1 / 4 & 1 / 2 & 1 / 4 & \text{1690d1} \\
\cline{2-6}
& 6 & 1 / 4 & 1 / 4 & 1 / 2 & \text{338d1} \\
\hline
\text{3.8.0.1} & any & 5 / 9 & 2 / 9 & 2 / 9 & \text{20a1} \\
\hline
\multirow{2}{*}{\begin{tabular}{c} 9.24.0.2, \ 9.72.0.(8,9,10), \\ 27.648.18.1, \ 27.1944.55.(43,44) \end{tabular}} & 1, 4, 7 & 1 / 3 & 2 / 3 & 0 & \text{108a1} \\
\cline{2-6}
& 2, 5, 8 & 1 / 3 & 0 & 2 / 3 & \text{36a1} \\
\hline
\multicolumn{2}{|c|}{any} & 1 & 0 & 0 & 14a1 \\
\hline
\end{array}
$$}

\end{frame}

\begin{frame}{Proof of algebraic result}

Manin's formalism for modular symbols compares $ L(E, 1) $ and $ L(E, \chi, 1) $.

\bigskip The Hecke action on the space of modular symbols gives
$$ -L(E, 1) \cdot \#E(\F_p) = \sum_{a = 1}^{p - 1} \int_0^{\tfrac{a}{p}} 2\pi if_E(z)\d z. $$
On the other hand, Birch's formula can be modified to give
$$ L(E, \chi, 1) = \dfrac{\tau(\chi)}{n}\sum_{a = 1}^{p - 1} \overline{\chi(a)}\int_0^{\tfrac{a}{p}} 2\pi if_E(z)\d z. $$
Scaling appropriately gives a $ \Z[\zeta_q] $ congruence
$$ -\LLL(E) \cdot \#E(\F_p) \equiv \LLL(E, \chi) \mod (1 - \zeta_q), $$
which proves the algebraic result.

\end{frame}

\begin{frame}{Proof of analytic result}

For the analytic result, note that $ \LLL(E, \chi) $ varies according to
$$ \#E(\F_p) = 1 + \det(\rho_{E, q}(\Fr_p)) - \tr(\rho_{E, q}(\Fr_p)) \mod q. $$
Chebotarev's density theorem says that $ \rho_{E, q}(\Fr_p) $ varies uniformly in
$$ G_{E, q^\infty} := \{M \in \im\rho_{E, q} : \det(M) \equiv 1 \mod q\}. $$
The following result reduces the computation from $ G_{E, q^\infty} $ to $ G_{E, q^2} $.

\begin{theorem}[A. 2024]
Let $ q $ be an odd prime. Then $ \ord_q(\LLL(E)) \ge -1 $.
\end{theorem}

\begin{proof}
\begin{itemize}
\item Cancellation of torsion and Tamagawa numbers (Lorenzini 2011)
\item Classification of $ \im(\rho_{E, 3}) $ (Rouse--Sutherland--Zureick-Brown 2022)
\end{itemize}
\end{proof}

\end{frame}

\end{document}