\ifx\type\undefined
  \documentclass[10pt, t]{beamer}
  \setbeamertemplate{footline}[page number]
\else
  \documentclass[10pt]{article}
  \usepackage[margin=1in]{geometry}
\fi

\usepackage{amsmath}
\usepackage{amssymb}
\usepackage{amsthm}
\usepackage{bbm}
\usepackage{cancel}
\usepackage{listings}
\usepackage{mathrsfs}
\usepackage{multirow}
\usepackage{soul}
\usepackage{stmaryrd}
\usepackage{tikz}
\usepackage{tikz-cd}
\usepackage{wrapfig}

\newtheorem*{algorithm}{Algorithm}
\newtheorem*{assumptions}{Assumptions}
\newtheorem*{conjecture}{Conjecture}
\newtheorem*{consequences}{Consequences}
\newtheorem*{exercise}{Exercise}
\newtheorem*{formalisation}{Formalisation}
\newtheorem*{proposition}{Proposition}
\newtheorem*{question}{Question}
\newtheorem*{remark}{Remark}

\ifx\type\undefined\else
  \newtheorem*{definition}{Definition}
  \newtheorem*{example}{Example}
  \newtheorem*{lemma}{Lemma}
  \newtheorem*{theorem}{Theorem}
\fi

\definecolor{keywordcolor}{rgb}{0.7, 0.1, 0.1}
\definecolor{tacticcolor}{rgb}{0.0, 0.1, 0.6}
\definecolor{commentcolor}{rgb}{0.4, 0.4, 0.4}
\definecolor{symbolcolor}{rgb}{0.0, 0.1, 0.6}
\definecolor{sortcolor}{rgb}{0.1, 0.5, 0.1}
\definecolor{attributecolor}{rgb}{0.7, 0.1, 0.1}
\def\lstlanguagefiles{lstlean.tex}
\lstset{language=lean}

\newcommand\A{\mathbb{A}}
\newcommand\C{\mathbb{C}}
\newcommand\F{\mathbb{F}}
\newcommand\G{\mathbb{G}}
\renewcommand\H{\mathbb{H}}
\newcommand\I{\mathbb{I}}
\newcommand\N{\mathbb{N}}
\renewcommand\P{\mathbb{P}}
\newcommand\Q{\mathbb{Q}}
\newcommand\R{\mathbb{R}}
\newcommand\Z{\mathbb{Z}}

\renewcommand\AA{\mathcal{A}}
\newcommand\BB{\mathcal{B}}
\newcommand\CC{\mathcal{C}}
\newcommand\DD{\mathcal{D}}
\newcommand\EE{\mathcal{E}}
\newcommand\FF{\mathcal{F}}
\newcommand\GG{\mathcal{G}}
\newcommand\HH{\mathcal{H}}
\newcommand\II{\mathcal{I}}
\newcommand\LL{\mathcal{L}}
\newcommand\MM{\mathcal{M}}
\newcommand\NN{\mathcal{N}}
\newcommand\OO{\mathcal{O}}
\newcommand\PP{\mathcal{P}}
\newcommand\RR{\mathcal{R}}
\renewcommand\SS{\mathcal{S}}
\newcommand\TT{\mathcal{T}}
\newcommand\XX{\mathcal{X}}

\renewcommand\aa{\mathfrak{a}}
\newcommand\cc{\mathfrak{c}}
\newcommand\dd{\mathfrak{d}}
\newcommand\ff{\mathfrak{f}}
\renewcommand\gg{\mathfrak{g}}
\newcommand\mm{\mathfrak{m}}
\newcommand\pp{\mathfrak{p}}
\newcommand\qq{\mathfrak{q}}
\renewcommand\ss{\mathfrak{s}}

\newcommand\LLL{\mathscr{L}}

\newcommand\ab{\mathrm{ab}}
\newcommand\Ab{\mathbf{Ab}}
\newcommand\Alg{\mathbf{Alg}}
\newcommand\Aff{\mathbf{Aff}}
\newcommand\Aut{\operatorname{Aut}}
\newcommand\Az{\mathrm{Az}}
\newcommand\Br{\operatorname{Br}}
\newcommand\BSD{\operatorname{BSD}}
\newcommand\ch{\operatorname{char}}
\newcommand\Cl{\operatorname{Cl}}
\newcommand\coker{\operatorname{coker}}
\newcommand\cris{\mathrm{cris}}
\renewcommand\d{\mathrm{d}}
\newcommand\Div{\operatorname{Div}}
\newcommand\dR{\mathrm{dR}}
\newcommand\EN{\operatorname{EN}}
\newcommand\End{\operatorname{End}}
\newcommand\ES{\operatorname{ES}}
\newcommand\et{\mathrm{\acute{e}t}}
\newcommand\Et{\mathbf{\acute{E}t}}
\newcommand\Ext{\operatorname{Ext}}
\newcommand\Fr{\operatorname{Fr}}
\newcommand\Frac{\operatorname{Frac}}
\newcommand\Gal{\operatorname{Gal}}
\newcommand\GL{\operatorname{GL}}
\newcommand\Gr{\mathrm{Gr}}
\newcommand\Hom{\operatorname{Hom}}
\newcommand\HT{\mathrm{HT}}
\newcommand\id{\operatorname{id}}
\newcommand\im{\operatorname{im}}
\newcommand\Ind{\operatorname{Ind}}
\renewcommand\inf{\operatorname{inf}}
\newcommand\inv{\operatorname{inv}}
\newcommand\Irr{\operatorname{Irr}}
\newcommand\Jac{\operatorname{Jac}}
\newcommand\lcm{\operatorname{lcm}}
\newcommand\Mat{\operatorname{Mat}}
\newcommand\Mod{\mathbf{Mod}}
\newcommand\Nm{\operatorname{Nm}}
\newcommand\nr{\mathrm{nr}}
\newcommand\NS{\operatorname{NS}}
\newcommand\Ob{\operatorname{Ob}}
\newcommand\ord{\operatorname{ord}}
\newcommand\op{\mathrm{op}}
\newcommand\PGL{\operatorname{PGL}}
\newcommand\Pic{\operatorname{Pic}}
\newcommand\Prob{\operatorname{Prob}}
\newcommand\Proj{\operatorname{Proj}}
\newcommand\PSh{\mathbf{PSh}}
\newcommand\Reg{\operatorname{Reg}}
\newcommand\res{\operatorname{res}}
\newcommand\rk{\operatorname{rk}}
\newcommand\Sch{\mathbf{Sch}}
\newcommand\Sel{\operatorname{Sel}}
\newcommand\Set{\mathbf{Set}}
\newcommand\sgn{\operatorname{sgn}}
\newcommand\Sh{\mathbf{Sh}}
\newcommand\SL{\operatorname{SL}}
\newcommand\Spec{\operatorname{Spec}}
\newcommand\supp{\operatorname{supp}}
\newcommand\Tam{\operatorname{Tam}}
\newcommand\Top{\mathbf{Top}}
\newcommand\tor{\operatorname{tor}}
\newcommand\tr{\operatorname{tr}}
\newcommand\tra{\operatorname{tra}}
\newcommand\WC{\operatorname{WC}}

\DeclareFontFamily{U}{wncyr}{}
\DeclareFontShape{U}{wncyr}{m}{n}{<->wncyr10}{}
\DeclareSymbolFont{cyr}{U}{wncyr}{m}{n}
\DeclareMathSymbol{\Sha}{\mathord}{cyr}{"58}

\newcommand{\function}[5][]{
  \if &#1&
    \begin{array}{rcl}
      #2 & \longrightarrow & #3 \\
      #4 & \longmapsto     & #5
    \end{array}
  \else
    \begin{array}{rcrcl}
      #1 & : & #2 & \longrightarrow & #3 \\
         &   & #4 & \longmapsto     & #5
    \end{array}
  \fi
}

\newcommand{\functions}[7][]{
  \if &#1&
    \begin{array}{rcl}
      #2 & \longrightarrow & #3 \\
      #4 & \longmapsto     & #5 \\
      #6 & \longmapsto     & #7 \\
    \end{array}
  \else
    \begin{array}{rcrcl}
      #1 & : & #2 & \longrightarrow & #3 \\
         &   & #4 & \longmapsto     & #5 \\
         &   & #6 & \longmapsto     & #7
    \end{array}
  \fi
}
\title{Formalisation of elliptic curves in Lean}
\subtitle{Young Researchers in Algebraic Number Theory}
\author{David Kurniadi Angdinata}
\institute{London School of Geometry and Number Theory}
\date{Wednesday, 24 August 2022}

\begin{document}

\frame\maketitle

\begin{frame}[c]{The Lean theorem prover}

\begin{center}
\includegraphics[width=\textwidth]{img/lean.png}
\end{center}

\begin{flushleft}
A functional programming language...
\end{flushleft}

\begin{flushright}
and an interactive theorem prover!
\end{flushright}

\end{frame}

\begin{frame}[fragile]{Programming in Lean}

\begin{center}
Idea: \emph{set theory} is replaced by \texttt{Type Theory}.
$$ \textit{element} \ \in \ \textit{set} \quad \implies \quad \texttt{Term : Type} $$
\end{center}

\bigskip Can define inductive types.

\begin{lstlisting}[basicstyle=\scriptsize, frame=single]
inductive Nat
  | zero : Nat
  | succ : Nat → Nat
\end{lstlisting}

\bigskip Can define functions recursively.

\begin{lstlisting}[basicstyle=\scriptsize, frame=single]
def add : Nat → Nat → Nat
  | n zero := n
  | n (succ m) := succ (add n m)
\end{lstlisting}

\end{frame}

\begin{frame}[fragile]{Programming in Lean}

How to prove $ \forall n \in \N, \ 0 + n = n $?

\bigskip A theorem is a \texttt{Type} (of type \texttt{Prop}).

\begin{lstlisting}[basicstyle=\scriptsize, frame=single]
theorem zero_add : ∀ (n : Nat), add zero n = n :=
\end{lstlisting}

\bigskip A proof of this theorem (if it exists) is the unique \texttt{Term} of this type.

\begin{lstlisting}[basicstyle=\scriptsize, frame=single]
begin
  intro n,
  induction n with m hm,
  { refl },
  { rw [add, hm] }
end
\end{lstlisting}

The keywords \texttt{intro}, \texttt{induction}, \texttt{refl}, and \texttt{rw} are \textbf{tactics}.

\bigskip Play \textbf{The Natural Number Game}!

\end{frame}

\begin{frame}[fragile]{Lean's mathematical library \texttt{mathlib}}

Community-driven unified library of mathematics formalised in Lean.

\bigskip
\begin{minipage}{0.49\textwidth}
\begin{itemize}
\item algebra
\item algebraic\_geometry
\item algebraic\_topology
\item analysis
\item category\_theory
\item combinatorics
\item computability
\item dynamics
\item field\_theory
\item geometry
\item group\_theory
\end{itemize}
\end{minipage}
\begin{minipage}{0.49\textwidth}
\begin{itemize}
\item information\_theory
\item linear\_algebra
\item measure\_theory
\item model\_theory
\item number\_theory
\item order
\item probability
\item representation\_theory
\item ring\_theory
\item set\_theory
\item topology
\end{itemize}
\end{minipage}

\bigskip 3k files, 1m lines, 40k definitions, 100k theorems, 270 contributors.

\end{frame}

\begin{frame}[fragile]{Lean's mathematical library \texttt{mathlib}}

Consider the following theorem in \texttt{group\_theory/quotient\_group}.

\begin{lstlisting}[basicstyle=\scriptsize, frame=single]
variables {G H : Type} [group G] [group H]
variables (φ : G →* H) (ψ : H → G) (hφ : right_inverse ψ φ)

def quotient_ker_equiv_of_right_inverse : G / ker φ ≃* H :=
  { to_fun := ker_lift φ,
    inv_fun := mk ∘ ψ,
    left_inv := ...,
    right_inv := hφ,
    map_mul' := ... }
\end{lstlisting}

Why is this a definition?

\bigskip Consider an immediate corollary.

\begin{lstlisting}[basicstyle=\scriptsize, frame=single]
def quotient_bot : G / (⊥ : subgroup G) ≃* G :=
  quotient_ker_equiv_of_right_inverse (monoid_hom.id G) id (λ _, rfl)
\end{lstlisting}

Why is this not trivial?

\bigskip Canonical isomorphisms are important data!

\end{frame}

\begin{frame}[fragile]{Elliptic curves in Lean}

What generality? Ideally, defined abstractly over a scheme or a ring... However, \texttt{mathlib}'s algebraic geometry is still quite primitive.

\bigskip Here is a working definition in \texttt{algebraic\_geometry/EllipticCurve}.

\begin{lstlisting}[basicstyle=\scriptsize, frame=single]
def Δ_aux {R : Type} [comm_ring R] (a₁ a₂ a₃ a₄ a₆ : R) : R :=
  let
    b₂ := a₁^2 + 4*a₂,
    b₄ := 2*a₄ + a₁*a₃,
    b₆ := a₃^2 + 4*a₆,
    b₈ := a₁^2*a₆ + 4*a₂*a₆ - a₁*a₃*a₄ + a₂*a₃^2 - a₄^2
  in
    -b₂^2*b₈ - 8*b₄^3 - 27*b₆^2 + 9*b₂*b₄*b₆

structure EllipticCurve (R : Type) [comm_ring R] :=
  (a₁ a₂ a₃ a₄ a₆ : R) (Δ : units R) (Δ_eq : ↑Δ = Δ_aux a₁ a₂ a₃ a₄ a₆)
\end{lstlisting}

Accurate for rings $ R $ with $ \Pic(R)[12] = 0 $, such as PIDs!

\bigskip Much can be done just with this definition.

\end{frame}

\begin{frame}[fragile]{Elliptic curves in Lean}

Can define $ K $-points.

\begin{lstlisting}[basicstyle=\scriptsize, frame=single]
variables {F : Type} [field F] (E : EllipticCurve F) (K : Type) [field K] [algebra F K]

inductive point
  | zero
  | some (x y : K) (w : y^2 + E.a₁*x*y + E.a₃*y = x^3 + E.a₂*x^2 + E.a₄*x + E.a₆)

notation E(K) := point E K
\end{lstlisting}

Can define zero.

\begin{lstlisting}[basicstyle=\scriptsize, frame=single]
instance : has_zero E(K) := ⟨zero⟩
\end{lstlisting}

Can define negation.

\begin{lstlisting}[basicstyle=\scriptsize, frame=single]
def neg : E(K) → E(K)
  | zero := zero
  | (some x y w) := some x (-y - E.a₁*x - E.a₃)
    begin
      rw [← w],
      ring
    end

instance : has_neg E(K) := ⟨neg⟩
\end{lstlisting}

\end{frame}

\begin{frame}[fragile]{Elliptic curves in Lean}

Can define $ K $-points.

\begin{lstlisting}[basicstyle=\scriptsize, frame=single]
variables {F : Type} [field F] (E : EllipticCurve F) (K : Type) [field K] [algebra F K]

inductive point
  | zero
  | some (x y : K) (w : y^2 + E.a₁*x*y + E.a₃*y = x^3 + E.a₂*x^2 + E.a₄*x + E.a₆)

notation E(K) := point E K
\end{lstlisting}

Can define addition.

\begin{lstlisting}[basicstyle=\scriptsize, frame=single]
def add : E(K) → E(K) → E(K)
  | zero P := P
  | P zero := P
  | (some x₁ y₁ w₁) (some x₂ y₂ w₂) :=
    if x_ne : x₁ ≠ x₂ then
      let
        L := (y₁ - y₂) / (x₁ - x₂),
        x₃ := L^2 + E.a₁*L - E.a₂ - x₁ - x₂,
        y₃ := -L*x₃ - E.a₁*x₃ - y₁ + L*x₁ - E.a₃
      in
        some x₃ y₃ ... -- 100 lines
    else ... -- 100 lines

instance : has_add E(K) := ⟨add⟩
\end{lstlisting}

\end{frame}

\begin{frame}[fragile]{Elliptic curves in Lean}

Can define $ K $-points.

\begin{lstlisting}[basicstyle=\scriptsize, frame=single]
variables {F : Type} [field F] (E : EllipticCurve F) (K : Type) [field K] [algebra F K]

inductive point
  | zero
  | some (x y : K) (w : y^2 + E.a₁*x*y + E.a₃*y = x^3 + E.a₂*x^2 + E.a₄*x + E.a₆)

notation E(K) := point E K
\end{lstlisting}

Can prove group axioms (except associativity, which is left as a \texttt{sorry}).

\begin{lstlisting}[basicstyle=\scriptsize, frame=single]
lemma zero_add (P : E(K)) : 0 + P = P := ...

lemma add_zero (P : E(K)) : P + 0 = P := ...

lemma add_left_neg (P : E(K)) : -P + P = 0 := ...

lemma add_comm (P Q : E(K)) : P + Q = Q + P := ... -- 100 lines

lemma add_assoc (P Q R : E(K)) : (P + Q) + R = P + (Q + R) := ... -- ?? lines
\end{lstlisting}

Can also prove Galois-theoretic properties and structure of torsion points.

\end{frame}

\begin{frame}{The Mordell--Weil theorem in Lean}

Can prove Mordell's theorem by complete $ 2 $-descent and na\"ive heights.

\begin{theorem}[Mordell]
$ E(\Q) $ is finitely generated.
\end{theorem}

\begin{proof}[Proof ($ E(\Q) / 2E(\Q) $ finite)]
\renewcommand\qedsymbol{}
\begin{itemize}
\item Reduce to $ K \supseteq E[2] $, so that $ y^2 = (x - e_1)(x - e_2)(x - e_3) $.
\item Define the complete $ 2 $-descent homomorphism
$$ \functions{E(K)}{K^\times / (K^\times)^2 \times K^\times / (K^\times)^2}{\OO}{(1, 1)}{(x, y)}{(x - e_1, x - e_2)}. $$
\item Prove its kernel is $ 2E(K) $.
\item Prove its image lies in a Selmer group $ K(S, 2) $.
\item Prove $ 0 \to \OO_K^\times / (\OO_K^\times)^n \to K(\emptyset, n) \to \Cl_K[n] \to 0 $ is exact.
\item Prove $ \Cl_K $ is finite (done) and $ \OO_K^\times $ is finitely generated (soon). $ \square $
\end{itemize}
\end{proof}

\end{frame}

\begin{frame}{The Mordell--Weil theorem in Lean}

Can prove Mordell's theorem by complete $ 2 $-descent and na\"ive heights.

\begin{theorem}[Mordell]
$ E(\Q) $ is finitely generated.
\end{theorem}

\begin{proof}[Proof ($ E(\Q) / 2E(\Q) $ finite $ \implies E(\Q) $ finitely generated)]
\renewcommand\qedsymbol{}
\begin{itemize}
\item Define the na\"ive height
$$ \functions[h]{E(\Q)}{\R}{\OO}{0}{(\tfrac{n}{d}, y)}{\log\max(|n|, |d|)}. $$
\item Prove $ \forall Q \in E(\Q), \ \exists C \in \R, \ \forall P \in E(\Q), \ h(P + Q) \le 2h(P) + C $.
\item Prove $ \exists C \in \R, \ \forall P \in E(\Q), \ 4h(P) \le h(2P) + C $.
\item Prove $ \forall C \in \R $, the set $ \{P \in E(\Q) : h(P) \le C\} $ is finite.
\item Prove the descent theorem (done). $ \square $
\end{itemize}
\vspace{-0.5cm}
\end{proof}

Can finally define the algebraic rank of $ E(\Q) $.

\end{frame}

\begin{frame}[c]{Algebraic number theory in Lean}

Here are some recent developments.

\bigskip

\begin{minipage}{0.49\textwidth}
Completed:
\begin{itemize}
\item Quadratic reciprocity
\item Hensel's lemma
\item UF in Dedekind domains
\item $ \#\Cl_K < \infty $ for global fields
\item Ad\`eles and id\`eles
\item Statement of global CFT
\item L-series of arithmetic functions
\item Bernoulli polynomials
\item Perfectoid spaces
\item Liquid tensor experiment
\end{itemize}
\end{minipage}
\begin{minipage}{0.49\textwidth}
Ongoing:
\begin{itemize}
\item S-unit theorem (\textbf{HELP})
\item FLT for regular primes
\item p-adic L-functions
\item $ B_{\dR} $, $ B_{\HT} $, and $ B_{\cris} $
\item Modular forms
\item \'Etale cohomology
\item Local CFT
\item Statement of BSD
\item Statement of GAGA
\item Statement of R=T
\end{itemize}
\end{minipage}

\end{frame}

\end{document}