\ifx\type\undefined
  \documentclass[10pt, t]{beamer}
  \setbeamertemplate{footline}[page number]
\else
  \documentclass[10pt]{article}
  \usepackage[margin=1in]{geometry}
\fi

\usepackage{amsmath}
\usepackage{amssymb}
\usepackage{amsthm}
\usepackage{bbm}
\usepackage{cancel}
\usepackage{listings}
\usepackage{mathrsfs}
\usepackage{multirow}
\usepackage{soul}
\usepackage{stmaryrd}
\usepackage{tikz}
\usepackage{tikz-cd}
\usepackage{wrapfig}

\newtheorem*{algorithm}{Algorithm}
\newtheorem*{assumptions}{Assumptions}
\newtheorem*{conjecture}{Conjecture}
\newtheorem*{consequences}{Consequences}
\newtheorem*{exercise}{Exercise}
\newtheorem*{formalisation}{Formalisation}
\newtheorem*{proposition}{Proposition}
\newtheorem*{question}{Question}
\newtheorem*{remark}{Remark}

\ifx\type\undefined\else
  \newtheorem*{definition}{Definition}
  \newtheorem*{example}{Example}
  \newtheorem*{lemma}{Lemma}
  \newtheorem*{theorem}{Theorem}
\fi

\definecolor{keywordcolor}{rgb}{0.7, 0.1, 0.1}
\definecolor{tacticcolor}{rgb}{0.0, 0.1, 0.6}
\definecolor{commentcolor}{rgb}{0.4, 0.4, 0.4}
\definecolor{symbolcolor}{rgb}{0.0, 0.1, 0.6}
\definecolor{sortcolor}{rgb}{0.1, 0.5, 0.1}
\definecolor{attributecolor}{rgb}{0.7, 0.1, 0.1}
\def\lstlanguagefiles{lstlean.tex}
\lstset{language=lean}

\newcommand\A{\mathbb{A}}
\newcommand\C{\mathbb{C}}
\newcommand\F{\mathbb{F}}
\newcommand\G{\mathbb{G}}
\renewcommand\H{\mathbb{H}}
\newcommand\I{\mathbb{I}}
\newcommand\N{\mathbb{N}}
\renewcommand\P{\mathbb{P}}
\newcommand\Q{\mathbb{Q}}
\newcommand\R{\mathbb{R}}
\newcommand\Z{\mathbb{Z}}

\renewcommand\AA{\mathcal{A}}
\newcommand\BB{\mathcal{B}}
\newcommand\CC{\mathcal{C}}
\newcommand\DD{\mathcal{D}}
\newcommand\EE{\mathcal{E}}
\newcommand\FF{\mathcal{F}}
\newcommand\GG{\mathcal{G}}
\newcommand\HH{\mathcal{H}}
\newcommand\II{\mathcal{I}}
\newcommand\LL{\mathcal{L}}
\newcommand\MM{\mathcal{M}}
\newcommand\NN{\mathcal{N}}
\newcommand\OO{\mathcal{O}}
\newcommand\PP{\mathcal{P}}
\newcommand\RR{\mathcal{R}}
\renewcommand\SS{\mathcal{S}}
\newcommand\TT{\mathcal{T}}
\newcommand\XX{\mathcal{X}}

\renewcommand\aa{\mathfrak{a}}
\newcommand\cc{\mathfrak{c}}
\newcommand\dd{\mathfrak{d}}
\newcommand\ff{\mathfrak{f}}
\renewcommand\gg{\mathfrak{g}}
\newcommand\mm{\mathfrak{m}}
\newcommand\pp{\mathfrak{p}}
\newcommand\qq{\mathfrak{q}}
\renewcommand\ss{\mathfrak{s}}

\newcommand\LLL{\mathscr{L}}

\newcommand\ab{\mathrm{ab}}
\newcommand\Ab{\mathbf{Ab}}
\newcommand\Alg{\mathbf{Alg}}
\newcommand\Aff{\mathbf{Aff}}
\newcommand\Aut{\operatorname{Aut}}
\newcommand\Az{\mathrm{Az}}
\newcommand\Br{\operatorname{Br}}
\newcommand\BSD{\operatorname{BSD}}
\newcommand\ch{\operatorname{char}}
\newcommand\Cl{\operatorname{Cl}}
\newcommand\coker{\operatorname{coker}}
\newcommand\cris{\mathrm{cris}}
\renewcommand\d{\mathrm{d}}
\newcommand\Div{\operatorname{Div}}
\newcommand\dR{\mathrm{dR}}
\newcommand\EN{\operatorname{EN}}
\newcommand\End{\operatorname{End}}
\newcommand\ES{\operatorname{ES}}
\newcommand\et{\mathrm{\acute{e}t}}
\newcommand\Et{\mathbf{\acute{E}t}}
\newcommand\Ext{\operatorname{Ext}}
\newcommand\Fr{\operatorname{Fr}}
\newcommand\Frac{\operatorname{Frac}}
\newcommand\Gal{\operatorname{Gal}}
\newcommand\GL{\operatorname{GL}}
\newcommand\Gr{\mathrm{Gr}}
\newcommand\Hom{\operatorname{Hom}}
\newcommand\HT{\mathrm{HT}}
\newcommand\id{\operatorname{id}}
\newcommand\im{\operatorname{im}}
\newcommand\Ind{\operatorname{Ind}}
\renewcommand\inf{\operatorname{inf}}
\newcommand\inv{\operatorname{inv}}
\newcommand\Irr{\operatorname{Irr}}
\newcommand\Jac{\operatorname{Jac}}
\newcommand\lcm{\operatorname{lcm}}
\newcommand\Mat{\operatorname{Mat}}
\newcommand\Mod{\mathbf{Mod}}
\newcommand\Nm{\operatorname{Nm}}
\newcommand\nr{\mathrm{nr}}
\newcommand\NS{\operatorname{NS}}
\newcommand\Ob{\operatorname{Ob}}
\newcommand\ord{\operatorname{ord}}
\newcommand\op{\mathrm{op}}
\newcommand\PGL{\operatorname{PGL}}
\newcommand\Pic{\operatorname{Pic}}
\newcommand\Prob{\operatorname{Prob}}
\newcommand\Proj{\operatorname{Proj}}
\newcommand\PSh{\mathbf{PSh}}
\newcommand\Reg{\operatorname{Reg}}
\newcommand\res{\operatorname{res}}
\newcommand\rk{\operatorname{rk}}
\newcommand\Sch{\mathbf{Sch}}
\newcommand\Sel{\operatorname{Sel}}
\newcommand\Set{\mathbf{Set}}
\newcommand\sgn{\operatorname{sgn}}
\newcommand\Sh{\mathbf{Sh}}
\newcommand\SL{\operatorname{SL}}
\newcommand\Spec{\operatorname{Spec}}
\newcommand\supp{\operatorname{supp}}
\newcommand\Tam{\operatorname{Tam}}
\newcommand\Top{\mathbf{Top}}
\newcommand\tor{\operatorname{tor}}
\newcommand\tr{\operatorname{tr}}
\newcommand\tra{\operatorname{tra}}
\newcommand\WC{\operatorname{WC}}

\DeclareFontFamily{U}{wncyr}{}
\DeclareFontShape{U}{wncyr}{m}{n}{<->wncyr10}{}
\DeclareSymbolFont{cyr}{U}{wncyr}{m}{n}
\DeclareMathSymbol{\Sha}{\mathord}{cyr}{"58}

\newcommand{\function}[5][]{
  \if &#1&
    \begin{array}{rcl}
      #2 & \longrightarrow & #3 \\
      #4 & \longmapsto     & #5
    \end{array}
  \else
    \begin{array}{rcrcl}
      #1 & : & #2 & \longrightarrow & #3 \\
         &   & #4 & \longmapsto     & #5
    \end{array}
  \fi
}

\newcommand{\functions}[7][]{
  \if &#1&
    \begin{array}{rcl}
      #2 & \longrightarrow & #3 \\
      #4 & \longmapsto     & #5 \\
      #6 & \longmapsto     & #7 \\
    \end{array}
  \else
    \begin{array}{rcrcl}
      #1 & : & #2 & \longrightarrow & #3 \\
         &   & #4 & \longmapsto     & #5 \\
         &   & #6 & \longmapsto     & #7
    \end{array}
  \fi
}
\title{Dual abelian varieties \footnote{J S Milne (2008) Abelian Varieties}}
\subtitle{Abelian varieties over finite fields}
\author{David Kurniadi Angdinata}
\institute{University College London}
\date{Wednesday, 18 January 2023}

\begin{document}

\frame\maketitle

\begin{frame}{Dual elliptic curves}

Let $ (E, O) $ be an elliptic curve over a field $ K $. Recall that
$$ \function[\lambda_{(O)}]{E}{\Cl^0(E) \le \Cl(E)}{P}{(-P) - (O)}. $$
Here $ \Cl(E) $ is the \textbf{class group} of Weil divisors $ \sum_{P \in E} n_P(P) $ modulo $ \sim $, where $ D \sim 0 $ if $ D $ is the divisor $ (f) $ of some rational function $ f \in \overline{K}(E)^\times $, and $ \Cl^0(E) $ is its subgroup with $ \sum_{P \in E} n_P = 0 $.

\bigskip Idea: for any $ D \in \Cl^0(E) $, the Riemann--Roch space $ \LLL(D + (O)) $, where
$$ \LLL(D) := \{f \in \overline{K}(E)^\times : (f) + D \ge 0\} \cup \{0\}, $$
is one-dimensional, so $ D \sim (-P) - (O) $ for some $ P \in E $.

\bigskip For an elliptic curve $ E $, its \emph{dual} is $ \Cl^0(E) $.

\end{frame}

\begin{frame}{Invertible sheaves on smooth varieties}

Let $ X / K $ be a smooth variety. Then identify
$$
\begin{array}{rcl}
\Cl(X) & \xrightarrow{\sim} & \Pic(X) \\
D & \mapsto & \LL(D).
\end{array}
$$
Here $ \Pic(X) $ is the \textbf{Picard group} of invertible sheaves $ \LL $ modulo $ \cong $, with
$$ \LL \cdot \LL' := \LL \otimes_{\OO_X} \LL', \qquad \LL^{-1} := \HH om(\LL, \OO_X), $$
and $ \LL(D) $ is the sheaf of $ \OO_X $-modules such that for any open $ U \subseteq X $,
$$ \Gamma(U, \LL(D)) := \{f \in K(X)^\times : (f) + D \ge 0 \ \text{in} \ U\} \cup \{0\}. $$
If $ f : Y \to X $ is a morphism, then there is also a \textbf{pull-back}
$$ f^*\LL := f^{-1}\LL \otimes_{f^{-1}\OO_Y} \OO_X \in \Pic(Y). $$

\end{frame}

\begin{frame}{Invertible sheaves on abelian varieties}

Let $ A / K $ be an abelian variety. For any $ a \in A(K) $, the translation map $ \tau_a : A \to A $ induces $ \tau_a^* : \Pic(A) \to \Pic(A) $. For any $ \LL \in \Pic(A) $, define
$$ \function[\lambda_\LL]{A(K)}{\Pic(A)}{a}{\tau_a^*\LL \cdot \LL^{-1}}. $$
This is a homomorphism, by \textbf{theorem of the square}
$$ \tau_{a + b}^*\LL \cdot \LL \cong \tau_a^*\LL \cdot \tau_b^*\LL, \qquad a, b \in A(K). $$
This follows from \textbf{theorem of the cube} \footnote{Theorem I.5.1} that
$$ (f + g + h)^*\LL \cdot (f + g)^*\LL^{-1} \cdot (f + h)^*\LL^{-1} \cdot (g + h)^*\LL^{-1} \cdot f^*\LL \cdot g^*\LL \cdot h^*\LL $$
is trivial for any regular maps $ f, g, h : V \to A $ from a variety $ V / K $.

\bigskip In fact, if $ \LL \in \Pic(A) $ is ample, then $ \ker(\lambda_{\LL}) \le A(K) $ is finite. \footnote{Proposition I.8.1}

\end{frame}

\begin{frame}{Invertible sheaves and Weil divisors}

\begin{remark}
Equivalently, $ \tau_a^* : \Cl(A) \to \Cl(A) $ translates a Weil divisor $ D $ by $ -a $, so
$$ \function[\lambda_{\LL(D)}]{A(K)}{\Cl(A)}{a}{D_{-a} - D}, $$
where $ D_{-a} $ is translation of $ D $ by $ -a $. Theorem of the square becomes
$$ D_{-(a + b)} + D \sim D_{-a} + D_{-b}, \qquad a, b \in A(K). $$
If $ A = E $, then
$$ \function[\lambda_{\LL((O))}]{E(K)}{\Cl(E)}{P}{(-P) - (O)}. $$
In fact, if $ D \in \Cl(E) $ is effective, then $ \deg D = 0 $ iff $ \lambda_{\LL(D)} = 0 $. \footnote{Example I.8.3}
\end{remark}

\end{frame}

\begin{frame}{Translation-invariant invertible sheaves}

Let $ + : A \times A \to A $ be the addition map, and let $ \pi_i : A \times A \to A $ be the projection map to the $ i $-th component.
For any $ \LL \in \Pic(A) $, define
$$ K(\LL) := \{a \in A : (+^*\LL \cdot \pi_1^*\LL^{-1})|_{A \times \{a\}} \cong \OO_A\}. $$
Then $ K(\LL)(K) = \ker(\lambda_\LL) $ as subgroups of $ A $, since
$$ (+^*\LL \cdot \pi_1^*\LL^{-1})|_{A \times \{a\}} = \tau_a^*\LL \cdot \LL^{-1}, \qquad a \in A(K). $$
In fact, $ K(\LL) $ is closed as a subvariety of $ A $. \footnote{Proposition I.5.19}

\bigskip Define the subgroup of \textbf{translation-invariant invertible sheaves}
$$ \Pic^0(A) := \{\LL \in \Pic(A) : K(\LL) = A\}. $$
Then $ \tau_a^*\LL \cdot \LL^{-1} \in \Pic^0(A) $ for any $ a \in A(K) $, so $ \im(\lambda_\LL) \subseteq \Pic^0(A) $.

\bigskip Need an abelian variety $ \widehat{A} $ such that $ \widehat{A}(K) \cong \Pic^0(A) $.

\end{frame}

\begin{frame}{Construction of dual abelian varieties}

Idea: $ \lambda_\LL : A(K) \to \Pic^0(A) $ has kernel $ K(\LL)(K) $, and in fact is surjective if $ \LL \in \Pic(A) $ is ample, \footnote{Proposition I.8.14} so $ \widehat{A} $ should be the quotient variety $ A / K(\LL) $.
\begin{itemize}
\item If $ \ch(K) = 0 $, then $ K(\LL) $ is a reduced subgroup variety of $ A $, and $ A / K(\LL) $ is simply defined as the $ K(\LL) $-orbits of $ A $.
\item If $ \ch(K) \ne 0 $, then $ K(\LL) $ may not be reduced in general, so redefine $ K(\LL) $ as the maximal subscheme of $ A $ such that $ (+^*\LL \cdot \pi_1^*\LL^{-1})|_{A \times K(\LL)} \cong \pi_2^*\LL' $ for some $ \LL' \in \Pic(K(\LL)) $, and $ A / K(\LL) $ is naturally an algebraic space quotient of $ A $.
\end{itemize}
The \textbf{dual abelian variety} of $ A $ is $ \widehat{A} := A / K(\LL) $.

\bigskip

\begin{remark}
Since $ \LL \in \Pic^0(A) $ iff $ +^*\LL \cong \pi_1^*\LL \cdot \pi_2^*\LL $, addition on $ A $ lifts to multiplication on $ \LL $ and makes $ \GG(\LL) := \LL \setminus \{0\} $ an abelian group scheme over $ K $. In fact, $ \GG(\LL) $ is an extension of $ A $ by $ \G_m $, and this defines an isomorphism $ \GG : \Pic^0(A) \xrightarrow{\sim} \Ext_K^1(A, \G_m) $ of abelian group schemes. \footnote{Proposition I.9.3}
\end{remark}

\end{frame}

\begin{frame}{Representability of dual abelian varieties}

Consider the functor $ \FF : \textbf{Var}_K \to \textbf{Set} $ that associates a variety $ V / K $ to the set of isomorphism classes of $ \LL \in \Pic(A \times V) $ such that
\begin{itemize}
\item $ \LL|_{A \times \{x\}} \in \Pic^0(A_x) $ for any $ x \in V $, and
\item $ \LL|_{\{0\} \times V} \cong \OO_V $.
\end{itemize}

\begin{theorem}
$ \widehat{A} $ represents $ \FF $. In other words $ \FF(V) = \Hom(V, \widehat{A}) $ for any variety $ V / K $.
\end{theorem}

\begin{proof}
Sketched in Section I.8.
\end{proof}

\bigskip By construction, $ \widehat{A}(L) = \Pic^0(A_L) $ for any field extension $ L / K $.

\bigskip By universality, $ \widehat{A} $ is unique up to unique isomorphism. Its corresponding universal element is the \textbf{Poincar\'e sheaf} $ \PP_A \in \FF(\widehat{A}) $, which associates any $ \LL \in \Pic^0(A) $ with a unique $ \PP_A|_{A \times \{a\}} $ for some $ a \in \widehat{A}(K) $.

\end{frame}

\begin{frame}{Dualities on abelian varieties}

The functor $ A \mapsto \widehat{A} $ is a duality theory in the sense that $ \widehat{\widehat{A}} \cong A $. This follows from $ \PP_{\widehat{A}} \cong \PP_A $, \footnote{Theorem I.8.9} since $ \PP_A $ parameterises $ \widehat{A}(K) \cong \Pic^0(A) $.

\bigskip Now let $ \phi : A \to B $ be a morphism. Then it has a dual morphism
$$ \function[\widehat{\phi}]{\widehat{B}}{\widehat{A}}{\LL}{\phi^*\LL}. $$
If $ \phi $ is an isogeny, then $ \ker(\widehat{\phi}) = \widehat{\ker(\phi)} $ is the \emph{Cartier dual} of $ \ker(\phi) $, \footnote{Theorem I.9.1} where $ \widehat{\widehat{\ker(\phi)}} \cong \ker(\phi) $. If $ K = K^s $ with $ \ch(K) \nmid n := \#\ker(\phi) $, then
$$ \widehat{\ker(\phi)} = \Hom(\ker(\phi), \mu_n). $$
This defines a \emph{Weil pairing}
$$ e_\phi : \ker(\phi) \times \ker(\widehat{\phi}) \to \mu_n. $$

\end{frame}

\begin{frame}{Polarisations on abelian varieties}

A \textbf{polarisation} on $ A $ is an isogeny $ \lambda : A \to \widehat{A} $ such that $ \lambda = \lambda_\LL $ over $ \overline{K} $ for some ample $ \LL \in \Pic(A_{\overline{K}}) $. It is \textbf{principal} if it has degree one.

\bigskip

\begin{remark}
Zarhin proved that $ (A \times \widehat{A})^4 $ is always principally polarised. \footnote{Theorem I.13.12}
\end{remark}

\bigskip Let $ \lambda : A \to \widehat{A} $ be a polarisation. This defines an involution on $ \End^0(A) $ called the \textbf{Rosati involution} $ (\cdot)^\dagger : \End^0(A) \to \End^0(A) $, where
$$ A \xrightarrow{\phi} A \qquad \longmapsto \qquad A \xrightarrow{\lambda} \widehat{A} \xrightarrow{\widehat{\phi}} \widehat{A} \xrightarrow{\lambda^{-1}} A, $$
which is well-defined since $ \lambda^{-1} \in \Hom^0(\widehat{A}, A) $. It satisfies
$$ (\phi + \psi)^\dagger = \phi^\dagger + \psi^\dagger, \qquad (\phi \circ \psi)^\dagger = \psi^\dagger \circ \phi^\dagger, \qquad \phi, \psi \in \End^0(A), $$
and $ a^\dagger = a $ for any $ a \in \Q $.

\end{frame}

\end{document}