\ifx\type\undefined
  \documentclass[10pt, t]{beamer}
  \setbeamertemplate{footline}[page number]
\else
  \documentclass[10pt]{article}
  \usepackage[margin=1in]{geometry}
\fi

\usepackage{amsmath}
\usepackage{amssymb}
\usepackage{amsthm}
\usepackage{bbm}
\usepackage{cancel}
\usepackage{listings}
\usepackage{mathrsfs}
\usepackage{multirow}
\usepackage{soul}
\usepackage{stmaryrd}
\usepackage{tikz}
\usepackage{tikz-cd}
\usepackage{wrapfig}

\newtheorem*{algorithm}{Algorithm}
\newtheorem*{assumptions}{Assumptions}
\newtheorem*{conjecture}{Conjecture}
\newtheorem*{consequences}{Consequences}
\newtheorem*{exercise}{Exercise}
\newtheorem*{formalisation}{Formalisation}
\newtheorem*{proposition}{Proposition}
\newtheorem*{question}{Question}
\newtheorem*{remark}{Remark}

\ifx\type\undefined\else
  \newtheorem*{definition}{Definition}
  \newtheorem*{example}{Example}
  \newtheorem*{lemma}{Lemma}
  \newtheorem*{theorem}{Theorem}
\fi

\definecolor{keywordcolor}{rgb}{0.7, 0.1, 0.1}
\definecolor{tacticcolor}{rgb}{0.0, 0.1, 0.6}
\definecolor{commentcolor}{rgb}{0.4, 0.4, 0.4}
\definecolor{symbolcolor}{rgb}{0.0, 0.1, 0.6}
\definecolor{sortcolor}{rgb}{0.1, 0.5, 0.1}
\definecolor{attributecolor}{rgb}{0.7, 0.1, 0.1}
\def\lstlanguagefiles{lstlean.tex}
\lstset{language=lean}

\newcommand\A{\mathbb{A}}
\newcommand\C{\mathbb{C}}
\newcommand\F{\mathbb{F}}
\newcommand\G{\mathbb{G}}
\renewcommand\H{\mathbb{H}}
\newcommand\I{\mathbb{I}}
\newcommand\N{\mathbb{N}}
\renewcommand\P{\mathbb{P}}
\newcommand\Q{\mathbb{Q}}
\newcommand\R{\mathbb{R}}
\newcommand\Z{\mathbb{Z}}

\renewcommand\AA{\mathcal{A}}
\newcommand\BB{\mathcal{B}}
\newcommand\CC{\mathcal{C}}
\newcommand\DD{\mathcal{D}}
\newcommand\EE{\mathcal{E}}
\newcommand\FF{\mathcal{F}}
\newcommand\GG{\mathcal{G}}
\newcommand\HH{\mathcal{H}}
\newcommand\II{\mathcal{I}}
\newcommand\LL{\mathcal{L}}
\newcommand\MM{\mathcal{M}}
\newcommand\NN{\mathcal{N}}
\newcommand\OO{\mathcal{O}}
\newcommand\PP{\mathcal{P}}
\newcommand\RR{\mathcal{R}}
\renewcommand\SS{\mathcal{S}}
\newcommand\TT{\mathcal{T}}
\newcommand\XX{\mathcal{X}}

\renewcommand\aa{\mathfrak{a}}
\newcommand\cc{\mathfrak{c}}
\newcommand\dd{\mathfrak{d}}
\newcommand\ff{\mathfrak{f}}
\renewcommand\gg{\mathfrak{g}}
\newcommand\mm{\mathfrak{m}}
\newcommand\pp{\mathfrak{p}}
\newcommand\qq{\mathfrak{q}}
\renewcommand\ss{\mathfrak{s}}

\newcommand\LLL{\mathscr{L}}

\newcommand\ab{\mathrm{ab}}
\newcommand\Ab{\mathbf{Ab}}
\newcommand\Alg{\mathbf{Alg}}
\newcommand\Aff{\mathbf{Aff}}
\newcommand\Aut{\operatorname{Aut}}
\newcommand\Az{\mathrm{Az}}
\newcommand\Br{\operatorname{Br}}
\newcommand\BSD{\operatorname{BSD}}
\newcommand\ch{\operatorname{char}}
\newcommand\Cl{\operatorname{Cl}}
\newcommand\coker{\operatorname{coker}}
\newcommand\cris{\mathrm{cris}}
\renewcommand\d{\mathrm{d}}
\newcommand\Div{\operatorname{Div}}
\newcommand\dR{\mathrm{dR}}
\newcommand\EN{\operatorname{EN}}
\newcommand\End{\operatorname{End}}
\newcommand\ES{\operatorname{ES}}
\newcommand\et{\mathrm{\acute{e}t}}
\newcommand\Et{\mathbf{\acute{E}t}}
\newcommand\Ext{\operatorname{Ext}}
\newcommand\Fr{\operatorname{Fr}}
\newcommand\Frac{\operatorname{Frac}}
\newcommand\Gal{\operatorname{Gal}}
\newcommand\GL{\operatorname{GL}}
\newcommand\Gr{\mathrm{Gr}}
\newcommand\Hom{\operatorname{Hom}}
\newcommand\HT{\mathrm{HT}}
\newcommand\id{\operatorname{id}}
\newcommand\im{\operatorname{im}}
\newcommand\Ind{\operatorname{Ind}}
\renewcommand\inf{\operatorname{inf}}
\newcommand\inv{\operatorname{inv}}
\newcommand\Irr{\operatorname{Irr}}
\newcommand\Jac{\operatorname{Jac}}
\newcommand\lcm{\operatorname{lcm}}
\newcommand\Mat{\operatorname{Mat}}
\newcommand\Mod{\mathbf{Mod}}
\newcommand\Nm{\operatorname{Nm}}
\newcommand\nr{\mathrm{nr}}
\newcommand\NS{\operatorname{NS}}
\newcommand\Ob{\operatorname{Ob}}
\newcommand\ord{\operatorname{ord}}
\newcommand\op{\mathrm{op}}
\newcommand\PGL{\operatorname{PGL}}
\newcommand\Pic{\operatorname{Pic}}
\newcommand\Prob{\operatorname{Prob}}
\newcommand\Proj{\operatorname{Proj}}
\newcommand\PSh{\mathbf{PSh}}
\newcommand\Reg{\operatorname{Reg}}
\newcommand\res{\operatorname{res}}
\newcommand\rk{\operatorname{rk}}
\newcommand\Sch{\mathbf{Sch}}
\newcommand\Sel{\operatorname{Sel}}
\newcommand\Set{\mathbf{Set}}
\newcommand\sgn{\operatorname{sgn}}
\newcommand\Sh{\mathbf{Sh}}
\newcommand\SL{\operatorname{SL}}
\newcommand\Spec{\operatorname{Spec}}
\newcommand\supp{\operatorname{supp}}
\newcommand\Tam{\operatorname{Tam}}
\newcommand\Top{\mathbf{Top}}
\newcommand\tor{\operatorname{tor}}
\newcommand\tr{\operatorname{tr}}
\newcommand\tra{\operatorname{tra}}
\newcommand\WC{\operatorname{WC}}

\DeclareFontFamily{U}{wncyr}{}
\DeclareFontShape{U}{wncyr}{m}{n}{<->wncyr10}{}
\DeclareSymbolFont{cyr}{U}{wncyr}{m}{n}
\DeclareMathSymbol{\Sha}{\mathord}{cyr}{"58}

\newcommand{\function}[5][]{
  \if &#1&
    \begin{array}{rcl}
      #2 & \longrightarrow & #3 \\
      #4 & \longmapsto     & #5
    \end{array}
  \else
    \begin{array}{rcrcl}
      #1 & : & #2 & \longrightarrow & #3 \\
         &   & #4 & \longmapsto     & #5
    \end{array}
  \fi
}

\newcommand{\functions}[7][]{
  \if &#1&
    \begin{array}{rcl}
      #2 & \longrightarrow & #3 \\
      #4 & \longmapsto     & #5 \\
      #6 & \longmapsto     & #7 \\
    \end{array}
  \else
    \begin{array}{rcrcl}
      #1 & : & #2 & \longrightarrow & #3 \\
         &   & #4 & \longmapsto     & #5 \\
         &   & #6 & \longmapsto     & #7
    \end{array}
  \fi
}
\usetheme{hannover}
\title{Twisted elliptic L-values}
\subtitle{Early Number Theory Researchers Workshop 2023}
\author[David Ang]{David Kurniadi Angdinata}
\institute{London School of Geometry and Number Theory}
\date{Friday, 25 August 2023}

\begin{document}

\frame\maketitle

\section{Introduction}

\begin{frame}{A tale of two ranks}

Let $ E $ be an elliptic curve over $ \Q $, and let $ K $ be a number field.

\bigskip

\begin{theorem}[Mordell--Weil]
The set of $ K $-points $ E(K) $ is a finitely generated abelian group.
\end{theorem}

In particular, $ E(K) \cong \tor_{E / K} \times \Z^{\rk_{E / K}} $, where
\begin{itemize}
\item $ \tor_{E / K} $ is the \emph{torsion subgroup}, and
\item $ \rk_{E / K} $ is the \emph{(algebraic) rank}.
\end{itemize}
While $ \tor_{E / K} $ is classified, $ \rk_{E / K} $ remains mysterious.

\bigskip

\begin{conjecture}[Birch--Swinnerton-Dyer, weak form]
The order of vanishing of $ L_{E / K}(s) $ at $ s = 1 $ is equal to $ \rk_{E / K} $.
\end{conjecture}

This is called the \emph{analytic rank}.

\end{frame}

\begin{frame}{L-functions}

For any $ G_K $-representation $ \rho $, its \textbf{local Euler factor} is given by
$$ L_\pp(\rho, T) := \det(1 - T \cdot \phi_\pp \mid \rho^{I_\pp}), $$
where $ \phi_\pp \in G_K $ is a Frobenius and $ I_\pp \le G_K $ is the inertia subgroup. The \textbf{(Hasse--Weil) L-function of $ E / K $} is given by
$$ L_{E / K}(s) := \prod_\pp \dfrac{1}{L_\pp(\rho_{E, \ell}, \Nm(\pp)^{-s})}, $$
where $ \rho_{E, \ell} $ is the rational $ \ell $-adic Tate module as a $ G_K $-representation.

\begin{example}[$ K = \Q $]
Let $ a_p := 1 + p - \#E(\F_p) $. Then
$$ L_p(\rho_{E, \ell}, p^{-s}) =
\begin{cases}
1 - a_pp^{-s} + p^{1 - 2s} & \text{if} \ p \nmid \Delta(E), \\
1 - a_pp^{-s} & \text{if} \ p \mid \Delta(E).
\end{cases}
$$
\end{example}

\end{frame}

\begin{frame}{The BSD conjecture}

\begin{conjecture}[Birch--Swinnerton-Dyer, strong form]
The leading term of $ L_{E / K}(s) $ at $ s = 1 $ satisfies
$$ \lim_{s \to 1} \dfrac{L_{E / K}(s)}{(s - 1)^{\rk_{E / K}}} \cdot \dfrac{\sqrt{|\Delta_K|}}{\Omega_{E / K}} = \dfrac{C_{E / K} \cdot R_{E / K} \cdot \#\Sha_{E / K}}{\#\tor_{E / K}^2}. $$
\vspace{-0.5cm}
\end{conjecture}

Here,
\begin{itemize}
\item $ \Omega_{E / K}$ is the \emph{global period},
\item $ C_{E / K} $ is the \emph{Tamagawa product},
\item $ R_{E / K} $ is the \emph{regulator}, where $ R_{E / K} = 1 $ if $ \rk_{E / K} = 0 $, and
\item $ \Sha_{E / K} $ is the \emph{Tate--Shafarevich group}, conjecturally finite.
\end{itemize}
If $ \rk_{E / K} = 0 $, the LHS is called the \textbf{algebraic L-value}, given by
$$ \LLL_{E / K} := L_{E / K}(1) \cdot \dfrac{\sqrt{\Delta_K}}{\Omega_{E / K}}. $$

\end{frame}

\section{Dirichlet twists}

\begin{frame}{Twisted L-functions}

Let $ K = \Q(\zeta_m) $, and let $ \chi : (\Z / m\Z)^\times \to \C^\times $ be a Dirichlet character.

\bigskip The \textbf{(Hasse--Weil) L-function of $ E $ twisted by $ \chi $} is given by
$$ L_{E, \chi}(s) := \prod_p \dfrac{1}{L_p(\rho_{E, \ell} \otimes \chi, p^{-s})}. $$
Note that
$$ L_E(s) = \sum_{n \in \N} \dfrac{a_n}{n^s} \quad \overset{\chi}{\rightsquigarrow} \quad L_{E, \chi}(s) = \sum_{n \in \N} \dfrac{a_n\chi(n)}{n^s}. $$
By representation theory, there is a factorisation
$$ L_{E / K}(s) = \prod_\chi L_{E, \chi}(s), $$
where $ \chi : (\Z / m\Z)^\times \to \C^\times $ runs over primitive Dirichlet characters.

\end{frame}

\begin{frame}{A twisted BSD conjecture}

\begin{conjecture}[Deligne--Gross]
The order of vanishing of $ L_{E, \chi}(s) $ at $ s = 1 $ is equal to $ \langle\chi, E(K)_\C\rangle $.
\end{conjecture}

\bigskip Unfortunately, a twisted version of strong BSD seems difficult.

\begin{example}[Dokchitser--Evans--Wiersema]
Let $ E_1 $ and $ E_2 $ be elliptic curves given by Cremona labels 307a1 and 307c1, and let $ \chi : (\Z / 11\Z)^\times \to \C^\times $ be the primitive Dirichlet character of order $ 5 $ and conductor $ 11 $ given by $ \chi(2) = \zeta_5 $. Then
$$ C_{E_i / K} = R_{E_i / K} = \Sha_{E_i / K} = \tor_{E_i / K} = 1, $$
for $ K \subseteq \Q(\zeta_{11})^+ $, but
$$ \LLL_{E_1, \chi} = 1, \qquad \LLL_{E_2, \chi} = \zeta_5(1 + \zeta_5^4)^2. $$
\end{example}

\end{frame}

\begin{frame}{Algebraic twisted L-values}

If $ \rk_E = 0 $, the \textbf{algebraic twisted L-value} is given by
$$ \LLL_{E, \chi} := L_{E, \chi}(1) \cdot \dfrac{\tau(\overline{\chi})}{\Omega_E}, $$
where $ \tau(\overline{\chi}) $ is the \textbf{Gauss sum}
$$ \tau(\overline{\chi}) := \sum_{n \in (\Z / m\Z)^\times} \overline{\chi}(n)\zeta_m^n. $$
In general $ \LLL_{E, \chi} \in \overline{\Q} $, but some integrality statements are known.

\bigskip

\begin{theorem}[Wiersema--Wuthrich]
If $ E $ is semistable optimal of conductor $ N_E $, and if $ \chi $ is primitive of order $ k $ and conductor coprime to $ N_E $, then $ \LLL_{E, \chi} \in \Z[\zeta_k] $.
\end{theorem}

There are stronger statements under the \emph{Manin constant conjecture}.

\end{frame}

\section{Real values}

\begin{frame}{Real algebraic twisted L-values}

Assume that $ \rk_E = 0 $, and that $ \chi $ is primitive of order $ k > 2 $.

\bigskip

\begin{lemma}[Kisilevsky--Nam]
Let $ \omega_E $ be the ``root number'' of $ E $. Then $ \lambda_\chi \cdot \LLL_{E, \chi} \in \Z[\zeta_k]^+ $, where
$$ \lambda_\chi :=
\begin{cases}
1 & \text{if} \ \omega_E \cdot \chi(-N_E) = 1, \\
\chi(m) - \overline{\chi(m)} & \text{if} \ \omega_E \cdot \chi(-N_E) = -1, \\
1 + \omega_E \cdot \overline{\chi(-N_E)} & \text{if} \ \omega_E \cdot \chi(-N_E) \ne \pm 1,
\end{cases}
\qquad m \in \Z. $$
\end{lemma}

This is called the \textbf{real algebraic twisted L-value} $ \LLL_{E, \chi}^+ $.

\begin{example}[$ k = 3 $]
$$ \Z \ni \LLL_{E, \chi}^+ =
\begin{cases}
\Re(\LLL_{E, \chi}) & \text{if} \ \omega_E \cdot \chi(N_E) = 1, \\
2\Re(\LLL_{E, \chi}) & \text{if} \ \omega_E \cdot \chi(N_E) \ne 1.
\end{cases}
$$
\end{example}

\end{frame}

\begin{frame}{Some observations}

Let $ \chi : (\Z / p\Z)^\times \to \C^\times $ run over primes $ p \equiv 1 \mod 3 $.

\begin{example}[Kisilevsky--Nam]
Let $ E $ be the elliptic curve given by the Cremona label 11a1.
{\scriptsize $$
\begin{array}{r|rrrrrrrrrrr}
p & 7 & 13 & 19 & 31 & 37 & 43 & 61 & 67 & 73 & 79 & 97 \\
\hline
\LLL_{E, \chi}^+ & 5 & -10 & -10 & 5 & 20 & 5 & -10 & 15 & 5 & 15 & -30 \\
\overline{\LLL}_{E, \chi}^+ & 1 & -2 & -2 & 1 & 4 & 1 & -2 & 3 & 1 & 3 & -6 \\
{[\overline{\LLL}_{E, \chi}^+]_3} & 1 & 1 & 1 & 1 & 1 & 1 & 1 & 0 & 1 & 0 & 0 \\
{[\#E(\F_p)]_3} & 1 & 1 & 2 & 1 & 2 & 2 & 2 & 0 & 1 & 0 & 0 \\
\chi(N_E) & \zeta_3 & \zeta_3 & 1 & \overline{\zeta_3} & 1 & 1 & 1 & \overline{\zeta_3} & \zeta_3 & \overline{\zeta_3} & \overline{\zeta_3}
\end{array}
$$
$$
\begin{array}{r|rrrrrrrrrr}
p & 103 & 109 & 127 & 139 & 151 & 157 & 163 & 181 & 193 & 199 \\
\hline
\LLL_{E, \chi}^+ & 30 & 5 & 15 & 5 & 0 & 0 & 80 & 50 & -5 & -55 \\
\overline{\LLL}_{E, \chi}^+ & 6 & 1 & 3 & 1 & 0 & 0 & 16 & 10 & -1 & -11 \\
{[\overline{\LLL}_{E, \chi}^+]_3} & 0 & 1 & 0 & 1 & 0 & 0 & 1 & 1 & 2 & 1 \\
{[\#E(\F_p)]_3} & 0 & 1 & 0 & 1 & 0 & 0 & 1 & 1 & 1 & 2 \\
\chi(N_E) & \zeta_3 & \overline{\zeta_3} & \overline{\zeta_3} & \zeta_3 & \zeta_3 & \zeta_3 & \overline{\zeta_3} & \overline{\zeta_3} & 1 & 1
\end{array}
$$}
Here, $ \overline{\LLL}_{E, \chi}^+ := \LLL_{E, \chi}^+ / \gcd_{\chi'}\{\LLL_{E, \chi'}^+\} $.
\end{example}

\end{frame}

\begin{frame}{Some phenomena}

If $ E $ is the elliptic curve given by the Cremona label 11a1,
$$ \overline{\LLL}_{E, \chi}^+ \equiv_3
\begin{cases}
0 & \text{if} \ \#E(\F_p) \equiv 0 \mod 3, \\
2 & \text{if} \ \#E(\F_p) \equiv 1 \mod 3 \ \text{and} \ \chi(N_E) = 1, \\
1 & \text{otherwise}.
\end{cases}
$$
KN computed $ \overline{\LLL}_{E, \chi}^+ $ modulo $ 3 $ for six elliptic curves.
\begin{itemize}
\item For 14a1, $ \overline{\LLL}_{E, \chi}^+ \equiv 2 \mod 3 $ often occurs.
\item For 11a1, 15a1, 17a1, $ \overline{\LLL}_{E, \chi}^+ \equiv 2 \mod 3 $ rarely occurs.
\item For 19a1, 37b1, $ \overline{\LLL}_{E, \chi}^+ \equiv 2 \mod 3 $ never occurs.
\end{itemize}

\bigskip

\begin{theorem}[A.]
I can partially explain the DEW and KN phenomena.
\end{theorem}

\end{frame}

\section{Modular symbols}

\begin{frame}{The modularity theorem}

Let $ E $ be a semistable optimal elliptic curve over $ \Q $ of conductor $ N_E $.

\bigskip

\begin{theorem}[Taylor--Wiles]
There is an eigenform $ f_E \in S_2(\Gamma_0(N_E)) $ with (Hecke) L-function $ L_{f_E}(s) = L_E(s) $, such that the Hecke operator $ T_p $ has eigenvalue $ a_p $.
\end{theorem}

\bigskip For any cusp form $ f \in S_k(\Gamma) $, its L-function is a Mellin transform
$$ L_f(s) := \dfrac{(-2\pi i)^s}{\Gamma(s)}\int_0^\infty z^{s - 1}f(z)\d z. $$
Set $ s = 1 $:
$$ L_f(1) = -2\pi i\int_0^\infty f(z)\d z =: -\langle\{0, \infty\}, f\rangle. $$
This is a \emph{period} of the \emph{modular symbol} $ \{0, \infty\} $.

\end{frame}

\begin{frame}{Classical modular symbols}

Let $ \HH $ be the extended upper half plane, and let $ \phi : \HH \twoheadrightarrow \HH / \Gamma =: X_\Gamma $.

\bigskip A \textbf{modular symbol} is a class $ \{x, y\} \in H_1(X_\Gamma, \R) $ for any $ x, y \in \HH $.
\begin{itemize}
\item If $ \Gamma \cdot x = \Gamma \cdot y $, then $ \phi(x \rightsquigarrow y) \in H_1(X_\Gamma, \Z) $, and conversely any $ \gamma \in H_1(X_\Gamma, \Z) $ arises from $ x, y \in \HH $ in the same $ \Gamma $-orbit. Define
$$ \{x, y\} := \phi(x \rightsquigarrow y). $$
\item The map $ H_1(X_\Gamma, \Z) \to \Hom_\C(S_2(\Gamma), \C) $ given by $ \gamma \mapsto \langle\gamma, \cdot\rangle $ extends to $ \psi : H_1(X_\Gamma, \R) \xrightarrow{\sim} \Hom_\C(S_2(\Gamma), \C) $. Define
$$ \{x, y\} := \psi^{-1}\langle\phi(x \rightsquigarrow y), \phi^*(\cdot)\rangle. $$
\end{itemize}
Note that
$$ \{x, x\} = 0, \quad \{x, y\} = -\{y, x\}, \quad \{x, y\} + \{y, z\} = \{x, z\}, $$
$$ \langle\{x, y\}, M \cdot f\rangle = \langle\{M \cdot x, M \cdot y\}, f\rangle, \qquad M \in \Gamma. $$

\end{frame}

\begin{frame}{L-values as periods}

Let $ p \nmid N_E $. The Hecke operator $ T_p $ acts on $ H_1(X_\Gamma, \Q) $ by
$$ T_p \cdot \{x, y\} = \{px, py\} + \sum_{n = 0}^{p - 1} \{\tfrac{x + n}{p}, \tfrac{y + n}{p}\}. $$

\begin{lemma}[Manin]
\vspace{-0.5cm}
$$ -\#E(\F_p) \cdot \LLL_E = \dfrac{1}{\Omega_E}\sum_{n = 1}^{p - 1} \langle\{0, \tfrac{n}{p}\}, f_E\rangle. $$
\vspace{-0.5cm}
\end{lemma}

\begin{proof}
Set $ \{x, y\} = \{0, \infty\} $ in the Hecke action and apply the pairing $ \langle\cdot, f_E\rangle $:
$$ \underbrace{(1 + p - a_p)}_{\#E(\F_p)} \cdot \underbrace{\langle\{0, \infty\}, f_E\rangle}_{-L_E(1)} = \sum_{n = 1}^{p - 1} \underbrace{\langle\{0, \tfrac{n}{p}\}, f_E\rangle}_{???}. $$
Multiply by $ \tfrac{1}{\Omega_E} $.
\end{proof}

\end{frame}

\begin{frame}{Twisted L-values as periods}

\begin{lemma}[Manin]
\vspace{-0.5cm}
$$ \LLL_{E, \chi} = \dfrac{1}{\Omega_E}\sum_{n = 1}^{p - 1} \overline{\chi}(n)\langle\{0, \tfrac{n}{p}\}, f_E\rangle. $$
\vspace{-0.5cm}
\end{lemma}

\begin{proof}
For any $ m \in (\Z / p\Z)^\times $, the discrete Fourier transform of $ \chi(m) $ is
$$ \chi(m) = \dfrac{1}{\tau(\overline{\chi})}\sum_{n = 1}^{p - 1} \overline{\chi}(n)\zeta_p^{mn}. $$
Substitute into $ \sum_m a_m\chi(m)q^m $ and apply the Mellin transform:
$$ L_{E, \chi}(1) = \dfrac{1}{\tau(\overline{\chi})}\sum_{n = 1}^{p - 1} \overline{\chi}(n)\underbrace{\langle\{0, \infty\}, M \cdot f_E\rangle}_{\text{apply properties}}, \qquad M := \begin{pmatrix} p & k \\ 0 & p \end{pmatrix}. $$
Multiply by $ \tfrac{\tau(\overline{\chi})}{\Omega_E} $.
\end{proof}

\end{frame}

\begin{frame}{A congruence for L-values}

Let $ E $ be a semistable optimal elliptic curve over $ \Q $ of conductor $ N_E $, let $ p \nmid N_E $, and let $ \chi : (\Z / p\Z)^\times \to \C^\times $ be a Dirichlet character. Then
\begin{align*}
-\#E(\F_p) \cdot \LLL_E & = \dfrac{1}{\Omega_E}\sum_{n = 1}^{p - 1} \langle\{0, \tfrac{n}{p}\}, f_E\rangle, \\
\LLL_{E, \chi} & = \dfrac{1}{\Omega_E}\sum_{n = 1}^{p - 1} \overline{\chi}(n)\langle\{0, \tfrac{n}{p}\}, f_E\rangle.
\end{align*}

\begin{corollary}[A]
If $ \chi $ has prime order $ k $, then
$$ -\#E(\F_p) \cdot \LLL_E \equiv \LLL_{E, \chi} \mod (1 - \zeta_k). $$
\vspace{-0.5cm}
\end{corollary}

\begin{proof}
By integrality, the lemmata, and $ \overline{\chi} \equiv 1 \mod (1 - \zeta_k) $.
\end{proof}

\end{frame}

\begin{frame}{A congruence for cubic twists}

Let $ \chi : (\Z / p\Z)^\times \to \C^\times $ be a cubic primitive Dirichlet character.

\begin{corollary}[B]
\vspace{-0.5cm}
$$ \LLL_{E, \chi}^+ \equiv_3 \#E(\F_p) \cdot \LLL_E \cdot
\begin{cases}
2 & \text{if} \ \omega_E \cdot \chi(N_E) = 1, \\
1 & \text{if} \ \omega_E \cdot \chi(N_E) \ne 1.
\end{cases}
$$
\vspace{-0.5cm}
\end{corollary}

\begin{proof}
By cases of corollary (A).
\end{proof}

\bigskip

\begin{corollary}[C]
\vspace{-0.5cm}
$$ \overline{\LLL}_{E, \chi}^+ \equiv_3 \#E(\F_p) \cdot \LLL_E \cdot \gcd_{\chi'}\{\LLL_{E, \chi'}^+\} \cdot
\begin{cases}
2 & \text{if} \ \omega_E \cdot \chi(N_E) = 1, \\
1 & \text{if} \ \omega_E \cdot \chi(N_E) \ne 1.
\end{cases}
$$
\vspace{-0.5cm}
\end{corollary}

\begin{proof}
By cases of corollary (B).
\end{proof}

\end{frame}

\section{Explanations}

\begin{frame}{DEW phenomena}

Recall that if $ E_1 $ and $ E_2 $ are elliptic curves given by Cremona labels 307a1 and 307c1, and $ \chi : (\Z / 11\Z)^\times \to \C^\times $ is the primitive Dirichlet character of order $ 5 $ and conductor $ 11 $ given by $ \chi(2) = \zeta_5 $, then
$$ C_{E_i / K} = R_{E_i / K} = \Sha_{E_i / K} = \tor_{E_i / K} = 1, $$
for $ K \subseteq \Q(\zeta_{11})^+ $, but
$$ \LLL_{E_1, \chi} = 1, \qquad \LLL_{E_2, \chi} = \zeta_5(1 + \zeta_5^4)^2. $$
Note that $ \LLL_{E_i} = 1 $, and
$$ \#E_1(\F_{11}) = 9, \qquad \#E_2(\F_{11}) = 16, $$
so corollary (A) says $ \LLL_{E_1, \chi} \not\equiv \LLL_{E_2, \chi} \mod (1 - \zeta_5) $.

\bigskip In fact, corollary (A) partially explains all examples in DEW where $ \chi $ is quintic, and fully explains all examples in DEW where $ \chi $ is cubic.

\end{frame}

\begin{frame}{Insufficiency of congruence}

Unfortunately, there are elliptic curves $ E_1 $ and $ E_2 $ over $ \Q $, where $ \LLL_{E_1, \chi} \equiv \LLL_{E_2, \chi} \mod (1 - \zeta_5) $, but $ \LLL_{E_1, \chi} \ne \LLL_{E_2, \chi} $.

\bigskip

\begin{example}[A.]
Let $ E_1 $ and $ E_2 $ be elliptic curves given by Cremona labels 130b3 and 312c3, and let $ \chi : (\Z / 11\Z)^\times \to \C^\times $ be the primitive Dirichlet character of order $ 5 $ and conductor $ 11 $ given by $ \chi(2) = \zeta_5 $. Then
$$ C_{E_i / K} = 2, \qquad R_{E_i / K} = 1, \qquad \Sha_{E_i / K} \cong \tor_{E_i / K} \cong (\Z / 2\Z)^2, $$
for $ K \subseteq \Q(\zeta_{11})^+ $, and furthermore $ \#E_i(\F_{11}) = 12 $ and $ \LLL_{E_i} = \tfrac{1}{2} $, but
$$ \LLL_{E_1, \chi} = -4\zeta_5^3, \qquad \LLL_{E_2, \chi} = -4\zeta_5, $$
which are not equal but congruent modulo $ (1 - \zeta_5) $.
\end{example}

\bigskip Heuristically, the norm of $ \overline{\LLL}_{E, \chi}^+ $ is the $ \chi $-component of $ \Sha_E $.

\end{frame}

\begin{frame}{KN phenomena}

Recall that if $ E $ is the elliptic curve given by the Cremona label 11a1,
$$ \overline{\LLL}_{E, \chi}^+ \equiv_3
\begin{cases}
0 & \text{if} \ \#E(\F_p) \equiv 0 \mod 3, \\
2 & \text{if} \ \#E(\F_p) \equiv 1 \mod 3 \ \text{and} \ \chi(N_E) = 1, \\
1 & \text{otherwise}.
\end{cases}
$$
Note that $ \LLL_E \cdot \gcd_{\chi'}\{\LLL_{E, \chi'}^+\} = 1 $ and $ \omega_E = 1 $, so corollary (C) says
$$ \overline{\LLL}_{E, \chi}^+ \equiv_3
\begin{cases}
2\#E(\F_p) & \text{if} \ \chi(N_E) = 1, \\
\#E(\F_p) & \text{if} \ \chi(N_E) \ne 1.
\end{cases}
$$
This fully explains the three cases, except for when $ \#E(\F_p) \equiv 1 \mod 3 $ or $ \chi(N_E) = 1 $. In fact, corollary (C) fully explains $ \overline{\LLL}_{E, \chi}^+ $ modulo $ 3 $ for any elliptic curve $ E $ over $ \Q $ where
\begin{itemize}
\item $ E $ does not have rational $ 3 $-isogenies, and
\item the $ 3 $-division field $ \Q(x(E[3])) $ of $ E $ contains $ \sqrt[3]{N_E} $.
\end{itemize}

\end{frame}

\begin{frame}{The missing piece}

\begin{theorem}[A.--Dokchitser]
Assume that
\begin{itemize}
\item $ E $ does not have rational $ 3 $-isogenies, and
\item the $ 3 $-division field $ \Q(x(E[3])) $ of $ E $ contains $ \sqrt[3]{N_E} $.
\end{itemize}
If $ \#E(\F_p) \equiv 2 \mod 3 $, then $ \chi(N_E) = 1 $.
\end{theorem}

\begin{proof}
The assumptions imply that
$$ \Gal(\Q(x(E[3])) / \Q) \cong \PGL_2(\F_3) \cong S_4. $$
This has a quotient
$$ \Gal(\Q(\sqrt[3]{N_E}, \zeta_3) / \Q) \cong S_4 / K_4 \cong S_3. $$
If $ \#E(\F_p) \equiv 2 \mod 3 $, then $ a_p \equiv 0 \mod 3 $, so $ \phi_p^2 = 1 $ in $ S_4 $. By group theory, $ \phi_p = 1 $ in $ S_3 $, but it acts as $ \chi(N_E) $ on $ \Q(\sqrt[3]{N_E}, \zeta_3) $.
\end{proof}

\end{frame}

\begin{frame}{Other KN phenomena}

Key idea: understand how $ \phi_p $ acts on $ \Q(x(E[3])) $ and $ \Q(\sqrt[3]{N_E}, \zeta_3) $.

\bigskip In general, $ \phi_p \ne 1 $ in $ \Gal(\Q(x(E[3])) / \Q) \le \PGL_2(\F_3) $.
\begin{itemize}
\item If $ E $ has rational $ 3 $-isogenies, then $ \overline{\LLL}_{E, \chi}^+ $ modulo $ 3 $ is partially explained by how $ \phi_p $ acts on the $ 9 $-division field $ \Q(x(E[9])) $.
\item If $ \Q(x(E[3])) $ does not contain $ \sqrt[3]{N_E} $, then $ \overline{\LLL}_{E, \chi}^+ $ modulo $ 3 $ is fully explained by how $ \phi_p $ acts on $ \Q(x(E[3]), \sqrt[3]{N_E}) $.
\end{itemize}
The elliptic curves given by Cremona labels 11a1, 15a1, 17a1 are generic, but those given by 14a1, 19a1, 37b1 are special.

\bigskip

\begin{theorem}[A.]
I understand how $ \phi_p $ acts on $ \Q(x(E[9]), \sqrt[3]{N_E}) $.
\end{theorem}

This crucially uses the classification of $ 3 $-adic images of Galois for elliptic curves over $ \Q $ by Rouse--Sutherland--Zureick-Brown.

\end{frame}

\section{}

\begin{frame}[c]{Future work}

Here are some potential extensions, listed in increasing difficulty:
\begin{itemize}
\item replace $ \Q $ with a global field
\item replace $ \chi $ with an Artin representation
\item replace $ E $ with the Jacobian of a higher genus curve
\item remove the $ \rk_E = 0 $ assumption
\end{itemize}

\end{frame}

\end{document}