\ifx\type\undefined
  \documentclass[10pt, t]{beamer}
  \setbeamertemplate{footline}[page number]
\else
  \documentclass[10pt]{article}
  \usepackage[margin=1in]{geometry}
\fi

\usepackage{amsmath}
\usepackage{amssymb}
\usepackage{amsthm}
\usepackage{bbm}
\usepackage{cancel}
\usepackage{listings}
\usepackage{mathrsfs}
\usepackage{multirow}
\usepackage{soul}
\usepackage{stmaryrd}
\usepackage{tikz}
\usepackage{tikz-cd}
\usepackage{wrapfig}

\newtheorem*{algorithm}{Algorithm}
\newtheorem*{assumptions}{Assumptions}
\newtheorem*{conjecture}{Conjecture}
\newtheorem*{consequences}{Consequences}
\newtheorem*{exercise}{Exercise}
\newtheorem*{formalisation}{Formalisation}
\newtheorem*{proposition}{Proposition}
\newtheorem*{question}{Question}
\newtheorem*{remark}{Remark}

\ifx\type\undefined\else
  \newtheorem*{definition}{Definition}
  \newtheorem*{example}{Example}
  \newtheorem*{lemma}{Lemma}
  \newtheorem*{theorem}{Theorem}
\fi

\definecolor{keywordcolor}{rgb}{0.7, 0.1, 0.1}
\definecolor{tacticcolor}{rgb}{0.0, 0.1, 0.6}
\definecolor{commentcolor}{rgb}{0.4, 0.4, 0.4}
\definecolor{symbolcolor}{rgb}{0.0, 0.1, 0.6}
\definecolor{sortcolor}{rgb}{0.1, 0.5, 0.1}
\definecolor{attributecolor}{rgb}{0.7, 0.1, 0.1}
\def\lstlanguagefiles{lstlean.tex}
\lstset{language=lean}

\newcommand\A{\mathbb{A}}
\newcommand\C{\mathbb{C}}
\newcommand\F{\mathbb{F}}
\newcommand\G{\mathbb{G}}
\renewcommand\H{\mathbb{H}}
\newcommand\I{\mathbb{I}}
\newcommand\N{\mathbb{N}}
\renewcommand\P{\mathbb{P}}
\newcommand\Q{\mathbb{Q}}
\newcommand\R{\mathbb{R}}
\newcommand\Z{\mathbb{Z}}

\renewcommand\AA{\mathcal{A}}
\newcommand\BB{\mathcal{B}}
\newcommand\CC{\mathcal{C}}
\newcommand\DD{\mathcal{D}}
\newcommand\EE{\mathcal{E}}
\newcommand\FF{\mathcal{F}}
\newcommand\GG{\mathcal{G}}
\newcommand\HH{\mathcal{H}}
\newcommand\II{\mathcal{I}}
\newcommand\LL{\mathcal{L}}
\newcommand\MM{\mathcal{M}}
\newcommand\NN{\mathcal{N}}
\newcommand\OO{\mathcal{O}}
\newcommand\PP{\mathcal{P}}
\newcommand\RR{\mathcal{R}}
\renewcommand\SS{\mathcal{S}}
\newcommand\TT{\mathcal{T}}
\newcommand\XX{\mathcal{X}}

\renewcommand\aa{\mathfrak{a}}
\newcommand\cc{\mathfrak{c}}
\newcommand\dd{\mathfrak{d}}
\newcommand\ff{\mathfrak{f}}
\renewcommand\gg{\mathfrak{g}}
\newcommand\mm{\mathfrak{m}}
\newcommand\pp{\mathfrak{p}}
\newcommand\qq{\mathfrak{q}}
\renewcommand\ss{\mathfrak{s}}

\newcommand\LLL{\mathscr{L}}

\newcommand\ab{\mathrm{ab}}
\newcommand\Ab{\mathbf{Ab}}
\newcommand\Alg{\mathbf{Alg}}
\newcommand\Aff{\mathbf{Aff}}
\newcommand\Aut{\operatorname{Aut}}
\newcommand\Az{\mathrm{Az}}
\newcommand\Br{\operatorname{Br}}
\newcommand\BSD{\operatorname{BSD}}
\newcommand\ch{\operatorname{char}}
\newcommand\Cl{\operatorname{Cl}}
\newcommand\coker{\operatorname{coker}}
\newcommand\cris{\mathrm{cris}}
\renewcommand\d{\mathrm{d}}
\newcommand\Div{\operatorname{Div}}
\newcommand\dR{\mathrm{dR}}
\newcommand\EN{\operatorname{EN}}
\newcommand\End{\operatorname{End}}
\newcommand\ES{\operatorname{ES}}
\newcommand\et{\mathrm{\acute{e}t}}
\newcommand\Et{\mathbf{\acute{E}t}}
\newcommand\Ext{\operatorname{Ext}}
\newcommand\Fr{\operatorname{Fr}}
\newcommand\Frac{\operatorname{Frac}}
\newcommand\Gal{\operatorname{Gal}}
\newcommand\GL{\operatorname{GL}}
\newcommand\Gr{\mathrm{Gr}}
\newcommand\Hom{\operatorname{Hom}}
\newcommand\HT{\mathrm{HT}}
\newcommand\id{\operatorname{id}}
\newcommand\im{\operatorname{im}}
\newcommand\Ind{\operatorname{Ind}}
\renewcommand\inf{\operatorname{inf}}
\newcommand\inv{\operatorname{inv}}
\newcommand\Irr{\operatorname{Irr}}
\newcommand\Jac{\operatorname{Jac}}
\newcommand\lcm{\operatorname{lcm}}
\newcommand\Mat{\operatorname{Mat}}
\newcommand\Mod{\mathbf{Mod}}
\newcommand\Nm{\operatorname{Nm}}
\newcommand\nr{\mathrm{nr}}
\newcommand\NS{\operatorname{NS}}
\newcommand\Ob{\operatorname{Ob}}
\newcommand\ord{\operatorname{ord}}
\newcommand\op{\mathrm{op}}
\newcommand\PGL{\operatorname{PGL}}
\newcommand\Pic{\operatorname{Pic}}
\newcommand\Prob{\operatorname{Prob}}
\newcommand\Proj{\operatorname{Proj}}
\newcommand\PSh{\mathbf{PSh}}
\newcommand\Reg{\operatorname{Reg}}
\newcommand\res{\operatorname{res}}
\newcommand\rk{\operatorname{rk}}
\newcommand\Sch{\mathbf{Sch}}
\newcommand\Sel{\operatorname{Sel}}
\newcommand\Set{\mathbf{Set}}
\newcommand\sgn{\operatorname{sgn}}
\newcommand\Sh{\mathbf{Sh}}
\newcommand\SL{\operatorname{SL}}
\newcommand\Spec{\operatorname{Spec}}
\newcommand\supp{\operatorname{supp}}
\newcommand\Tam{\operatorname{Tam}}
\newcommand\Top{\mathbf{Top}}
\newcommand\tor{\operatorname{tor}}
\newcommand\tr{\operatorname{tr}}
\newcommand\tra{\operatorname{tra}}
\newcommand\WC{\operatorname{WC}}

\DeclareFontFamily{U}{wncyr}{}
\DeclareFontShape{U}{wncyr}{m}{n}{<->wncyr10}{}
\DeclareSymbolFont{cyr}{U}{wncyr}{m}{n}
\DeclareMathSymbol{\Sha}{\mathord}{cyr}{"58}

\newcommand{\function}[5][]{
  \if &#1&
    \begin{array}{rcl}
      #2 & \longrightarrow & #3 \\
      #4 & \longmapsto     & #5
    \end{array}
  \else
    \begin{array}{rcrcl}
      #1 & : & #2 & \longrightarrow & #3 \\
         &   & #4 & \longmapsto     & #5
    \end{array}
  \fi
}

\newcommand{\functions}[7][]{
  \if &#1&
    \begin{array}{rcl}
      #2 & \longrightarrow & #3 \\
      #4 & \longmapsto     & #5 \\
      #6 & \longmapsto     & #7 \\
    \end{array}
  \else
    \begin{array}{rcrcl}
      #1 & : & #2 & \longrightarrow & #3 \\
         &   & #4 & \longmapsto     & #5 \\
         &   & #6 & \longmapsto     & #7
    \end{array}
  \fi
}
\title{Computing L-functions over global function fields}
\subtitle{Elliptic Curves in the Cotswolds}
\author{David Kurniadi Angdinata}
\institute{London School of Geometry and Number Theory}
\date{Monday, 3 March 2025}

\begin{document}

\frame\maketitle

\begin{frame}{Global fields}

Let $ E $ be an elliptic curve over a global field $ K $. Its L-function is given by
$$ L(E, s) := \prod_v \dfrac{1}{\LL_v(E, p_v^{-s\deg v})}, $$
where $ p_v $ is the residue characteristic at each place $ v $ of $ K $.

\bigskip Here, the local Euler factors are given by
$$ \LL_v(E, T) := \det(1 - T \cdot \phi_v^{-1} \mid \rho_{E, \ell}^{I_v}) \in 1 + T \cdot \Q[T], $$
where $ \ell $ is some prime different from $ p_v $.

\bigskip

\begin{conjecture}[Birch and Swinnerton-Dyer]
The arithmetic of $ E $ is determined by the analysis of $ L(E, s) $ at $ s = 1 $.
\end{conjecture}

\bigskip There is much numerical evidence, which requires computing $ L(E, s) $!

\end{frame}

\begin{frame}{Computing special values}

Over a number field $ K $, Dokchitser \footnote{Tim Dokchitser. ``Computing special values of motivic L-functions'' Experimental Mathematics 13 (2) 137--150, 2004} gave an algorithm to compute the special values of $ L(E, s) $ assuming the functional equation
$$ \Lambda(E, s) = \epsilon_E\Nm(\ff_E)^{1 - s}\Delta_K^{1 - s}\Lambda(E, 2 - s), $$
where its completed L-function is given by
$$ \Lambda(E, s) := \left(\dfrac{\Gamma(s)}{(2\pi)^s}\right)^{[K : \Q]}L(E, s). $$
This was originally the \texttt{ComputeL} package in PARI/GP, but later ported to Magma as \texttt{LSeries()} and SageMath as \texttt{lseries().dokchitser()}.

\bigskip Over a global function field, Magma has \texttt{LFunction()}, which uses the theory of Mordell--Weil lattices on elliptic surfaces to give a polynomial.

\bigskip I claim that there is a much easier way to compute the same polynomial!

\end{frame}

\begin{frame}{Global function fields}

Let $ K := k(C) $ be the global function field of a smooth proper geometrically irreducible curve $ C $ over a finite field $ k := \F_q $.

\bigskip The formal L-function of an elliptic curve $ E $ over $ K $ is given by
$$ \LL(E, T) := \prod_v \dfrac{1}{\LL_v(E, T^{\deg v})} \in \Q[[T]], $$
so that $ L(E, s) = \LL(E, q^{-s}) $.

\bigskip If $ \{a_{v, i}\}_{i = 0}^\infty $ are the coefficients of $ \LL_v(E, T^{\deg v})^{-1} $, then
$$ \LL(E, T) = \prod_v \left(\sum_{i = 0}^\infty a_{v, i}T^{i\deg v}\right) = \sum_{j = 0}^\infty \left(\sum_{\deg D = j} a_D\right)T^j, $$
where $ a_D := \prod_v a_{v, i_v} $ for any effective Weil divisor $ D = \sum_v i_v[v] $ on $ C $.

\end{frame}

\begin{frame}{Rationality}

\begin{corollary}[of the Weil conjectures \footnote{Grothendieck--Lefschetz trace formula and Grothendieck--Ogg--Shafarevich formula}]
There are polynomials $ P_0(T), P_1(T), P_2(T) \in 1 + T \cdot \Q[T] $ such that
$$ \LL(E, T) = \dfrac{P_1(T)}{P_0(T) \cdot P_2(T)} \in \Q(T), $$
and
$$ -\deg P_0(T) + \deg P_1(T) - \deg P_2(T) = 4g_C - 4 + \deg\ff_E. $$
Furthermore, there are simple expressions for $ P_0(T) $ and $ P_2(T) $ in terms of $ \LL(C, T) $, and in fact $ P_0(T) = P_2(T) = 1 $ whenever $ E $ is not constant.
\end{corollary}

\bigskip Thus $ \LL(E, T) $ is completely determined by the coefficients $ a_D $ for all effective Weil divisors $ D $ on $ C $ with $ \deg D \le d_E $, where
$$ d_E := 4g_C - 4 + \deg\ff_E + \deg P_0(T) + \deg P_2(T). $$

\end{frame}

\begin{frame}{Quadratic example}

Let $ E $ be the elliptic curve $ y^2 = x^3 + x^2 + t^2 + 2 $ over $ K = \F_3(t) $. Then
$$ \deg\LL(E, T) = d_E = 4(0) - 4 + \deg(4[\tfrac{1}{t}] + [t + 1] + [t + 2]) = 2. $$
\vspace{-0.5cm}
$$
\begin{array}{|c|c|c|c|}
\hline
v & \LL_v(E, T) & \LL_v(E, T^{\deg v}) & \LL_v(E, T^{\deg v})^{-1} \\
\hline
\tfrac{1}{t} & 1 & 1 & 1 \\
\hline
t & 1 - T + 3T^2 & 1 - T + 3T^2 & 1 + T - 2T^2 + \dots \\
\hline
t + 1 & 1 - T & 1 - T & 1 + T + T^2 + \dots \\
\hline
t + 2 & 1 - T & 1 - T & 1 + T + T^2 + \dots \\
\hline
t^2 + 1 & 1 + 2T + 3T^2 & 1 + 2T^2 + \dots & 1 - 2T^2 + \dots \\
\hline
t^2 + t + 2 & 1 - 4T + 3T^2 & 1 - 4T^2 + \dots & 1 + 4T^2 + \dots \\
\hline
t^2 + 2t + 2 & 1 - 4T + 3T^2 & 1 - 4T^2 + \dots & 1 + 4T^2 + \dots \\
\hline
\end{array}
$$
Thus
\begin{align*}
\LL(E, T)
& \equiv (1 + T - 2T^2 + \dots) \cdot \dots \cdot (1 + 4T^2 + \dots) \mod T^3 \\
& \equiv 1 + 3T + 9T^2 \mod T^3,
\end{align*}
which forces $ \LL(E, T) = 1 + 3T + 9T^2 $.

\end{frame}

\begin{frame}{Functional equation}

\begin{corollary}[of the Weil conjectures and root number results \footnote{by the works of Deligne, Rohrlich, Kobayashi, and Imai}]
There is a global root number $ \epsilon_E \in \{\pm1\} $ such that
$$ \LL(E, T) = \epsilon_Eq^{d_E}T^{d_E}\LL(E, 1 / q^2T). $$
Furthermore, there is a simple algorithm to compute $ \epsilon_E $ in terms of the reduction type of $ E $ at each place in the support of $ \ff_E $.
\end{corollary}

\bigskip If $ \{b_i\}_{i = 0}^{d_E} $ are the coefficients of $ \LL(E, T) $, then
$$ \sum_{i = 0}^{d_E} b_iT^i = \sum_{i = 0}^{d_E} \epsilon_Eb_iq^{d_E - 2i}T^{d_E - i} = \sum_{i = 0}^{d_E} \epsilon_Eb_{d_E - i}q^{2i - d_E}T^i, $$
so that $ b_i $ can be computed as $ \epsilon_Eb_{d_E - i}q^{2i - d_E} $ when $ \lceil d_E / 2\rceil \le i \le d_E $.

\end{frame}

\begin{frame}{Quintic example}

Let $ E $ be the elliptic curve $ y^2 = x^3 + x^2 + t^4 + t^2 $ over $ K = \F_3(t) $. Then
$$ \deg\LL(E, T) = d_E = 4(0) - 4 + \deg(6[\tfrac{1}{t}] + [t] + [t^2 + 1]) = 5. $$
By computing $ \LL_v(E, T^{\deg v})^{-1} $ for all places $ v $ of $ K $ with $ \deg v \le 2 $,
$$ \LL(E, T) \equiv 1 + 3T + 9T^2 \mod T^3, $$
which forces $ \LL(E, T) = 1 + 3T + 9T^2 + 27\epsilon_ET^3 + 81\epsilon_ET^4 + 243\epsilon_ET^5 $.

\bigskip In fact, $ \epsilon_E = -1 $, since $ \epsilon_{E, t} = \epsilon_{E, t^2 + 1} = -1 $ and
\begin{align*}
\epsilon_{E, \frac{1}{t}}
& = -(\Delta_{E'}, a_{6, E'}) \cdot \left(\dfrac{v_{\frac{1}{t}}(a_{6, E'})}{3}\right)^{v_{\frac{1}{t}}(\Delta_{E'})} \cdot \left(\dfrac{-1}{3}\right)^{\tfrac{v_{\frac{1}{t}}(\Delta_{E'})(v_{\frac{1}{t}}(\Delta_{E'}) - 1)}{2}} \\
& = -1,
\end{align*}
where $ E' $ is the elliptic curve $ y^2 = x^3 + (\tfrac{1}{t})^2x^2 + (\tfrac{1}{t})^4 + (\tfrac{1}{t})^2 $ over $ K_{\frac{1}{t}} $.

\end{frame}

\begin{frame}{$ \ell $-adic representations}

In general, the formal L-function of an almost everywhere unramified $ \ell $-adic representation $ \rho : G_K \to \GL_n(\overline{\Q_\ell}) $ is given by
$$ \LL(\rho, T) := \prod_v \dfrac{1}{\LL_v(\rho, T^{\deg v})} \in \overline{\Q_\ell}[[T]], $$
where $ \LL_v(\rho, T) $ is defined similarly as before.

\begin{corollary}[of the Weil conjectures \footnote{by the works of Grothendieck and Deligne}]
If $ \rho $ has no $ G_{\overline{k}K} $-invariants, then $ \LL(\rho, T) \in \overline{\Q_\ell}[T] $ has degree
$$ d_\rho := (2g_C - 2)\dim\rho + \deg\ff_\rho, $$
and satisfies the functional equation
$$ \LL(\rho, T) = \epsilon_\rho q^{d_\rho(\tfrac{w_\rho + 1}{2})}T^{d_\rho}\LL(\rho, 1 / q^{w_\rho + 1}T)^{\sigma_\rho}, $$
where $ w_\rho $ is the weight of $ \rho $ and $ \sigma_\rho $ is some automorphism on $ \overline{\Q_\ell} $.
\end{corollary}

\end{frame}

\begin{frame}{Magma implementation}

I have implemented \texttt{intrinsic}s for computing formal L-functions of arbitrary $ \ell $-adic representations with or without functional equations.

\bigskip This includes specific examples of motives over $ k(t) $:
\begin{itemize}
\item elliptic curves, with functional equation except when $ \ch(k) = 2, 3 $
\begin{itemize}
\item functional equation when $ \ch(k) = 2, 3 $ require Hilbert symbols
\item faster than \texttt{LFunction()} when $ \ch(k) = 2, 3, 5, 7 $
\end{itemize}
\item Dirichlet characters, without functional equation
\begin{itemize}
\item functional equation requires efficient computations of Gauss sums
\item non-square-free modulus is surprisingly tricky
\end{itemize}
\item tensor products assuming their conductors are disjoint
\begin{itemize}
\item degree computation requires $ \ff_{\rho \otimes \tau} $ in terms of $ \ff_\rho $ and $ \ff_\tau $
\item functional equation requires $ \epsilon_{\rho \otimes \tau} $ in terms of $ \epsilon_\rho $ and $ \epsilon_\tau $
\end{itemize}
\item any other nice motives?
\begin{itemize}
\item hyperelliptic curves?
\item Artin representations?
\end{itemize}
\end{itemize}

\end{frame}

\end{document}