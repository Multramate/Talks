\ifx\type\undefined
  \documentclass[10pt, t]{beamer}
  \setbeamertemplate{footline}[page number]
\else
  \documentclass[10pt]{article}
  \usepackage[margin=1in]{geometry}
\fi

\usepackage{amsmath}
\usepackage{amssymb}
\usepackage{amsthm}
\usepackage{bbm}
\usepackage{cancel}
\usepackage{listings}
\usepackage{mathrsfs}
\usepackage{multirow}
\usepackage{soul}
\usepackage{stmaryrd}
\usepackage{tikz}
\usepackage{tikz-cd}
\usepackage{wrapfig}

\newtheorem*{algorithm}{Algorithm}
\newtheorem*{assumptions}{Assumptions}
\newtheorem*{conjecture}{Conjecture}
\newtheorem*{consequences}{Consequences}
\newtheorem*{exercise}{Exercise}
\newtheorem*{formalisation}{Formalisation}
\newtheorem*{proposition}{Proposition}
\newtheorem*{question}{Question}
\newtheorem*{remark}{Remark}

\ifx\type\undefined\else
  \newtheorem*{definition}{Definition}
  \newtheorem*{example}{Example}
  \newtheorem*{lemma}{Lemma}
  \newtheorem*{theorem}{Theorem}
\fi

\definecolor{keywordcolor}{rgb}{0.7, 0.1, 0.1}
\definecolor{tacticcolor}{rgb}{0.0, 0.1, 0.6}
\definecolor{commentcolor}{rgb}{0.4, 0.4, 0.4}
\definecolor{symbolcolor}{rgb}{0.0, 0.1, 0.6}
\definecolor{sortcolor}{rgb}{0.1, 0.5, 0.1}
\definecolor{attributecolor}{rgb}{0.7, 0.1, 0.1}
\def\lstlanguagefiles{lstlean.tex}
\lstset{language=lean}

\newcommand\A{\mathbb{A}}
\newcommand\C{\mathbb{C}}
\newcommand\F{\mathbb{F}}
\newcommand\G{\mathbb{G}}
\renewcommand\H{\mathbb{H}}
\newcommand\I{\mathbb{I}}
\newcommand\N{\mathbb{N}}
\renewcommand\P{\mathbb{P}}
\newcommand\Q{\mathbb{Q}}
\newcommand\R{\mathbb{R}}
\newcommand\Z{\mathbb{Z}}

\renewcommand\AA{\mathcal{A}}
\newcommand\BB{\mathcal{B}}
\newcommand\CC{\mathcal{C}}
\newcommand\DD{\mathcal{D}}
\newcommand\EE{\mathcal{E}}
\newcommand\FF{\mathcal{F}}
\newcommand\GG{\mathcal{G}}
\newcommand\HH{\mathcal{H}}
\newcommand\II{\mathcal{I}}
\newcommand\LL{\mathcal{L}}
\newcommand\MM{\mathcal{M}}
\newcommand\NN{\mathcal{N}}
\newcommand\OO{\mathcal{O}}
\newcommand\PP{\mathcal{P}}
\newcommand\RR{\mathcal{R}}
\renewcommand\SS{\mathcal{S}}
\newcommand\TT{\mathcal{T}}
\newcommand\XX{\mathcal{X}}

\renewcommand\aa{\mathfrak{a}}
\newcommand\cc{\mathfrak{c}}
\newcommand\dd{\mathfrak{d}}
\newcommand\ff{\mathfrak{f}}
\renewcommand\gg{\mathfrak{g}}
\newcommand\mm{\mathfrak{m}}
\newcommand\pp{\mathfrak{p}}
\newcommand\qq{\mathfrak{q}}
\renewcommand\ss{\mathfrak{s}}

\newcommand\LLL{\mathscr{L}}

\newcommand\ab{\mathrm{ab}}
\newcommand\Ab{\mathbf{Ab}}
\newcommand\Alg{\mathbf{Alg}}
\newcommand\Aff{\mathbf{Aff}}
\newcommand\Aut{\operatorname{Aut}}
\newcommand\Az{\mathrm{Az}}
\newcommand\Br{\operatorname{Br}}
\newcommand\BSD{\operatorname{BSD}}
\newcommand\ch{\operatorname{char}}
\newcommand\Cl{\operatorname{Cl}}
\newcommand\coker{\operatorname{coker}}
\newcommand\cris{\mathrm{cris}}
\renewcommand\d{\mathrm{d}}
\newcommand\Div{\operatorname{Div}}
\newcommand\dR{\mathrm{dR}}
\newcommand\EN{\operatorname{EN}}
\newcommand\End{\operatorname{End}}
\newcommand\ES{\operatorname{ES}}
\newcommand\et{\mathrm{\acute{e}t}}
\newcommand\Et{\mathbf{\acute{E}t}}
\newcommand\Ext{\operatorname{Ext}}
\newcommand\Fr{\operatorname{Fr}}
\newcommand\Frac{\operatorname{Frac}}
\newcommand\Gal{\operatorname{Gal}}
\newcommand\GL{\operatorname{GL}}
\newcommand\Gr{\mathrm{Gr}}
\newcommand\Hom{\operatorname{Hom}}
\newcommand\HT{\mathrm{HT}}
\newcommand\id{\operatorname{id}}
\newcommand\im{\operatorname{im}}
\newcommand\Ind{\operatorname{Ind}}
\renewcommand\inf{\operatorname{inf}}
\newcommand\inv{\operatorname{inv}}
\newcommand\Irr{\operatorname{Irr}}
\newcommand\Jac{\operatorname{Jac}}
\newcommand\lcm{\operatorname{lcm}}
\newcommand\Mat{\operatorname{Mat}}
\newcommand\Mod{\mathbf{Mod}}
\newcommand\Nm{\operatorname{Nm}}
\newcommand\nr{\mathrm{nr}}
\newcommand\NS{\operatorname{NS}}
\newcommand\Ob{\operatorname{Ob}}
\newcommand\ord{\operatorname{ord}}
\newcommand\op{\mathrm{op}}
\newcommand\PGL{\operatorname{PGL}}
\newcommand\Pic{\operatorname{Pic}}
\newcommand\Prob{\operatorname{Prob}}
\newcommand\Proj{\operatorname{Proj}}
\newcommand\PSh{\mathbf{PSh}}
\newcommand\Reg{\operatorname{Reg}}
\newcommand\res{\operatorname{res}}
\newcommand\rk{\operatorname{rk}}
\newcommand\Sch{\mathbf{Sch}}
\newcommand\Sel{\operatorname{Sel}}
\newcommand\Set{\mathbf{Set}}
\newcommand\sgn{\operatorname{sgn}}
\newcommand\Sh{\mathbf{Sh}}
\newcommand\SL{\operatorname{SL}}
\newcommand\Spec{\operatorname{Spec}}
\newcommand\supp{\operatorname{supp}}
\newcommand\Tam{\operatorname{Tam}}
\newcommand\Top{\mathbf{Top}}
\newcommand\tor{\operatorname{tor}}
\newcommand\tr{\operatorname{tr}}
\newcommand\tra{\operatorname{tra}}
\newcommand\WC{\operatorname{WC}}

\DeclareFontFamily{U}{wncyr}{}
\DeclareFontShape{U}{wncyr}{m}{n}{<->wncyr10}{}
\DeclareSymbolFont{cyr}{U}{wncyr}{m}{n}
\DeclareMathSymbol{\Sha}{\mathord}{cyr}{"58}

\newcommand{\function}[5][]{
  \if &#1&
    \begin{array}{rcl}
      #2 & \longrightarrow & #3 \\
      #4 & \longmapsto     & #5
    \end{array}
  \else
    \begin{array}{rcrcl}
      #1 & : & #2 & \longrightarrow & #3 \\
         &   & #4 & \longmapsto     & #5
    \end{array}
  \fi
}

\newcommand{\functions}[7][]{
  \if &#1&
    \begin{array}{rcl}
      #2 & \longrightarrow & #3 \\
      #4 & \longmapsto     & #5 \\
      #6 & \longmapsto     & #7 \\
    \end{array}
  \else
    \begin{array}{rcrcl}
      #1 & : & #2 & \longrightarrow & #3 \\
         &   & #4 & \longmapsto     & #5 \\
         &   & #6 & \longmapsto     & #7
    \end{array}
  \fi
}
\title{Schinzel's hypothesis H}
\subtitle{Open problems in number theory}
\author{David Kurniadi Angdinata}
\institute{University College London}
\date{Thursday, 31 October 2024}

\begin{document}

\frame\maketitle

\begin{frame}[c]{Some fun quotes}

\begin{center}
Skorobogatov--Morgan (2024):

\bigskip \emph{A notoriously difficult conjecture on prime values of polynomials, deemed to be inaccessible in the current state of analytic number theory}.
\end{center}

\bigskip

\begin{center}
Bunyakovsky (1857):

\bigskip \emph{Il est \`a pr\'esumer que la d\'emonstration rigoureuse du th\'eor\`eme \'enonc\'e sur les progressions arithm\'etiques des ordres sup\'erieurs conduirait, dans l'\'etat actuel de la th\'eorie des nombres, \`a des difficult\'es insurmontables; n\'eanmoins, sa r\'ealit\'e ne peut pas \^etre r\'evoqu\'ee en doute.}
\end{center}

\end{frame}

\begin{frame}{Primes in arithmetic progressions}

\begin{theorem}[Dirichlet, 1837]
Let $ a, b \in \Z $. Assume no primes $ p $ satisfy $ p \mid a $ and $ p \mid b $. Then there are infinitely many $ n $ such that $ an + b $ is prime.
\end{theorem}

\bigskip

\begin{example}[$ 4X + 3 $]
\vspace{-0.5cm}
$$
\begin{array}{|c|c|c|c|c|c|c|c|c|c|c|c|c|c|}
\hline
n & 0 & 1 & 2 & 3 & 4 & 5 & 6 & 7 & 8 & 9 & 10 & 11 & 12 \\
\hline
4n + 3 & 3 & 7 & 11 & 15 & 19 & 23 & 27 & 31 & 35 & 39 & 43 & 47 & 51 \\
\hline
\text{prime} & $ \checkmark $ & \checkmark & \checkmark & & \checkmark & \checkmark & & \checkmark & & & \checkmark & \checkmark & \\
\hline
\end{array}
$$
If there were a finite set $ S := \{p \ \text{prime} : p \equiv 3 \mod 4\} $, then
$$ N := 2 + \prod_{p \in S} p^2 \equiv 3 \mod 4, $$
so $ N $ has a prime factor $ q \equiv 3 \mod 4 $ not in $ S $, which is a contradiction.
\end{example}

\end{frame}

\begin{frame}{Primes in polynomial sequences}

\begin{conjecture}[Bunyakovsky, 1857]
Let $ f \in \Z[X] $ be irreducible. Assume no primes $ p $ satisfy ``$ p \mid f(n) $ for all $ n $''. Then there are infinitely many $ n $ such that $ f(n) $ is prime.
\end{conjecture}

\bigskip This is Dirichlet's theorem when $ f(X) = aX + b $.

\bigskip

\begin{example}[$ X^2 + 1 $]
\vspace{-0.5cm}
$$
\begin{array}{|c|c|c|c|c|c|c|c|c|c|c|c|c|c|}
\hline
n & 0 & 1 & 2 & 3 & 4 & 5 & 6 & 7 & 8 & 9 & 10 & 11 & 12 \\
\hline
n^2 + 1 & 1 & 2 & 5 & 10 & 17 & 26 & 37 & 50 & 65 & 82 & 101 & 122 & 145 \\
\hline
\text{prime} & & \checkmark & \checkmark & & \checkmark & & \checkmark & & & & \checkmark & & \\
\hline
\end{array}
$$
This is one of the four Landau's problems, amongst Goldbach's conjecture, the twin prime conjecture, and Legendre's conjecture.
\end{example}

\end{frame}

\begin{frame}{Simultaneous primes in arithmetic progressions}

\begin{conjecture}[Dickson, 1904]
Let $ a_1, \dots, a_k, b_1, \dots, b_k \in \Z $. Set $ f(X) := (a_1X + b_1) \cdot \dots \cdot (a_kX + b_k) $. Assume no primes $ p $ satisfy ``$ p \mid f(n) $ for all $ n $''. Then there are infinitely many $ n $ such that $ a_1n + b_1, \dots, a_kn + b_k $ are simultaneously prime.
\end{conjecture}

\bigskip This is the twin prime conjecture for $ X $ and $ X + 2 $.

\bigskip

\begin{example}[$ X $ and $ 2X + 1 $]
\vspace{-0.5cm}
$$
\begin{array}{|c|c|c|c|c|c|c|c|c|c|c|c|c|c|}
\hline
p & 2 & 3 & 5 & 7 & 11 & 13 & 17 & 19 & 23 & 29 & 31 & 37 & 41 \\
\hline
2p + 1 & 5 & 7 & 11 & 15 & 23 & 27 & 35 & 39 & 47 & 59 & 63 & 75 & 83 \\
\hline
\text{prime} & \checkmark & \checkmark & \checkmark & & \checkmark & & & & \checkmark & \checkmark & & & \checkmark \\
\hline
\end{array}
$$
This is the Germain prime conjecture, which implies that there are infinitely many composite Mersenne numbers, since $ 2p + 1 \mid 2^p - 1 $ whenever $ p \equiv 3 \mod 4 $ is a Germain prime.
\end{example}

\end{frame}

\begin{frame}{Density of simultaneous primes}

\begin{conjecture}[Hardy--Littlewood, 1923]
Let $ a_1, \dots, a_k, b_1, \dots, b_k \in \Z $. Set $ f(X) := (a_1X + b_1) \cdot \dots \cdot (a_kX + b_k) $. Assume no primes $ p $ satisfy ``$ p \mid f(n) $ for all $ n $''. Then
$$ \#\left\{n \le N : \begin{array}{c} a_1n + b_1, \dots, a_kn + b_k \\ \text{are simultaneously prime} \end{array}\right\} \sim C \cdot \dfrac{N}{\log^k N}. $$
Here,
$$ C := \prod_p \left(1 - \dfrac{1}{p}\right)^{-k}\left(1 - \dfrac{\#\{n \in \F_p : f(n) = 0\}}{p}\right). $$
\end{conjecture}

If $ f_1(X) = X $, then this is the prime number theorem that
$$ \#\{n \le N : n \ \text{is prime}\} \sim \dfrac{N}{\log N}. $$
If $ f_1(X) = X $ and $ f_2(X) = X + 2 $, then $ C $ is the twin prime constant.

\end{frame}

\begin{frame}{Simultaneous primes in polynomial sequences}

\begin{conjecture}[Schinzel's hypothesis H, 1958]
Let $ f_1, \dots, f_k \in \Z[X] $ be irreducible. Set $ f := f_1 \cdot \dots \cdot f_k $. Assume no primes $ p $ satisfy ``$ p \mid f(n) $ for all $ n $''. Then there are infinitely many $ n $ such that $ f_1(n), \dots, f_k(n) $ are simultaneously prime.
\end{conjecture}

\bigskip

\begin{conjecture}[Bateman--Horn, 1962]
Let $ f_1, \dots, f_k \in \Z[X] $ be irreducible. Set $ f := f_1 \cdot \dots \cdot f_k $. Assume no primes $ p $ satisfy ``$ p \mid f(n) $ for all $ n $''. Then
$$ \#\left\{n \le N : \begin{array}{c} f_1(n), \dots, f_k(n) \\ \text{are simultaneously prime} \end{array}\right\} \sim C \cdot \dfrac{N}{\prod_i \deg f_i \cdot \log^k N}. $$
Here,
$$ C := \prod_p \left(1 - \dfrac{1}{p}\right)^{-k}\left(1 - \dfrac{\#\{n \in \F_p : f(n) = 0\}}{p}\right). $$
\end{conjecture}

\end{frame}

\begin{frame}{Multivariate variants}

\begin{theorem}[Friedlander--Iwaniec, 1997]
There are infinitely many $ (x, y) \in \Z^2 $ such that $ x^2 + y^4 $ is prime.
\end{theorem}

\begin{theorem}[Green--Tao--Ziegler, 2006]
Let $ f_1, \dots, f_k \in \Z[X] $ such that $ f_i(0) = 0 $. Then there are infinitely many $ (x, y) \in \Z^2 $ such that $ x + f_1(y), \dots, x + f_k(y) $ are simultaneously prime.
\end{theorem}

\begin{theorem}[Bodin--D\`ebes--Najib, 2019]
Let $ R $ be a characteristic zero UFD whose fraction field satisfies the product formula, and let $ f_1, \dots, f_k \in R[X, Y] $. Then there are $ y \in R[X] $ such that $ f_1(X, y(X)), \dots, f_k(X, y(X)) $ are simultaneously irreducible.
\end{theorem}

\bigskip

\begin{example}[{$ X^8 + t^3 $ over $ \F_2[t] $}]
$ (t^2 + t + 1)^8 + t^3 = (t + 1)(t^{15} + t^{14} + t^{13} + t^{12} + t^{11} + t^{10} + t^9 + t^8 + t^2 + t + 1) $.
\end{example}

\end{frame}

\begin{frame}{Genericity of simultaneous primes}

Let $ P_{d, N} $ be the set of $ a_dX^d + \dots + a_0 \in \Z[X] $ such that $ |a_i| \le N $.

\begin{theorem}[Skorobogatov--Sofos, 2023]
Let $ S_{d, N} $ be the set of $ f \in P_{d, N} $ such that $ X^p - X \nmid f $ in all $ \F_p[X] $. Then
$$ \lim_{N \to \infty} \dfrac{\#\left\{(f_1, \dots, f_k) \in S_{d, N}^k : \begin{array}{c} \exists n \in \Z, \ f_1(n), \dots, f_k(n) \\ \text{are simultaneously prime} \end{array}\right\}}{\#S_{d, N}^k} = 1. $$
\end{theorem}

\begin{theorem}[Skorobogatov--Sofos, 2023]
Let $ K $ be a cyclic number field with integral basis $ e_1, \dots, e_m $ of $ \OO_K $. Then
$$ \lim_{N \to \infty} \dfrac{\#\left\{f \in P_{d, N} : \begin{array}{c} \Nm_\Q^K(e_1X_1 + \dots + e_mX_m) = f(X) \\ \text{has a rational point} \end{array}\right\}}{\#P_{d, N}} = 1. $$
\end{theorem}

\end{frame}

\begin{frame}{The Hasse principle}

The Hasse principle holds for a variety $ V $ over a global field $ K $ if it has a point in $ K $ whenever it has points in $ K_v $ for all places $ v $ of $ K $.

\begin{theorem}[Hasse--Minkowski theorem]
Let $ a_1, \dots, a_m \in \Q $. Then the Hasse principle holds for
$$ a_1X_1^2 + \dots + a_mX_m^2. $$
\end{theorem}

The proof for $ m = 4 $ reduces to the proof for $ m = 3 $ by Dirichlet's theorem and the fundamental exact sequence of global class field theory.

\begin{theorem}[Hasse norm theorem]
Let $ K $ be a cyclic number field. Then there is a short exact sequence
$$ 1 \to \Q^\times / \Nm_\Q^K(K^\times) \to \bigoplus_{p \le \infty} \Q_p^\times / \Nm_\Q^K((K \otimes_\Q \Q_p)^\times) \to \Gal(K / \Q) \to 1. $$
\end{theorem}

Thus a local norm everywhere except possibly one place is a global norm.

\end{frame}

\begin{frame}{Application of Dirichlet's theorem}

\begin{example}[$ Y^2 + 3Z^2 = 5X + 7 $]
Claim that the Hasse principle holds. By the Hasse norm theorem, it suffices to find some $ x \in \Q $ such that $ Y^2 + 3Z^2 = 5x + 7 $ has points in $ \Q_p $ for all places $ p $ of $ \Q $ except possibly one prime. Observe that
$$ (1)^2 + 3(1)^2 \equiv 5(1) + 7 \mod 2^3, $$
$$ (3)^2 + 3(1)^2 \equiv 5(1) + 7 \mod 3^3, $$
so it has points in $ \Q_2 $ and $ \Q_3 $ by Hensel's lemma. It suffices to find some $ x \in \Q $ such that $ x \equiv 1 \mod 2^3 $ and $ x \equiv 1 \mod 3^3 $, so that
$$ 5x + 7 = 5(2^3 \cdot 3^3 \cdot n + 1) + 7 = 2^2 \cdot 3 \cdot (90n + 1). $$
By Dirichlet's theorem, there is some $ n $ such that $ 90n + 1 $ is prime. For instance, $ n = 2 $ gives $ Y^2 + 3Z^2 = 2^2 \cdot 3 \cdot 181 $, which has points in $ \Q_2 $, $ \Q_3 $, and $ \R $, but also $ \Q_p $ for all primes $ p $ except $ 181 $.
\end{example}

\end{frame}

\begin{frame}{Application of Schinzel's hypothesis H}

Dirichlet's theorem can be replaced by assuming Schinzel's hypothesis H.

\bigskip

\begin{theorem}[Colliot-Th\'el\`ene--Sansuc, 1982]
Let $ a_1, \dots, a_k \in \Q^\times $, and let $ f_1, \dots, f_k \in \Q[X] $ be irreducible. Assume Schinzel's hypothesis H. Then the Hasse principle holds for
$$ Y_1^2 + a_1Z_1^2 = f_1(X), \qquad \dots, \qquad Y_k^2 + a_kZ_k^2 = f_k(X). $$
\end{theorem}

Thus the Hasse principle conditionally holds for conic bundles over $ \P_\Q^1 $.

\bigskip

\begin{example}[Iskovskikh, 1971]
Let $ V $ be the variety over $ \Q $ given by $ Y^2 + Z^2 = -(X - 2)(X - 3) $. Then $ V $ has points in $ \R $ and $ \Q_p $ for all primes $ p $ but no points in $ \Q $. The failure of the Hasse principle can be detected by $ (3 - X^2, - 1) \in \Br(V)[2] $.
\end{example}

\end{frame}

\begin{frame}{The Brauer--Manin obstruction}

Let $ V $ be a variety over a global field $ K $. There is a commutative diagram
$$
\begin{tikzcd}[ampersand replacement=\&]
\& V(K) \arrow{r} \arrow{d} \& V(\A_K) \arrow{d}[swap]{(-)^*} \& \& \\
0 \arrow{r} \& \Br(K) \arrow{r} \& \displaystyle\bigoplus_v \Br(K_v) \arrow{r}[swap]{\inv_v} \& \Q / \Z \arrow{r} \& 0.
\end{tikzcd}
$$
For any $ A \in \Br(V) $, the Brauer--Manin set is
$$ V(\A_K)^A := \left\{(x_v)_v \in V(\A_K) : \sum_v \inv_v(x_v^*A) = 0\right\}. $$

\begin{example}[Iskovskikh, 1971]
Let $ A := (3 - X^2, -1) \in \Br(V) $. For any $ (x_v)_v \in V(\A_K) $, it can be shown that $ \sum_v \inv_v(x_v^*A) = \tfrac{1}{2} $, so that $ V(K) \subseteq V(\A_K)^A = \emptyset $.
\end{example}

\end{frame}

\begin{frame}{Rationally connected varieties}

A rationally connected variety is a smooth projective variety such that any two geometric points are connected by a rational curve.

\bigskip

\begin{conjecture}[Colliot-Th\'el\`ene, 2003]
Let $ V $ be a rationally connected variety over a number field $ K $. If $ V(K) = \emptyset $, then $ V(\A_K)^A = \emptyset $ for some $ A \in \Br(V) $.
\end{conjecture}

\bigskip This is known for conic bundles over $ \P_\Q^1 $ with at most five geometric degenerate fibres, due to Colliot-Th\'el\`ene--Sansuc--Swinnerton-Dyer (1987), Colliot-Th\'el\`ene (1990), and Salberger--Skorobogatov (1991).

\bigskip

\begin{theorem}[Colliot-Th\'el\`ene--Swinnerton-Dyer, 1994]
Assume Schinzel's hypothesis H. Then Colliot-Th\'el\`ene's conjecture holds for Severi--Brauer bundles over $ \P_\Q^1 $.
\end{theorem}

\end{frame}

\end{document}