\ifx\type\undefined
  \documentclass[10pt, t]{beamer}
  \setbeamertemplate{footline}[page number]
\else
  \documentclass[10pt]{article}
  \usepackage[margin=1in]{geometry}
\fi

\usepackage{amsmath}
\usepackage{amssymb}
\usepackage{amsthm}
\usepackage{bbm}
\usepackage{cancel}
\usepackage{listings}
\usepackage{mathrsfs}
\usepackage{multirow}
\usepackage{soul}
\usepackage{stmaryrd}
\usepackage{tikz}
\usepackage{tikz-cd}
\usepackage{wrapfig}

\newtheorem*{algorithm}{Algorithm}
\newtheorem*{assumptions}{Assumptions}
\newtheorem*{conjecture}{Conjecture}
\newtheorem*{consequences}{Consequences}
\newtheorem*{exercise}{Exercise}
\newtheorem*{formalisation}{Formalisation}
\newtheorem*{proposition}{Proposition}
\newtheorem*{question}{Question}
\newtheorem*{remark}{Remark}

\ifx\type\undefined\else
  \newtheorem*{definition}{Definition}
  \newtheorem*{example}{Example}
  \newtheorem*{lemma}{Lemma}
  \newtheorem*{theorem}{Theorem}
\fi

\definecolor{keywordcolor}{rgb}{0.7, 0.1, 0.1}
\definecolor{tacticcolor}{rgb}{0.0, 0.1, 0.6}
\definecolor{commentcolor}{rgb}{0.4, 0.4, 0.4}
\definecolor{symbolcolor}{rgb}{0.0, 0.1, 0.6}
\definecolor{sortcolor}{rgb}{0.1, 0.5, 0.1}
\definecolor{attributecolor}{rgb}{0.7, 0.1, 0.1}
\def\lstlanguagefiles{lstlean.tex}
\lstset{language=lean}

\newcommand\A{\mathbb{A}}
\newcommand\C{\mathbb{C}}
\newcommand\F{\mathbb{F}}
\newcommand\G{\mathbb{G}}
\renewcommand\H{\mathbb{H}}
\newcommand\I{\mathbb{I}}
\newcommand\N{\mathbb{N}}
\renewcommand\P{\mathbb{P}}
\newcommand\Q{\mathbb{Q}}
\newcommand\R{\mathbb{R}}
\newcommand\Z{\mathbb{Z}}

\renewcommand\AA{\mathcal{A}}
\newcommand\BB{\mathcal{B}}
\newcommand\CC{\mathcal{C}}
\newcommand\DD{\mathcal{D}}
\newcommand\EE{\mathcal{E}}
\newcommand\FF{\mathcal{F}}
\newcommand\GG{\mathcal{G}}
\newcommand\HH{\mathcal{H}}
\newcommand\II{\mathcal{I}}
\newcommand\LL{\mathcal{L}}
\newcommand\MM{\mathcal{M}}
\newcommand\NN{\mathcal{N}}
\newcommand\OO{\mathcal{O}}
\newcommand\PP{\mathcal{P}}
\newcommand\RR{\mathcal{R}}
\renewcommand\SS{\mathcal{S}}
\newcommand\TT{\mathcal{T}}
\newcommand\XX{\mathcal{X}}

\renewcommand\aa{\mathfrak{a}}
\newcommand\cc{\mathfrak{c}}
\newcommand\dd{\mathfrak{d}}
\newcommand\ff{\mathfrak{f}}
\renewcommand\gg{\mathfrak{g}}
\newcommand\mm{\mathfrak{m}}
\newcommand\pp{\mathfrak{p}}
\newcommand\qq{\mathfrak{q}}
\renewcommand\ss{\mathfrak{s}}

\newcommand\LLL{\mathscr{L}}

\newcommand\ab{\mathrm{ab}}
\newcommand\Ab{\mathbf{Ab}}
\newcommand\Alg{\mathbf{Alg}}
\newcommand\Aff{\mathbf{Aff}}
\newcommand\Aut{\operatorname{Aut}}
\newcommand\Az{\mathrm{Az}}
\newcommand\Br{\operatorname{Br}}
\newcommand\BSD{\operatorname{BSD}}
\newcommand\ch{\operatorname{char}}
\newcommand\Cl{\operatorname{Cl}}
\newcommand\coker{\operatorname{coker}}
\newcommand\cris{\mathrm{cris}}
\renewcommand\d{\mathrm{d}}
\newcommand\Div{\operatorname{Div}}
\newcommand\dR{\mathrm{dR}}
\newcommand\EN{\operatorname{EN}}
\newcommand\End{\operatorname{End}}
\newcommand\ES{\operatorname{ES}}
\newcommand\et{\mathrm{\acute{e}t}}
\newcommand\Et{\mathbf{\acute{E}t}}
\newcommand\Ext{\operatorname{Ext}}
\newcommand\Fr{\operatorname{Fr}}
\newcommand\Frac{\operatorname{Frac}}
\newcommand\Gal{\operatorname{Gal}}
\newcommand\GL{\operatorname{GL}}
\newcommand\Gr{\mathrm{Gr}}
\newcommand\Hom{\operatorname{Hom}}
\newcommand\HT{\mathrm{HT}}
\newcommand\id{\operatorname{id}}
\newcommand\im{\operatorname{im}}
\newcommand\Ind{\operatorname{Ind}}
\renewcommand\inf{\operatorname{inf}}
\newcommand\inv{\operatorname{inv}}
\newcommand\Irr{\operatorname{Irr}}
\newcommand\Jac{\operatorname{Jac}}
\newcommand\lcm{\operatorname{lcm}}
\newcommand\Mat{\operatorname{Mat}}
\newcommand\Mod{\mathbf{Mod}}
\newcommand\Nm{\operatorname{Nm}}
\newcommand\nr{\mathrm{nr}}
\newcommand\NS{\operatorname{NS}}
\newcommand\Ob{\operatorname{Ob}}
\newcommand\ord{\operatorname{ord}}
\newcommand\op{\mathrm{op}}
\newcommand\PGL{\operatorname{PGL}}
\newcommand\Pic{\operatorname{Pic}}
\newcommand\Prob{\operatorname{Prob}}
\newcommand\Proj{\operatorname{Proj}}
\newcommand\PSh{\mathbf{PSh}}
\newcommand\Reg{\operatorname{Reg}}
\newcommand\res{\operatorname{res}}
\newcommand\rk{\operatorname{rk}}
\newcommand\Sch{\mathbf{Sch}}
\newcommand\Sel{\operatorname{Sel}}
\newcommand\Set{\mathbf{Set}}
\newcommand\sgn{\operatorname{sgn}}
\newcommand\Sh{\mathbf{Sh}}
\newcommand\SL{\operatorname{SL}}
\newcommand\Spec{\operatorname{Spec}}
\newcommand\supp{\operatorname{supp}}
\newcommand\Tam{\operatorname{Tam}}
\newcommand\Top{\mathbf{Top}}
\newcommand\tor{\operatorname{tor}}
\newcommand\tr{\operatorname{tr}}
\newcommand\tra{\operatorname{tra}}
\newcommand\WC{\operatorname{WC}}

\DeclareFontFamily{U}{wncyr}{}
\DeclareFontShape{U}{wncyr}{m}{n}{<->wncyr10}{}
\DeclareSymbolFont{cyr}{U}{wncyr}{m}{n}
\DeclareMathSymbol{\Sha}{\mathord}{cyr}{"58}

\newcommand{\function}[5][]{
  \if &#1&
    \begin{array}{rcl}
      #2 & \longrightarrow & #3 \\
      #4 & \longmapsto     & #5
    \end{array}
  \else
    \begin{array}{rcrcl}
      #1 & : & #2 & \longrightarrow & #3 \\
         &   & #4 & \longmapsto     & #5
    \end{array}
  \fi
}

\newcommand{\functions}[7][]{
  \if &#1&
    \begin{array}{rcl}
      #2 & \longrightarrow & #3 \\
      #4 & \longmapsto     & #5 \\
      #6 & \longmapsto     & #7 \\
    \end{array}
  \else
    \begin{array}{rcrcl}
      #1 & : & #2 & \longrightarrow & #3 \\
         &   & #4 & \longmapsto     & #5 \\
         &   & #6 & \longmapsto     & #7
    \end{array}
  \fi
}
\title{The Tate--Shafarevich and Brauer groups}
\subtitle{Curves over function fields}
\author{David Kurniadi Angdinata}
\institute{University College London}
\date{Tuesday, 5 July 2022}

\begin{document}

\frame\maketitle

\begin{frame}[c]{Overview}

Part I
\begin{itemize}
\item The Tate--Shafarevich group of a $ \begin{cases} \text{number field} \\ \text{function field} \end{cases} $
\item The Artin--Tate conjecture
\end{itemize}

\bigskip Part II
\begin{itemize}
\item The Brauer--$ \begin{cases} \text{Grothendieck} \\ \text{Azumaya} \end{cases} $ group of a $ \begin{cases} \text{field} \\ \text{scheme} \end{cases} $
\item The Brauer--Manin obstruction
\end{itemize}

\end{frame}

\begin{frame}{The Tate--Shafarevich group of a number field}

Let $ E $ be an elliptic curve over a number field $ K $. Let
$$ V_K := \{\text{closed points of} \ \Spec(\OO_K)\} \cup V_K^\infty. $$
The \textbf{Tate--Shafarevich group} is
$$ \Sha(E / K) := \ker\left(H^1(K, E) \to \prod_{v \in V_K} H^1(K_v, E)\right). $$
Note that there is a bijection
$$ H^1(K, E) \xrightarrow{\sim} \WC(E / K), $$
the \textbf{Weil--Ch\^atelet group} of torsors for $ E / K $. Thus $ 0 \ne C \in \Sha(E / K) $ is a $ K $-twist of $ E $ that is everywhere locally soluble but globally insoluble.

\begin{example}[Selmer]
The curve $ 3X^3 + 4Y^3 + 5Z^3 = 0 $ is a $ \Q $-twist of $ E : X^3 + Y^3 + 60Z^3 = 0 $ that is everywhere locally soluble but globally insoluble, so $ \Sha(E / \Q) \ne 0 $.
\end{example}

\end{frame}

\begin{frame}{The Tate--Shafarevich group of a number field}

Let $ E $ be an elliptic curve over a number field $ K $. Let
$$ V_K := \{\text{closed points of} \ \Spec(\OO_K)\} \cup V_K^\infty. $$
The \textbf{Tate--Shafarevich group} is
$$ \Sha(E / K) := \ker\left(H^1(K, E) \to \prod_{v \in V_K} H^1(K_v, E)\right). $$

\begin{conjecture}[Tate--Shafarevich]
$ \#\Sha(E / K) $ is finite.
\end{conjecture}

\begin{conjecture}[Birch--Swinnerton-Dyer]
Assuming TS holds,
$$ \lim_{s \to 1} \dfrac{L(E / K, s)}{(s - 1)^{\rk(E / K)}} = \dfrac{R \cdot \#\Sha(E / K) \cdot \tau}{\#E(K)_{\tor}^2}. $$
\end{conjecture}

\end{frame}

\begin{frame}{The Tate--Shafarevich group of a function field}

Let $ E $ be an elliptic curve over a function field $ K = \F_q(C) $. Let
$$ V_K := \{\text{closed points of} \ C\}. $$
The \textbf{Tate--Shafarevich group} is
$$ \Sha(E / K) := \ker\left(H^1(K, E) \to \prod_{v \in V_K} H^1(K_v, E)\right). $$

\begin{conjecture}[Tate--Shafarevich]
$ \#\Sha(E / K) $ is finite.
\end{conjecture}

\begin{theorem}[KT03]
Assuming TS$ [\ell^\infty] $ holds for some $ \ell $,
$$ \lim_{s \to 1} \dfrac{L(E / K, s)}{(s - 1)^{\rk(E / K)}} = \dfrac{R \cdot \#\Sha(E / K) \cdot \tau}{\#E(K)_{\tor}^2}. $$
\end{theorem}

\end{frame}

\begin{frame}{The Tate--Shafarevich group of a function field}

Let $ E $ be an elliptic curve over a function field $ K = \F_q(C) $. Let
$$ V_K := \{\text{closed points of} \ C\}. $$
The \textbf{Tate--Shafarevich group} is
$$ \Sha(E / K) := \ker\left(H^1(K, E) \to \prod_{v \in V_K} H^1(K_v, E)\right). $$

\begin{theorem}[Mil68, Mil70, ASD73]
TS holds if $ \EE $ is constant, rational, or K3.
\end{theorem}

\bigskip

\begin{theorem}[Ulm12, Proposition 5.3.1]
$ \Br(\EE) \xrightarrow{\sim} \Sha(E / K) $.
\end{theorem}

\end{frame}

\begin{frame}{The Artin--Tate conjecture}

Let $ \EE \to C $ be an elliptic surface over $ \F_q $ with generic fibre $ E / K $. Then
\begin{align*}
\text{BSD holds for} \ E \qquad
& \overset{\text{KT03}}{\iff} \qquad \#\Sha(E / K)[\ell^\infty] \ \text{is finite} \ \text{for some} \ \ell \\
& \overset{\text{Gro79}}{\iff} \qquad \#\Br(\EE)[\ell^\infty] \ \text{is finite} \ \text{for some} \ \ell \\
& \overset{\text{Mil75}}{\iff} \qquad \text{AT (and T) holds for} \ \EE.
\end{align*}

\begin{conjecture}[Artin--Tate]
Let $ X $ be a smooth projective geometrically-connected surface over $ \F_q $. Then $ \#\Br(X) $ is finite, and if $ \NS(X)_{/ \tor} = \langle D_i\rangle $, then
$$ \lim_{s \to 1} \dfrac{P_2(X, q^{-s})}{(1 - q^{1 - s})^{\rk(\NS(X))}} = \dfrac{\#\Br(X) \cdot |\det(\langle D_i, D_j\rangle_{i, j})|}{\#\NS(X)_{\tor}^2 \cdot q^{\chi(X, \OO_X) - 1 + \dim\Pic(X)}}. $$
\end{conjecture}

Note that if $ X \to C $ is flat proper with smooth geometrically-connected generic fibre $ X_K / K $, then $ \#\Sha(\Jac(X_K) / K) \sim \#\Br(X) $ (LLR18).

\end{frame}

\begin{frame}{The Brauer--Azumaya group of a field}

Let $ K $ be a field. The \textbf{classical Brauer group} of $ K $ is
$$ \Br(K) := \{\text{central simple algebras over} \ K\} / \sim. $$
A \textbf{central simple algebra} over $ K $ is a finite-dimensional associative $ K $-algebra with centre $ K $ and no non-trivial proper two-sided ideals.

\begin{examples}
\begin{itemize}
\item Algebra of $ n \times n $ matrices $ \Mat_n(K) $ over $ K $.
\item Algebra of $ n \times n $ matrices $ \Mat_n(D) $ over a central division algebra $ D $.
\item Tensor product $ A \otimes_K B $ of two CSAs $ A $ and $ B $.
\item Opposite algebra $ A^\op $ of a CSA $ A $.
\end{itemize}
\end{examples}

Two CSAs $ A $ and $ B $ over $ K $ are \textbf{equivalent} if there are $ n, m \in \N $ such that $ A \otimes_K \Mat_n(K) \cong B \otimes_K \Mat_m(K) $.

\begin{example}
If $ n, m \in \N $ and $ D $ is a CDA, then $ \Mat_n(D) \sim \Mat_m(D) $.
\end{example}

\end{frame}

\begin{frame}{The Brauer--Azumaya group of a field}

Let $ K $ be a field. The \textbf{classical Brauer group} of $ K $ is
$$ \Br(K) := \{\text{central simple algebras over} \ K\} / \sim. $$

\begin{examples}
\begin{itemize}
\item $ \Br(\F_q) = 0 $. Suffices to prove a CDA $ D $ over $ \F_q $ is $ \F_q $. A finite division algebra $ D $ is a field $ K $. A field $ K $ with centre $ \F_q $ is $ \F_q $.
\item $ \Br(\C) = 0 $. Suffices to prove a CDA $ D $ over $ \C $ is $ \C $. If $ x \in D $, then $ \C[x] $ is an integral domain and a finite-dimensional $ \C $-vector space. Thus $ \C[x] $ is a field, but $ \C $ does not have finite extensions.
\item $ \Br(\C(X)) = 0 $ for a curve $ X / \C $. This is Tsen's theorem.
\end{itemize}
\end{examples}

\end{frame}

\begin{frame}{The Brauer--Grothendieck group of a field}

Let $ K $ be a field. The \textbf{cohomological Brauer group} of $ K $ is
$$ \Br'(K) := H^2(K, \G_m). $$

\begin{theorem}[CTS19, Theorem 1.3.5]
$ \Br(K) \xrightarrow{\sim} \Br'(K) $.
\end{theorem}

\begin{examples}
\begin{itemize}
\item $ \Br'(\R) = \tfrac{1}{2}\Z / \Z $. By cohomology of cyclic groups,
$$ \Br'(\R) = H^2(\Gal(\C / \R), \C^\times) \cong \R^\times / \Nm_{\C / \R}(\C^\times) \cong \{\pm\}. $$
In fact, $ \Br'(\R) = \{\R, \H\} $.
\item Local class field theory gives isomorphisms
$$ \inv_p : \Br'(\Q_p) \xrightarrow{\sim} \Q / \Z, \qquad \inv_q : \Br'(\F_q((T))) \xrightarrow{\sim} \Q / \Z. $$
\end{itemize}
\end{examples}

\end{frame}

\begin{frame}{The Brauer--Grothendieck group of a field}

Let $ K $ be a field. The \textbf{cohomological Brauer group} of $ K $ is
$$ \Br'(K) := H^2(K, \G_m). $$

\begin{theorem}[CTS19, Theorem 1.3.5]
$ \Br(K) \xrightarrow{\sim} \Br'(K) $.
\end{theorem}

\begin{examples}
\begin{itemize}
\item Global class field theory gives short exact sequences
$$ 0 = \varinjlim_{L / K} H^1(L / K, C_L) \to \Br'(\Q) \to \bigoplus_{v \in V_\Q} \Br'(\Q_v) \xrightarrow{\sum_v \inv_v} \Q / \Z \to 0, $$
$$ 0 = H^1(\F_q, \Jac(C_{\overline{\F_q}})) \to \Br'(K) \to \bigoplus_{v \in V_K} \Br'(K_v) \xrightarrow{\sum_v \inv_v} \Q / \Z \to 0, $$
where $ K = \F_q(C) $.
\end{itemize}
\end{examples}

\end{frame}

\begin{frame}{The Brauer--Azumaya group of a scheme}

Let $ X $ be a scheme. The \textbf{Brauer--Azumaya group} of $ X $ is
$$ \Br_\Az(X) := \{\text{Azumaya algebras on} \ X\} / \sim. $$
An \textbf{Azumaya algebra $ \AA $ on $ X $} is a locally free $ \OO_X $-algebra of finite type such that $ \AA_x \otimes_{\OO_{X, x}} \kappa_x $ is a CSA over $ \kappa_x $ for all closed points $ x \in X $.

\begin{examples}
\begin{itemize}
\item Trivial, tensor product, opposite algebra sheaves of AAs.
\item ($ X = \Spec(K) $) For a CSA $ A $ over $ K $, the constant sheaf $ A $.
\item ($ X = \P_K^n $) For a CSA $ A $ over $ K $, the sheaf $ A \otimes_K \EE nd_K(\bigoplus_{n_i} \OO_X(n_i)) $.
\end{itemize}
\end{examples}

Two AAs $ \AA $ and $ \BB $ are \textbf{equivalent} if there are locally free $ \OO_X $-modules $ A $ and $ B $ of finite rank such that $ \AA \otimes_{\OO_X} \EE nd_{\OO_X}(A) \cong \BB \otimes_{\OO_X} \EE nd_{\OO_X}(B) $.

\begin{examples}
\begin{itemize}
\item $ \Br_\Az(\Spec(K)) = \Br(K) $.
\item (Fis17) $ \Br_\Az(C) $ for an smooth curve of genus one $ C / K $.
\end{itemize}
\end{examples}

\end{frame}

\begin{frame}{The Brauer--Grothendieck group of a scheme}

Let $ X $ be a scheme. The \textbf{Brauer--Grothendieck group} of $ X $ is
$$ \Br_\Gr(X) := H_\et^2(X, \G_m). $$
Unlike for fields, in general $ \Br_\Az(X) \hookrightarrow \Br_\Gr(X) $ is not surjective.

\begin{theorem}[CTS19, Theorem 3.3.2]
Assume $ X $ is quasi-compact separated with an ample line bundle. Then
$$ \Br(X) := \Br_\Az(X) \xrightarrow{\sim} \Br_\Gr(X)_{\tor}. $$
\end{theorem}

\begin{example}
A quasi-projective scheme over an affine scheme, such as $ E / \F_q(C) $ or $ \EE / \F_q $. If $ X $ is regular integral noetherian, then $ \Br_\Gr(X) $ is already torsion.
\end{example}

\begin{theorem}[CTS19, Theorem 3.5.4]
Assume $ X $ is regular integral over a field $ K $. Then $ \Br(X) \hookrightarrow \Br(K(X)) $.
\end{theorem}

\end{frame}

\begin{frame}{The Brauer--Grothendieck group of a scheme}

Let $ X $ be a variety over a perfect field $ K $, and write $ \overline{X} := X \times_K \overline{K} $. The first seven terms of the Leray spectral sequence form an exact sequence
{\scriptsize $$
\begin{tikzcd}[ampersand replacement=\&, column sep=tiny]
0 \arrow{r} \& H^1(K, \overline{K}[X]^\times) \arrow{r} \& \Pic(X) \arrow{r} \& \Pic(\overline{X})^{G_K} \arrow{r} \& H^2(K, \overline{K}[X]^\times) \arrow[in=180, out=0, overlay]{dlll} \\
\& \ker(\Br(X) \to \Br(\overline{X})) \arrow{r} \& H^1(K, \Pic(\overline{X})) \arrow{r} \& \ker(H^3(K, \overline{K}[X]^\times) \arrow{r} \& H_\et^3(X, \G_m)).
\end{tikzcd}
$$}

\begin{examples}
\begin{itemize}
\item If $ X = \A_K^1 $ or $ X = \P_K^1 $, then $ \Br(X) \cong \Br(K) $.
\begin{itemize}
\item $ H^2(K, \overline{K}[X]^\times) \cong \Br(K) $ since $ \overline{K}[X]^\times = \overline{K}^\times $.
\item $ \Br(\overline{X}) \hookrightarrow \Br(\overline{K}(X)) = 0 $ by Tsen's theorem.
\item $ \Br(K) \to \Br(X) $ and $ H^3(K, \overline{K}[X]^\times) \to H_\et^3(X, \G_m) $ are injective since $ X(K) \ne \emptyset $ gives retractions.
\item $ H^1(K, \Pic(\overline{X})) = 0 $ since $ \Pic(\A_{\overline{K}}^1) = 0 $ and $ \deg : \Pic(\P_{\overline{K}}^1) \xrightarrow{\sim} \Z $.
\end{itemize}
In fact, $ \Br(\A_K^n) \cong \Br(\P_K^n) \cong \Br(K) $ by induction.
\end{itemize}
\end{examples}

\end{frame}

\begin{frame}{The Brauer--Grothendieck group of a scheme}

Let $ X $ be a variety over a perfect field $ K $, and write $ \overline{X} := X \times_K \overline{K} $. The first seven terms of the Leray spectral sequence form an exact sequence
{\scriptsize $$
\begin{tikzcd}[ampersand replacement=\&, column sep=tiny]
0 \arrow{r} \& H^1(K, \overline{K}[X]^\times) \arrow{r} \& \Pic(X) \arrow{r} \& \Pic(\overline{X})^{G_K} \arrow{r} \& H^2(K, \overline{K}[X]^\times) \arrow[in=180, out=0, overlay]{dlll} \\
\& \ker(\Br(X) \to \Br(\overline{X})) \arrow{r} \& H^1(K, \Pic(\overline{X})) \arrow{r} \& \ker(H^3(K, \overline{K}[X]^\times) \arrow{r} \& H_\et^3(X, \G_m)).
\end{tikzcd}
$$}

\begin{examples}
\begin{itemize}
\item If $ X = E $ is an elliptic curve, then there is a short exact sequence
$$ 0 \to \Br(K) \to \Br(E) \to H^1(K, E) \to 0. $$
As before, with $ H^1(K, \Pic(\overline{E})) = H^1(K, \Jac(\overline{E})) = H^1(K, E) $.
\item (Tho10) $ \Br(\EE)[\ell^\infty] $ for an elliptic K3 surface $ \EE / \F_q $ given by $ t(t - 1)y^2 = x(x - 1)(x - t) $. Uses the short exact sequence
$$ 0 \to \NS(\EE) \otimes_\Z \Z_\ell \to H_\et^2(\EE, \Z_\ell(1)) \to T_\ell\Br(\EE) \to 0. $$
\end{itemize}
\end{examples}

\end{frame}

\begin{frame}{The Brauer--Manin obstruction}

Let $ X $ be a scheme over a global field $ K $. A point $ x_v : \Spec(K_v) \to X $ induces a map $ x_v^* : \Br(X) \to \Br(K_v) $. The \textbf{Brauer--Manin pairing} is
$$ \function[\langle-, -\rangle_{\Br}]{\Br(X) \times X(\A_K)}{\Q / \Z}{(A, (x_v)_v)}{\displaystyle\sum_{v \in V_K} \inv_v(x_v^*(A))}. $$
The \textbf{Brauer--Manin set} for $ A \in \Br(X) $ is
$$ X(\A_K)^A := \{(x_v) \in X(\A_K) : \langle A, (x_v)_v\rangle_{\Br} = 0\}. $$
By global class field theory,
$$ \overline{X(K)} \hookrightarrow \bigcap_{A \in \Br(X)} X(\A_K)^A \subseteq X(\A_K). $$
If $ X(\A_K)^A \ne \emptyset $ but $ X(\A_K) = \emptyset $, then there is a \textbf{Brauer--Manin obstruction to the Hasse principle} for $ X $ due to $ A \in \Br(X) $.

\end{frame}

\begin{frame}{The Brauer--Manin obstruction}

Let $ X $ be a scheme over a global field $ K $. A point $ x_v : \Spec(K_v) \to X $ induces a map $ x_v^* : \Br(X) \to \Br(K_v) $. The \textbf{Brauer--Manin pairing} is
$$ \function[\langle-, -\rangle_{\Br}]{\Br(X) \times X(\A_K)}{\Q / \Z}{(A, (x_v)_v)}{\displaystyle\sum_{v \in V_K} \inv_v(x_v^*(A))}. $$

\begin{theorem}[Wit15]
Let $ \EE $ be an elliptic K3 surface over $ \Q $ given by
$$ y^2 = x(x - 3(t - 1)^3(3 + t))(x + 3(t + 1)^3(3 - t)). $$
There is a Brauer--Manin obstruction to the Hasse principle for $ \EE $ due to
$$ (x + 3(t - 1)^3(3 + t), 6t(t + 1)) + (x - 3(t + 1)^3(3 - t), 6t(t - 1)) \in \Br(\EE). $$
\end{theorem}

\end{frame}

\begin{frame}[c]{References}

\scriptsize
\begin{itemize}
\item[ASD73] Artin, Swinnerton-Dyer (1973) \emph{The Shafarevich--Tate conjecture for pencils of elliptic curves on K3 surfaces}
\item[CTS19] Colliot-Th\'el\`ene, Skorobogatov (2019) \emph{The Brauer--Grothendieck group}
\item[Fis17] Fisher (2017) \emph{On some algebras associated to genus one curves}
\item[KT03] Kato, Trihan (2003) \emph{On the conjectures of Birch and Swinnerton-Dyer in characteristic p}
\item[LLR18] Liu, Lorenzini, Raynaud (2018) \emph{Corrigendum to N\'eron models, Lie algebras, and reduction of curves of genus one and the Brauer group of a surface}
\item[Mil68] Milne (1968) \emph{The Tate-\v Safarevi\v c group of a constant abelian variety}
\item[Mil70] Milne (1970) \emph{The Brauer group of a rational surface}
\item[Mil75] Milne (1975) \emph{On a conjecture by Artin and Tate}
\item[Tat66] Tate (1966) \emph{On the conjectures of Birch and Swinnerton-Dyer and a geometric analog}
\item[Tho10] Thorne (2010) \emph{On the Tate--Shafarevich groups of certain elliptic curves}
\item[Ulm12] Ulmer (2012) \emph{Curves and Jacobians over function fields}
\item[Wit15] Wittenberg (2015) \emph{Transcendental Brauer--Manin obstruction on a pencil of elliptic curves}
\end{itemize}

\end{frame}

\end{document}