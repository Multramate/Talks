\ifx\type\undefined
  \documentclass[10pt, t]{beamer}
  \setbeamertemplate{footline}[page number]
\else
  \documentclass[10pt]{article}
  \usepackage[margin=1in]{geometry}
\fi

\usepackage{amsmath}
\usepackage{amssymb}
\usepackage{amsthm}
\usepackage{bbm}
\usepackage{cancel}
\usepackage{listings}
\usepackage{mathrsfs}
\usepackage{multirow}
\usepackage{soul}
\usepackage{stmaryrd}
\usepackage{tikz}
\usepackage{tikz-cd}
\usepackage{wrapfig}

\newtheorem*{algorithm}{Algorithm}
\newtheorem*{assumptions}{Assumptions}
\newtheorem*{conjecture}{Conjecture}
\newtheorem*{consequences}{Consequences}
\newtheorem*{exercise}{Exercise}
\newtheorem*{formalisation}{Formalisation}
\newtheorem*{proposition}{Proposition}
\newtheorem*{question}{Question}
\newtheorem*{remark}{Remark}

\ifx\type\undefined\else
  \newtheorem*{definition}{Definition}
  \newtheorem*{example}{Example}
  \newtheorem*{lemma}{Lemma}
  \newtheorem*{theorem}{Theorem}
\fi

\definecolor{keywordcolor}{rgb}{0.7, 0.1, 0.1}
\definecolor{tacticcolor}{rgb}{0.0, 0.1, 0.6}
\definecolor{commentcolor}{rgb}{0.4, 0.4, 0.4}
\definecolor{symbolcolor}{rgb}{0.0, 0.1, 0.6}
\definecolor{sortcolor}{rgb}{0.1, 0.5, 0.1}
\definecolor{attributecolor}{rgb}{0.7, 0.1, 0.1}
\def\lstlanguagefiles{lstlean.tex}
\lstset{language=lean}

\newcommand\A{\mathbb{A}}
\newcommand\C{\mathbb{C}}
\newcommand\F{\mathbb{F}}
\newcommand\G{\mathbb{G}}
\renewcommand\H{\mathbb{H}}
\newcommand\I{\mathbb{I}}
\newcommand\N{\mathbb{N}}
\renewcommand\P{\mathbb{P}}
\newcommand\Q{\mathbb{Q}}
\newcommand\R{\mathbb{R}}
\newcommand\Z{\mathbb{Z}}

\renewcommand\AA{\mathcal{A}}
\newcommand\BB{\mathcal{B}}
\newcommand\CC{\mathcal{C}}
\newcommand\DD{\mathcal{D}}
\newcommand\EE{\mathcal{E}}
\newcommand\FF{\mathcal{F}}
\newcommand\GG{\mathcal{G}}
\newcommand\HH{\mathcal{H}}
\newcommand\II{\mathcal{I}}
\newcommand\LL{\mathcal{L}}
\newcommand\MM{\mathcal{M}}
\newcommand\NN{\mathcal{N}}
\newcommand\OO{\mathcal{O}}
\newcommand\PP{\mathcal{P}}
\newcommand\RR{\mathcal{R}}
\renewcommand\SS{\mathcal{S}}
\newcommand\TT{\mathcal{T}}
\newcommand\XX{\mathcal{X}}

\renewcommand\aa{\mathfrak{a}}
\newcommand\cc{\mathfrak{c}}
\newcommand\dd{\mathfrak{d}}
\newcommand\ff{\mathfrak{f}}
\renewcommand\gg{\mathfrak{g}}
\newcommand\mm{\mathfrak{m}}
\newcommand\pp{\mathfrak{p}}
\newcommand\qq{\mathfrak{q}}
\renewcommand\ss{\mathfrak{s}}

\newcommand\LLL{\mathscr{L}}

\newcommand\ab{\mathrm{ab}}
\newcommand\Ab{\mathbf{Ab}}
\newcommand\Alg{\mathbf{Alg}}
\newcommand\Aff{\mathbf{Aff}}
\newcommand\Aut{\operatorname{Aut}}
\newcommand\Az{\mathrm{Az}}
\newcommand\Br{\operatorname{Br}}
\newcommand\BSD{\operatorname{BSD}}
\newcommand\ch{\operatorname{char}}
\newcommand\Cl{\operatorname{Cl}}
\newcommand\coker{\operatorname{coker}}
\newcommand\cris{\mathrm{cris}}
\renewcommand\d{\mathrm{d}}
\newcommand\Div{\operatorname{Div}}
\newcommand\dR{\mathrm{dR}}
\newcommand\EN{\operatorname{EN}}
\newcommand\End{\operatorname{End}}
\newcommand\ES{\operatorname{ES}}
\newcommand\et{\mathrm{\acute{e}t}}
\newcommand\Et{\mathbf{\acute{E}t}}
\newcommand\Ext{\operatorname{Ext}}
\newcommand\Fr{\operatorname{Fr}}
\newcommand\Frac{\operatorname{Frac}}
\newcommand\Gal{\operatorname{Gal}}
\newcommand\GL{\operatorname{GL}}
\newcommand\Gr{\mathrm{Gr}}
\newcommand\Hom{\operatorname{Hom}}
\newcommand\HT{\mathrm{HT}}
\newcommand\id{\operatorname{id}}
\newcommand\im{\operatorname{im}}
\newcommand\Ind{\operatorname{Ind}}
\renewcommand\inf{\operatorname{inf}}
\newcommand\inv{\operatorname{inv}}
\newcommand\Irr{\operatorname{Irr}}
\newcommand\Jac{\operatorname{Jac}}
\newcommand\lcm{\operatorname{lcm}}
\newcommand\Mat{\operatorname{Mat}}
\newcommand\Mod{\mathbf{Mod}}
\newcommand\Nm{\operatorname{Nm}}
\newcommand\nr{\mathrm{nr}}
\newcommand\NS{\operatorname{NS}}
\newcommand\Ob{\operatorname{Ob}}
\newcommand\ord{\operatorname{ord}}
\newcommand\op{\mathrm{op}}
\newcommand\PGL{\operatorname{PGL}}
\newcommand\Pic{\operatorname{Pic}}
\newcommand\Prob{\operatorname{Prob}}
\newcommand\Proj{\operatorname{Proj}}
\newcommand\PSh{\mathbf{PSh}}
\newcommand\Reg{\operatorname{Reg}}
\newcommand\res{\operatorname{res}}
\newcommand\rk{\operatorname{rk}}
\newcommand\Sch{\mathbf{Sch}}
\newcommand\Sel{\operatorname{Sel}}
\newcommand\Set{\mathbf{Set}}
\newcommand\sgn{\operatorname{sgn}}
\newcommand\Sh{\mathbf{Sh}}
\newcommand\SL{\operatorname{SL}}
\newcommand\Spec{\operatorname{Spec}}
\newcommand\supp{\operatorname{supp}}
\newcommand\Tam{\operatorname{Tam}}
\newcommand\Top{\mathbf{Top}}
\newcommand\tor{\operatorname{tor}}
\newcommand\tr{\operatorname{tr}}
\newcommand\tra{\operatorname{tra}}
\newcommand\WC{\operatorname{WC}}

\DeclareFontFamily{U}{wncyr}{}
\DeclareFontShape{U}{wncyr}{m}{n}{<->wncyr10}{}
\DeclareSymbolFont{cyr}{U}{wncyr}{m}{n}
\DeclareMathSymbol{\Sha}{\mathord}{cyr}{"58}

\newcommand{\function}[5][]{
  \if &#1&
    \begin{array}{rcl}
      #2 & \longrightarrow & #3 \\
      #4 & \longmapsto     & #5
    \end{array}
  \else
    \begin{array}{rcrcl}
      #1 & : & #2 & \longrightarrow & #3 \\
         &   & #4 & \longmapsto     & #5
    \end{array}
  \fi
}

\newcommand{\functions}[7][]{
  \if &#1&
    \begin{array}{rcl}
      #2 & \longrightarrow & #3 \\
      #4 & \longmapsto     & #5 \\
      #6 & \longmapsto     & #7 \\
    \end{array}
  \else
    \begin{array}{rcrcl}
      #1 & : & #2 & \longrightarrow & #3 \\
         &   & #4 & \longmapsto     & #5 \\
         &   & #6 & \longmapsto     & #7
    \end{array}
  \fi
}
\usetheme{singapore}
\title{A blueprint for the Birch and Swinnerton-Dyer conjecture in Lean}
\subtitle{Meeting for Key Laboratory of Mathematics Mechanization (Academy of Mathematics and System Sciences, Chinese Academy of Sciences)}
\author{David Kurniadi Angdinata}
\institute{University of East Anglia}
\date{Tuesday, 13 January 2026}

\begin{document}

\frame\maketitle

\section{Algebraic theory}

\begin{frame}{Introduction}

Let $ E_K $ be an elliptic curve over a number field $ K $.

\begin{conjecture}[Birch--Swinnerton-Dyer]
Assume that $ L(E_K, s) $ has meromophic continuation at $ s = 1 $.
\begin{enumerate}
\item The order of vanishing of $ L(E_K, s) $ at $ s = 1 $ is equal to $ \rk(E_K) $.
\item The group $ \Sha(E_K) $ is finite.
\item The leading term of $ L(E_K, s) $ at $ s = 1 $ satisfies
$$ \lim_{s \to 1} \dfrac{L(E_K, s)}{(s - 1)^{\rk(E_K)}} = \dfrac{\Omega(E_K) \cdot \Reg(E_K) \cdot \#\Sha(E_K) \cdot \Tam(E_K)}{\delta_K \cdot \#\tor(E_K)^2}, $$
where $ \delta_K $ is the absolute discriminant of $ K $.
\end{enumerate}
\end{conjecture}

In this talk, I will describe each of these invariants in detail.

\bigskip Note that this generalises to abelian varieties over global fields.

\end{frame}

\begin{frame}{Weierstrass equations}

An \emph{elliptic curve} $ E_F $ over a field $ F $ is a smooth projective curve of genus one over $ F $ with a distinguished point $ \OO $ over $ F $.

\bigskip By the Riemann--Roch theorem, $ E_F $ is isomorphic to a curve of the form
$$ y^2 + a_1xy + a_3y = x^3 + a_2x^2 + a_4x + a_6, $$
for some $ a_i \in F $ such that $ \Delta \ne 0 $, and $ \OO $ is its unique point at infinity.

\bigskip In \texttt{mathlib}, a \emph{Weierstrass curve} over $ F $ is a tuple $ (a_1, a_2, a_3, a_4, a_6) \in F^5 $, and an elliptic curve is a Weierstrass curve such that $ \Delta \ne 0 $.

\bigskip A \emph{point} over $ F $ is either $ \OO $ or an affine point $ (x, y) \in F^2 $ satisfying
$$ y^2 + a_1xy + a_3y = x^3 + a_2x^2 + a_4x + a_6, $$
and a nonsingularity condition.

\end{frame}

\begin{frame}{The group law}

With this definition, addition on points is given by explicit rational functions, where associativity is known to be \emph{computationally difficult}: generic associativity is an equality of polynomials with 26,082 terms!

\bigskip

\begin{formalisation}[A.--Xu, 2022]
The type of nonsingular $ F $-points $ E_F(F) $ is an abelian group.
\end{formalisation}

\bigskip It suffices to show that the homomorphism $ E_F(F) \to \Cl(F[E_F]) $ mapping $ (x, y) $ to $ [(X - x, Y - y)] $ is injective. If it were not, then there are polynomials $ f, g \in F[X] $ such that $ (X - x, Y - y) = (f + gY) $. Then
$$ \deg(\Nm(f + gY)) =
\begin{cases}
\max(2\deg(f), 2\deg(g) + 3), \\
\dim_F(F[E_F] / (f + gY)),
\end{cases}
$$
which give a contradiction.

\end{frame}

\begin{frame}{The Tate module}

I attempted to formalise the isomorphism $ E_F(F^s)[n] \cong (\Z / n\Z)^2 $ in 2023.

\bigskip Silverman defines polynomials $ \psi_n, \phi_n, \omega_n \in F^s[X, Y] $ and \emph{claims} that there is a computational proof for the multiplication-by-$ n $ formula
$$ [n](x, y) = \left(\dfrac{\phi_n(x)}{\psi_n^2(x)}, \dfrac{\omega_n(x, y)}{\psi_n^3(x, y)}\right). $$
Computing $ \deg(\phi_n) = n^2 $ and $ \deg(\psi_n^2) = n^2 - 1 $, and proving that $ (\phi_n, \psi_n^2) = 1 $, imply that $ \#\ker[n] = n^2 $, and the result follows formally.

\bigskip

\begin{formalisation}[A.--Wu--Xu, 2026?]
For any $ \ell \ne \ch(F) $, the $ \ell $-adic Tate module $ T_\ell E_{F^s} $ defines a two-dimensional Galois representation $ \rho_{E_F, \ell} : G_F \to \GL(T_\ell E_{F^s}) $.
\end{formalisation}

\bigskip The proof is much trickier than he claims!

\end{frame}

\begin{frame}{The L-function}

Let $ E_K $ be an elliptic curve over a number field $ K $.

\bigskip The \textbf{Euler factor} of $ E_K $ at a finite place $ v $ of $ K $ is
$$ L_v(E_K, s) := \det(1 - \rho_{E_K, \ell}^{\vee I_v}(\phi_v) \cdot q_v^{-s}), $$
where $ \ell \nmid q_v $ is any prime number.

\bigskip The \textbf{L-function} of $ E_K $ is
$$ L(E_K, s) := \prod_\pp \dfrac{1}{L_v(E_K, s)}, $$
where the product runs over all finite places $ v $ of $ K $.

\bigskip Assuming an appropriate modularity conjecture for $ E_K $ over $ K $, the L-function has analytic continuation to all of $ \C $.

\end{frame}

\section{Local theory}

\begin{frame}{Non-archimedean local fields}

Let $ E_F $ be an elliptic curve over a non-archimedean local field $ F $ with normalised valuation $ v $, valuation ring $ R $, and residue field $ k $.

\bigskip By the valuative criterion for properness, there is a \emph{reduction map}
$$ \widetilde{(\cdot)} : E_F \xleftarrow{\sim} E_R \twoheadrightarrow E_k, $$
which induces a map on points $ E_F(F) \to \widetilde{E_F}(k) $.

\bigskip Note that this generalises to the fraction field $ F $ of a B\'ezout domain $ R $ with $ k := R / m $ for any maximal ideal $ m $ of $ R $.

\bigskip Say that $ E_F $ is \emph{minimal} if $ v(\Delta) \in \N $ is minimal subject to $ a_i \in R $. Any elliptic curve over $ F $ is isomorphic to one that is minimal.

\bigskip If $ E_K $ is an elliptic curve over a number field $ K $ with $ \Cl(K) = 1 $, then $ E_K $ is isomorphic to an elliptic curve that is minimal everywhere.

\end{frame}

\begin{frame}{Reduction types}

Say that $ E_F $ is
\begin{itemize}
\item \emph{good} if $ \widetilde{E_F} $ is elliptic,
\item \emph{split multiplicative} if $ \widetilde{E_F} $ is nodal with tangent over $ k $,
\item \emph{non-split multiplicative} if $ \widetilde{E_F} $ is nodal with tangent not over $ k $, and
\item \emph{additive} if $ \widetilde{E_F} $ is cuspidal.
\end{itemize}

\bigskip Let $ E_K $ be an elliptic curve over a number field $ K $. Then
$$ L_v(E_K, s) =
\begin{cases}
1 - a_vq_v^{-s} + q_v^{1 - 2s} & \text{if} \ E_{K_v} \ \text{is good}, \\
1 - q_v^{-s} & \text{if} \ E_{K_v} \ \text{is split multiplicative}, \\
1 + q_v^{-s} & \text{if} \ E_{K_v} \ \text{is non-split multiplicative}, \\
1 & \text{if} \ E_{K_v} \ \text{is additive},
\end{cases}
$$
where $ a_v := 1 + q_v - \#\widetilde{E_{K_v}}(k_v) $ is the trace of Frobenius of $ E_K $ at $ v $.

\end{frame}

\begin{frame}{Tamagawa numbers}

The \textbf{Tamagawa number} of $ E_F $ is
$$ \Tam(E_F) := [E_F(F) : E_F^0(F)], $$
where $ E_F^0(F) $ is the subgroup of $ E_F(F) $ with nonsingular reduction.

\bigskip Let $ E_K $ be an elliptic curve over a number field $ K $, and let
$$ \omega := \dfrac{\d x}{2y + a_1x + a_3}. $$
For each place $ v $ of $ K $, let $ \omega_v $ be a non-zero invariant differential of a minimal elliptic curve isomorphic to $ E_{K_v} $. Then its \textbf{Tamagawa number} is
$$ \Tam(E_K) := \prod_v \Tam(E_{K_v}) \cdot \left|\dfrac{\omega_v}{\omega}\right|_v, $$
where the product runs over all finite places $ v $ of $ K $.

\end{frame}

\begin{frame}{Complex fields}

Let $ E_\C $ be an elliptic curve over $ \C $ given by $ y^2 = x^3 + Ax + B $.

\bigskip There is a $ \C $-lattice $ \Lambda_{A, B} $ that is unique up to homothety such that
$$ \function{\C / \Lambda_{A, B}}{E_\C(\C)}{z}{(\wp(z), \tfrac{1}{2}\wp'(z))} $$
is an isomorphism of complex Lie groups.

\bigskip The \textbf{period} of $ E_\C $ is
$$ \Omega(E_\C) := \int_{\C / \Lambda_{A, B}} 2\d x\d y = \int_{E_\C(\C)} \omega \wedge \overline{\omega}, $$
which is just the area of $ \Lambda_{A, B} $.

\bigskip See Silverman's \emph{Advanced Topics in the Arithmetic of Elliptic Curves}.

\end{frame}

\begin{frame}{Real fields}

Let $ E_\R $ be an elliptic curve over $ \R $. Then there is an isomorphism
$$ E_\R(\R) \cong
\begin{cases}
S^1 & \text{if} \ \Delta < 0 \\
S^1 \oplus C_2 & \text{if} \ \Delta > 0
\end{cases}
$$
of real Lie groups.

\bigskip The \textbf{period} of $ E_\R $ is
$$ \Omega(E_\R) := \int_{E_\R(\R)} \omega. $$
If $ E_K $ is an elliptic curve over a number field $ K $, its \textbf{period} is
$$ \Omega(E_K) := \prod_v \Omega(E_{K_v}), $$
where the product runs over all infinite places $ v $ of $ K $.

\end{frame}

\section{Global theory}

\begin{frame}{The Mordell--Weil theorem}

Let $ E_K $ be an elliptic curve over a number field $ K $.

\bigskip

\begin{theorem}[Mordell--Weil]
$ E_K(K) $ is finitely generated.
\end{theorem}

\bigskip By the structure theorem of finitely generated abelian groups,
$$ E_K(K) \cong \tor(E_K) \oplus \Z^{\rk(E_K)}. $$
where $ \tor(E_K) $ is the \textbf{torsion subgroup} and $ \rk(E_K) $ is the \textbf{rank}.

\bigskip The torsion subgroup can be computed via the reduction map.

\bigskip The rank is conjecturally the order of vanishing of $ L(E_K, s) $ at $ s = 1 $.

\end{frame}

\begin{frame}{Na\"ive heights}

The proof that $ E_K(K) $ is finitely generated reduces to a proof of the \emph{weak Mordell--Weil theorem} that $ E_K(K) / n $ is finite and the existence of a \emph{na\"ive height} $ h : E_K(K) \to \R $ satisfying the following.
\begin{itemize}
\item For all $ Q \in E_K(K) $, there exists $ C_1 \in \R $ such that for all $ P \in E_K(K) $,
$$ h(P + Q) \le 2h(P) + C_1. $$
\item There exists $ C_2 \in \R $ such that for all $ P \in E_K(K) $,
$$ n^2h(P) \le h(nP) + C_2. $$
\item For all $ C_3 \in \R $, the set $ \{P \in E_K(K) : h(P) \le C_3\} $ is finite.
\end{itemize}
For instance, when $ K = \Q $,
$$ \functions[h]{E_\Q(\Q)}{\R}{(n / d, y)}{\log\max(|n|, |d|)}{\OO}{0}. $$

\end{frame}

\begin{frame}{Canonical heights}

Any na\"ive height defines the \emph{canonical height} $ \widehat{h} : E_K(K) \to \R $ given by
$$ \widehat{h}(P) := \lim_{n \to \infty} \dfrac{h([2^n]P)}{4^n}, $$
which is independent of the choice of na\"ive height.

\bigskip This is a quadratic form on $ E_K(K) $, with associated bilinear pairing
$$ \langle P, Q\rangle := \tfrac{1}{2}(\widehat{h}(P + Q) - \widehat{h}(P) - \widehat{h}(Q)). $$
The \textbf{regulator} of $ E_K $ is
$$ \Reg(E_K) := \left|\det(\langle P_i, P_j\rangle)_{i, j = 0}^{\rk(E_K)}\right|, $$
where $ \{P_n\}_{n = 0}^{\rk(E_K)} $ is any $ \Z $-basis of $ E_K(K) / \tor(E_K) $.

\end{frame}

\begin{frame}{Galois cohomology}

For any field $ F $, multiplication by $ n \in \Z $ gives
$$ 0 \to E_F[n] \to E_F \to E_F \to 0, $$
which induces a long exact sequence that truncates to
$$ 0 \to E_F(F) / n \to H^1(F, E_F[n]) \to H^1(F, E_F)[n] \to 0. $$
Applying this to $ F = K $ and to $ F = K_v $ for each place $ v $ of $ K $ gives
$$
\begin{tikzcd}[ampersand replacement=\&, column sep=tiny]
0 \arrow{r} \& E_K(K) / n \arrow{r} \arrow{d} \& H^1(K, E_K[n]) \arrow{r} \arrow{d} \arrow[dashed]{dr}[swap]{\sigma} \& H^1(K, E_K)[n] \arrow{r} \arrow{d}{\tau[n]} \& 0 \\
0 \arrow{r} \& \displaystyle\prod_v E_K(K_v) / n \arrow{r} \& \displaystyle\prod_v H^1(K_v, E_K[n]) \arrow{r} \& \displaystyle\prod_v H^1(K_v, E_K)[n] \arrow{r} \& 0.
\end{tikzcd}
$$
Note that $ H^1(K, E_K[n]) $ is not finite in general.

\end{frame}

\begin{frame}{The weak Mordell--Weil theorem}

The \emph{$ n $-Selmer group} $ \Sel_n(E_K) := \ker\sigma $ and the \textbf{Tate--Shafarevich group}
$$ \Sha(E_K) := \ker\left(\tau : H^1(K, E_K) \to \prod_v H^1(K_v, E_K)\right) $$
fit in a short exact sequence
$$ 0 \to E_K(K) / n \to \Sel_n(E_K) \to \Sha(E_K)[n] \to 0. $$
The weak Mordell--Weil theorem then reduces to showing that
$$ \Sel_n(E_K) \subseteq \Sel_n(K, S) \times \Sel_n(K, S), $$
where $ \Sel_n(K, S) $ is the $ n $-Selmer group of $ K $ unramified outside an explicit finite set $ S $ of bad places of $ K $, which is finite since
$$ 0 \to \OO_{K, S}^\times / (\OO_{K, S}^\times)^n \to \Sel_n(K, S) \to \Cl_S(K)[n] \to 0. $$

\end{frame}

\end{document}