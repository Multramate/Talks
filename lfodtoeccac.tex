\ifx\type\undefined
  \documentclass[10pt, t]{beamer}
  \setbeamertemplate{footline}[page number]
\else
  \documentclass[10pt]{article}
  \usepackage[margin=1in]{geometry}
\fi

\usepackage{amsmath}
\usepackage{amssymb}
\usepackage{amsthm}
\usepackage{bbm}
\usepackage{cancel}
\usepackage{listings}
\usepackage{mathrsfs}
\usepackage{multirow}
\usepackage{soul}
\usepackage{stmaryrd}
\usepackage{tikz}
\usepackage{tikz-cd}
\usepackage{wrapfig}

\newtheorem*{algorithm}{Algorithm}
\newtheorem*{assumptions}{Assumptions}
\newtheorem*{conjecture}{Conjecture}
\newtheorem*{consequences}{Consequences}
\newtheorem*{exercise}{Exercise}
\newtheorem*{formalisation}{Formalisation}
\newtheorem*{proposition}{Proposition}
\newtheorem*{question}{Question}
\newtheorem*{remark}{Remark}

\ifx\type\undefined\else
  \newtheorem*{definition}{Definition}
  \newtheorem*{example}{Example}
  \newtheorem*{lemma}{Lemma}
  \newtheorem*{theorem}{Theorem}
\fi

\definecolor{keywordcolor}{rgb}{0.7, 0.1, 0.1}
\definecolor{tacticcolor}{rgb}{0.0, 0.1, 0.6}
\definecolor{commentcolor}{rgb}{0.4, 0.4, 0.4}
\definecolor{symbolcolor}{rgb}{0.0, 0.1, 0.6}
\definecolor{sortcolor}{rgb}{0.1, 0.5, 0.1}
\definecolor{attributecolor}{rgb}{0.7, 0.1, 0.1}
\def\lstlanguagefiles{lstlean.tex}
\lstset{language=lean}

\newcommand\A{\mathbb{A}}
\newcommand\C{\mathbb{C}}
\newcommand\F{\mathbb{F}}
\newcommand\G{\mathbb{G}}
\renewcommand\H{\mathbb{H}}
\newcommand\I{\mathbb{I}}
\newcommand\N{\mathbb{N}}
\renewcommand\P{\mathbb{P}}
\newcommand\Q{\mathbb{Q}}
\newcommand\R{\mathbb{R}}
\newcommand\Z{\mathbb{Z}}

\renewcommand\AA{\mathcal{A}}
\newcommand\BB{\mathcal{B}}
\newcommand\CC{\mathcal{C}}
\newcommand\DD{\mathcal{D}}
\newcommand\EE{\mathcal{E}}
\newcommand\FF{\mathcal{F}}
\newcommand\GG{\mathcal{G}}
\newcommand\HH{\mathcal{H}}
\newcommand\II{\mathcal{I}}
\newcommand\LL{\mathcal{L}}
\newcommand\MM{\mathcal{M}}
\newcommand\NN{\mathcal{N}}
\newcommand\OO{\mathcal{O}}
\newcommand\PP{\mathcal{P}}
\newcommand\RR{\mathcal{R}}
\renewcommand\SS{\mathcal{S}}
\newcommand\TT{\mathcal{T}}
\newcommand\XX{\mathcal{X}}

\renewcommand\aa{\mathfrak{a}}
\newcommand\cc{\mathfrak{c}}
\newcommand\dd{\mathfrak{d}}
\newcommand\ff{\mathfrak{f}}
\renewcommand\gg{\mathfrak{g}}
\newcommand\mm{\mathfrak{m}}
\newcommand\pp{\mathfrak{p}}
\newcommand\qq{\mathfrak{q}}
\renewcommand\ss{\mathfrak{s}}

\newcommand\LLL{\mathscr{L}}

\newcommand\ab{\mathrm{ab}}
\newcommand\Ab{\mathbf{Ab}}
\newcommand\Alg{\mathbf{Alg}}
\newcommand\Aff{\mathbf{Aff}}
\newcommand\Aut{\operatorname{Aut}}
\newcommand\Az{\mathrm{Az}}
\newcommand\Br{\operatorname{Br}}
\newcommand\BSD{\operatorname{BSD}}
\newcommand\ch{\operatorname{char}}
\newcommand\Cl{\operatorname{Cl}}
\newcommand\coker{\operatorname{coker}}
\newcommand\cris{\mathrm{cris}}
\renewcommand\d{\mathrm{d}}
\newcommand\Div{\operatorname{Div}}
\newcommand\dR{\mathrm{dR}}
\newcommand\EN{\operatorname{EN}}
\newcommand\End{\operatorname{End}}
\newcommand\ES{\operatorname{ES}}
\newcommand\et{\mathrm{\acute{e}t}}
\newcommand\Et{\mathbf{\acute{E}t}}
\newcommand\Ext{\operatorname{Ext}}
\newcommand\Fr{\operatorname{Fr}}
\newcommand\Frac{\operatorname{Frac}}
\newcommand\Gal{\operatorname{Gal}}
\newcommand\GL{\operatorname{GL}}
\newcommand\Gr{\mathrm{Gr}}
\newcommand\Hom{\operatorname{Hom}}
\newcommand\HT{\mathrm{HT}}
\newcommand\id{\operatorname{id}}
\newcommand\im{\operatorname{im}}
\newcommand\Ind{\operatorname{Ind}}
\renewcommand\inf{\operatorname{inf}}
\newcommand\inv{\operatorname{inv}}
\newcommand\Irr{\operatorname{Irr}}
\newcommand\Jac{\operatorname{Jac}}
\newcommand\lcm{\operatorname{lcm}}
\newcommand\Mat{\operatorname{Mat}}
\newcommand\Mod{\mathbf{Mod}}
\newcommand\Nm{\operatorname{Nm}}
\newcommand\nr{\mathrm{nr}}
\newcommand\NS{\operatorname{NS}}
\newcommand\Ob{\operatorname{Ob}}
\newcommand\ord{\operatorname{ord}}
\newcommand\op{\mathrm{op}}
\newcommand\PGL{\operatorname{PGL}}
\newcommand\Pic{\operatorname{Pic}}
\newcommand\Prob{\operatorname{Prob}}
\newcommand\Proj{\operatorname{Proj}}
\newcommand\PSh{\mathbf{PSh}}
\newcommand\Reg{\operatorname{Reg}}
\newcommand\res{\operatorname{res}}
\newcommand\rk{\operatorname{rk}}
\newcommand\Sch{\mathbf{Sch}}
\newcommand\Sel{\operatorname{Sel}}
\newcommand\Set{\mathbf{Set}}
\newcommand\sgn{\operatorname{sgn}}
\newcommand\Sh{\mathbf{Sh}}
\newcommand\SL{\operatorname{SL}}
\newcommand\Spec{\operatorname{Spec}}
\newcommand\supp{\operatorname{supp}}
\newcommand\Tam{\operatorname{Tam}}
\newcommand\Top{\mathbf{Top}}
\newcommand\tor{\operatorname{tor}}
\newcommand\tr{\operatorname{tr}}
\newcommand\tra{\operatorname{tra}}
\newcommand\WC{\operatorname{WC}}

\DeclareFontFamily{U}{wncyr}{}
\DeclareFontShape{U}{wncyr}{m}{n}{<->wncyr10}{}
\DeclareSymbolFont{cyr}{U}{wncyr}{m}{n}
\DeclareMathSymbol{\Sha}{\mathord}{cyr}{"58}

\newcommand{\function}[5][]{
  \if &#1&
    \begin{array}{rcl}
      #2 & \longrightarrow & #3 \\
      #4 & \longmapsto     & #5
    \end{array}
  \else
    \begin{array}{rcrcl}
      #1 & : & #2 & \longrightarrow & #3 \\
         &   & #4 & \longmapsto     & #5
    \end{array}
  \fi
}

\newcommand{\functions}[7][]{
  \if &#1&
    \begin{array}{rcl}
      #2 & \longrightarrow & #3 \\
      #4 & \longmapsto     & #5 \\
      #6 & \longmapsto     & #7 \\
    \end{array}
  \else
    \begin{array}{rcrcl}
      #1 & : & #2 & \longrightarrow & #3 \\
         &   & #4 & \longmapsto     & #5 \\
         &   & #6 & \longmapsto     & #7
    \end{array}
  \fi
}
\title{L-functions of Dirichlet twists of elliptic curves: computations and congruences}
\subtitle{PhD viva examination}
\author{David Kurniadi Angdinata}
\institute{University College London}
\date{Monday, 1 December 2025}

\begin{document}

\frame\maketitle

\begin{frame}{Notation}

Let $ K $ be a global field.

\bigskip For each place $ v \in \Upsilon_K $,
\begin{itemize}
\item let $ q_v $ be the size of its residue field,
\item let $ I_v $ be its inertia group, and
\item let $ \varphi_v $ be a choice of geometric Frobenius.
\end{itemize}
For a $ \lambda $-adic representation $ \rho $ of $ K $,
\begin{itemize}
\item let $ \aa(\rho) $ be its global Artin conductor,
\item let $ \epsilon(\rho) $ be its global epsilon factor, and
\item let $ W(\rho) $ be its global root number.
\end{itemize}
Examples of $ \lambda $-adic representations of $ K $ will include
\begin{itemize}
\item the $ \ell $-adic cohomology $ \rho_{A, \ell}^\vee $ of an abelian variety $ A $,
\item the $ \ell $-adic Tate module $ \rho_{E, \ell} $ of an elliptic curve $ E $,
\item an Artin representation $ \varrho $, and
\item a primitive Dirichlet character $ \chi $.
\end{itemize}

\end{frame}

\begin{frame}{Classical L-functions}

The \textbf{L-function} of an abelian variety $ A $ over $ K $ is the complex function
$$ L(A, s) := \prod_{v \in \Upsilon_K} \dfrac{1}{L_v(A, s)}, $$
where for each place $ v \in \Upsilon_K $, the \textbf{local Euler factor} of $ A $ is given by
$$ L_v(A, s) := \det(1 - (\rho_{A, \ell}^\vee)^{I_v}(\varphi_v) \cdot q_v^{-s}), $$
for some prime $ \ell \nmid q_v $.

\bigskip

\begin{conjecture}[Birch--Swinnerton-Dyer (BSD)]
Assume that $ L(A, s) $ has meromorphic continuation at $ s = 1 $. Then its order of vanishing at $ s = 1 $ is $ \rk(A) $, and its leading term is
$$ L^*(A, 1) = \dfrac{\Omega(A) \cdot \Reg(A) \cdot \#\Sha(A) \cdot \Tam(A)}{\mu_K \cdot \#\tor(A) \cdot \#\tor(A^\vee)}. $$
\end{conjecture}

\end{frame}

\begin{frame}{Twisted L-functions}

Over a finite Galois extension $ K' $ of $ K $, Artin's formalism gives
$$ L(A / K', s) = \prod_\varrho L(A, \varrho, s), $$
where $ \varrho $ runs over Artin representations of $ K $ that factor through $ K' $ and $ L(A, \varrho, s) $ are certain \textbf{twisted L-functions} of $ A $.

\bigskip One may ask a variety of theoretical and computational questions.
\begin{itemize}
\item Are there algebraic or integral versions of $ L^*(A, \varrho, 1) $?
\item Can $ L^*(A, \varrho, 1) $ be expressed in terms of BSD invariants?
\item Does $ L^*(A, \varrho, 1) $ have a predictable asymptotic distribution?
\item Can $ L^*(A, \varrho, 1) $ be computed numerically or algorithmically?
\item Is $ L^*(A, \varrho, 1) $ directly related to $ L^*(A, 1) $?
\end{itemize}
I provide partial answers when $ A = E $ is an elliptic curve and $ \varrho = \chi $ is a primitive Dirichlet character over the global fields $ K = \Q $ and $ K = \F_q(t) $.

\end{frame}

\begin{frame}{Algebraic L-values}

When $ K = \Q $, the \textbf{algebraic L-value} of $ A $ twisted by $ \varrho $ is defined by
$$ \LLL(A, \varrho) := \dfrac{L^*(A, \varrho, 1) \cdot \sqrt{\aa(\varrho)}^{\dim(A)}}{W(\varrho)^{\dim(A)} \cdot \Omega_+(A)^{\dim(\varrho^{\varsigma = +})} \cdot \Omega_-(A)^{\dim(\varrho^{\varsigma = -})}}, $$
where $ \varsigma $ is a lift of complex conjugation in $ G_\Q $, and denote
$$ \LLL(A) := \LLL(A, 1). $$
If $ A = E $ and $ \varrho = \chi $, then
$$ \LLL(E, \chi) = \dfrac{L^*(E, \chi, 1) \cdot \aa(\chi)}{\gg(\chi) \cdot \Omega_{\chi(-1)}(E)}, $$
where $ \gg(\chi) $ is the Gauss sum of $ \chi $, and
$$ \LLL(E) = \dfrac{L^*(E, 1)}{\Omega(E)}. $$

\end{frame}

\begin{frame}{Formal L-functions}

When $ K = \F_q(C) $, rationality gives
$$ L(A, \varrho, s) = \dfrac{P_1(\rho_{A, \ell}^\vee \otimes \varrho, q^{-s})}{P_0(\rho_{A, \ell}^\vee \otimes \varrho, q^{-s}) \cdot P_2(\rho_{A, \ell}^\vee \otimes \varrho, q^{-s})}, $$
where there are canonical $ \overline{\Q_\ell} $-representations $ H^n(\rho) $ such that
$$ P_n(\rho, T) := \det(1 - T \cdot H^n(\rho)(\varphi_q)) \in \overline{\Q}[T]. $$
Define the \textbf{formal L-function} of $ A $ twisted by $ \varrho $ by
$$ \LL(A, \varrho, T) := \dfrac{P_1(\rho_{A, \ell}^\vee \otimes \varrho, T)}{P_0(\rho_{A, \ell}^\vee \otimes \varrho, T) \cdot P_2(\rho_{A, \ell}^\vee \otimes \varrho, T)}, $$
so that $ L(A, \varrho, s) = \LL(A, \varrho, q^{-s}) $, and denote
$$ \LL(A, T) := \LL(A, 1, T). $$

\end{frame}

\begin{frame}{Algebraicity of L-functions}

Assuming an appropriate automorphic correspondence for $ E $ over $ \Q^\chi $, a local argument shows that $ \LLL(E, \varrho) $ is the algebraic version of $ L^*(E, \varrho, 1) $.

\begin{theorem}[Theorem 4.2 of Bouganis--Dokchitser 2007]
Let $ K = \Q $. If $ (\aa(E), \aa(\chi)) = 1 $, then
\begin{itemize}
\item $ \LLL(E, \chi) \in \Q(\chi) $, and
\item $ \LLL(E, \chi)^\varsigma = \LLL(E, \varsigma \circ \chi) $ for all $ \varsigma \in G_\Q $.
\end{itemize}
\end{theorem}

They deduced this from the corresponding result for Rankin--Selberg convolutions of certain parallel weight primitive Hilbert modular forms.

\bigskip A similar local argument works for $ \LL(E, \chi, T) $ without assumptions.

\begin{theorem}[Theorem 5.7 of thesis]
Let $ K = \F_q(C) $. Then
\begin{itemize}
\item $ \LL(E, \chi, T) \in \Q(\chi)(T) $, and
\item $ \LL(E, \chi, T)^\varsigma = \LL(E, \varsigma \circ \chi, T) $ for all $ \varsigma \in G_\Q $.
\end{itemize}
\end{theorem}

\end{frame}

\begin{frame}{Integrality of L-functions}

Under assumptions on the Manin constant $ \cc_0(E) $, Wiersema--Wuthrich 2022 proved that $ \LLL(E, \chi) $ is integral in many cases, by formally manipulating its expression as period sums of modular symbols.

\begin{theorem}[Proposition 3.8 of thesis]
Let $ K = \Q $. If $ \chi $ has prime order $ \ell \nmid \cc_0(E) $ and $ (\aa(E), \aa(\chi)) = 1 $, then
\begin{itemize}
\item $ \LLL(E, \chi) \in \Z_\ell[\zeta_\ell] $, and
\item $ \LLL(E) \cdot \#E(\F_v) \in \Z_\ell $ for any odd prime $ v \nmid \aa(E) $.
\end{itemize}
\end{theorem}

\bigskip A similar result holds for $ \LL(E, \chi, T) $ when $ E $ and $ \chi $ are generic.

\begin{theorem}[Proposition 5.10 of thesis]
Let $ K = \F_q(C) $. If $ \chi $ is separable geometric and $ (\aa(E), \aa(\chi)) = 1 $, then
\begin{itemize}
\item $ \LL(E, \chi, T) \in \Q(\chi)[T] $, and
\item $ \LL(E, T) \in \Q[T] $ if $ E $ is non-constant.
\end{itemize}
\end{theorem}

\end{frame}

\begin{frame}{Congruences of L-functions}

When $ \chi $ has prime order $ \ell $, a bit of further work gives a congruence with $ \LLL(E) $ or $ \LL(E, T) $ modulo the prime $ (1 - \zeta_\ell) $ of $ \Z[\zeta_\ell] $ above $ \ell $.

\begin{theorem}[Corollary 3.9 of thesis]
Let $ K = \Q $. If $ \ell \nmid \cc_0(E) \cdot \aa(\chi) $ and $ (\aa(E), \aa(\chi)) = 1 $, then
$$ \LLL(E, \chi) \equiv \LLL(E) \cdot \prod_{v \mid \aa(\chi)} (-L_v(E, 1)) \mod (1 - \zeta_\ell). $$
\end{theorem}

\begin{theorem}[Theorem 5.12 of thesis]
Let $ K = \F_q(t) $. If $ E $ is non-constant and $ \chi $ is separable geometric, and furthermore $ (\aa(E), \aa(\chi)) = 1 $, then
$$ \LL(E, \chi, T) \equiv \LL(E, T) \cdot \prod_{v \mid \aa(\chi)} \LL_v(E, T) \mod (1 - \zeta_\ell). $$
\end{theorem}

\end{frame}

\begin{frame}{Ideals of L-values}

The ideal of $ \Z[\chi] $ generated by $ \LLL(E, \chi) $ and $ \LL(E, \chi, q^{-1}) $ can be expressed in terms of $ \chi $-isotypic components of $ \Reg(E) $ and $ \Sha(E) $.

\begin{theorem}[Proposition 7.3 of Burns--Castillo 2024]
Let $ K = \Q $. Assume that the refined BSD conjecture holds over $ K^\chi / K $. If $ (\aa(E), \aa(\chi)) = 1 $, then there is an explicit finite set $ S(E, \chi) \subseteq \Upsilon_{\Q(\chi)} $ such that for all $ \lambda \in \Upsilon_{\Q(\chi)} \setminus S(E, \chi) $,
$$ \LLL(E, \chi) \cdot \prod_{v \mid \aa(\chi)} L_v(E, \chi, 1) \cdot \Z[\chi]_\lambda = \Reg(E, \chi) \cdot \ch(\Sha(E, \chi)). $$
\end{theorem}

\begin{theorem}[Theorem 7.12 of Kim--Tan--Trihan--Tsoi 2024]
Let $ K = \F_q(C) $. Assume that $ \Sha(E / K^\chi) $ is finite. Then there is an explicit finite set $ S(E, \chi) \subseteq \Upsilon_{\Q(\chi)} $ such that for all $ \lambda \in \Upsilon_{\Q(\chi)} \setminus S(E, \chi) $,
$$ \LL(E, \chi, q^{-1}) \cdot \prod_{v \mid \aa(\chi)} L_v(E, \chi, 1) \cdot \Z[\chi]_\lambda = \Reg_\lambda(E, \chi) \cdot \ch(\Sha_\lambda(E, \chi)). $$
\end{theorem}

\end{frame}

\begin{frame}{Norms of L-values}

When $ K = \Q $, Dokchitser--Evans--Wiersema 2021 computed the norm of $ \LLL(E, \chi) $ in terms of $ \BSD(E) $ and $ \BSD(E / \Q^\chi) $, which are invariants such that the BSD conjecture over $ \Q $ and over $ \Q^\chi $ respectively read
$$ \LLL(E) = \BSD(E), \qquad \LLL(E / \Q^\chi) = \BSD(E / \Q^\chi). $$

\begin{theorem}[Proposition 3.13 of thesis]
Let $ K = \Q $. Assume the Manin constant conjecture $ \cc_1(E) = 1 $ and the BSD conjecture hold over $ \Q $ and over $ \Q^\chi $. If $ L(E, 1), L(E, \chi, 1) \ne 0 $, $ \chi $ has prime order $ \ell $, and $ (\aa(E), \aa(\chi)) = 1 $, then
$$ \Nm_\Q^{\Q(\zeta_\ell)^+}(\LLL(E, \chi) \cdot \chi(\aa(E))^{(\ell - 1) / 2}) = \sqrt{\BSD(E / \Q^\chi) / \BSD(E)}. $$
\end{theorem}

\bigskip There is an ongoing project led by Maistret and Wiersema as part of Women In Numbers Europe 2025 for the $ K = \F_q(C) $ analogue.

\end{frame}

\begin{frame}{Predicting algebraic L-values}

Dokchitser--Evans--Wiersema 2021 also gave examples of arithmetically identical elliptic curves $ E_1 $ and $ E_2 $ such that $ \LLL(E_1, \chi) \ne \LLL(E_2, \chi) $.

\bigskip When $ \ell = 3 $, this difference can be explained by the congruence.

\begin{theorem}[Corollary 3.14 of thesis]
Let $ K = \Q $. Assume the Manin constant conjecture $ \cc_1(E) = 1 $ and the BSD conjecture hold over $ \Q $ and over $ \Q^\chi $. If $ L(E, 1), L(E, \chi, 1) \ne 0 $, $ \chi $ is cubic, and $ (\aa(E), \aa(\chi)) = 1 $, then
$$ \LLL(E, \chi) = u \cdot \overline{\chi}(\aa(E)) \cdot \sqrt{\BSD(E / \Q^\chi) / \BSD(E)}, $$
where $ u \in \{\pm1\} $ is such that
$$ u \equiv \dfrac{\BSD(E) \cdot \prod_{v \mid \aa(\chi)} (-\#E(\F_v))}{\sqrt{\BSD(E / \Q^\chi) / \BSD(E)}} \mod 3. $$
\end{theorem}

\end{frame}

\begin{frame}{Biases of algebraic L-values}

Kisilevsky--Nam 2025 observed biases in the distribution of
$$ \widetilde{\LLL}^+(E, \chi) := \dfrac{\Nm_\Q^{\Q(\zeta_\ell)^+}(\LLL(E, \chi) \cdot (1 + \overline{\chi}(\aa(E))))}{\gcd\left\{\Nm_\Q^{\Q(\zeta_\ell)^+}(\LLL(E, \chi) \cdot (1 + \overline{\chi}(\aa(E)))) : \chi \in \XX_\ell^{< N}\right\}}, $$
as $ \chi $ varies over the set $ \XX_\ell^{< N} $ of primitive Dirichlet characters of $ \Q $ of odd prime order $ \ell \nmid \cc_0(E) $ and prime $ \aa(\chi) < N $ with $ N \to \infty $.

\bigskip

\begin{center}
\includegraphics[width=\textwidth]{img/kisilevskynam.png}
\end{center}

\end{frame}

\begin{frame}{Predicting residual L-densities}

This distribution can be quantified by computing the \textbf{residual L-density} of $ E $ modulo an odd prime $ \ell \nmid \cc_0(E) $ defined by
$$ \dd_{E, \ell}(n) := \lim_{N \to \infty} \dfrac{\#\{\chi \in \XX_\ell^{< N} : \LLL(E, \chi) \equiv n \mod (1 - \zeta_\ell)\}}{\#\XX_\ell^{< N}}. $$
Chebotarev's density theorem reduces this to computations in $ \im(\rho_{E, \ell}) $.

\begin{theorem}[Theorem 4.11 of thesis]
Let $ K = \Q $. Assume that the BSD conjecture holds over $ \Q $. If $ L(E, 1) \ne 0 $, then $ \dd_{E, \ell} $ only depends on $ \ord_\ell(\BSD(E)) $ and on $ \im(\overline{\rho}_{E, \ell^2}) $.
\end{theorem}

\bigskip A similar argument recovers the distribution of Kisilevsky--Nam 2025.

\begin{theorem}[Proposition 4.19 of thesis]
Let $ K = \Q $. If $ E $ has Cremona label 11a1, 15a1, or 17a1, and $ \chi $ is cubic, then the distribution of $ \widetilde{\LLL}^+(E, \chi) $ can be predicted precisely.
\end{theorem}

\end{frame}

\begin{frame}{Bounding denominators of L-values}

Lorenzini 2011 described the cancellations between $ \tor(E) $ and $ \Tam(E) $.

\begin{theorem}[Proposition 4.5 of thesis]
Let $ K = \Q $. If $ \ell \nmid 3 \cdot \cc_0(E) $ is an odd prime, then
$$ \ord_\ell(\#\tor(E)) \le \ord_\ell(\Tam(E)). $$
\end{theorem}

The $ \ell = 3 $ analogue can be deduced from the integrality of $ \LLL(E) $ and the classification of $ \im(\rho_{E, 3}) $ by Rouse--Sutherland--Zureick-Brown 2022.

\begin{theorem}[Theorem 4.9 of thesis]
Let $ K = \Q $. Assume that the BSD conjecture holds over $ \Q $. If $ L(E, 1) \ne 0 $ and $ \ell \nmid \cc_0(E) $, then
$$ \ord_\ell(\LLL(E)) = \ord_\ell(\BSD(E)) \ge -1. $$
\end{theorem}

There is an ongoing project by Melistas and I for the $ K = \F_q(t) $ analogue.

\end{frame}

\begin{frame}{Computations of L-values}

Much of the previous explorations were only possible thanks to efficient algorithms to compute $ \LLL(E, \chi) $ in computer algebra systems.

\begin{algorithm}[Dokchitser 2004]
Computes $ L(M, 0) $ where $ M $ is a motive over a number field.
\end{algorithm}

\bigskip There are almost no public implementations for global function fields.

\begin{algorithm}[Comeau-Lapointe--David--Lal\'in--Li 2022]
Computes $ \LL(E, \chi, T) $ where $ E $ and $ \chi $ are defined over $ \F_q(t) $.
\end{algorithm}

\bigskip The proof of the Weil conjectures gives an algorithm for general $ \lambda $-adic representations, which is used by Maistret and Wiersema in their project.

\begin{algorithm}[Algorithm 5.15 of thesis]
Computes $ \LL(\rho, T) $ where $ \rho $ is an almost everywhere unramified $ \lambda $-adic representation of $ \F_q(C) $ (that is pure of weight $ w $ and $ \rho^\vee \cong \rho^\varsigma \otimes \overline{\Q}(w) $).
\end{algorithm}

\end{frame}

\begin{frame}{Computing formal L-functions}

Let $ \rho $ be an almost everywhere unramified $ \lambda $-adic representation of $\F_q(C) $.

\begin{theorem}[Proposition 5.13 of thesis]
If $ \rho^{G_{\overline{\F_q}(C)}} = 0 $, then $ \LL(\rho, T) $ is a polynomial of degree
$$ d := \deg\aa(\rho) + (2g(C) - 2)\dim\rho, $$
where $ g(C) $ is the genus of $ C $. Furthermore, if $ \rho $ is pure of weight $ w $ and $ \rho^\vee \cong \rho^\varsigma \otimes \overline{\Q}(w) $, then the functional equation gives $ \epsilon(\rho) \in \C^\times $ such that
$$ \LL(\rho, T) = \epsilon(\rho) \cdot T^d \cdot \LL(\rho, (q^{w + 1}T)^{-1})^\varsigma. $$
In particular, if $ \{c_n\}_{n \in \N} $ denotes the coefficients of $ \LL(\rho, T) $, then
$$ c_n =
\begin{cases}
1 & \text{if} \ n = 1, \\
q^{(w + 1)(n - d)} \cdot \epsilon(\rho) \cdot c_{d - n}^\varsigma & \text{if} \ 0 < n < d, \\
\epsilon(\rho) & \text{if} \ n = d, \\
0 & \text{otherwise}.
\end{cases}
$$
\end{theorem}

\end{frame}

\begin{frame}{Computing twisted L-functions}

There is a refinement of the algorithm for tensor products $ \rho \otimes \sigma $.

\begin{theorem}[Theorem 2.7 of thesis]
Under the previous assumptions, if $ (\aa(\rho), \aa(\sigma)) = 1 $, then
$$ \epsilon(\rho \otimes \sigma) = \dfrac{\epsilon(\rho)^{\dim\sigma} \cdot \epsilon(\sigma)^{\dim\rho} \cdot \det\sigma(\aa(\rho)) \cdot \det\rho(\aa(\sigma))}{q^{(g(C) - 1)\dim\rho\dim\sigma}}. $$
\end{theorem}

The remainder of the thesis provides explicit examples of $ \LL(\rho \otimes \sigma, T) $ when $ \rho $ and $ \sigma $ arise from elliptic curves or Dirichlet characters.

\bigskip In particular, the examples use an alternative implementation of Dirichlet characters of $ \F_q(t) $ that is more amenable to computation.

\begin{theorem}[Theorem 6.6 of thesis]
Let $ K = \F_q(t) $. Then there is a canonical representation of any $ u \in (\F_q[t] / m)^\times $ that allows for an efficient computation of $ \chi(u) $.
\end{theorem}

\end{frame}

\end{document}