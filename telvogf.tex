\ifx\type\undefined
  \documentclass[10pt, t]{beamer}
  \setbeamertemplate{footline}[page number]
\else
  \documentclass[10pt]{article}
  \usepackage[margin=1in]{geometry}
\fi

\usepackage{amsmath}
\usepackage{amssymb}
\usepackage{amsthm}
\usepackage{bbm}
\usepackage{cancel}
\usepackage{listings}
\usepackage{mathrsfs}
\usepackage{multirow}
\usepackage{soul}
\usepackage{stmaryrd}
\usepackage{tikz}
\usepackage{tikz-cd}
\usepackage{wrapfig}

\newtheorem*{algorithm}{Algorithm}
\newtheorem*{assumptions}{Assumptions}
\newtheorem*{conjecture}{Conjecture}
\newtheorem*{consequences}{Consequences}
\newtheorem*{exercise}{Exercise}
\newtheorem*{formalisation}{Formalisation}
\newtheorem*{proposition}{Proposition}
\newtheorem*{question}{Question}
\newtheorem*{remark}{Remark}

\ifx\type\undefined\else
  \newtheorem*{definition}{Definition}
  \newtheorem*{example}{Example}
  \newtheorem*{lemma}{Lemma}
  \newtheorem*{theorem}{Theorem}
\fi

\definecolor{keywordcolor}{rgb}{0.7, 0.1, 0.1}
\definecolor{tacticcolor}{rgb}{0.0, 0.1, 0.6}
\definecolor{commentcolor}{rgb}{0.4, 0.4, 0.4}
\definecolor{symbolcolor}{rgb}{0.0, 0.1, 0.6}
\definecolor{sortcolor}{rgb}{0.1, 0.5, 0.1}
\definecolor{attributecolor}{rgb}{0.7, 0.1, 0.1}
\def\lstlanguagefiles{lstlean.tex}
\lstset{language=lean}

\newcommand\A{\mathbb{A}}
\newcommand\C{\mathbb{C}}
\newcommand\F{\mathbb{F}}
\newcommand\G{\mathbb{G}}
\renewcommand\H{\mathbb{H}}
\newcommand\I{\mathbb{I}}
\newcommand\N{\mathbb{N}}
\renewcommand\P{\mathbb{P}}
\newcommand\Q{\mathbb{Q}}
\newcommand\R{\mathbb{R}}
\newcommand\Z{\mathbb{Z}}

\renewcommand\AA{\mathcal{A}}
\newcommand\BB{\mathcal{B}}
\newcommand\CC{\mathcal{C}}
\newcommand\DD{\mathcal{D}}
\newcommand\EE{\mathcal{E}}
\newcommand\FF{\mathcal{F}}
\newcommand\GG{\mathcal{G}}
\newcommand\HH{\mathcal{H}}
\newcommand\II{\mathcal{I}}
\newcommand\LL{\mathcal{L}}
\newcommand\MM{\mathcal{M}}
\newcommand\NN{\mathcal{N}}
\newcommand\OO{\mathcal{O}}
\newcommand\PP{\mathcal{P}}
\newcommand\RR{\mathcal{R}}
\renewcommand\SS{\mathcal{S}}
\newcommand\TT{\mathcal{T}}
\newcommand\XX{\mathcal{X}}

\renewcommand\aa{\mathfrak{a}}
\newcommand\cc{\mathfrak{c}}
\newcommand\dd{\mathfrak{d}}
\newcommand\ff{\mathfrak{f}}
\renewcommand\gg{\mathfrak{g}}
\newcommand\mm{\mathfrak{m}}
\newcommand\pp{\mathfrak{p}}
\newcommand\qq{\mathfrak{q}}
\renewcommand\ss{\mathfrak{s}}

\newcommand\LLL{\mathscr{L}}

\newcommand\ab{\mathrm{ab}}
\newcommand\Ab{\mathbf{Ab}}
\newcommand\Alg{\mathbf{Alg}}
\newcommand\Aff{\mathbf{Aff}}
\newcommand\Aut{\operatorname{Aut}}
\newcommand\Az{\mathrm{Az}}
\newcommand\Br{\operatorname{Br}}
\newcommand\BSD{\operatorname{BSD}}
\newcommand\ch{\operatorname{char}}
\newcommand\Cl{\operatorname{Cl}}
\newcommand\coker{\operatorname{coker}}
\newcommand\cris{\mathrm{cris}}
\renewcommand\d{\mathrm{d}}
\newcommand\Div{\operatorname{Div}}
\newcommand\dR{\mathrm{dR}}
\newcommand\EN{\operatorname{EN}}
\newcommand\End{\operatorname{End}}
\newcommand\ES{\operatorname{ES}}
\newcommand\et{\mathrm{\acute{e}t}}
\newcommand\Et{\mathbf{\acute{E}t}}
\newcommand\Ext{\operatorname{Ext}}
\newcommand\Fr{\operatorname{Fr}}
\newcommand\Frac{\operatorname{Frac}}
\newcommand\Gal{\operatorname{Gal}}
\newcommand\GL{\operatorname{GL}}
\newcommand\Gr{\mathrm{Gr}}
\newcommand\Hom{\operatorname{Hom}}
\newcommand\HT{\mathrm{HT}}
\newcommand\id{\operatorname{id}}
\newcommand\im{\operatorname{im}}
\newcommand\Ind{\operatorname{Ind}}
\renewcommand\inf{\operatorname{inf}}
\newcommand\inv{\operatorname{inv}}
\newcommand\Irr{\operatorname{Irr}}
\newcommand\Jac{\operatorname{Jac}}
\newcommand\lcm{\operatorname{lcm}}
\newcommand\Mat{\operatorname{Mat}}
\newcommand\Mod{\mathbf{Mod}}
\newcommand\Nm{\operatorname{Nm}}
\newcommand\nr{\mathrm{nr}}
\newcommand\NS{\operatorname{NS}}
\newcommand\Ob{\operatorname{Ob}}
\newcommand\ord{\operatorname{ord}}
\newcommand\op{\mathrm{op}}
\newcommand\PGL{\operatorname{PGL}}
\newcommand\Pic{\operatorname{Pic}}
\newcommand\Prob{\operatorname{Prob}}
\newcommand\Proj{\operatorname{Proj}}
\newcommand\PSh{\mathbf{PSh}}
\newcommand\Reg{\operatorname{Reg}}
\newcommand\res{\operatorname{res}}
\newcommand\rk{\operatorname{rk}}
\newcommand\Sch{\mathbf{Sch}}
\newcommand\Sel{\operatorname{Sel}}
\newcommand\Set{\mathbf{Set}}
\newcommand\sgn{\operatorname{sgn}}
\newcommand\Sh{\mathbf{Sh}}
\newcommand\SL{\operatorname{SL}}
\newcommand\Spec{\operatorname{Spec}}
\newcommand\supp{\operatorname{supp}}
\newcommand\Tam{\operatorname{Tam}}
\newcommand\Top{\mathbf{Top}}
\newcommand\tor{\operatorname{tor}}
\newcommand\tr{\operatorname{tr}}
\newcommand\tra{\operatorname{tra}}
\newcommand\WC{\operatorname{WC}}

\DeclareFontFamily{U}{wncyr}{}
\DeclareFontShape{U}{wncyr}{m}{n}{<->wncyr10}{}
\DeclareSymbolFont{cyr}{U}{wncyr}{m}{n}
\DeclareMathSymbol{\Sha}{\mathord}{cyr}{"58}

\newcommand{\function}[5][]{
  \if &#1&
    \begin{array}{rcl}
      #2 & \longrightarrow & #3 \\
      #4 & \longmapsto     & #5
    \end{array}
  \else
    \begin{array}{rcrcl}
      #1 & : & #2 & \longrightarrow & #3 \\
         &   & #4 & \longmapsto     & #5
    \end{array}
  \fi
}

\newcommand{\functions}[7][]{
  \if &#1&
    \begin{array}{rcl}
      #2 & \longrightarrow & #3 \\
      #4 & \longmapsto     & #5 \\
      #6 & \longmapsto     & #7 \\
    \end{array}
  \else
    \begin{array}{rcrcl}
      #1 & : & #2 & \longrightarrow & #3 \\
         &   & #4 & \longmapsto     & #5 \\
         &   & #6 & \longmapsto     & #7
    \end{array}
  \fi
}
\usetheme{frankfurt}
\title{Twisted elliptic L-values over global fields}
\subtitle{Algebraic Number Theory}
\author{David Kurniadi Angdinata}
\institute{London School of Geometry and Number Theory}
\date{Thursday, 5 September 2024}

\begin{document}

\frame\maketitle

\section{The BSD formula}

\begin{frame}{Twisted L-series}

Let $ A $ be an abelian variety over a global field $ K $, and let $ \chi $ be a character of a finite group $ G $. The \textbf{Artin--Hasse--Weil L-series} of $ (A, \chi) $ is
$$ L(A, \chi, s) := \prod_\pp \dfrac{1}{L_\pp(\rho_{A, \ell}^\vee \otimes \chi, q_\pp^{-s})}, $$
where $ L_\pp(\rho, T) := \det(1 - T \cdot \phi_\pp \mid \rho^{I_\pp}) $.

\bigskip If $ \chi = 1 $, then $ L(A, \chi, s) = L(A, s) $ is the \textbf{Hasse--Weil L-series} of $ A $.

\bigskip If $ L $ is a finite Galois extension of $ K $ with Galois group $ G $, then
$$ L(A / L, s) = L(A, \Ind_{\{1\}}^G 1, s) = \prod_\chi L(A, \chi, s), $$
where $ \chi $ runs over all the irreducible characters $ \Irr(G) $ of $ G $.

\end{frame}

\begin{frame}{The Birch--Swinnerton-Dyer conjecture}

\begin{conjecture}[Birch--Swinnerton-Dyer]
The order of vanishing of $ L(A, s) $ at $ s = 1 $ is equal to $ \rk(A) $. Furthermore, the leading term of $ L(A, s) $ at $ s = 1 $ is equal to
$$ L^*(A, 1) = \dfrac{\Omega(A) \cdot \Reg(A) \cdot \Tam(A) \cdot \#\Sha(A)}{\sqrt{|\Delta_K|} \cdot \#\tor(A) \cdot \#\tor(\widehat{A})}. $$
\end{conjecture}

This is known in some cases when $ A $ is an elliptic curve.
\begin{itemize}
\item If $ K = \Q $ and the order of vanishing is at most $ 1 $, then the rank conjecture is proven by Gross--Zagier 1986 and Kolyvagin 1988, and much of the $ \ell $-part of the leading term conjecture is proven by Keller--Yin 2024 and Burungale--Castella--Skinner 2024.
\item If $ K = \F_p(C) $, then Kato--Trihan 2003 proved that the rank conjecture is equivalent to the finiteness of $ \Sha(A)[\ell^\infty] $ for some prime $ \ell \ne p $ and implies the leading term conjecture.
\end{itemize}

\end{frame}

\begin{frame}{The Deligne--Gross conjecture}

\begin{conjecture}[Deligne--Gross]
The order of vanishing of $ L(A, \chi, s) $ at $ s = 1 $ is equal to $ \langle\chi, A(L)_\C\rangle $.
\end{conjecture}

This is known in some cases when $ A $ is an elliptic curve over $ \Q $.
\begin{itemize}
\item If $ A $ has no potential complex multiplication, then Kato 2004 proved this for one-dimensional Artin representations.
\item If the order of vanishing is $ 0 $, then Bertolini--Darmon--Rotger 2015 proved this for odd irreducible two-dimensional Artin representations.
\item If the order of vanishing is $ 0 $, then Darmon--Rotger 2017 proved this for certain self-dual Artin representations of dimension at most $ 4 $.
\end{itemize}

\begin{theorem}[Bisatt--Dokchitser 2018]
Assume the Deligne--Gross conjecture. If $ \chi \in \Irr(C_q \rtimes C_{p^n}) $ with $ q \not\equiv 1 \mod p^n $, then $ p $ divides the order of vanishing of $ L(A, \chi, s) $ at $ s = 1 $.
\end{theorem}

\end{frame}

\begin{frame}{A twisted leading term conjecture}

There seems to be a barrier to a leading term conjecture for $ L(A, \chi, s) $.

\begin{example}
Let $ A_1 $ and $ A_2 $ be elliptic curves over $ \Q $ given by Cremona labels 1356d1 and 1356f1, and let $ \chi $ be the primitive Dirichlet character of conductor $ 7 $ and order $ 3 $ given by $ \chi(3) = \zeta_3^2 $. Then
$$ \Reg(A_i / K) = \Tam(A_i / K) = \Sha(A_i / K) = \tor(A_i / K) = 1, $$
for $ K = \Q $ and $ K = \Q(\zeta_7)^+ $, but $ \LLL(A_1, \chi) = \zeta_3^2 $ and $ \LLL(A_2, \chi) = -\zeta_3^2 $.
\end{example}

\begin{theorem}[Dokchitser--Evans--Wiersema 2021]
Assume there is a conjecture $ \LLL(A, \chi) = \BSD(A, \chi) $ for a semistable elliptic curve $ A $ over $ \Q $. If $ \chi \in \Irr(D_{pq}) $ with $ p \equiv q \equiv 3 \mod 4 $, then $ \langle\chi, A(L)_\C\rangle > 0 $ if the order of vanishing of $ L(A, \chi, s) $ at $ s = 1 $ is odd.
\end{theorem}

\end{frame}

\section{Algebraic L-values}

\begin{frame}{Algebraic L-values}

Define the \textbf{algebraic L-value} of $ A $ by
$$ \LLL(A) := \dfrac{L^*(A, 1)}{\Omega(A) \cdot \Reg(A)}. $$
If $ L(A, 1) \ne 0 $, then
$$ \LLL(A) = \dfrac{L(A, 1)}{\Omega(A)}. $$
If $ A $ is an elliptic curve over $ \Q $, then modularity gives
$$ -(1 + p - a_p(A)) \cdot L(f_A, 1) = \sum_{n = 1}^{p - 1} \int_0^{\tfrac{n}{p}} f_A(q)\d q, $$
which is a rational multiple of $ \Omega(A) $.

\bigskip In general, the algebraicity of $ \LLL(A) $ is Deligne's period conjecture.

\end{frame}

\begin{frame}{Deligne's period conjecture}

A motive $ M $ over a global field $ K $ is a collection of $ K $-vector space realisations $ H_B(M) $, $ H_{dR}(M) $, $ H_\lambda(M) $, and $ H_p(M) $, equipped with comparison isomorphisms between their complexifications.

\begin{conjecture}[Deligne]
Let $ M $ be a critical motive over a number field $ K $ such that $ L(M, 0) \ne 0 $. Then there is some $ x \in K^\times $ such that
$$ L(M, 0) = x^\sigma \cdot c^+(M), \qquad \sigma \in \Gal(K / \Q). $$
\end{conjecture}

Here, $ c^+(M) $ is the determinant of the period map
$$ H_B(M)^+ \otimes \C \hookrightarrow H_B(M) \otimes \C \xrightarrow{\sim} H_{dR}(M) \otimes \C \twoheadrightarrow H_{dR}(M)^+ \otimes \C. $$
If $ M = h^1(A)(1) $, then this says that there is some $ x \in \Q^\times $ such that
$$ L(A, 1) = x \cdot \Omega(A). $$

\end{frame}

\begin{frame}{Algebraic twisted L-values}

Define the \textbf{algebraic twisted L-value} of $ (A, \chi) $ by
$$ \LLL(A, \chi) := \dfrac{L^*(A, \chi, 1)}{\Omega(A, \chi) \cdot \Reg(A, \chi)}. $$
Define the twisted period of $ (A, \chi) $ by
$$ \Omega(A, \chi) := \dfrac{\Omega_+(A)^{\dim^+(\chi)} \cdot \Omega_-(A)^{\dim^-(\chi)} \cdot w_\chi^{\dim(A)}}{\sqrt{N_\chi}^{\dim(A)}}. $$
Define the twisted regulator of $ (A, \chi) $ by
$$ \Reg(A, \chi) := \det(\langle e_i(\chi), e_j(\widehat{\chi})\rangle)_{i, j}, $$
where $ \{e_i(\chi)\}_i $ is a basis of
$$ A(L)[\chi] := \Hom_{\Z[\chi]}(\rho_\chi, A(L) \otimes_\Z \Z[\chi])^{\Gal(L / K)}. $$

\end{frame}

\begin{frame}{Algebraicity of twisted L-values}

If $ L(A, \chi, 1) \ne 0 $, then $ \Reg(A, \chi) = 1 $. Then Deligne's period conjecture for $ M = h^1(A)(1) \otimes \chi $ says that there is some $ x \in \Q(\chi)^\times $ such that
$$ L(A, \chi^\sigma, 1) = x^\sigma \cdot \Omega(A, \chi^\sigma), \qquad \sigma \in \Gal(\Q(\chi) / \Q). $$
Thus $ \LLL(A, \chi^\sigma) = \LLL(A, \chi)^\sigma $ for all $ \sigma \in \Gal(\Q(\chi) / \Q) $.

\begin{theorem}[Bouganis--Dokchitser 2007, Wiersema--Wuthrich 2021]
Let $ L $ be a finite abelian extension of $ \Q $ with Galois group $ G $, and let $ A $ be an elliptic curve over $ \Q $ such that $ L(A, \chi, 1) \ne 0 $. Then for any non-trivial $ \chi \in \Irr(G) $ such that $ (N_\chi, N_A) = 1 $,
$$ \LLL(A, \chi^\sigma) = \LLL(A, \chi)^\sigma, \qquad \sigma \in \Gal(\Q(\chi) / \Q). $$
Furthermore, $ \LLL(A, \chi) \in \Z[\chi] $ assuming that $ c_1(A) = 1 $.
\end{theorem}

Castillo--Evans--Wiersema 2023 gave numerical evidence for $ A = \Jac(C) $.

\end{frame}

\section{A twisted BSD formula}

\begin{frame}{Ideals of twisted L-values}

The ideal generated by $ \LLL(A, \chi) $ has a conjectural twisted BSD formula.

\begin{theorem}[Burns--Castillo 2019]
Let $ L $ be a finite Galois extension of $ \Q $ with Galois group $ G $, and let $ A $ be an abelian variety over $ \Q $ such that $ (\Delta_L, N_A) = 1 $. Assume that the refined Birch--Swinnerton-Dyer conjecture holds for $ (A, L, \Q) $. Let $ \chi \in \Irr(G) $, and let $ \lambda $ be a prime of $ \Q(\chi) $ not dividing
$$ 2, \quad |G|, \quad \Delta_L, \quad N_A, \quad \Tam(A), \quad \#\tor(A / L), \quad \#\tor(\widehat{A} / L). $$
Then there is an equality of fractional $ \Z[\chi]_\lambda $-ideals
$$ \LLL(A, \chi) \cdot \Z[\chi]_\lambda = \dfrac{\ch_\lambda(\Sha(A / L, \chi))}{\prod_{v \mid \Delta_L} L_v(A, \chi, 1)}. $$
\end{theorem}

Here, $ \Sha(A / L, \chi) := \Hom_{\Z[\chi]}(\rho_\chi, \Sha(A / L) \otimes_\Z \Z[\chi])^{\Gal(L / K)} $.

\end{frame}

\begin{frame}{Norms of twisted L-values}

The norm of $ \LLL(A, \chi) $ has a conjectural expression when $ L(A, \chi, 1) \ne 0 $.

\begin{theorem}[Dokchitser--Evans--Wiersema 2021]
Let $ L $ be a finite abelian extension of $ \Q $ with Galois group $ G $, and let $ A $ be an elliptic curve over $ \Q $ such that $ c_1(A) = 1 $. Assume that the Birch--Swinnerton-Dyer conjecture holds for $ (A, L) $ and $ (A, \Q) $. Let $ \chi \in \Irr(G) $ have odd prime conductor $ p \nmid N_A $ and odd prime order $ q \nmid \#A(\F_p) \cdot \LLL(A) $ such that $ L(A, \chi, 1) \ne 0 $. Then
$$ \Nm_\Q^{\Q(\zeta_q)^+}(\LLL(A, \chi) \cdot \zeta) = B(K), $$
where $ \zeta := \chi(N_A)^{(q - 1) / 2} $ and $ K $ is the subfield of $ \Q(\zeta_p) $ cut out by $ \chi $.
\end{theorem}

Here,
$$ B(K) := \dfrac{\#\tor(A)}{\#\tor(A / K)}\sqrt{\dfrac{\Tam(A / K) \cdot \#\Sha(A / K)}{\Tam(A) \cdot \#\Sha(A)}}. $$

\end{frame}

\begin{frame}{Values of twisted L-values}

The value of $ \LLL(A, \chi) $ can be predicted precisely when $ \chi $ is cubic.

\begin{theorem}[A. 2023]
Let $ L $ be a finite abelian extension of $ \Q $ with Galois group $ G $, and let $ A $ be an elliptic curve over $ \Q $ such that $ c_1(A) = 1 $. Assume that the Birch--Swinnerton-Dyer conjecture holds for $ (A, L) $ and $ (A, \Q) $. Let $ \chi \in \Irr(G) $ have odd prime conductor $ p \nmid N_A $ and order $ 3 \nmid \#A(\F_p) \cdot \LLL(A) $ such that $ L(A, \chi, 1) \ne 0 $. Then
$$ \LLL(A, \chi) \cdot \zeta =
\begin{cases}
B(K) & \text{if} \ \#A(\F_p) \cdot \LLL(A) \cdot B(K)^{-1} \equiv 2 \mod 3, \\
-B(K) & \text{if} \ \#A(\F_p) \cdot \LLL(A) \cdot B(K)^{-1} \equiv 1 \mod 3,
\end{cases}
$$
where $ \zeta := \chi(N_A)^{(q - 1) / 2} $.
\end{theorem}

This follows from $ -\#A(\F_p) \cdot \LLL(A) \equiv \LLL(A, \chi) \mod (1 - \zeta_q) $, which arises from a congruence in Manin's formalism for modular symbols.

\end{frame}

\begin{frame}{Example of twisted L-values}

This explains a barrier to a leading term conjecture for $ L(A, \chi, s) $.

\begin{example}
Let $ A_1 $ and $ A_2 $ be elliptic curves over $ \Q $ given by Cremona labels 1356d1 and 1356f1, and let $ \chi $ be the primitive Dirichlet character of conductor $ 7 $ and order $ 3 $ given by $ \chi(3) = \zeta_3^2 $. Then
$$ \LLL(A_1, \chi) = \zeta_3^2, \qquad \LLL(A_2, \chi) = -\zeta_3^2. $$
Now Dokchitser--Evans--Wiersema 2021 says that
$$ \LLL(A_i, \chi) = \pm\chi(1356)^2 = \pm\zeta_3^2. $$
On the other hand $ \#A_1(\F_7) = 11 $ and $ \#A_2(\F_7) = 7 $, so A. 2023 says that
$$ \LLL(A_1, \chi) \equiv -\#A_1(\F_7) \equiv 1 \equiv \zeta_3^2 \mod (1 - \zeta_3), $$
$$ \LLL(A_2, \chi) \equiv -\#A_2(\F_7) \equiv -1 \equiv -\zeta_3^2 \mod (1 - \zeta_3). $$
\end{example}

\end{frame}

\section{Global function fields}

\begin{frame}{Algebraic twisted L-values}

Now let $ A $ be an abelian variety over a global function field $ K = \F_p(C) $. By the Grothendieck--Lefschetz trace formula,
$$ L(A, \chi, s) = \prod_{i = 0}^2 \det(1 - p^{-s} \cdot \phi_p \mid H_{\et, c}^i(\overline{C}, \FF))^{(-1)^{i + 1}}, $$
where $ \FF $ is the constructible sheaf on $ C $ given by the pushforward of the lisse sheaf $ V_\ell(A) \otimes \rho_\chi $ defined over any unramified open subset of $ C $.

\bigskip Since $ L(A, \chi, s) $ is already algebraic, define $ \LLL(A, \chi) := L^*(A, \chi, 1) $.

\begin{theorem}[A. 2024]
Let $ L $ be a finite Galois extension of $ K = \F_p(C) $ with Galois group $ G $, and let $ A $ be an abelian variety over $ K $. Then for any $ \chi \in \Irr(G) $,
$$ \LLL(A, \chi^\sigma) = \LLL(A, \chi)^\sigma, \qquad \sigma \in \Gal(\Q(\chi) / \Q). $$
\end{theorem}

\end{frame}

\begin{frame}{Ideals of twisted L-values}

The ideal generated by $ \LLL(A, \chi) $ has a conjectural twisted BSD formula.

\begin{theorem}[Kim--Tan--Trihan--Tsoi 2024]
Let $ L $ be a finite Galois extension of $ K = \F_p(C) $ with Galois group $ G $, and let $ A $ be an abelian variety over $ K $. Assume that $ \Sha(A / L) $ is finite. Let $ \chi \in \Irr(G) $, and let $ \lambda $ be a prime of $ \Q(\chi) $ not dividing
$$ p, \qquad |G|, \qquad \Tam(A), \qquad \#\tor(A / L), \qquad \#\tor(\widehat{A} / L). $$
Then there is an equality of fractional $ \Z[\chi]_\lambda $-ideals
$$ \LLL(A, \chi) \cdot \Z[\chi]_\lambda = \dfrac{\Reg_\lambda(A, \chi) \cdot \ch_\lambda(\Sha_\lambda(A / L, \chi))}{\prod_{v \mid \Delta_L} L_v(A, \chi, 1)}. $$
\end{theorem}

This involves the $ \Z[\chi]_\lambda $-modules $ \Reg_\lambda(A, \chi) $ and $ \Sha_\lambda(A / L, \chi) $, which are necessary to generalise the statement to primes $ \lambda $ of $ \Q(\chi) $ dividing $ p $.

\end{frame}

\begin{frame}{Values of twisted L-values}

Can we predict the value of $ \LLL(A, \chi) $ analogous to A. 2023?

\bigskip I am currently working on this.

\bigskip If $ A $ is an elliptic curve over $ \Q $, then modularity gives a congruence of classical modular symbols, which proves A. 2023. In contrast, there are modularity results for abelian varieties over $ K = \F_p(C) $, due to the global Langlands conjectures proven by Drinfeld 1989 and Lafforgue 1998.

\bigskip On the other hand, $ L(A, \chi, s) $ is already a rational function in $ p^{-s} $ with coefficients in $ \Q(\chi) $, which can be determined by investigating the action of $ \phi_p $ on $ H_{\et, c}^i(\overline{C}, \FF) $. In fact, $ L(A, \chi, s) $ is a polynomial in $ p^{-s} $ with coefficients in $ \Z[\chi] $ under certain conditions on $ (A, \chi) $.

\bigskip Can we understand the action of $ \phi_p $ from the geometry of $ (A, \chi) $?

\end{frame}

\end{document}