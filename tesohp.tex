\ifx\type\undefined
  \documentclass[10pt, t]{beamer}
  \setbeamertemplate{footline}[page number]
\else
  \documentclass[10pt]{article}
  \usepackage[margin=1in]{geometry}
\fi

\usepackage{amsmath}
\usepackage{amssymb}
\usepackage{amsthm}
\usepackage{bbm}
\usepackage{cancel}
\usepackage{listings}
\usepackage{mathrsfs}
\usepackage{multirow}
\usepackage{soul}
\usepackage{stmaryrd}
\usepackage{tikz}
\usepackage{tikz-cd}
\usepackage{wrapfig}

\newtheorem*{algorithm}{Algorithm}
\newtheorem*{assumptions}{Assumptions}
\newtheorem*{conjecture}{Conjecture}
\newtheorem*{consequences}{Consequences}
\newtheorem*{exercise}{Exercise}
\newtheorem*{formalisation}{Formalisation}
\newtheorem*{proposition}{Proposition}
\newtheorem*{question}{Question}
\newtheorem*{remark}{Remark}

\ifx\type\undefined\else
  \newtheorem*{definition}{Definition}
  \newtheorem*{example}{Example}
  \newtheorem*{lemma}{Lemma}
  \newtheorem*{theorem}{Theorem}
\fi

\definecolor{keywordcolor}{rgb}{0.7, 0.1, 0.1}
\definecolor{tacticcolor}{rgb}{0.0, 0.1, 0.6}
\definecolor{commentcolor}{rgb}{0.4, 0.4, 0.4}
\definecolor{symbolcolor}{rgb}{0.0, 0.1, 0.6}
\definecolor{sortcolor}{rgb}{0.1, 0.5, 0.1}
\definecolor{attributecolor}{rgb}{0.7, 0.1, 0.1}
\def\lstlanguagefiles{lstlean.tex}
\lstset{language=lean}

\newcommand\A{\mathbb{A}}
\newcommand\C{\mathbb{C}}
\newcommand\F{\mathbb{F}}
\newcommand\G{\mathbb{G}}
\renewcommand\H{\mathbb{H}}
\newcommand\I{\mathbb{I}}
\newcommand\N{\mathbb{N}}
\renewcommand\P{\mathbb{P}}
\newcommand\Q{\mathbb{Q}}
\newcommand\R{\mathbb{R}}
\newcommand\Z{\mathbb{Z}}

\renewcommand\AA{\mathcal{A}}
\newcommand\BB{\mathcal{B}}
\newcommand\CC{\mathcal{C}}
\newcommand\DD{\mathcal{D}}
\newcommand\EE{\mathcal{E}}
\newcommand\FF{\mathcal{F}}
\newcommand\GG{\mathcal{G}}
\newcommand\HH{\mathcal{H}}
\newcommand\II{\mathcal{I}}
\newcommand\LL{\mathcal{L}}
\newcommand\MM{\mathcal{M}}
\newcommand\NN{\mathcal{N}}
\newcommand\OO{\mathcal{O}}
\newcommand\PP{\mathcal{P}}
\newcommand\RR{\mathcal{R}}
\renewcommand\SS{\mathcal{S}}
\newcommand\TT{\mathcal{T}}
\newcommand\XX{\mathcal{X}}

\renewcommand\aa{\mathfrak{a}}
\newcommand\cc{\mathfrak{c}}
\newcommand\dd{\mathfrak{d}}
\newcommand\ff{\mathfrak{f}}
\renewcommand\gg{\mathfrak{g}}
\newcommand\mm{\mathfrak{m}}
\newcommand\pp{\mathfrak{p}}
\newcommand\qq{\mathfrak{q}}
\renewcommand\ss{\mathfrak{s}}

\newcommand\LLL{\mathscr{L}}

\newcommand\ab{\mathrm{ab}}
\newcommand\Ab{\mathbf{Ab}}
\newcommand\Alg{\mathbf{Alg}}
\newcommand\Aff{\mathbf{Aff}}
\newcommand\Aut{\operatorname{Aut}}
\newcommand\Az{\mathrm{Az}}
\newcommand\Br{\operatorname{Br}}
\newcommand\BSD{\operatorname{BSD}}
\newcommand\ch{\operatorname{char}}
\newcommand\Cl{\operatorname{Cl}}
\newcommand\coker{\operatorname{coker}}
\newcommand\cris{\mathrm{cris}}
\renewcommand\d{\mathrm{d}}
\newcommand\Div{\operatorname{Div}}
\newcommand\dR{\mathrm{dR}}
\newcommand\EN{\operatorname{EN}}
\newcommand\End{\operatorname{End}}
\newcommand\ES{\operatorname{ES}}
\newcommand\et{\mathrm{\acute{e}t}}
\newcommand\Et{\mathbf{\acute{E}t}}
\newcommand\Ext{\operatorname{Ext}}
\newcommand\Fr{\operatorname{Fr}}
\newcommand\Frac{\operatorname{Frac}}
\newcommand\Gal{\operatorname{Gal}}
\newcommand\GL{\operatorname{GL}}
\newcommand\Gr{\mathrm{Gr}}
\newcommand\Hom{\operatorname{Hom}}
\newcommand\HT{\mathrm{HT}}
\newcommand\id{\operatorname{id}}
\newcommand\im{\operatorname{im}}
\newcommand\Ind{\operatorname{Ind}}
\renewcommand\inf{\operatorname{inf}}
\newcommand\inv{\operatorname{inv}}
\newcommand\Irr{\operatorname{Irr}}
\newcommand\Jac{\operatorname{Jac}}
\newcommand\lcm{\operatorname{lcm}}
\newcommand\Mat{\operatorname{Mat}}
\newcommand\Mod{\mathbf{Mod}}
\newcommand\Nm{\operatorname{Nm}}
\newcommand\nr{\mathrm{nr}}
\newcommand\NS{\operatorname{NS}}
\newcommand\Ob{\operatorname{Ob}}
\newcommand\ord{\operatorname{ord}}
\newcommand\op{\mathrm{op}}
\newcommand\PGL{\operatorname{PGL}}
\newcommand\Pic{\operatorname{Pic}}
\newcommand\Prob{\operatorname{Prob}}
\newcommand\Proj{\operatorname{Proj}}
\newcommand\PSh{\mathbf{PSh}}
\newcommand\Reg{\operatorname{Reg}}
\newcommand\res{\operatorname{res}}
\newcommand\rk{\operatorname{rk}}
\newcommand\Sch{\mathbf{Sch}}
\newcommand\Sel{\operatorname{Sel}}
\newcommand\Set{\mathbf{Set}}
\newcommand\sgn{\operatorname{sgn}}
\newcommand\Sh{\mathbf{Sh}}
\newcommand\SL{\operatorname{SL}}
\newcommand\Spec{\operatorname{Spec}}
\newcommand\supp{\operatorname{supp}}
\newcommand\Tam{\operatorname{Tam}}
\newcommand\Top{\mathbf{Top}}
\newcommand\tor{\operatorname{tor}}
\newcommand\tr{\operatorname{tr}}
\newcommand\tra{\operatorname{tra}}
\newcommand\WC{\operatorname{WC}}

\DeclareFontFamily{U}{wncyr}{}
\DeclareFontShape{U}{wncyr}{m}{n}{<->wncyr10}{}
\DeclareSymbolFont{cyr}{U}{wncyr}{m}{n}
\DeclareMathSymbol{\Sha}{\mathord}{cyr}{"58}

\newcommand{\function}[5][]{
  \if &#1&
    \begin{array}{rcl}
      #2 & \longrightarrow & #3 \\
      #4 & \longmapsto     & #5
    \end{array}
  \else
    \begin{array}{rcrcl}
      #1 & : & #2 & \longrightarrow & #3 \\
         &   & #4 & \longmapsto     & #5
    \end{array}
  \fi
}

\newcommand{\functions}[7][]{
  \if &#1&
    \begin{array}{rcl}
      #2 & \longrightarrow & #3 \\
      #4 & \longmapsto     & #5 \\
      #6 & \longmapsto     & #7 \\
    \end{array}
  \else
    \begin{array}{rcrcl}
      #1 & : & #2 & \longrightarrow & #3 \\
         &   & #4 & \longmapsto     & #5 \\
         &   & #6 & \longmapsto     & #7
    \end{array}
  \fi
}
\title{The Euler system of Heegner points \footnote{Victor Kolyvagin, 1989. \textbf{Euler Systems}, in \emph{Grothendieck Festschrift}}}
\subtitle{London Junior Number Theory Seminar}
\author{David Kurniadi Angdinata}
\institute{London School of Geometry and Number Theory}
\date{Tuesday, 10 May 2022}

\begin{document}

\frame\maketitle

\begin{frame}[c]{Overview}

\begin{itemize}
\item Introduction
\begin{itemize}
\item From Gross--Zagier to Kolyvagin
\item Application to BSD
\item The main result
\end{itemize}
\item Generalised Selmer groups
\begin{itemize}
\item Selmer structures
\item Application of Tate duality
\item Application of Chebotarev density
\end{itemize}
\item The Euler system of Heegner points
\begin{itemize}
\item Heegner points of higher conductors
\item Derived Kolyvagin classes
\item Computing the Selmer group
\end{itemize}
\end{itemize}

\end{frame}

\begin{frame}{From Gross--Zagier to Kolyvagin}

\begin{assumptions}
\begin{itemize}
\item Elliptic curve $ E / \Q $ with modular parameterisation $ \phi : X_0(N) \twoheadrightarrow E $.
\item Imaginary quadratic field $ K = \Q(\sqrt{-D}) $ with \textbf{Heegner condition}: \footnote{assume $ \End(E) \cong \Z $ and $ D \ne 1, 3 $}
$$ p \mid N \qquad \implies \qquad p \ \text{is split in} \ K. $$
\end{itemize}
\vspace{-0.5cm}
\end{assumptions}

\begin{consequences}
\begin{itemize}
\item An ideal $ \NN_K \trianglelefteq \OO_K $ such that $ \OO_K / \NN_K \cong \Z / N $.
\item A cyclic $ N $-isogeny $ \C / \OO_K \to \C / \NN_K^{-1} $.
\item A point $ x_1 \in X_0(N)(K^1) $ by CM theory.
\item A \textbf{Heegner point} $ P_1 := \phi(x_1) \in E(K^1) $.
\item A \textbf{basic Heegner point}
$$ P_K := \sum_{\sigma \in \Gal(K^1 / K)} \sigma(P_1) \in E(K). $$
\end{itemize}
\end{consequences}

\end{frame}

\begin{frame}{From Gross--Zagier to Kolyvagin}

Recall the Gross--Zagier formula.

\bigskip

\begin{theorem}[Gross--Zagier, 1986]
There is some $ c \ne 0 $ such that $ L'(E / K, 1) = c \cdot \widehat{h}(P_K) $.
\end{theorem}

\begin{corollary}
If $ L'(E / K, 1) \ne 0 $, then $ \rk_\Z E(K) \ge 1 $.
\end{corollary}

\bigskip

\begin{theorem}[Kolyvagin, 1989]
If $ \widehat{h}(P_K) \ne 0 $, then $ E(K)_{/ \tor} = \Z \cdot \tfrac{1}{n}P_K $.
\end{theorem}

\begin{corollary}
If $ L'(E / K, 1) \ne 0 $, then $ \rk_\Z E(K) = 1 $.
\end{corollary}

\bigskip This \emph{almost} proves weak BSD for analytic rank $ \le 1 $!

\end{frame}

\begin{frame}{Application to BSD}

\begin{theorem}[Weak BSD for analytic rank $ \le 1 $]
Assume $ \ord_{s = 1} \ L(E / \Q, s) \le 1 $. Then $ \ord_{s = 1} \ L(E / \Q, s) = \rk_\Z E(\Q) $.
\end{theorem}

\begin{proof}
\renewcommand\qedsymbol{}
Consider the functional equation
$$ \Lambda(E / \Q, s) = \epsilon \cdot \Lambda(E / \Q, 2 - s). $$
Differentiating $ k $ times and evaluating at $ s = 1 $ gives
$$ L^{(k)}(E / \Q, 1) = \epsilon \cdot (-1)^k \cdot L^{(k)}(E / \Q, 1). $$
Then
$$ \ord_{s = 1} \ L(E / \Q, s) =
\begin{cases}
0 & \text{if} \ \epsilon = +, \\
1 & \text{if} \ \epsilon = -.
\end{cases}
$$
Consider cases for $ \epsilon $.
\end{proof}

\end{frame}

\begin{frame}{Application to BSD}

\begin{theorem}[Weak BSD for analytic rank $ \le 1 $]
Assume $ \ord_{s = 1} \ L(E / \Q, s) \le 1 $. Then $ \ord_{s = 1} \ L(E / \Q, s) = \rk_\Z E(\Q) $.
\end{theorem}

\begin{proof}[Proof (for $ \epsilon = - $)]
\underline{Fact}: There is Heegner $ K = \Q(\sqrt{-D}) $ such that $ L(E_D / \Q, 1) \ne 0 $. Then
$$ \ord_{s = 1} \ L(E / K, s) = \underset{1}{\underbrace{\ord_{s = 1} \ L(E / \Q, s)}} + \underset{0}{\underbrace{\ord_{s = 1} \ L(E_D / \Q, s)}}. $$
In particular
$$ L'(E / K, 1) \ne 0 \ \overset{\text{G--Z}}{\implies} \ \widehat{h}(P_K) \ne 0 \ \overset{\text{K}}{\implies} \ E(K)_{/ \tor} = \Z \cdot \tfrac{1}{n}P_K. $$
\underline{Fact}: complex conjugation of $ K $ acts like $ -\epsilon $ on $ E(K)_{/ \tor} $.

\bigskip Thus $ E(\Q)_{/ \tor} = \Z \cdot \tfrac{1}{n}P_K $, so $ \rk_\Z E(\Q) = 1 $.
\end{proof}

\end{frame}

\begin{frame}{The main result}

\begin{theorem}[Kolyvagin, 1989]
If $ \widehat{h}(P_K) \ne 0 $, then $ E(K)_{/ \tor} = \Z \cdot \tfrac{1}{n}P_K $.
\end{theorem}

\begin{theorem}[main result \footnote{Benedict Gross, 1991. \textbf{Kolyvagin's work on modular elliptic curves}}]
Let $ \ell \in \N $ be an odd prime of good reduction such that
$$ \Gal(\Q(E[\ell]) / \Q) \cong \GL_2(\F_\ell), \qquad P_K \notin \ell E(K). $$
Then $ \Sel(K, E[\ell]) = \F_\ell \cdot \delta(P_K) $.
\end{theorem}

\begin{proof}[Proof (of Kolyvagin)]
For any $ \ell \in \N $, there is a short exact sequence
$$ 0 \to E(K) / \ell E(K) \xrightarrow{\delta} \Sel(K, E[\ell]) \to \Sha(K, E)[\ell] \to 0. $$
Choose any $ \ell \in \N $ such that $ K $ and $ \Q(E[\ell]) $ are linearly disjoint over $ \Q $. Then $ E(K)[\ell] = 0 $, so that $ \dim_{\F_\ell} E(K) / \ell E(K) = \rk_\Z E(K) $.
\end{proof}

\end{frame}

\begin{frame}{Selmer structures}

Selmer groups can be defined in general.

\bigskip Let $ M $ be a (non-scalar, simple) self-dual $ \F_\ell[\Gal(L / K)] $-module.

\begin{example}
Let $ M = E[\ell] $.
\begin{itemize}
\item \underline{Fact}: Galois equivariance of $ \ell $-Weil pairing implies $ M $ is non-scalar.
\item \underline{Fact}: surjective $ \ell $-adic representation implies $ M $ is simple.
\end{itemize}
\end{example}

By inflation-restriction, there is a short exact sequence
$$ 0 \to H^1(G_v^\nr, M^{I_v}) \to H^1(K_v, M) \to H^1(I_v, M)^{G_v^\nr} \to 0. $$

\begin{example}
Let $ v \nmid \ell $ have good reduction. Then there is a short exact sequence
$$ 0 \to E(K_v) / \ell E(K_v) \xrightarrow{\delta} H^1(K_v, M) \to H^1(K_v, E)[\ell] \to 0. $$
\end{example}

\end{frame}

\begin{frame}{Selmer structures}

A \textbf{Selmer structure} on $ M $ is an assignment
$$ v \longmapsto H_f^1(K_v, M) \subseteq H^1(K_v, M), $$
such that $ H_f^1(K_v, M) = H^1(G_v^\nr, M^{I_v}) $ for almost all places $ v $ of $ K $.

Its \textbf{singular quotient} $ H_s^1(K_v, M) $ sits in
$$ 0 \to H_f^1(K_v, M) \to H^1(K_v, M) \xrightarrow{(\cdot)^s} H_s^1(K_v, M) \to 0. $$

\begin{example}
\begin{itemize}
\item The \textbf{unramified} Selmer structure has
$$ H_f^1(K_v, M) := H^1(G_v^\nr, M^{I_v}), \qquad H_s^1(K_v, M) := H^1(I_v, M)^{G_v^\nr}. $$
\item The \textbf{geometric} Selmer structure has
$$ H_f^1(K_v, M) := E(K_v) / \ell E(K_v), \qquad H_s^1(K_v, M) := H^1(K_v, E)[\ell]. $$
\end{itemize}
\end{example}

\end{frame}

\begin{frame}{Selmer structures}

There is a localisation map
$$ (\cdot)_v : H^1(K, M) \to H^1(K_v, M). $$
\begin{itemize}
\item The \textbf{classical} Selmer group $ \Sel(K, M) $ sits in
$$ 0 \to \Sel(K, M) \to H^1(K, M) \xrightarrow{\prod_v (\cdot)_v^s} \prod_v H_s^1(K_v, M). $$
\item The \textbf{relaxed} Selmer group $ \Sel^S(K, M) $ sits in
$$ 0 \to \Sel(K, M) \to \Sel^S(K, M) \xrightarrow{\prod_{v \in S} (\cdot)_v^s} \bigoplus_{v \in S} H_s^1(K_v, M). $$
\item The \textbf{restricted} Selmer group $ \Sel_S(K, M) $ sits in
$$ 0 \to \Sel_S(K, M) \to \Sel(K, M) \xrightarrow{\prod_{v \in S} (\cdot)_v} \bigoplus_{v \in S} H_f^1(K_v, M). $$
\end{itemize}

\end{frame}

\begin{frame}{Application of Tate duality}

\begin{proposition}
Let $ S' \subseteq S $ be finite sets of places of $ K $. There is an exact sequence
$$ 0 \longrightarrow \Sel^{S'} \longrightarrow \Sel^S \longrightarrow \bigoplus_{v \in S \setminus S'} H_s^1(K_v, M) \longrightarrow \Sel_{S'}^\vee \longrightarrow \Sel_S^\vee \longrightarrow 0. $$
\vspace{-0.5cm}
\end{proposition}

\begin{proof}
Local Tate duality gives a perfect pairing
$$ H_s^1(K_v, M) \times H_f^1(K_v, M) \to \F_\ell. $$
By the snake lemma, may assume that $ S $ and $ S' $ contain all bad places. The Poitou--Tate exact sequence gives exactness at
$$ \Sel^S \to \bigoplus_{v \in S} H^1(K_v, M) \to \Sel^{S\vee}. $$
Diagram chase.
\end{proof}

\end{frame}

\begin{frame}{Application of Tate duality}

\begin{proposition}
Let $ S' \subseteq S $ be finite sets of places of $ K $. There is an exact sequence
$$ 0 \longrightarrow \Sel^{S'} \longrightarrow \Sel^S \longrightarrow \bigoplus_{v \in S \setminus S'} H_s^1(K_v, M) \longrightarrow \Sel_{S'}^\vee \longrightarrow \Sel_S^\vee \longrightarrow 0. $$
\end{proposition}

\underline{Fact}: complex conjugation of $ K $ respects the exact sequence. Thus
$$ 0 \to \Sel^{S'\pm} \to \Sel^{S\pm} \to \bigoplus_{v \in S \setminus S'} H_s^1(K_v, M)^\pm \to \Sel_{S'}^{\vee\pm} \to \Sel_S^{\vee\pm} \to 0. $$
Specialising to $ S' = \emptyset $ and $ M = E[\ell] $,
$$ 0 \to \coker\left(\Sel^{S\pm} \to \bigoplus_{v \in S} H_s^1(K_v, E[\ell])^\pm\right) \to \Sel^{\vee\pm} \to \Sel_S^{\vee\pm} \to 0. $$
\underline{Idea}: choose appropriate $ S $.

\end{frame}

\begin{frame}{Application of Chebotarev density}

Assume $ M $ is non-scalar and simple.

\bigskip Let $ K(E[\ell]) \subseteq L \subseteq L' $ be finite extensions, and fix $ \sigma \in \Gal(L' / L)^- $. Choose a lift of complex conjugation $ \tau \in \Gal(L' / \Q) $.

\begin{lemma}
There is a finite set $ S $ of inert primes of $ K / \Q $ such that
\begin{enumerate}
\item $ \left(\tfrac{p}{L' / \Q}\right) \sim \sigma\tau $ for all $ p \in S $, and
\item $ \Sel_S^\pm \subseteq H^1(L' / K, E[\ell])^\pm $.
\end{enumerate}
\end{lemma}

\begin{proof}
\begin{itemize}
\item Chebotarev density gives $ S $ satisfying $ 1 $.
\item \underline{Fact}: non-scalar and simple imply $ 2 $.
\end{itemize}
\end{proof}

\underline{Idea}: choose appropriate $ L' / L $ to bound $ \Sel_S^\pm $.

\end{frame}

\begin{frame}{Heegner points of higher conductors}

Both $ \Sel^{S\pm} $ and $ H_s^1(K_v, E[\ell])^\pm $ in
$$ 0 \to \coker\left(\Sel^{S\pm} \to \bigoplus_{v \in S} H_s^1(K_v, E[\ell])^\pm\right) \to \Sel^{\vee\pm} \to \Sel_S^{\vee\pm} \to 0 $$
are generated by some $ c(n) \in H^1(K, E[\ell])^\pm $ indexed by $ n \in \N $.

\bigskip Each $ c(n) $ is generated by a \textbf{Heegner point of conductor $ n $}.
$$
\renewcommand{\arraystretch}{2}
\begin{array}{c|c}
\text{conductor} \ 1 & \text{conductor} \ n \\
\hline
\text{ring of integers} \ \OO_K & \text{order} \ \OO_{K, n} \\
\text{Hilbert class field} \ K^1 & \text{ring class field} \ K^n \\
\text{Heegner point} \ P_1 \in E(K^1) & \text{Heegner point} \ P_n \in E(K^n)
\end{array}
$$

\end{frame}

\begin{frame}{Heegner points of higher conductors}

The Heegner points $ P_n \in E(K^n) $ satisfy ``Euler system'' relations.

\bigskip Consider only the square-free $ n \in \N $ (coprime to $ ND\ell $) such that:
$$ p \mid n \qquad \implies \qquad p \ \text{is inert in} \ K. $$
By class field theory,
$$ \Gal(K^n / K^1) \cong \Cl(\OO_{K, n}) / \Cl(\OO_K) \cong (\OO_K / n)^\times / (\Z / n)^\times. $$
Since $ n $ is square-free,
$$ \Gal(K^n / K^1) \cong \prod_{p \mid n} \Gal(K^p / K^1). $$
Since $ p \mid n $ is inert in $ K $,
$$ \Gal(K^p / K^1) = \Z / (p + 1) \cdot \sigma_p. $$

\end{frame}

\begin{frame}{Heegner points of higher conductors}

\begin{proposition}[AX3]
Let $ n = pq $. Then
\begin{enumerate}
\item $ \sum_{i = 0}^p \sigma_p^iP_{pq} = a_pP_q $ in $ E(K^q) $, and
\item $ \overline{P_{pq}} = \overline{\left(\tfrac{\pp_\qq}{K^q / K}\right)P_q} $ in $ \overline{E}(\F_{\pp_\qq}) $.
\end{enumerate}
\end{proposition}

\begin{proof}[Proof (sketch of 1)]
If $ H_p : \Div(X_0(N)) \to \Div(X_0(N)) $ is the Hecke correspondence, then
$$ \sum_{i = 0}^p \sigma_p^ix_{pq} = H_px_q. $$
By Eichler--Shimura theory, for any $ D \in \Div(X_0(N)) $,
$$ \phi(H_pD) = a_p\phi(D). $$
\end{proof}

\end{frame}

\begin{frame}{Derived Kolyvagin classes}

Given $ P_n \in E(K^n) $, how to derive $ c(n) \in H^1(K, E[\ell]) $?

\bigskip Define a ``trace''
$$ T_n := \sum_{\tau \in T} \tau \in \Z[\Gal(K^n / K)], $$
where $ T $ is a set of coset representatives for $ \Gal(K^n / K^1) \le \Gal(K^n / K) $.

\bigskip Define the \textbf{Kolyvagin derivative}
$$ D_n := \prod_{p \mid n} D_p \in \Z[\Gal(K^n / K^1)], $$
where $ D_p $ is any solution in $ \Z[\Gal(K^n / K)] $ to
$$ (\sigma_p - 1)D_p = p + 1 - T_p. $$
Define $ \PP_n := [T_nD_nP_n] \in E(K^n) / \ell E(K^n) $.

\end{frame}

\begin{frame}{Derived Kolyvagin classes}

\underline{Fact}: By AX3,
\begin{itemize}
\item $ \PP_n $ is fixed by $ G_n := \Gal(K^n / K) $, and
\item $ \PP_n $ lies in the $ \epsilon_n := -\epsilon \cdot (-1)^{\#\{p \mid n\}} $ eigenspace.
\end{itemize}
There is an exact diagram
$$
\begin{tikzcd}[ampersand replacement=\&, column sep=tiny]
\& \& 0 \arrow{d}{\inf_n} \& \& \\
0 \arrow{r} \& H_f^1(K, E[\ell])^{\epsilon_n} \arrow{r}{\delta} \arrow{d} \& H^1(K, E[\ell])^{\epsilon_n} \arrow{r} \arrow{d}{\res_n} \& H_s^1(K, E[\ell])^{\epsilon_n} \arrow{r} \arrow{d} \& 0 \\
0 \arrow{r} \& H_f^1(K^n, E[\ell])^{G_n\epsilon_n} \arrow{r}[swap]{\delta_n} \& H^1(K^n, E[\ell])^{G_n\epsilon_n} \arrow{r} \arrow{d}{\tra_n} \& H_s^1(K^n, E[\ell])^{G_n\epsilon_n} \& \\
\& \& 0. \& \&
\end{tikzcd}
$$
Define $ c(n) \in H^1(K, E[\ell])^{\epsilon_n} $ by $ \res_n(c(n)) = \delta_n(\PP_n) $.

\end{frame}

\begin{frame}{Derived Kolyvagin classes}

\begin{lemma}
\begin{enumerate}
\item If $ v \nmid n $, then $ c(n)_v^s = 0 $ (\emph{i.e.} $ c(n) \in \Sel^{\{p \mid n\}\epsilon_n} $).
\item If $ v \mid n $, then $ c(n)_v^s = 0 $ if and only if $ \PP_{n / v} \in \ell E(K_v) $.
\end{enumerate}
\end{lemma}

\begin{proof}[Proof (sketch of 1)]
Assume $ v \nmid \ell $ has good reduction. Then $ K_v^n / K_v $ is unramified, so
$$
\begin{tikzcd}[ampersand replacement=\&]
0 \arrow{r} \& H_f^1(K_v, E[\ell]) \arrow{r} \arrow{d} \& H^1(K_v, E[\ell]) \arrow{r}{(\cdot)^s} \arrow{d}{\res_n} \& \Hom(I_v, E[\ell]) \arrow{d} \arrow[dash]{d}{\sim} \\
0 \arrow{r} \& H_f^1(K_v^n, E[\ell]) \arrow{r}[swap]{\delta_n} \& H^1(K_v^n, E[\ell]) \arrow{r}[swap]{(\cdot)^s} \& \Hom(I_v, E[\ell]).
\end{tikzcd}
$$
Thus $ (\res_n(c(n)_v))^s = 0 $ by exactness.
\end{proof}

\end{frame}

\begin{frame}{Computing the Selmer group}

Compute $ \Sel^\epsilon $ and $ \Sel^{-\epsilon} $ separately.

\bigskip Use the short exact sequence
$$ 0 \to \coker\left(\Sel^{S\pm} \to \bigoplus_{p \in S} H_s^1(K_p, E[\ell])^\pm\right) \to \Sel^\pm \to \Sel_S^\pm \to 0. $$

\bigskip Restricted:
\begin{itemize}
\item Choose $ L' / L $ to get $ S $ such that $ \Sel_S^\pm \subseteq H^1(L' / K, E[\ell])^\pm $.
\item Compute $ H^1(L' / K, E[\ell])^\pm $.
\end{itemize}

\bigskip Relaxed:
\begin{itemize}
\item \underline{Fact}: each $ H_s^1(K_p, E[\ell])^\pm $ is one-dimensional.
\item Show $ c(n) \in \Sel^{S\epsilon_n} $ is non-zero in $ H_s^1(K_p, E[\ell]) $ for some $ n $.
\end{itemize}

\end{frame}

\begin{frame}{Computing the Selmer group}

Compute $ \Sel^\epsilon $.

\bigskip Let $ L := K(E[\ell]) $ and $ L' := K(E[\ell], \tfrac{1}{\ell}P_K) $. Get $ S $ such that
$$ \Sel_S^\epsilon \subseteq H^1(L' / K, E[\ell])^\epsilon \cong \underset{-\epsilon}{\underbrace{\F_\ell \cdot \delta(P_K)}}. $$
By Frobenius computations,
$$ \forall p \in S, \qquad c(p) \in \Sel^{S\epsilon}, \qquad c(p)_p^s \ne 0. $$
Thus
$$ 0 \to \underset{0}{\underbrace{\coker\left(\Sel^{S\epsilon} \to \bigoplus_{p \in S} H_s^1(K_p, E[\ell])^\epsilon\right)}} \to \Sel^\epsilon \to \underset{0}{\underbrace{\Sel_S^\epsilon}} \to 0. $$

\end{frame}

\begin{frame}{Computing the Selmer group}

Compute $ \Sel^{-\epsilon} $. Fix $ p \in S $.

\bigskip Let $ L := K(E[\ell], \tfrac{1}{\ell}P_K) $ and $ L' := \ker(G_L \xrightarrow{c(p)} E[\ell]) $. Get $ S' $ such that
$$ \Sel_{S'}^{-\epsilon} \subseteq H^1(L' / K, E[\ell])^{-\epsilon} \cong \underset{-\epsilon}{\underbrace{\F_\ell \cdot \delta(P_K)}} \oplus \underset{\epsilon}{\underbrace{\F_\ell \cdot c(p)}}. $$
By Frobenius computations,
$$ \forall q \in S', \qquad c(pq) \in \Sel^{S'-\epsilon}, \qquad c(pq)_q^s \ne 0. $$
Thus
$$ 0 \to \underset{0}{\underbrace{\coker\left(\Sel^{S'-\epsilon} \to \bigoplus_{q \in S'} H_s^1(K_q, E[\ell])^{-\epsilon}\right)}} \to \Sel^{-\epsilon} \to \underset{\subseteq \F_\ell \cdot \delta(P_K)}{\underbrace{\Sel_{S'}^{-\epsilon}}} \to 0. $$

\end{frame}

\end{document}