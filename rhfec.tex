\ifx\type\undefined
  \documentclass[10pt, t]{beamer}
  \setbeamertemplate{footline}[page number]
\else
  \documentclass[10pt]{article}
  \usepackage[margin=1in]{geometry}
\fi

\usepackage{amsmath}
\usepackage{amssymb}
\usepackage{amsthm}
\usepackage{bbm}
\usepackage{cancel}
\usepackage{listings}
\usepackage{mathrsfs}
\usepackage{multirow}
\usepackage{soul}
\usepackage{stmaryrd}
\usepackage{tikz}
\usepackage{tikz-cd}
\usepackage{wrapfig}

\newtheorem*{algorithm}{Algorithm}
\newtheorem*{assumptions}{Assumptions}
\newtheorem*{conjecture}{Conjecture}
\newtheorem*{consequences}{Consequences}
\newtheorem*{exercise}{Exercise}
\newtheorem*{formalisation}{Formalisation}
\newtheorem*{proposition}{Proposition}
\newtheorem*{question}{Question}
\newtheorem*{remark}{Remark}

\ifx\type\undefined\else
  \newtheorem*{definition}{Definition}
  \newtheorem*{example}{Example}
  \newtheorem*{lemma}{Lemma}
  \newtheorem*{theorem}{Theorem}
\fi

\definecolor{keywordcolor}{rgb}{0.7, 0.1, 0.1}
\definecolor{tacticcolor}{rgb}{0.0, 0.1, 0.6}
\definecolor{commentcolor}{rgb}{0.4, 0.4, 0.4}
\definecolor{symbolcolor}{rgb}{0.0, 0.1, 0.6}
\definecolor{sortcolor}{rgb}{0.1, 0.5, 0.1}
\definecolor{attributecolor}{rgb}{0.7, 0.1, 0.1}
\def\lstlanguagefiles{lstlean.tex}
\lstset{language=lean}

\newcommand\A{\mathbb{A}}
\newcommand\C{\mathbb{C}}
\newcommand\F{\mathbb{F}}
\newcommand\G{\mathbb{G}}
\renewcommand\H{\mathbb{H}}
\newcommand\I{\mathbb{I}}
\newcommand\N{\mathbb{N}}
\renewcommand\P{\mathbb{P}}
\newcommand\Q{\mathbb{Q}}
\newcommand\R{\mathbb{R}}
\newcommand\Z{\mathbb{Z}}

\renewcommand\AA{\mathcal{A}}
\newcommand\BB{\mathcal{B}}
\newcommand\CC{\mathcal{C}}
\newcommand\DD{\mathcal{D}}
\newcommand\EE{\mathcal{E}}
\newcommand\FF{\mathcal{F}}
\newcommand\GG{\mathcal{G}}
\newcommand\HH{\mathcal{H}}
\newcommand\II{\mathcal{I}}
\newcommand\LL{\mathcal{L}}
\newcommand\MM{\mathcal{M}}
\newcommand\NN{\mathcal{N}}
\newcommand\OO{\mathcal{O}}
\newcommand\PP{\mathcal{P}}
\newcommand\RR{\mathcal{R}}
\renewcommand\SS{\mathcal{S}}
\newcommand\TT{\mathcal{T}}
\newcommand\XX{\mathcal{X}}

\renewcommand\aa{\mathfrak{a}}
\newcommand\cc{\mathfrak{c}}
\newcommand\dd{\mathfrak{d}}
\newcommand\ff{\mathfrak{f}}
\renewcommand\gg{\mathfrak{g}}
\newcommand\mm{\mathfrak{m}}
\newcommand\pp{\mathfrak{p}}
\newcommand\qq{\mathfrak{q}}
\renewcommand\ss{\mathfrak{s}}

\newcommand\LLL{\mathscr{L}}

\newcommand\ab{\mathrm{ab}}
\newcommand\Ab{\mathbf{Ab}}
\newcommand\Alg{\mathbf{Alg}}
\newcommand\Aff{\mathbf{Aff}}
\newcommand\Aut{\operatorname{Aut}}
\newcommand\Az{\mathrm{Az}}
\newcommand\Br{\operatorname{Br}}
\newcommand\BSD{\operatorname{BSD}}
\newcommand\ch{\operatorname{char}}
\newcommand\Cl{\operatorname{Cl}}
\newcommand\coker{\operatorname{coker}}
\newcommand\cris{\mathrm{cris}}
\renewcommand\d{\mathrm{d}}
\newcommand\Div{\operatorname{Div}}
\newcommand\dR{\mathrm{dR}}
\newcommand\EN{\operatorname{EN}}
\newcommand\End{\operatorname{End}}
\newcommand\ES{\operatorname{ES}}
\newcommand\et{\mathrm{\acute{e}t}}
\newcommand\Et{\mathbf{\acute{E}t}}
\newcommand\Ext{\operatorname{Ext}}
\newcommand\Fr{\operatorname{Fr}}
\newcommand\Frac{\operatorname{Frac}}
\newcommand\Gal{\operatorname{Gal}}
\newcommand\GL{\operatorname{GL}}
\newcommand\Gr{\mathrm{Gr}}
\newcommand\Hom{\operatorname{Hom}}
\newcommand\HT{\mathrm{HT}}
\newcommand\id{\operatorname{id}}
\newcommand\im{\operatorname{im}}
\newcommand\Ind{\operatorname{Ind}}
\renewcommand\inf{\operatorname{inf}}
\newcommand\inv{\operatorname{inv}}
\newcommand\Irr{\operatorname{Irr}}
\newcommand\Jac{\operatorname{Jac}}
\newcommand\lcm{\operatorname{lcm}}
\newcommand\Mat{\operatorname{Mat}}
\newcommand\Mod{\mathbf{Mod}}
\newcommand\Nm{\operatorname{Nm}}
\newcommand\nr{\mathrm{nr}}
\newcommand\NS{\operatorname{NS}}
\newcommand\Ob{\operatorname{Ob}}
\newcommand\ord{\operatorname{ord}}
\newcommand\op{\mathrm{op}}
\newcommand\PGL{\operatorname{PGL}}
\newcommand\Pic{\operatorname{Pic}}
\newcommand\Prob{\operatorname{Prob}}
\newcommand\Proj{\operatorname{Proj}}
\newcommand\PSh{\mathbf{PSh}}
\newcommand\Reg{\operatorname{Reg}}
\newcommand\res{\operatorname{res}}
\newcommand\rk{\operatorname{rk}}
\newcommand\Sch{\mathbf{Sch}}
\newcommand\Sel{\operatorname{Sel}}
\newcommand\Set{\mathbf{Set}}
\newcommand\sgn{\operatorname{sgn}}
\newcommand\Sh{\mathbf{Sh}}
\newcommand\SL{\operatorname{SL}}
\newcommand\Spec{\operatorname{Spec}}
\newcommand\supp{\operatorname{supp}}
\newcommand\Tam{\operatorname{Tam}}
\newcommand\Top{\mathbf{Top}}
\newcommand\tor{\operatorname{tor}}
\newcommand\tr{\operatorname{tr}}
\newcommand\tra{\operatorname{tra}}
\newcommand\WC{\operatorname{WC}}

\DeclareFontFamily{U}{wncyr}{}
\DeclareFontShape{U}{wncyr}{m}{n}{<->wncyr10}{}
\DeclareSymbolFont{cyr}{U}{wncyr}{m}{n}
\DeclareMathSymbol{\Sha}{\mathord}{cyr}{"58}

\newcommand{\function}[5][]{
  \if &#1&
    \begin{array}{rcl}
      #2 & \longrightarrow & #3 \\
      #4 & \longmapsto     & #5
    \end{array}
  \else
    \begin{array}{rcrcl}
      #1 & : & #2 & \longrightarrow & #3 \\
         &   & #4 & \longmapsto     & #5
    \end{array}
  \fi
}

\newcommand{\functions}[7][]{
  \if &#1&
    \begin{array}{rcl}
      #2 & \longrightarrow & #3 \\
      #4 & \longmapsto     & #5 \\
      #6 & \longmapsto     & #7 \\
    \end{array}
  \else
    \begin{array}{rcrcl}
      #1 & : & #2 & \longrightarrow & #3 \\
         &   & #4 & \longmapsto     & #5 \\
         &   & #6 & \longmapsto     & #7
    \end{array}
  \fi
}
\title{Rank heuristics for elliptic curves \footnote{partially based on the VaNTAGe seminar on `Heuristics for the arithmetic of elliptic curves' by Bjorn Poonen on 1 September 2020}}
\subtitle{Part III Seminar Series}
\author{David Kurniadi Angdinata}
\institute{University of Cambridge}
\date{Friday, 4 December 2020}

\begin{document}

\frame\maketitle

\begin{frame}{Elliptic curves}

Let $ E $ be an elliptic curve over a number field $ K $.

\bigskip

\begin{theorem}[Mordell--Weil]
$ E(K) $ is a finitely generated abelian group of the form
$$ E(K) \cong \tor(E / K) \oplus \Z^{\rk(E / K)}. $$
\end{theorem}

\bigskip The \textbf{torsion subgroup} $ \tor(E / K) $ is effectively computable.

\begin{theorem}[Lutz--Nagell]
If $ (x, y) \in \tor(E / \Q) $, then $ y \in \Z $ and either $ y = 0 $ or $ y^2 \mid \Delta(E / \Q) $.
\end{theorem}

\begin{theorem}[Mazur, Kamienny, Merel]
There are finitely many possibilities for $ \tor(E / K) $.
\end{theorem}

\end{frame}

\begin{frame}{Elliptic curves}

Let $ E $ be an elliptic curve over a number field $ K $.

\bigskip

\begin{theorem}[Mordell--Weil]
$ E(K) $ is a finitely generated abelian group of the form
$$ E(K) \cong \tor(E / K) \oplus \Z^{\rk(E / K)}. $$
\end{theorem}

\bigskip The \textbf{rank} $ \rk(E / K) $ is computationally harder and more mysterious.

\begin{conjecture}[Birch--Swinnerton-Dyer]
If $ K = \Q $, then $ \ord_{s = 1} L(E, s) = \rk(E / \Q) $.
\end{conjecture}

\begin{theorem}[Kolyvagin]
BSD holds for modular elliptic curves with analytic rank zero and one.
\end{theorem}

\end{frame}

\begin{frame}{Rank distribution conjecture}

How is the rank distributed?

\bigskip Consider the set $ \EE(\Q) $ of unique minimal representatives of isomorphism classes of elliptic curves over $ \Q $, ordered by the height function
$$ H(E : y^2 = x^3 + Ax + B) = \max(4|A|^3, 27|B|^2). $$

\begin{conjecture}[Rank distribution]
The average rank of $ \EE(\Q) $ is $ \tfrac{1}{2} $.
\end{conjecture}

\begin{theorem}[Bhargava--Shankar 2015]
The average rank of $ \EE(\Q) $ is at most $ \tfrac{7}{6} $.
\end{theorem}

\bigskip Combining these shows that BSD holds for a positive proportion of $ \EE(\Q) $ (Kolyvagin 1989, Breuil--Conrad--Diamond--Taylor 2001, Nekov\'a\v r 2009, Dokchitser--Dokchitser 2010, Skinner--Urban 2015).

\end{frame}

\begin{frame}{Rank boundedness conjecture}

Is the rank bounded? Probably not...

\bigskip

\begin{conjecture}[Rank boundedness]
There are $ E \in \EE(\Q) $ of arbitrarily large rank.
\end{conjecture}

\begin{theorem}[Shafarevich--Tate 1967, Ulmer 2002]
There are $ E \in \EE(\F_p(T)) $ of arbitrarily large rank.
\end{theorem}

\begin{theorem}[Elkies 2006]
There is $ E \in \EE(\Q) $ with rank at least $ 28 $.
\end{theorem}

\begin{theorem}[Elkies--Klagsbrun 2020]
There is $ E \in \EE(\Q) $ with rank exactly $ 20 $.
\end{theorem}

\bigskip Many proponents (Cassels 1966, Tate 1974, Mestre 1982, Silverman 1986, Brumer 1992, Ulmer 2002, Farmer--Gonek--Hughes 2007).

\end{frame}

\begin{frame}{Rank boundedness conjecture}

Is the rank bounded? Probably!

\bigskip

\begin{conjecture}[Poonen et al \footnote{B. Poonen and E. Rains. `Random maximal isotropic subspaces and Selmer groups'. In: J. Amer. Math. Soc 25 (2012)} \footnote{M. Bhargava, D. Kane, H. Lenstra, B. Poonen and E. Rains. `Modelling the distribution of ranks, Selmer groups, and Shafarevich--Tate groups of elliptic curves'. In: Camb. J. Math. 3 (2015)} \footnote{J. Park, B. Poonen, J. Voight and M. Wood. `A heuristic for boundedness of ranks of elliptic curves'. In: J. Eur. Math. Soc (2019)}]
There are finitely many $ E \in \EE(\Q) $ with rank greater than $ 21 $.
\end{conjecture}

\begin{itemize}
\item Model $ p^e $-Selmer groups using intersection of quadratic submodules.
\item Model Tate--Shafarevich groups using matrices with a fixed rank.
\item Model the Mordell--Weil rank using matrices without fixing the rank.
\end{itemize}

\bigskip A few others also predict boundedness (N\'eron 1950, Honda 1960, Rubin--Silverberg 2000, Granville 2006, Watkins 2015).

\end{frame}

\begin{frame}{The Selmer and Tate--Shafarevich groups}

Multiplication by $ n \in \N^+ $ gives
$$ 0 \to E[n] \to E \xrightarrow{[n]} E \to 0. $$
Applying $ \Gal(\overline{K} / K) $ cohomology gives
$$
\begin{tikzcd}[ampersand replacement=\&, column sep=tiny]
0 \arrow{r} \& E(K)[n] \arrow{r} \& E(K) \arrow{r} \& E(K) \arrow[in=180, out=0]{dll}[swap]{\delta} \& \\
\& H^1(K, E[n]) \arrow{r} \& H^1(K, E) \arrow{r} \& H^1(K, E) \arrow{r} \& \dots.
\end{tikzcd}
$$
Truncating at $ H^1(K, E[n]) $ gives a short exact sequence
$$ 0 \to E(K) / n \to H^1(K, E[n]) \to H^1(K, E)[n] \to 0. $$
Similarly, there are short exact sequences
$$ 0 \to E(K_v) / n \to H^1(K_v, E[n]) \to H^1(K_v, E)[n] \to 0. $$

\end{frame}

\begin{frame}{The Selmer and Tate--Shafarevich groups}

There is a row-exact commutative diagram
$$
\begin{tikzcd}[ampersand replacement=\&, column sep=tiny]
0 \arrow{r} \& E(K) / n \arrow{r} \arrow{d} \& H^1(K, E[n]) \arrow{r} \arrow{d}[swap]{\lambda} \arrow[dashed]{dr}{\sigma} \& H^1(K, E)[n] \arrow{r} \arrow{d}{\tau[n]} \& 0 \\
0 \arrow{r} \& \displaystyle\prod_v E(K_v) / n \arrow{r}[swap]{\kappa} \& \displaystyle\prod_v H^1(K_v, E[n]) \arrow{r} \& \displaystyle\prod_v H^1(K_v, E)[n] \arrow{r} \& 0.
\end{tikzcd}
$$
The \textbf{$ n $-Selmer group} is
$$ \Sel_n(E / K) = \ker(\sigma : H^1(K, E[n]) \to \textstyle\prod_v H^1(K_v, E)[n]). $$
The \textbf{Tate--Shafarevich group} is
$$ \Sha(E / K) = \ker(\tau : H^1(K, E) \to \textstyle\prod_v H^1(K_v, E)). $$
There is an exact sequence
$$ 0 \to E(K) / n \to \Sel_n(E / K) \to \Sha(E / K)[n] \to 0. $$

\end{frame}

\begin{frame}{Modelling $ p^e $-Selmer groups}

\begin{theorem}
For almost all $ E \in \EE(K) $, the $ p^e $-Selmer group $ \Sel_{p^e}(E / K) $ is the intersection of two maximal totally isotropic direct summands in a non-degenerate quadratic $ \Z / p^e $-module of infinite rank.
\end{theorem}

\bigskip Consider $ (\Z / p^e)^{2n} $, equipped with hyperbolic quadratic form
$$ (x_1, \dots, x_n, y_1, \dots, y_n) \mapsto \sum_{i = 1}^n x_iy_i, $$
with two MTIDS's $ (\Z / p^e)^n \oplus 0^n $ and $ 0^n \oplus (\Z / p^e)^n $.

\bigskip The result was known for a finite-dimensional vector space over $ \F_2 $ (Colliot-Th\'el\`ene--Skorobogatov--Swinnerton-Dyer 2002).

\end{frame}

\begin{frame}{Modelling $ p^e $-Selmer groups}

By the first isomorphism theorem,
$$ \Sel_n(E / K) / \ker\lambda \cong \im\kappa \cap \im\lambda. $$

\begin{theorem}
For almost all $ E \in \EE(K) $, the $ p^e $-Selmer group $ \Sel_{p^e}(E / K) $ is the intersection of two maximal totally isotropic direct summands in a non-degenerate quadratic $ \Z / p^e $-module of infinite rank.
\end{theorem}

\begin{conjecture}
The distribution of $ \Sel_{p^e}(E / \Q) $ coincides with the distribution of $ S_1 \cap S_2 $ for two randomly chosen MTIDS's $ S_1, S_2 \subseteq (\Z / p^e)^{2n} $ as $ n \to \infty $.
\end{conjecture}

\begin{itemize}
\item Variant for function fields is known (Feng--Landesman--Rains 2020).
\item Variant for quadratic twist families over $ \Q $ is known for $ p^e = 2 $ (Heath-Brown 1994, Swinnerton-Dyer 2008, Kane 2013).
\item Average of $ \#(S_1 \cap S_2) $ is $ \sigma_1(p^e) $, and average of $ \#\Sel_{p^e}(E / \Q) $ is $ \sigma_1(p^e) $ for $ p^e \le 5 $ (Bhargava--Shankar 2013-2015).
\end{itemize}

\end{frame}

\begin{frame}{Modelling short exact sequences}

Recall that
$$ 0 \to E(K) / n \to \Sel_n(E / K) \to \Sha(E / K)[n] \to 0. $$
Setting $ n = p^e $ and taking direct limits gives
$$ 0 \to E(K) \otimes \Q_p / \Z_p \to \varinjlim_e \Sel_{p^e}(E / K) \to \Sha(E / K)[p^\infty] \to 0. $$
Randomly choosing two MTIDS's $ S_1, S_2 \subseteq (\Z_p)^{2n} $ gives
$$ 0 \to \RR \to \SS \to \TT \to 0, $$
where $ \RR = (S_1 \cap S_2) \otimes \Q_p / \Z_p $ and $ \SS = (S_1 \otimes \Q_p / \Z_p) \cap (S_2 \otimes \Q_p / \Z_p) $.
\begin{itemize}
\item Both $ \varinjlim_e \Sel_{p^e}(E / K) $ and $ \SS $ are compatible with $ p^e $-parts.
\item Both $ \Sha(E / K)[p^\infty] $ and $ \TT $ are finite with an alternating pairing.
\item Both $ E(K) \otimes \Q_p / \Z_p $ and $ \RR $ satisfy the rank distribution conjecture.
\item Variant for quadratic twist families is known for $ p = 2 $ (Smith 2020).
\end{itemize}

\end{frame}

\begin{frame}{Modelling Tate--Shafarevich groups}

The rank distribution conjecture gives
$$ \P(\rk_{\Z_p}(S_1 \cap S_2) = 0) = \P(\rk_{\Z_p}(S_1 \cap S_2) = 1) = \dfrac{1}{2}. $$
If $ r \ge 2 $, then
$$ \{S_1, S_2 \subseteq \Z_p^{2n} : \rk_{\Z_p}(S_1 \cap S_2) = r\} $$
has measure zero as $ n \to \infty $.

\bigskip Instead choose $ M $ randomly from
$$ \{M \in \Mat_n \Z_p : M^\intercal = -M, \ \rk_{\Z_p}(\ker M) = r\}, \qquad n \equiv r \mod 2, $$
and let $ n \to \infty $. Use distribution of $ \tor(\coker M) $ to model $ \TT $.
\begin{itemize}
\item Coincides with original $ \Z_p^{2n} $ distribution for $ \TT $ for rank zero and one.
\item Coincides with Delaunay's distribution for $ \Sha(E / \Q)[p^\infty] $ (Delaunay--Jouhet 2000-2014).
\end{itemize}

\end{frame}

\begin{frame}{Modelling ranks}

How to model an elliptic curve $ E $ over $ \Q $ of height $ h $?
\begin{itemize}
\item Choose functions $ X : \N \to \R $ and $ Y : \N \to \R $ such that
$$ X(x)^{Y(x)} = x^{\tfrac{1}{12} + o(1)}, \qquad x \to \infty. $$
\item Choose $ n $ randomly from $ \{\lceil Y(h)\rceil, \lceil Y(h)\rceil + 1\} $.
\item Choose $ M $ randomly from
$$ \{M \in \Mat_n \Z : M^\intercal = -M, \ M_{ij} \le X(h)\}. $$
\item Model $ \Sha(E / \Q) $ by $ \tor(\coker M) $ and $ \rk(E / \Q) $ by $ \rk_\Z(\ker M) $.
\end{itemize}
Conditions are chosen such that the average size of
$$ \#\coker_0' M =
\begin{cases}
\#\tor(\coker M) & \text{if} \ \rk_\Z(\ker M) = 0, \\
0 & \text{if} \ \rk_\Z(\ker M) > 0,
\end{cases}
$$
is $ h^{1 / 12 + o(1)} $. The same is predicted for $ \Sha(E / \Q) $ by strong BSD.

\end{frame}

\begin{frame}{Modelling ranks}

Denote the model for $ \rk(E / \Q) $ by $ \rk'(E / \Q) $.

\bigskip

\begin{theorem}[Poonen et al]
The following hold with probability $ 1 $.
\begin{align*}
\#\{E \in \EE(\Q) : H(E) \le h, \ \rk'(E / \Q) = 0\} & = h^{20 / 24 + o(1)} \\
\#\{E \in \EE(\Q) : H(E) \le h, \ \rk'(E / \Q) = 1\} & = h^{20 / 24 + o(1)} \\
\#\{E \in \EE(\Q) : H(E) \le h, \ \rk'(E / \Q) \ge 2\} & = h^{19 / 24 + o(1)} \\
& \vdots \\
\#\{E \in \EE(\Q) : H(E) \le h, \ \rk'(E / \Q) \ge 20\} & = h^{1 / 24 + o(1)} \\
\#\{E \in \EE(\Q) : H(E) \le h, \ \rk'(E / \Q) \ge 21\} & \le h^{o(1)} \\
\#\{E \in \EE(\Q) : \rk'(E / \Q) > 21\} & \ \text{is finite}.
\end{align*}
\end{theorem}

\end{frame}

\end{document}