\ifx\type\undefined
  \documentclass[10pt, t]{beamer}
  \setbeamertemplate{footline}[page number]
\else
  \documentclass[10pt]{article}
  \usepackage[margin=1in]{geometry}
\fi

\usepackage{amsmath}
\usepackage{amssymb}
\usepackage{amsthm}
\usepackage{bbm}
\usepackage{cancel}
\usepackage{listings}
\usepackage{mathrsfs}
\usepackage{multirow}
\usepackage{soul}
\usepackage{stmaryrd}
\usepackage{tikz}
\usepackage{tikz-cd}
\usepackage{wrapfig}

\newtheorem*{algorithm}{Algorithm}
\newtheorem*{assumptions}{Assumptions}
\newtheorem*{conjecture}{Conjecture}
\newtheorem*{consequences}{Consequences}
\newtheorem*{exercise}{Exercise}
\newtheorem*{formalisation}{Formalisation}
\newtheorem*{proposition}{Proposition}
\newtheorem*{question}{Question}
\newtheorem*{remark}{Remark}

\ifx\type\undefined\else
  \newtheorem*{definition}{Definition}
  \newtheorem*{example}{Example}
  \newtheorem*{lemma}{Lemma}
  \newtheorem*{theorem}{Theorem}
\fi

\definecolor{keywordcolor}{rgb}{0.7, 0.1, 0.1}
\definecolor{tacticcolor}{rgb}{0.0, 0.1, 0.6}
\definecolor{commentcolor}{rgb}{0.4, 0.4, 0.4}
\definecolor{symbolcolor}{rgb}{0.0, 0.1, 0.6}
\definecolor{sortcolor}{rgb}{0.1, 0.5, 0.1}
\definecolor{attributecolor}{rgb}{0.7, 0.1, 0.1}
\def\lstlanguagefiles{lstlean.tex}
\lstset{language=lean}

\newcommand\A{\mathbb{A}}
\newcommand\C{\mathbb{C}}
\newcommand\F{\mathbb{F}}
\newcommand\G{\mathbb{G}}
\renewcommand\H{\mathbb{H}}
\newcommand\I{\mathbb{I}}
\newcommand\N{\mathbb{N}}
\renewcommand\P{\mathbb{P}}
\newcommand\Q{\mathbb{Q}}
\newcommand\R{\mathbb{R}}
\newcommand\Z{\mathbb{Z}}

\renewcommand\AA{\mathcal{A}}
\newcommand\BB{\mathcal{B}}
\newcommand\CC{\mathcal{C}}
\newcommand\DD{\mathcal{D}}
\newcommand\EE{\mathcal{E}}
\newcommand\FF{\mathcal{F}}
\newcommand\GG{\mathcal{G}}
\newcommand\HH{\mathcal{H}}
\newcommand\II{\mathcal{I}}
\newcommand\LL{\mathcal{L}}
\newcommand\MM{\mathcal{M}}
\newcommand\NN{\mathcal{N}}
\newcommand\OO{\mathcal{O}}
\newcommand\PP{\mathcal{P}}
\newcommand\RR{\mathcal{R}}
\renewcommand\SS{\mathcal{S}}
\newcommand\TT{\mathcal{T}}
\newcommand\XX{\mathcal{X}}

\renewcommand\aa{\mathfrak{a}}
\newcommand\cc{\mathfrak{c}}
\newcommand\dd{\mathfrak{d}}
\newcommand\ff{\mathfrak{f}}
\renewcommand\gg{\mathfrak{g}}
\newcommand\mm{\mathfrak{m}}
\newcommand\pp{\mathfrak{p}}
\newcommand\qq{\mathfrak{q}}
\renewcommand\ss{\mathfrak{s}}

\newcommand\LLL{\mathscr{L}}

\newcommand\ab{\mathrm{ab}}
\newcommand\Ab{\mathbf{Ab}}
\newcommand\Alg{\mathbf{Alg}}
\newcommand\Aff{\mathbf{Aff}}
\newcommand\Aut{\operatorname{Aut}}
\newcommand\Az{\mathrm{Az}}
\newcommand\Br{\operatorname{Br}}
\newcommand\BSD{\operatorname{BSD}}
\newcommand\ch{\operatorname{char}}
\newcommand\Cl{\operatorname{Cl}}
\newcommand\coker{\operatorname{coker}}
\newcommand\cris{\mathrm{cris}}
\renewcommand\d{\mathrm{d}}
\newcommand\Div{\operatorname{Div}}
\newcommand\dR{\mathrm{dR}}
\newcommand\EN{\operatorname{EN}}
\newcommand\End{\operatorname{End}}
\newcommand\ES{\operatorname{ES}}
\newcommand\et{\mathrm{\acute{e}t}}
\newcommand\Et{\mathbf{\acute{E}t}}
\newcommand\Ext{\operatorname{Ext}}
\newcommand\Fr{\operatorname{Fr}}
\newcommand\Frac{\operatorname{Frac}}
\newcommand\Gal{\operatorname{Gal}}
\newcommand\GL{\operatorname{GL}}
\newcommand\Gr{\mathrm{Gr}}
\newcommand\Hom{\operatorname{Hom}}
\newcommand\HT{\mathrm{HT}}
\newcommand\id{\operatorname{id}}
\newcommand\im{\operatorname{im}}
\newcommand\Ind{\operatorname{Ind}}
\renewcommand\inf{\operatorname{inf}}
\newcommand\inv{\operatorname{inv}}
\newcommand\Irr{\operatorname{Irr}}
\newcommand\Jac{\operatorname{Jac}}
\newcommand\lcm{\operatorname{lcm}}
\newcommand\Mat{\operatorname{Mat}}
\newcommand\Mod{\mathbf{Mod}}
\newcommand\Nm{\operatorname{Nm}}
\newcommand\nr{\mathrm{nr}}
\newcommand\NS{\operatorname{NS}}
\newcommand\Ob{\operatorname{Ob}}
\newcommand\ord{\operatorname{ord}}
\newcommand\op{\mathrm{op}}
\newcommand\PGL{\operatorname{PGL}}
\newcommand\Pic{\operatorname{Pic}}
\newcommand\Prob{\operatorname{Prob}}
\newcommand\Proj{\operatorname{Proj}}
\newcommand\PSh{\mathbf{PSh}}
\newcommand\Reg{\operatorname{Reg}}
\newcommand\res{\operatorname{res}}
\newcommand\rk{\operatorname{rk}}
\newcommand\Sch{\mathbf{Sch}}
\newcommand\Sel{\operatorname{Sel}}
\newcommand\Set{\mathbf{Set}}
\newcommand\sgn{\operatorname{sgn}}
\newcommand\Sh{\mathbf{Sh}}
\newcommand\SL{\operatorname{SL}}
\newcommand\Spec{\operatorname{Spec}}
\newcommand\supp{\operatorname{supp}}
\newcommand\Tam{\operatorname{Tam}}
\newcommand\Top{\mathbf{Top}}
\newcommand\tor{\operatorname{tor}}
\newcommand\tr{\operatorname{tr}}
\newcommand\tra{\operatorname{tra}}
\newcommand\WC{\operatorname{WC}}

\DeclareFontFamily{U}{wncyr}{}
\DeclareFontShape{U}{wncyr}{m}{n}{<->wncyr10}{}
\DeclareSymbolFont{cyr}{U}{wncyr}{m}{n}
\DeclareMathSymbol{\Sha}{\mathord}{cyr}{"58}

\newcommand{\function}[5][]{
  \if &#1&
    \begin{array}{rcl}
      #2 & \longrightarrow & #3 \\
      #4 & \longmapsto     & #5
    \end{array}
  \else
    \begin{array}{rcrcl}
      #1 & : & #2 & \longrightarrow & #3 \\
         &   & #4 & \longmapsto     & #5
    \end{array}
  \fi
}

\newcommand{\functions}[7][]{
  \if &#1&
    \begin{array}{rcl}
      #2 & \longrightarrow & #3 \\
      #4 & \longmapsto     & #5 \\
      #6 & \longmapsto     & #7 \\
    \end{array}
  \else
    \begin{array}{rcrcl}
      #1 & : & #2 & \longrightarrow & #3 \\
         &   & #4 & \longmapsto     & #5 \\
         &   & #6 & \longmapsto     & #7
    \end{array}
  \fi
}
\title{Congruences of twisted L-values}
\subtitle{What am I doing at the moment}
\author{David Kurniadi Angdinata}
\institute{University College London}
\date{Thursday, 19 October 2023}

\begin{document}

\frame\maketitle

\begin{frame}[c]{Overview}

Notation:
\begin{itemize}
\item $ N $ is an integer
\item $ p $ and $ q $ are odd primes such that $ p \nmid N $ (and $ p \equiv 1 \mod q $)
\item $ E $ is an elliptic curve over $ \Q $ of conductor $ N $ (with analytic rank zero)
\item $ \chi $ is a Dirichlet character of conductor $ p $ and order $ q $
\end{itemize}

\bigskip Outline:
\begin{itemize}
\item Twisted L-values
\item Modular symbols
\item Arithmetic consequences
\item Asymptotic distribution
\end{itemize}

\end{frame}

\begin{frame}{The L-function of $ E $}

Recall that the \textbf{L-function of $ E $} is given by
$$ L(E, s) := \prod_p \dfrac{1}{\det(1 - p^{-s} \cdot \phi_p \ | \ \rho_{E, \ell}^{I_p})}, $$
where $ \phi_p \in G_\Q $ is an arithmetic Frobenius and $ \rho_{E, \ell} : G_\Q \to \Aut(T_\ell(E)) $ is the representation of the $ \ell $-adic Tate module $ T_\ell(E) $ for some $ \ell \ne p $.

\begin{conjecture}[Birch--Swinnerton-Dyer]
The order of vanishing of $ L(E, s) $ at $ s = 1 $ is $ \rk(E) $, and
$$ \lim_{s \to 1} \dfrac{L(E, s)}{(s - 1)^{\rk(E)}} \cdot \dfrac{1}{\Omega(E)} = \dfrac{\Reg(E) \cdot \Tam(E) \cdot \#\Sha(E)}{\#\tor(E)^2}. $$
\end{conjecture}

When $ \rk(E) = 0 $, the LHS is the \textbf{algebraic L-value of $ E $}, given by
$$ \LLL(E) := L(E, 1) \cdot \dfrac{1}{\Omega(E)}. $$

\end{frame}

\begin{frame}{The L-function of $ E / K $}

Let $ K / \Q $ be finite Galois. The \textbf{L-function of $ E / K $} is given by
$$ L(E / K, s) := \prod_\pp \dfrac{1}{\det(1 - \Nm(\pp)^{-s} \cdot \phi_\pp \ | \ \rho_{E / K, \ell}^{I_\pp})}. $$

\begin{conjecture}[Birch--Swinnerton-Dyer]
The order of vanishing of $ L(E / K, s) $ at $ s = 1 $ is $ \rk(E / K) $, and
$$ \lim_{s \to 1} \dfrac{L(E / K, s)}{(s - 1)^{\rk(E / K)}} \cdot \dfrac{\sqrt{\Delta(K)}}{\Omega(E / K)} = \dfrac{\Reg(E / K) \cdot \Tam(E / K) \cdot \#\Sha(E / K)}{\#\tor(E / K)^2}. $$
\end{conjecture}

On the other hand, Artin formalism gives a factorisation
$$ L(E / K, s) = \prod_{\rho : \Gal(K / \Q) \to \C^\times} L(E, \rho, s)^{\dim(\rho)}. $$

\end{frame}

\begin{frame}{Twisted L-functions of $ E $}

Let $ K = \Q(\zeta_p) $. Then
$$ \left\{\begin{array}{c} \text{Artin representations} \\ \Gal(K / \Q) \to \C^\times \end{array}\right\} \quad \leftrightsquigarrow \quad \left\{\begin{array}{c} \text{Dirichlet characters} \\ (\Z / p\Z)^\times \to \C^\times \end{array}\right\}. $$
The \textbf{L-function of $ E $ twisted by $ \chi $} is given by
$$ L(E, \chi, s) := \prod_p \dfrac{1}{\det(1 - p^{-s} \cdot \phi_p \ | \ (\rho_{E, \ell} \otimes \rho_\chi)^{I_p})}. $$
More concretely,
$$ L(E, s) = \sum_{n \in \N} \dfrac{a_n}{n^s} \quad \overset{\chi}{\rightsquigarrow} \quad L(E, \chi, s) = \sum_{n \in \N} \dfrac{a_n\chi(n)}{n^s}. $$

\begin{conjecture}[Deligne--Gross]
The order of vanishing of $ L(E, \chi, s) $ at $ s = 1 $ is $ \langle\chi, E(K)_\C\rangle $.
\end{conjecture}

\end{frame}

\begin{frame}{A twisted BSD-type formula}

Is there a conjectural leading term?

\bigskip When $ \rk(E) = 0 $, the \textbf{algebraic L-value of $ E $ twisted by $ \chi $} is given by
$$ \LLL(E, \chi) := L(E, \chi, 1) \cdot \dfrac{p}{\tau(\chi) \cdot \Omega(E)}, $$
where $ \tau(\chi) $ is the Gauss sum of $ \chi $.

\begin{example}[Dokchitser--Evans--Wiersema]
Let $ E_1 $ and $ E_2 $ be given by 307a1 and 307c1, and let $ \chi $ be the quintic character of conductor $ 11 $ given by $ \chi(2) = \zeta_5 $. Then $ \Delta(E_i) = -307 $, and
$$ \Reg(E_i / K) = \Tam(E_i / K) = \Sha(E_i / K) = \tor(E_i / K) = 1, $$
for all $ K \subseteq \Q(\zeta_{11})^+ $. However
$$ \LLL(E_1, \chi) = 1, \qquad \LLL(E_2, \chi) = \zeta_5(\zeta_5 + \zeta_5^2 + \zeta_5^3)^2. $$
\end{example}

\end{frame}

\begin{frame}{Varying the character}

Fix $ E $ and $ q $. As $ p $ varies, how does $ \LLL(E, \chi) $ vary?

\begin{example}
Let $ E $ be given by 67a1, and let $ q = 3 $.
$$
\begin{array}{r|rrrrrrrrr}
p & 7 & 13 & 19 & 31 & 37 & 43 & 61 & 73 & 79 \\
\hline
\LLL(E, \chi) & 2\zeta_3 & 3\zeta_3 & -\zeta_3 & -27\zeta_3 & 3\zeta_3 & -4\zeta_3 & -\zeta_3 & -3\zeta_3 & 8 \\
\zeta_3 \mapsto 1 & 2 & 3 & -1 & -27 & 3 & -4 & -1 & -3 & 8 \\
\#E(\F_p) & 10 & 12 & 13 & 42 & 39 & 46 & 64 & 81 & 88 \\
\text{sum} & 12 & 15 & 12 & 15 & 42 & 42 & 63 & 78 & 96
\end{array}
$$
$$
\begin{array}{r|rrrrrrrrr}
p & 97 & 103 & 109 & 127 & 139 & 151 & 157 & 163 \\
\hline
\LLL(E, \chi) & -17 & 3\zeta_3 & -90\zeta_3 & 74\zeta_3 & 23\zeta_3 & -2 & 16 & -43\zeta_3 \\
\zeta_3 \mapsto 1 & -17 & 3 & -90 & 74 & 23 & -2 & 16 & -43 \\
\#E(\F_p) & 98 & 120 & 108 & 121 & 118 & 149 & 149 & 145 \\
\text{sum} & 81 & 123 & 18 & 195 & 141 & 147 & 165 & 102
\end{array}
$$
\end{example}

\end{frame}

\begin{frame}{The modularity theorem}

Write L-values of $ E $ as L-values of modular forms.

\bigskip Recall that the \textbf{Hecke L-function} of a cusp form $ f \in S_k(\Gamma) $ is given by
$$ L(f, s) := -\dfrac{(-z)^{s - 1}}{\Gamma(s)}\int_0^\infty (2\pi i)^sf(z)\d z. $$

\begin{theorem}[Carayol, Eichler, Shimura, BCDT, Edixhoven]
There is a finite surjective morphism $ \phi_E : X_0(N) \to E $ defined over $ \Q $, and a cuspidal eigenform $ f_E \in S_2(\Gamma_0(N)) $, such that
\begin{itemize}
\item the Hecke operator $ T_p $ has eigenvalue $ a_p(E) $,
\item the Hecke L-function of $ f_E $ is $ L(E, s) $, and
\item the pullback of $ \omega_E $ under $ \phi_E $ is a positive multiple of $ 2\pi if_E(z)\d z $.
\end{itemize}
\end{theorem}

This positive multiple is called the \textbf{Manin constant} $ c_0(E) $ of $ E $.

\end{frame}

\begin{frame}{Classical modular symbols}

A \textbf{modular symbol} is a path $ \{x, y\} \in \HH / \Gamma $, whose \textbf{period} is
$$ \mu_f(x, y) := \int_x^y 2\pi if(z)\d z, $$
so that $ \mu_f(0, \infty) = -L(f, 1) $. For any $ x \in \Q $,
$$ \mu_f(0, x + \Z) = \mu_f(0, x), \qquad \mu_f(0, -x) = \overline{\mu_f(0, x)}. $$
In particular, for any $ x \in \Q $,
$$ \mu_f(0, x) + \mu_f(0, 1 - x) = 2\Re(\mu_f(0, x)). $$

\begin{lemma}[Manin]
$$ \dfrac{2\Re(\mu_{f_E}(0, x))}{\Omega(E)} \in \dfrac{1}{c_0(E)}\Z. $$
\end{lemma}

\end{frame}

\begin{frame}{L-values as periods}

The Hecke operator $ T_p $ acts on the space of modular symbols such that
$$ -L(E, 1) \cdot \#E(\F_p) = \sum_{n = 1}^{p - 1} \mu_{f_E}(0, \tfrac{n}{p}). $$
Dividing by $ \Omega(E) $ gives
$$ -\LLL(E) \cdot \#E(\F_p) = \sum_{n = 1}^{p - 1} \dfrac{\mu_{f_E}(0, \tfrac{n}{p})}{\Omega(E)}. $$
Combining the $ n $-th and ($ p - n $)-th terms gives
$$ -\LLL(E) \cdot \#E(\F_p) = \sum_{n = 1}^{\tfrac{p - 1}{2}} \dfrac{2\Re(\mu_{f_E}(0, \tfrac{n}{p}))}{\Omega(E)}. $$
Multiplying by $ c_0(E) $ gives an equality in $ \Z $.

\end{frame}

\begin{frame}{Twisted L-values as periods}

Applying the Mellin transform to the Dirichlet series of $ f_E \otimes \chi $ yields
$$ L(E, \chi, 1) \cdot \dfrac{p}{\tau(\chi)} = \sum_{n = 1}^{p - 1} \overline{\chi}(n)\mu_{f_E}(0, \tfrac{n}{p}). $$
A similar rearrangement gives
$$ \LLL(E, \chi) = \sum_{n = 1}^{\tfrac{p - 1}{2}} \overline{\chi}(n)\dfrac{2\Re(\mu_{f_E}(0, \tfrac{n}{p}))}{\Omega(E)}. $$
Multiplying by $ c_0(E) $ gives an equality in $ \Z[\zeta_q] $.

\bigskip

\begin{theorem}[Manin]
$$ -c_0(E) \cdot \LLL(E) \cdot \#E(\F_p) \equiv c_0(E) \cdot \LLL(E, \chi) \mod (1 - \zeta_q). $$
\end{theorem}

\end{frame}

\begin{frame}{Revisiting the example}

\begin{example}[Dokchitser--Evans--Wiersema]
Let $ E_1 $ and $ E_2 $ be given by 307a1 and 307c1, and let $ \chi $ be the quintic character of conductor $ 11 $ given by $ \chi(2) = \zeta_5 $. Then $ \Delta(E_i) = -307 $, and
$$ \Reg(E_i / K) = \Tam(E_i / K) = \Sha(E_i / K) = \tor(E_i / K) = 1, $$
for all $ K \subseteq \Q(\zeta_{11})^+ $. However
$$ \LLL(E_1, \chi) = 1, \qquad \LLL(E_2, \chi) = \zeta_5(\zeta_5 + \zeta_5^2 + \zeta_5^3)^2. $$
Now $ c_0(E_i) = \LLL(E_i) = 1 $, but
$$ \#E_1(\F_{11}) = 9, \qquad \#E_2(\F_{11}) = 16, $$
so the congruence says $ \LLL(E_1, \chi) \not\equiv \LLL(E_2, \chi) \mod (1 - \zeta_5) $.
\end{example}

\bigskip In fact, the congruence clarifies all 30 pairs of examples in the paper.

\end{frame}

\begin{frame}{Insufficiency of congruence}

In general, the congruence only serves as a sanity check for the L-value.

\begin{example}
Let $ E_1 $ and $ E_2 $ be given by 182d1 and 460a1, and let $ \chi $ be the quintic character of conductor $ 11 $ given by $ \chi(2) = \zeta_5 $. Then $ \Delta(E_i) < 0 $, and
$$ \Reg(E_i / K) = \Tam(E_i / K) = \Sha(E_i / K) = \tor(E_i / K) = 1, $$
for all $ K \subseteq \Q(\zeta_{11})^+ $. Furthermore $ c_0(E_i) = \LLL(E_i) = 1 $, and
$$ \#E_1(\F_{11}) = 11, \qquad \#E_2(\F_{11}) = 6, $$
so the congruence says $ \LLL(E_1, \chi) \equiv \LLL(E_2, \chi) \mod (1 - \zeta_5) $. However
$$ \LLL(E_1, \chi) = -\zeta_5^2, \qquad \LLL(E_2, \chi) = -\zeta_5^3. $$
\end{example}

In certain cases, the congruence can be interpreted as an equality.

\end{frame}

\begin{frame}{Congruence for units}

Let $ K \subseteq \Q(\zeta_p) $ be the subfield of degree $ q $ where $ \chi $ factors through $ K / \Q $. Assume further that the Birch--Swinnerton-Dyer conjecture holds for $ E $ over $ \Q $ and over $ K $, and that $ c_0(E) = 1 $ and $ \LLL(E) \cdot \#E(\F_p) \not\equiv 0 \mod q $.

\begin{theorem}[Dokchitser--Evans--Wiersema]
$ \LLL(E, \chi) = \overline{\chi}(N) \cdot \ell $ for some $ \ell \in \Z[\zeta_q + \overline{\zeta_q}] $, has norm $ \pm\BB(E, \chi) $, where
$$ \BB(E, \chi) := \dfrac{\Tam(E / K) \cdot \#\Sha(E / K) \cdot \#\tor(E / K)^{-2}}{\Tam(E / \Q) \cdot \#\Sha(E / \Q) \cdot \#\tor(E / \Q)^{-2}} \in \Z, $$
and generates an ideal of $ \Z[\zeta_q] $ invariant under complex conjugation.
\end{theorem}

\begin{corollary}
If $ \BB(E, \chi) = 1 $, then $ \ell \in \Z[\zeta_q + \overline{\zeta_q}]^\times $, and
$$ \ell \equiv -\LLL(E) \cdot \#E(\F_p) \mod (2 - (\zeta_q + \overline{\zeta_q})). $$
\end{corollary}

If $ q = 3 $, the congruence determines $ \ell $ exactly.

\end{frame}

\begin{frame}{Congruence for non-units}

In general, the ideal generated by $ \LLL(E, \chi) $ has finitely many possibilities.

\begin{example}[Dokchitser--Evans--Wiersema]
Let $ E_1 $ and $ E_2 $ be given by 291d1 and 139a1, and let $ \chi $ be the quintic character of conductor $ 31 $ given by $ \chi(3) = \zeta_5^3 $. Then $ \BB(E_i, \chi) = 11^2 $, so $ \LLL(E_i, \chi) $ generate ideals of norm $ 11^2 $ that are invariant under complex conjugation. There are only two such ideals, generated by
$$ \ell_1 := 3\zeta_5^3 + \zeta_5^2 + 3\zeta_5, \qquad \ell_2 := \zeta_5^3 + 3\zeta_5 + 3. $$
In fact, $ (\LLL(E_i, \chi)) = (\ell_i) $ by Burns--Castillo. Furthermore $ \LLL(E_i) = 1 $, $ \#E_1(\F_{31}) = 33 $, and $ \#E_2(\F_{31}) = 23 $, so the congruence says
$$ \LLL(E_1, \chi) = u_1 \cdot \ell_1, \qquad u_1 \cdot (3 + 1 + 3) \equiv -33 \mod (1 - \zeta_5), $$
$$ \LLL(E_2, \chi) = u_2 \cdot \ell_2, \qquad u_2 \cdot (1 + 3 + 3) \equiv -23 \mod (1 - \zeta_5). $$
In fact, $ u_1 = \zeta_5^4 $ and $ u_2 = \zeta_5^2 - \zeta_5 + 1 $.
\end{example}

\end{frame}

\begin{frame}{Asymptotic distribution}

Fix $ E $ and $ q $. As $ p $ varies, how does $ \LLL(E, \chi) $ modulo $ (1 - \zeta_q) $ vary?

\bigskip The congruence says $ \LLL(E, \chi) $ varies according to $ \#E(\F_p) $ modulo $ q $.

\bigskip On the other hand, by considering $ \rho_{E, q}(\phi_p) \in \GL_2(\Z_q) $,
$$ \#E(\F_p) = 1 + \det(\rho_{E, q}(\phi_p)) - \tr(\rho_{E, q}(\phi_p)). $$
As $ p \equiv 1 \mod q $ varies, $ \rho_{E, q}(\phi_p) $ varies over the group
$$ G_{E, q^\infty} := \{M \in \im(\rho_{E, q}) : \det(M) \equiv 1 \mod q\}. $$
By Chebotarev, $ \rho_{E, q}(\phi_p) $ is asymptotically distributed uniformly in $ G_{E, q^\infty} $.

\bigskip Thus the asymptotic density of $ \#E(\F_p) \equiv \ell \mod q $ is the asymptotic density of matrices $ M \in G_{E, q^\infty} $ with $ 1 + \det(M) - \tr(M) \equiv \ell \mod q $.

\end{frame}

\begin{frame}{Maximal Galois image}

For most $ E $, suffices to consider $ \overline{\rho_{E, q}} : G_\Q \to \Aut(E[q]) $ and
$$ G_{E, q} := \{M \in \im(\overline{\rho_{E, q}}) : \det(M) = 1\}. $$

\begin{example}
Let $ E $ be given by 11a1. Then $ c_0(E) = 1 $ and $ \LLL(E) = \tfrac{1}{5} \equiv -1 \mod 3 $, so
$$ \LLL(E, \chi) \equiv \#E(\F_p) \equiv 2 - \tr(\overline{\rho_{E, 3}}(\phi_p)) \mod (1 - \zeta_3). $$
Now $ \overline{\rho_{E, 3}} $ is surjective, so $ G_{E, 3} = \SL_2(\F_3) $. This consists of:
{\scriptsize $$
\begin{array}{c}
\begin{pmatrix} 1 & 0 \\ 0 & 1 \end{pmatrix} \ \begin{pmatrix} 0 & 2 \\ 1 & 2 \end{pmatrix} \ \begin{pmatrix} 1 & 2 \\ 0 & 1 \end{pmatrix} \ \begin{pmatrix} 2 & 2 \\ 1 & 0 \end{pmatrix} \ \begin{pmatrix} 0 & 1 \\ 2 & 2 \end{pmatrix} \ \begin{pmatrix} 1 & 0 \\ 2 & 1 \end{pmatrix} \ \begin{pmatrix} 1 & 1 \\ 0 & 1 \end{pmatrix} \ \begin{pmatrix} 1 & 0 \\ 1 & 1 \end{pmatrix} \ \begin{pmatrix} 2 & 1 \\ 2 & 0 \end{pmatrix} \\ \\
\begin{pmatrix} 2 & 0 \\ 0 & 2 \end{pmatrix} \ \begin{pmatrix} 0 & 2 \\ 1 & 1 \end{pmatrix} \ \begin{pmatrix} 2 & 0 \\ 2 & 2 \end{pmatrix} \ \begin{pmatrix} 0 & 1 \\ 2 & 1 \end{pmatrix} \ \begin{pmatrix} 2 & 0 \\ 1 & 2 \end{pmatrix} \ \begin{pmatrix} 2 & 1 \\ 0 & 2 \end{pmatrix} \ \begin{pmatrix} 1 & 1 \\ 2 & 0 \end{pmatrix} \ \begin{pmatrix} 1 & 2 \\ 1 & 0 \end{pmatrix} \ \begin{pmatrix} 2 & 2 \\ 0 & 2 \end{pmatrix} \\ \\
\begin{pmatrix} 0 & 2 \\ 1 & 0 \end{pmatrix} \ \begin{pmatrix} 0 & 1 \\ 2 & 0 \end{pmatrix} \ \begin{pmatrix} 2 & 1 \\ 1 & 1 \end{pmatrix} \ \begin{pmatrix} 1 & 1 \\ 1 & 2 \end{pmatrix} \ \begin{pmatrix} 1 & 2 \\ 2 & 2 \end{pmatrix} \ \begin{pmatrix} 2 & 2 \\ 2 & 1 \end{pmatrix}
\end{array}
$$}
Thus $ \LLL(E, \chi) \equiv 0, 1, 2 \mod (1 - \zeta_3) $ with densities $ \tfrac{9}{24} $, $ \tfrac{9}{24} $, $ \tfrac{6}{24} $.
\end{example}

\end{frame}

\begin{frame}{Small Galois image}

For other $ E $, need to consider $ \overline{\rho_{E, q^n}} : G_\Q \to \Aut(E[q^n]) $ and
$$ G_{E, q^n} := \{M \in \im(\overline{\rho_{E, q^n}}) : \det(M) \equiv 1 \mod q\}. $$

\begin{example}
Let $ E $ be given by 14a1. Then $ c_0(E) = 1 $ and $ \LLL(E) = \tfrac{1}{6} $, so
$$ \LLL(E, \chi) \equiv -\tfrac{1}{6} \cdot \#E(\F_p) \mod (1 - \zeta_3). $$
In other words, $ \LLL(E, \chi) \equiv \ell \mod (1 - \zeta_3) $ precisely if
$$ 1 + \det(\overline{\rho_{E, 9}}(\phi_p)) - \tr(\overline{\rho_{E, 9}}(\phi_p)) \equiv -6\ell \mod 9. $$
However, $ 1 + \det(M) - \tr(M) \equiv 0 \mod 9 $ for all matrices $ M $ in
$$ G_{E, 9} = \{M \in \GL_2(\Z / 9\Z) : M \equiv 1 \mod 3\}. $$
Thus $ \LLL(E, \chi) \equiv 0, 1, 2 \mod (1 - \zeta_3) $ with densities $ 1 $, $ 0 $, $ 0 $.
\end{example}

\end{frame}

\begin{frame}{Large Galois image}

For some $ E $, the density of $ \#E(\F_p) $ might be visible in $ G_{E, q^n} $.

\bigskip

\begin{example}
Let $ E $ be given by 20a1. Then $ c_0(E) = 1 $ and $ \LLL(E) = \tfrac{1}{6} $, so similarly
$$ 1 + \det(\overline{\rho_{E, 9}}(\phi_p)) - \tr(\overline{\rho_{E, 9}}(\phi_p)) \equiv -6\ell \mod 9 $$
precisely if $ \LLL(E, \chi) \equiv \ell \mod (1 - \zeta_3) $. Now
$$ G_{E, 9} = \left\{M \in \GL_2(\Z / 9\Z) : M \equiv \begin{pmatrix} 1 & * \\ 0 & 1 \end{pmatrix} \mod 3\right\}. $$
There are $ 135 $, $ 54 $, $ 54 $ matrices $ M \in G_{E, 9} $ such that
$$ 1 + \det(M) - \tr(M) \equiv -6(0), -6(1), -6(2) \mod 9. $$
Thus $ \LLL(E, \chi) \equiv 0, 1, 2 \mod (1 - \zeta_3) $ with densities $ \tfrac{135}{243} $, $ \tfrac{54}{243} $, $ \tfrac{54}{243} $.
\end{example}

\end{frame}

\begin{frame}{The density theorem}

Define the \textbf{natural density}
$$ \delta_{E, q}(\ell) := \lim_{n \to \infty} \dfrac{\#\{p \in P_n : c_0(E) \cdot \LLL(E, \chi) \equiv \ell \mod (1 - \zeta_q)\}}{\#P_n}, $$
where $ P_n $ is the set of primes $ p \equiv 1 \mod q $ less than $ n $.

\begin{theorem}[A.]
Let $ c := (c_0(E) \cdot \LLL(E))^{-1} $, and let $ n := \nu_q(c) + 1 $. If $ n \le 0 $, then $ \delta_{E, q}(0) = 1 $. Otherwise, $ c $ is well-defined and non-zero modulo $ q^n $, and
$$ \delta_{E, q}(\ell) = \dfrac{\#\{M \in G_{E, q^n} : 1 + \det(M) - \tr(M) \equiv -c\ell \mod q^n\}}{\#G_{E, q^n}}. $$
In particular, if $ \overline{\rho_{E, q}} $ is surjective, then $ n = 1 $, and
$$ \delta_{E, q}(\ell) = \left\{
\begin{array}{lcr}
\tfrac{1}{q - 1} & & 1 \\
\tfrac{q}{q^2 - 1} & \qquad \text{if} \qquad & 0 \\
\tfrac{1}{q + 1} & & -1
\end{array}
\right\} = \left(\dfrac{c\ell}{q}\right)\left(\dfrac{c\ell + 4}{q}\right). $$
\end{theorem}

\end{frame}

\begin{frame}[c]{Current status}

Paper is in preparation.
\begin{itemize}
\item Stated congruence for non-trivial even Dirichlet characters of arbitrary conductor and order, but with an error term of periods.
\item Classified natural densities for cubic characters, thanks to classification of $ 3 $-adic images by Rouse--Sutherland--Zureick-Brown.
\item Explained some distributions for cubic characters in Kisilevsky--Nam, where the normalisation of $ \LLL(E, \chi) $ depends crucially on $ \chi(N) $.
\end{itemize}

\end{frame}

\end{document}