\ifx\type\undefined
  \documentclass[10pt, t]{beamer}
  \setbeamertemplate{footline}[page number]
\else
  \documentclass[10pt]{article}
  \usepackage[margin=1in]{geometry}
\fi

\usepackage{amsmath}
\usepackage{amssymb}
\usepackage{amsthm}
\usepackage{bbm}
\usepackage{cancel}
\usepackage{listings}
\usepackage{mathrsfs}
\usepackage{multirow}
\usepackage{soul}
\usepackage{stmaryrd}
\usepackage{tikz}
\usepackage{tikz-cd}
\usepackage{wrapfig}

\newtheorem*{algorithm}{Algorithm}
\newtheorem*{assumptions}{Assumptions}
\newtheorem*{conjecture}{Conjecture}
\newtheorem*{consequences}{Consequences}
\newtheorem*{exercise}{Exercise}
\newtheorem*{formalisation}{Formalisation}
\newtheorem*{proposition}{Proposition}
\newtheorem*{question}{Question}
\newtheorem*{remark}{Remark}

\ifx\type\undefined\else
  \newtheorem*{definition}{Definition}
  \newtheorem*{example}{Example}
  \newtheorem*{lemma}{Lemma}
  \newtheorem*{theorem}{Theorem}
\fi

\definecolor{keywordcolor}{rgb}{0.7, 0.1, 0.1}
\definecolor{tacticcolor}{rgb}{0.0, 0.1, 0.6}
\definecolor{commentcolor}{rgb}{0.4, 0.4, 0.4}
\definecolor{symbolcolor}{rgb}{0.0, 0.1, 0.6}
\definecolor{sortcolor}{rgb}{0.1, 0.5, 0.1}
\definecolor{attributecolor}{rgb}{0.7, 0.1, 0.1}
\def\lstlanguagefiles{lstlean.tex}
\lstset{language=lean}

\newcommand\A{\mathbb{A}}
\newcommand\C{\mathbb{C}}
\newcommand\F{\mathbb{F}}
\newcommand\G{\mathbb{G}}
\renewcommand\H{\mathbb{H}}
\newcommand\I{\mathbb{I}}
\newcommand\N{\mathbb{N}}
\renewcommand\P{\mathbb{P}}
\newcommand\Q{\mathbb{Q}}
\newcommand\R{\mathbb{R}}
\newcommand\Z{\mathbb{Z}}

\renewcommand\AA{\mathcal{A}}
\newcommand\BB{\mathcal{B}}
\newcommand\CC{\mathcal{C}}
\newcommand\DD{\mathcal{D}}
\newcommand\EE{\mathcal{E}}
\newcommand\FF{\mathcal{F}}
\newcommand\GG{\mathcal{G}}
\newcommand\HH{\mathcal{H}}
\newcommand\II{\mathcal{I}}
\newcommand\LL{\mathcal{L}}
\newcommand\MM{\mathcal{M}}
\newcommand\NN{\mathcal{N}}
\newcommand\OO{\mathcal{O}}
\newcommand\PP{\mathcal{P}}
\newcommand\RR{\mathcal{R}}
\renewcommand\SS{\mathcal{S}}
\newcommand\TT{\mathcal{T}}
\newcommand\XX{\mathcal{X}}

\renewcommand\aa{\mathfrak{a}}
\newcommand\cc{\mathfrak{c}}
\newcommand\dd{\mathfrak{d}}
\newcommand\ff{\mathfrak{f}}
\renewcommand\gg{\mathfrak{g}}
\newcommand\mm{\mathfrak{m}}
\newcommand\pp{\mathfrak{p}}
\newcommand\qq{\mathfrak{q}}
\renewcommand\ss{\mathfrak{s}}

\newcommand\LLL{\mathscr{L}}

\newcommand\ab{\mathrm{ab}}
\newcommand\Ab{\mathbf{Ab}}
\newcommand\Alg{\mathbf{Alg}}
\newcommand\Aff{\mathbf{Aff}}
\newcommand\Aut{\operatorname{Aut}}
\newcommand\Az{\mathrm{Az}}
\newcommand\Br{\operatorname{Br}}
\newcommand\BSD{\operatorname{BSD}}
\newcommand\ch{\operatorname{char}}
\newcommand\Cl{\operatorname{Cl}}
\newcommand\coker{\operatorname{coker}}
\newcommand\cris{\mathrm{cris}}
\renewcommand\d{\mathrm{d}}
\newcommand\Div{\operatorname{Div}}
\newcommand\dR{\mathrm{dR}}
\newcommand\EN{\operatorname{EN}}
\newcommand\End{\operatorname{End}}
\newcommand\ES{\operatorname{ES}}
\newcommand\et{\mathrm{\acute{e}t}}
\newcommand\Et{\mathbf{\acute{E}t}}
\newcommand\Ext{\operatorname{Ext}}
\newcommand\Fr{\operatorname{Fr}}
\newcommand\Frac{\operatorname{Frac}}
\newcommand\Gal{\operatorname{Gal}}
\newcommand\GL{\operatorname{GL}}
\newcommand\Gr{\mathrm{Gr}}
\newcommand\Hom{\operatorname{Hom}}
\newcommand\HT{\mathrm{HT}}
\newcommand\id{\operatorname{id}}
\newcommand\im{\operatorname{im}}
\newcommand\Ind{\operatorname{Ind}}
\renewcommand\inf{\operatorname{inf}}
\newcommand\inv{\operatorname{inv}}
\newcommand\Irr{\operatorname{Irr}}
\newcommand\Jac{\operatorname{Jac}}
\newcommand\lcm{\operatorname{lcm}}
\newcommand\Mat{\operatorname{Mat}}
\newcommand\Mod{\mathbf{Mod}}
\newcommand\Nm{\operatorname{Nm}}
\newcommand\nr{\mathrm{nr}}
\newcommand\NS{\operatorname{NS}}
\newcommand\Ob{\operatorname{Ob}}
\newcommand\ord{\operatorname{ord}}
\newcommand\op{\mathrm{op}}
\newcommand\PGL{\operatorname{PGL}}
\newcommand\Pic{\operatorname{Pic}}
\newcommand\Prob{\operatorname{Prob}}
\newcommand\Proj{\operatorname{Proj}}
\newcommand\PSh{\mathbf{PSh}}
\newcommand\Reg{\operatorname{Reg}}
\newcommand\res{\operatorname{res}}
\newcommand\rk{\operatorname{rk}}
\newcommand\Sch{\mathbf{Sch}}
\newcommand\Sel{\operatorname{Sel}}
\newcommand\Set{\mathbf{Set}}
\newcommand\sgn{\operatorname{sgn}}
\newcommand\Sh{\mathbf{Sh}}
\newcommand\SL{\operatorname{SL}}
\newcommand\Spec{\operatorname{Spec}}
\newcommand\supp{\operatorname{supp}}
\newcommand\Tam{\operatorname{Tam}}
\newcommand\Top{\mathbf{Top}}
\newcommand\tor{\operatorname{tor}}
\newcommand\tr{\operatorname{tr}}
\newcommand\tra{\operatorname{tra}}
\newcommand\WC{\operatorname{WC}}

\DeclareFontFamily{U}{wncyr}{}
\DeclareFontShape{U}{wncyr}{m}{n}{<->wncyr10}{}
\DeclareSymbolFont{cyr}{U}{wncyr}{m}{n}
\DeclareMathSymbol{\Sha}{\mathord}{cyr}{"58}

\newcommand{\function}[5][]{
  \if &#1&
    \begin{array}{rcl}
      #2 & \longrightarrow & #3 \\
      #4 & \longmapsto     & #5
    \end{array}
  \else
    \begin{array}{rcrcl}
      #1 & : & #2 & \longrightarrow & #3 \\
         &   & #4 & \longmapsto     & #5
    \end{array}
  \fi
}

\newcommand{\functions}[7][]{
  \if &#1&
    \begin{array}{rcl}
      #2 & \longrightarrow & #3 \\
      #4 & \longmapsto     & #5 \\
      #6 & \longmapsto     & #7 \\
    \end{array}
  \else
    \begin{array}{rcrcl}
      #1 & : & #2 & \longrightarrow & #3 \\
         &   & #4 & \longmapsto     & #5 \\
         &   & #6 & \longmapsto     & #7
    \end{array}
  \fi
}
\title{Denominators of BSD quotients}
\subtitle{Young Researchers in Algebraic Number Theory}
\author{David Kurniadi Angdinata}
\institute{London School of Geometry and Number Theory}
\date{Wednesday, 31 July 2024}

\begin{document}

\frame\maketitle

\begin{frame}{Mordell's theorem}

Let $ E $ be an elliptic curve over $ \Q $ given by a Weierstrass equation
$$ y^2 + a_1xy + a_3y = x^3 + a_2x^2 + a_4x + a_6, \qquad a_i \in \Q. $$
Its rational points forms a group $ E(\Q) $ under a geometric addition law.

\begin{theorem}[Mordell]
$$ E(\Q) \cong \tor(E) \oplus \Z^{\rk(E)}. $$
\end{theorem}

The \textbf{torsion subgroup} $ \tor(E) $ is well understood.

\begin{theorem}[Mazur]
$$ \tor(E) \cong
\begin{cases}
C_n & \text{for} \ n = 1, 2, 3, 4, 5, 6, 7, 8, 9, 10, 12, \\
C_2 \oplus C_{2n} & \text{for} \ n = 1, 2, 3, 4.
\end{cases}
$$
\end{theorem}

The \textbf{rank} $ \rk(E) $ is somewhat mysterious.

\end{frame}

\begin{frame}{The Birch--Swinnerton-Dyer conjecture}

Assume $ E $ has conductor $ N $. The L-function of $ E $ is the infinite product
$$ L(E, s) := \prod_p \dfrac{1}{L_p(E, p^{-s})}. $$
Here,
$$ L_p(E, T) :=
\begin{cases}
1 \pm \epsilon T & \text{if} \ p \mid N, \\
1 - a_p(E)T + pT^2 & \text{if} \ p \nmid N,
\end{cases}
$$
where $ a_p(E) := 1 + p - \#E(\F_p) $ and $ \epsilon \in \{-1, 0, 1\} $.

\bigskip

\begin{conjecture}[weak Birch--Swinnerton-Dyer]
$$ \ord_{s = 1} L(E, s) = \rk(E). $$
\end{conjecture}

\bigskip This is known for $ \ord_{s = 1} L(E, s) \le 1 $. Assume that $ \ord_{s = 1} L(E, s) = 0 $.

\end{frame}

\begin{frame}{The Birch--Swinnerton-Dyer quotient}

\begin{conjecture}[strong Birch--Swinnerton-Dyer]
$$ \dfrac{L(E, 1)}{\Omega(E)} = \dfrac{\Tam(E) \cdot \#\Sha(E)}{\#\tor(E)^2}. $$
\end{conjecture}

The LHS is the \textbf{algebraic L-value} and the RHS is the \textbf{BSD quotient}.
\begin{itemize}
\item The \textbf{Tamagawa product} is the finite product
$$ \Tam(E) := \prod_{p \mid N} [E(\Q_p) : E_0(\Q_p)], $$
where $ E_0(\Q_p) $ is the subgroup of points of $ E(\Q_p) $ whose reduction is \emph{nonsingular}. It can be computed by \emph{Tate's algorithm}.
\item The \textbf{Tate--Shafarevich group} is the finite group
$$ \Sha(E) := \ker\left(H^2(\Q, E) \to H^2(\R, E) \times \prod_p H^2(\Q_p, E)\right). $$
\end{itemize}

\end{frame}

\begin{frame}{The Birch--Swinnerton-Dyer quotient}

\begin{conjecture}[strong Birch--Swinnerton-Dyer]
$$ \dfrac{L(E, 1)}{\Omega(E)} = \dfrac{\Tam(E) \cdot \#\Sha(E)}{\#\tor(E)^2}. $$
\end{conjecture}

The LHS is the \textbf{algebraic L-value} and the RHS is the \textbf{BSD quotient}.
\begin{itemize}
\item The \textbf{real period} is the integral
$$ \Omega(E) := \int_{E(\R)} \omega_E, $$
where $ \omega_E $ is the \textbf{N\'eron differential}. If $ E $ is given by a \emph{minimal} Weierstrass equation $ y^2 + a_1xy + a_3y = x^3 + a_2x^2 + a_4x + a_6 $,
$$ \omega_E = \dfrac{\d x}{2y + a_1x + a_3}. $$
It is the least positive element of the \emph{real period lattice} of $ E $.
\end{itemize}

\end{frame}

{\usebackgroundtemplate{\includegraphics[width=\paperwidth]{img/lmfdb1.png}}
\begin{frame}[b]

\begin{minipage}{0.54\textwidth}\hfill\end{minipage}
\begin{minipage}{0.45\textwidth}\fbox{\tiny $ \dfrac{L(E, 1)}{\Omega(E)} = \dfrac{\Tam(E) \cdot \#\Sha(E)}{\#\tor(E)^2} $}\end{minipage}

\bigskip

\end{frame}
}

\begin{frame}{Denominator bounds}

Observe that BSD quotients have bounded denominators.

\bigskip

\begin{theorem}[Mazur]
$$ \tor(E) \cong
\begin{cases}
C_n & \text{for} \ n = 1, 2, 3, 4, 5, 6, 7, 8, 9, 10, 12, \\
C_2 \oplus C_{2n} & \text{for} \ n = 1, 2, 3, 4.
\end{cases}
$$
\end{theorem}

\begin{corollary}
$$ \ord_p\left(\dfrac{\Tam(E) \cdot \#\Sha(E)}{\#\tor(E)^2}\right) \ge
\begin{cases}
-8 & \text{if} \ p = 2, \\
-4 & \text{if} \ p = 3, \\
-2 & \text{if} \ p = 5, 7, \\
0 & \text{if} \ p \ge 11.
\end{cases}
$$
\end{corollary}

There are typically cancellations between $ \tor(E) $ and $ \Tam(E) $.

\end{frame}

\begin{frame}{Torsion cancellations}

\begin{theorem}[Lorenzini, 2010]
Assume that $ \tor(E) $ has a point of order $ n \ge 4 $.
\begin{itemize}
\item If $ n = 4 $, then $ 2 \mid \Tam(E) $, except for 15a7, 15a8, 17a4.
\item If $ n \ge 5 $, then $ n \mid \Tam(E) $, except for 11a3, 14a4, 14a6, 20a2.
\item If $ n = 9 $, then $ 27 \mid \Tam(E) $.
\end{itemize}
\end{theorem}

\begin{corollary}
With seven exceptions,
$$ \ord_p\left(\dfrac{\Tam(E) \cdot \#\Sha(E)}{\#\tor(E)^2}\right) \ge
\begin{cases}
-5 & \text{if} \ p = 2 \ \text{and} \ \tor(E) \cong C_2 \oplus C_{2n}, \\
-3 & \text{if} \ p = 2 \ \text{and} \ \tor(E) \not\cong C_2 \oplus C_{2n}, \\
-2 & \text{if} \ p = 3 \ \text{and} \ \tor(E) \cong C_3, \\
-1 & \text{if} \ p = 3 \ \text{and} \ \tor(E) \not\cong C_3, \\
-1 & \text{if} \ p = 5, 7, \\
0 & \text{if} \ p \ge 11.
\end{cases}
$$
\end{corollary}

\end{frame}

\begin{frame}{The seven exceptions}

Let $ \BSD(E) $ denote the BSD quotient.
$$
\renewcommand{\arraystretch}{2}
\begin{array}{|c|c|c|c|c|c|c|c|}
\hline
E & 11a3 & 14a4 & 14a6 & 15a7 & 15a8 & 17a4 & 20a2 \\
\hline
\tor(E) & C_5 & C_6 & C_6 & C_4 & C_4 & C_4 & C_6 \\
\hline
\Tam(E) & 1 & 2 & 2 & 1 & 1 & 1 & 3 \\
\hline
\Sha(E) & 1 & 1 & 1 & 1 & 1 & 1 & 1 \\
\hline
\BSD(E) & \tfrac{1}{5^2} & \tfrac{1}{2 \cdot 3^2} & \tfrac{1}{2 \cdot 3^2} & \tfrac{1}{2^4} & \tfrac{1}{2^4} & \tfrac{1}{2^4} & \tfrac{1}{2^2 \cdot 3} \\
\hline
c_0(E) & 5 & 3 & 3 & 2 & 4 & 4 & 2 \\
\hline
c_0(E)\BSD(E) & \tfrac{1}{5} & \tfrac{1}{2 \cdot 3} & \tfrac{1}{2 \cdot 3} & \tfrac{1}{2^3} & \tfrac{1}{2^2} & \tfrac{1}{2^2} & \tfrac{1}{2 \cdot 3} \\
\hline
\end{array}
$$
Here, $ c_0(E) $ is the \textbf{Manin constant} in the LMFDB.

\end{frame}

{\usebackgroundtemplate{\includegraphics[width=\paperwidth]{img/lmfdb2.png}}
\begin{frame}[c]

\vspace{3cm}

\begin{minipage}{0.09\textwidth}\hfill\end{minipage}
\fbox{\begin{minipage}{0.13\textwidth}\bigskip\hfill\end{minipage}}

\end{frame}
}

\begin{frame}{The Manin constant}

\begin{theorem}[Modularity, version $ L $]
There is an eigenform $ f_E \in S_2(\Gamma_0(N)) $ with eigenvalues $ a_p(E) $ such that
$$ L(f_E, s) = L(E, s). $$
\end{theorem}

In particular, this defines a differential $ f_E(q)\d q $ on $ X_0(N) $.

\begin{theorem}[Modularity, version $ X_\Q $]
There is a finite morphism $ \phi_E : X_0(N) \twoheadrightarrow E $ defined over $ \Q $ such that
$$ \phi_E^*\omega_E = c_0(E) \cdot f_E(q)\d q, $$
for some positive integer $ c_0(E) $.
\end{theorem}

\bigskip Conjecturally $ c_0(E) = 1 $ for all \emph{$ \Gamma_0(N) $-optimal} elliptic curves (known in the semistable case!), but the seven exceptions are not $ \Gamma_0(N) $-optimal.

\end{frame}

\begin{frame}{A refined conjecture}

\begin{conjecture}
With no exceptions,
$$ \ord_p\left(\dfrac{c_0(E) \cdot \Tam(E) \cdot \#\Sha(E)}{\#\tor(E)^2}\right) \ge
\begin{cases}
-3 & \text{if} \ p = 2, \\
-1 & \text{if} \ p = 3, 5, 7, \\
0 & \text{if} \ p \ge 11.
\end{cases}
$$
\end{conjecture}

This follows from Lorenzini's theorem, but the bound for $ p = 2 $ holds for $ \tor(E) \cong C_2 \oplus C_{2n} $, and the bound for $ p = 3 $ holds for $ \tor(E) \cong C_3 $.

\bigskip

\begin{conjecture}
Assume that $ \tor(E) \cong C_3 $. Then $ 3 \mid c_0(E) \cdot \Tam(E) \cdot \#\Sha(E) $.
\end{conjecture}

\bigskip I can prove this under the strong Birch--Swinnerton-Dyer conjecture.

\end{frame}

\begin{frame}{Modular symbols}

If $ f \in S_2(\Gamma_0(N)) $ and $ p \nmid N $, the Hecke operator $ T_p $ acts on periods by
$$ (1 + p - T_p) \cdot \int_0^\infty f(q)\d q = \sum_{a = 1}^{p - 1} \int_0^{\tfrac{a}{p}} f(q)\d q. $$
If $ f = f_E $ and $ p $ is odd, this says that
$$ (1 + p - a_p(E)) \cdot (-L(E, 1)) = \dfrac{\Omega(E)}{c_0(E)} \cdot n, \quad n \in \Z. $$
If the strong Birch--Swinnerton-Dyer conjecture holds,
$$ (1 + p - a_p(E)) \cdot \dfrac{c_0(E) \cdot \Tam(E) \cdot \#\Sha(E)}{\#\tor(E)^2} \in \Z. $$
If $ \tor(E) \cong C_3 $, it suffices to find an odd prime $ p \nmid N $ such that
$$ 1 + p - a_p(E) \equiv 3 \mod 9. $$

\end{frame}

\begin{frame}{3-adic Galois images}

In terms of $ \rho_{E, 3} : \Gal(\overline{\Q} / \Q) \to \GL_2(\Q_3) $,
$$ p = \det(\rho_{E, 3}(\Fr_p)), \qquad a_p(E) = \tr(\rho_{E, 3}(\Fr_p)). $$
Chebotarev's density theorem says that $ \Fr_p $ is uniformly distributed in $ \im(\rho_{E, 3}) $, so it suffices to find a matrix $ M \in \im(\rho_{E, 3}) $ such that
$$ 3 = 1 + \det(M) - \tr(M). $$

\begin{theorem}[Rouse--Sutherland--Zureick-Brown, 2022]
Assume that $ \tor(E) \cong C_3 $. Then $ \im(\rho_{E, 3}) $ is one of the explicit matrix subgroups 3.8.0.1, 3.24.0.1, 9.24.0.1/2, 9.72.0.1/2/3/4/6/7/8/9/10, 27.72.0.1, 27.648.13.25, 27.648.18.1, or 27.1944.55.31/37/43/44.
\end{theorem}

\bigskip Each $ \im(\rho_{E, 3}) $ contains a matrix $ M $ such that $ 3 = 1 + \det(M) - \tr(M) $, except for 9.72.0.1, but Tate's algorithm shows $ 3 \mid \Tam(E) $ in this case.

\end{frame}

\begin{frame}{Concluding remarks}

\begin{theorem}[A., 2023]
Assume the $ 3 $-part of the strong Birch--Swinnerton-Dyer conjecture. Then
$$ \ord_p\left(\dfrac{c_0(E) \cdot \Tam(E) \cdot \#\Sha(E)}{\#\tor(E)^2}\right) \ge
\begin{cases}
-3 & \text{if} \ p = 2, \\
-1 & \text{if} \ p = 3, 5, 7, \\
0 & \text{if} \ p \ge 11.
\end{cases}
$$
\end{theorem}

Note the similarity to a conjecture by Agashe--Stein (2005) that
$$ \dfrac{2 \cdot c_0(E) \cdot \Tam(E) \cdot \#\Sha(E)}{\#\tor(E)} \in \Z. $$
This is known for semistable optimal elliptic curves by Melistas (2023), building upon \v Cesnavi\v cius (2018) and Byeon--Kim--Yhee (2020).

\bigskip Does this generalise to $ \F_q(C) $ or $ \ord_{s = 1} L(E, s) \ge 1 $?

\end{frame}

\end{document}