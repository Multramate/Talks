\ifx\type\undefined
  \documentclass[10pt, t]{beamer}
  \setbeamertemplate{footline}[page number]
\else
  \documentclass[10pt]{article}
  \usepackage[margin=1in]{geometry}
\fi

\usepackage{amsmath}
\usepackage{amssymb}
\usepackage{amsthm}
\usepackage{bbm}
\usepackage{cancel}
\usepackage{listings}
\usepackage{mathrsfs}
\usepackage{multirow}
\usepackage{soul}
\usepackage{stmaryrd}
\usepackage{tikz}
\usepackage{tikz-cd}
\usepackage{wrapfig}

\newtheorem*{algorithm}{Algorithm}
\newtheorem*{assumptions}{Assumptions}
\newtheorem*{conjecture}{Conjecture}
\newtheorem*{consequences}{Consequences}
\newtheorem*{exercise}{Exercise}
\newtheorem*{formalisation}{Formalisation}
\newtheorem*{proposition}{Proposition}
\newtheorem*{question}{Question}
\newtheorem*{remark}{Remark}

\ifx\type\undefined\else
  \newtheorem*{definition}{Definition}
  \newtheorem*{example}{Example}
  \newtheorem*{lemma}{Lemma}
  \newtheorem*{theorem}{Theorem}
\fi

\definecolor{keywordcolor}{rgb}{0.7, 0.1, 0.1}
\definecolor{tacticcolor}{rgb}{0.0, 0.1, 0.6}
\definecolor{commentcolor}{rgb}{0.4, 0.4, 0.4}
\definecolor{symbolcolor}{rgb}{0.0, 0.1, 0.6}
\definecolor{sortcolor}{rgb}{0.1, 0.5, 0.1}
\definecolor{attributecolor}{rgb}{0.7, 0.1, 0.1}
\def\lstlanguagefiles{lstlean.tex}
\lstset{language=lean}

\newcommand\A{\mathbb{A}}
\newcommand\C{\mathbb{C}}
\newcommand\F{\mathbb{F}}
\newcommand\G{\mathbb{G}}
\renewcommand\H{\mathbb{H}}
\newcommand\I{\mathbb{I}}
\newcommand\N{\mathbb{N}}
\renewcommand\P{\mathbb{P}}
\newcommand\Q{\mathbb{Q}}
\newcommand\R{\mathbb{R}}
\newcommand\Z{\mathbb{Z}}

\renewcommand\AA{\mathcal{A}}
\newcommand\BB{\mathcal{B}}
\newcommand\CC{\mathcal{C}}
\newcommand\DD{\mathcal{D}}
\newcommand\EE{\mathcal{E}}
\newcommand\FF{\mathcal{F}}
\newcommand\GG{\mathcal{G}}
\newcommand\HH{\mathcal{H}}
\newcommand\II{\mathcal{I}}
\newcommand\LL{\mathcal{L}}
\newcommand\MM{\mathcal{M}}
\newcommand\NN{\mathcal{N}}
\newcommand\OO{\mathcal{O}}
\newcommand\PP{\mathcal{P}}
\newcommand\RR{\mathcal{R}}
\renewcommand\SS{\mathcal{S}}
\newcommand\TT{\mathcal{T}}
\newcommand\XX{\mathcal{X}}

\renewcommand\aa{\mathfrak{a}}
\newcommand\cc{\mathfrak{c}}
\newcommand\dd{\mathfrak{d}}
\newcommand\ff{\mathfrak{f}}
\renewcommand\gg{\mathfrak{g}}
\newcommand\mm{\mathfrak{m}}
\newcommand\pp{\mathfrak{p}}
\newcommand\qq{\mathfrak{q}}
\renewcommand\ss{\mathfrak{s}}

\newcommand\LLL{\mathscr{L}}

\newcommand\ab{\mathrm{ab}}
\newcommand\Ab{\mathbf{Ab}}
\newcommand\Alg{\mathbf{Alg}}
\newcommand\Aff{\mathbf{Aff}}
\newcommand\Aut{\operatorname{Aut}}
\newcommand\Az{\mathrm{Az}}
\newcommand\Br{\operatorname{Br}}
\newcommand\BSD{\operatorname{BSD}}
\newcommand\ch{\operatorname{char}}
\newcommand\Cl{\operatorname{Cl}}
\newcommand\coker{\operatorname{coker}}
\newcommand\cris{\mathrm{cris}}
\renewcommand\d{\mathrm{d}}
\newcommand\Div{\operatorname{Div}}
\newcommand\dR{\mathrm{dR}}
\newcommand\EN{\operatorname{EN}}
\newcommand\End{\operatorname{End}}
\newcommand\ES{\operatorname{ES}}
\newcommand\et{\mathrm{\acute{e}t}}
\newcommand\Et{\mathbf{\acute{E}t}}
\newcommand\Ext{\operatorname{Ext}}
\newcommand\Fr{\operatorname{Fr}}
\newcommand\Frac{\operatorname{Frac}}
\newcommand\Gal{\operatorname{Gal}}
\newcommand\GL{\operatorname{GL}}
\newcommand\Gr{\mathrm{Gr}}
\newcommand\Hom{\operatorname{Hom}}
\newcommand\HT{\mathrm{HT}}
\newcommand\id{\operatorname{id}}
\newcommand\im{\operatorname{im}}
\newcommand\Ind{\operatorname{Ind}}
\renewcommand\inf{\operatorname{inf}}
\newcommand\inv{\operatorname{inv}}
\newcommand\Irr{\operatorname{Irr}}
\newcommand\Jac{\operatorname{Jac}}
\newcommand\lcm{\operatorname{lcm}}
\newcommand\Mat{\operatorname{Mat}}
\newcommand\Mod{\mathbf{Mod}}
\newcommand\Nm{\operatorname{Nm}}
\newcommand\nr{\mathrm{nr}}
\newcommand\NS{\operatorname{NS}}
\newcommand\Ob{\operatorname{Ob}}
\newcommand\ord{\operatorname{ord}}
\newcommand\op{\mathrm{op}}
\newcommand\PGL{\operatorname{PGL}}
\newcommand\Pic{\operatorname{Pic}}
\newcommand\Prob{\operatorname{Prob}}
\newcommand\Proj{\operatorname{Proj}}
\newcommand\PSh{\mathbf{PSh}}
\newcommand\Reg{\operatorname{Reg}}
\newcommand\res{\operatorname{res}}
\newcommand\rk{\operatorname{rk}}
\newcommand\Sch{\mathbf{Sch}}
\newcommand\Sel{\operatorname{Sel}}
\newcommand\Set{\mathbf{Set}}
\newcommand\sgn{\operatorname{sgn}}
\newcommand\Sh{\mathbf{Sh}}
\newcommand\SL{\operatorname{SL}}
\newcommand\Spec{\operatorname{Spec}}
\newcommand\supp{\operatorname{supp}}
\newcommand\Tam{\operatorname{Tam}}
\newcommand\Top{\mathbf{Top}}
\newcommand\tor{\operatorname{tor}}
\newcommand\tr{\operatorname{tr}}
\newcommand\tra{\operatorname{tra}}
\newcommand\WC{\operatorname{WC}}

\DeclareFontFamily{U}{wncyr}{}
\DeclareFontShape{U}{wncyr}{m}{n}{<->wncyr10}{}
\DeclareSymbolFont{cyr}{U}{wncyr}{m}{n}
\DeclareMathSymbol{\Sha}{\mathord}{cyr}{"58}

\newcommand{\function}[5][]{
  \if &#1&
    \begin{array}{rcl}
      #2 & \longrightarrow & #3 \\
      #4 & \longmapsto     & #5
    \end{array}
  \else
    \begin{array}{rcrcl}
      #1 & : & #2 & \longrightarrow & #3 \\
         &   & #4 & \longmapsto     & #5
    \end{array}
  \fi
}

\newcommand{\functions}[7][]{
  \if &#1&
    \begin{array}{rcl}
      #2 & \longrightarrow & #3 \\
      #4 & \longmapsto     & #5 \\
      #6 & \longmapsto     & #7 \\
    \end{array}
  \else
    \begin{array}{rcrcl}
      #1 & : & #2 & \longrightarrow & #3 \\
         &   & #4 & \longmapsto     & #5 \\
         &   & #6 & \longmapsto     & #7
    \end{array}
  \fi
}
\title{An unusual cubic representation problem}
\subtitle{Undergraduate mathematics colloquium}
\author{David Kurniadi Angdinata}
\institute{Imperial College London}
\date{Wednesday, 16 January 2019}

\begin{document}

\frame\maketitle

\begin{frame}{An unusual cubic representation problem}

\begin{center}
\includegraphics[width=0.7\textwidth]{img/positivewholevalues.png}
\end{center}

\vspace{-0.5cm}
\begin{align*}
\textsc{apple} & = {\scriptscriptstyle 154476802108746166441951315019919837485664325669565431700026634898253202035277999} \\
\textsc{banana} & = {\scriptscriptstyle 36875131794129999827197811565225474825492979968971970996283137471637224634055579} \\
\textsc{pineapple} & = {\scriptscriptstyle 4373612677928697257861252602371390152816537558161613618621437993378423467772036}
\end{align*}

\end{frame}

\begin{frame}{A trivial cubic representation problem}

\begin{center}
\includegraphics[width=0.7\textwidth]{img/values.png}
\end{center}

\begin{align*}
\C, \R & : (a, b, c) = (2 + \sqrt{3}, 1, 0) \\
\Q, \Z & : (a, b, c) = (11, 4, -1), (-11, -4, 1), (1, -4, -11), \dots
\end{align*}

\end{frame}

\begin{frame}{A less unusual cubic representation problem}

\framebox[\linewidth]{$ \dfrac{a}{b + c} + \dfrac{b}{a + c} + \dfrac{c}{a + b} = 4, \qquad a, b, c \in \Z. $}

\bigskip

\begin{itemize}
\item Require $ a, b, c > 0 $.
\end{itemize}

\bigskip Clear denominators:
$$ a^3 + b^3 + c^3 - 5abc - 3(a^2b + ab^2 + a^2c + ac^2 + b^2c + bc^2) = 0. $$
\textbf{Trivial} solutions:
$$ (a, b, c) = (11, 4, -1), (-11, -4, 1), (1, -4, -11), \dots $$
\textbf{Invalid} solutions:
$$ (a, b, c) = (1, -1, 0), (-1, 1, 0), (-1, 1, -1), \dots $$
\vspace{-0.5cm}
\begin{itemize}
\item Require $ a + b, a + c, b + c > 0 $.
\end{itemize}

\end{frame}

\begin{frame}{Dimensionality of solution space}

\framebox[\linewidth]{$ a^3 + b^3 + c^3 - 5abc - 3(a^2b + ab^2 + a^2c + ac^2 + b^2c + bc^2) = 0 $}

\begin{definition}
A polynomial is \textbf{homogeneous} if its monomials have the same degree.
\end{definition}

\bigskip If $ (a_0, b_0, c_0) $ is a solution in $ \Z $, then $ (\lambda a_0, \lambda b_0, \lambda c_0) $ is a solution in $ \Q $ for any $ \lambda \in \Q^* $. Define the equivalence relation $ \sim $ by
$$ (a_0, b_0, c_0) \sim (a_0', b_0', c_0') \iff (a_0, b_0, c_0) = (\lambda a_0', \lambda b_0', \lambda c_0'), \qquad \lambda \in \Q^*. $$
Write the equivalence class as $ [a_0, b_0, c_0] $.
\begin{itemize}
\item Modulo $ \sim $, the space of solutions is only two-dimensional.
\end{itemize}

\bigskip If $ c \ne 0 $, the equation is equivalent to
$$ a^3 + b^3 + 1 - 5ab - 3(a^2b + ab^2 + a^2 + a + b^2 + b) = 0, \qquad a, b \in \Q. $$
\vspace{-0.5cm}
\begin{itemize}
\item Modulo $ \sim $, the equation is cubic of two variables.
\end{itemize}

\end{frame}

\begin{frame}{Elliptic curves: informally}

\begin{center}
\includegraphics[width=0.2\textwidth]{img/ellipticcurve.png}
\end{center}

An \emph{elliptic curve} is the space of solutions to a cubic equation
$$ y^2 = x^3 + Ax + B, $$
where $ A $ and $ B $ are in some field such that $ 4A^3 + 27B^2 \ne 0 $.
\begin{itemize}
\item Simplest non-trivial structures in algebraic geometry.
\item Topic of the \emph{Birch and Swinnerton-Dyer conjecture}.
\item Tool in Wiles' proof of \emph{Fermat's last theorem}.
\item Methods for primality testing and integer factorisation.
\item Applications in \emph{elliptic curve cryptography}.
\end{itemize}

\end{frame}

\begin{frame}{Elliptic curves: formally}

\begin{definition}
An \textbf{elliptic curve} over a field $ K $ is a \emph{smooth projective plane algebraic curve} $ E $ of \emph{genus one} with a \emph{$ K $-rational base point} $ \OO_E $.
\begin{itemize}
\item \emph{algebraic curve}: space of solutions to equation
\item \emph{plane}: two variables
\item \emph{projective}: consider equivalence classes of solutions
\item \emph{smooth}: no kinks
\item \emph{genus one}: degree three
\item \emph{$ K $-rational base point}: coordinates in $ K $
\end{itemize}
\end{definition}

\begin{theorem}
An elliptic curve over $ \Q $ is of the form
$$ E = \{(x, y) \in \Q^2 : y^2 = x^3 + Ax + B\} \cup \{\OO\}, $$
for some $ A, B \in \Q $ such that $ 4A^3 + 27B^2 \ne 0 $, where $ \OO = [0, 1, 0] $.
\end{theorem}

\end{frame}

\begin{frame}{Weierstrass representations}

\framebox[\linewidth]{$ a^3 + b^3 + c^3 - 5abc - 3(a^2b + ab^2 + a^2c + ac^2 + b^2c + bc^2) = 0 $}

\begin{proposition}
The curve given by the equation is isomorphic to the following curves.
\begin{itemize}
\item $ \{(x, y) \in \Q^2 : 6y^2 + 6xy + 6y = -91x^3 + 141x^2 + 15x - 1\} \cup \{\OO\} $.
\item $ \{(x, y) \in \Q^2 : y^2 + xy - \tfrac{91}{6}y = x^3 + \tfrac{47}{2}x^2 - \tfrac{455}{12}x - \tfrac{8281}{216}\} \cup \{\OO\} $.
\item $ \{(x, y) \in \Q^2 : y^2 + xy + y = x^3 - 234x + 1352\} \cup \{\OO\} $.
\item $ \{(x, y) \in \Q^2 : y^2 = x^3 + \tfrac{1}{4}x^2 - \tfrac{467}{2}x + \tfrac{5409}{4}\} \cup \{\OO\} $.
\item $ \{(x, y) \in \Q^2 : y^2 = x^3 - \tfrac{11209}{48}x + \tfrac{1185157}{864}\} \cup \{\OO\} $.
\item $ \{(x, y) \in \Q^2 : y^2 = x^3 - 302643x + 63998478\} \cup \{\OO\} $.
\end{itemize}
\end{proposition}

Let $ A = -302643 $ and $ B = 63998478 $. Overall invertible transformations:
$$
\begin{cases}
a = \tfrac{1}{72}x + \tfrac{1}{216}y - \tfrac{277}{24} \\
b = \tfrac{1}{72}x - \tfrac{1}{216}y - \tfrac{277}{24} \\
c = \tfrac{1}{6}x - \tfrac{95}{2}
\end{cases}
\qquad
\begin{cases}
x = \tfrac{1710a + 1710b - 831c}{6a + 6b - c} \\
y = \tfrac{-9828a + 9828b}{6a + 6b - c}
\end{cases}
$$

\end{frame}

\begin{frame}{A group law}

\framebox[\linewidth]{$ E = \{(x, y) \in \Q^2 : y^2 = x^3 + Ax + B\} \cup \{\OO\} $}

\begin{theorem}
$ E $ is an abelian group $ (E, +) $.
\begin{itemize}
\item The identity point is $ \OO \in E $.
\item The inverse of a point is obtained by reflecting the point vertically.
\item The sum of two points is obtained by inverting the third point of intersection between the curve and the line joining the two points:
$$ P + Q =
\begin{cases}
S & \text{if} \ P = (x, y), \ Q = (x', y'), \ x \ne x', \\
R & \text{if} \ P = Q = (x, y), \ y \ne 0, \\
P & \text{if} \ Q = \OO, \\
\OO & \text{if} \ P = Q = (x, 0),
\end{cases}
$$
{\scriptsize $$ S = \left(\tfrac{(A + xx')(x + x') + 2(B - yy')}{(x - x')^2}, \tfrac{(Ay' - x'^2y)(3x + x') + (x^2y' - Ay)(x + 3x') - 4B(y - y')}{(x - x')^3}\right), $$}
{\scriptsize $$ R = \left(\tfrac{x^4 - 2Ax^2 - 8Bx + A^2}{4y^2}, \tfrac{x^6 + 5Ax^4 + 20Bx^3 - 5A^2x^2 - 4ABx - A^3 - 8B^2}{8y^3}\right). $$}
\end{itemize}
\end{theorem}

\end{frame}

\begin{frame}{Proof of the group law}

\framebox[\linewidth]{$ E = \{(x, y) \in \Q^2 : y^2 = x^3 + Ax + B\} \cup \{\OO\} $}

\begin{lemma}[B\'ezout's theorem]
Let $ C $ and $ D $ be projective algebraic curves over an algebraically closed field $ \overline{K} $. Then $ C $ and $ D $ intersect at exactly $ \deg C\deg D $ points counted with intersection multiplicity.
\end{lemma}

\begin{lemma}[Cayley--Bacharach theorem]
Let $ C, D, E $ be projective algebraic cubic curves over an algebraically closed field $ \overline{K} $ such that
$$ C \cap E = \{P_1, \dots, P_8, Q\}, \qquad D \cap E = \{P_1, \dots, P_8, R\}, $$
counted with intersection multiplicity. Then $ Q = R $.
\end{lemma}

\begin{itemize}
\item Well-definition of addition in $ K $ holds by explicit equations.
\item Commutativity of addition holds by symmetry.
\item Associativity of addition holds by intimidation.
\end{itemize}

\end{frame}

\begin{frame}{Procedure}

\begin{algorithm}
Generate new solutions from old solutions.
\begin{itemize}
\item Choose an initial solution $ (a, b, c) $.
\item Apply the change of variables:
$$
\begin{cases}
x = \tfrac{1710a + 1710b - 831c}{6a + 6b - c} \\
y = \tfrac{-9828a + 9828b}{6a + 6b - c}
\end{cases}
$$
\item Compute multiples of the point $ (x, y) $.
\item Apply the change of variables:
$$
\begin{cases}
a = \tfrac{1}{72}x + \tfrac{1}{216}y - \tfrac{277}{24} \\
b = \tfrac{1}{72}x - \tfrac{1}{216}y - \tfrac{277}{24} \\
c = \tfrac{1}{6}x - \tfrac{95}{2}
\end{cases}
$$
\item Terminate or repeat.
\end{itemize}
\end{algorithm}

\end{frame}

\begin{frame}{Computation: failure}

Choose an invalid solution:
$$ (a, b, c) = (-1, 1, -1). $$
Apply the change of variables:
$$ (x, y) = (831, 19656). $$
Compute multiples of point:
\begin{itemize}
\item $ 2(x, y) = (363, 1404) $.
\item $ 3(x, y) = (327, 0) $.
\item $ 4(x, y) = (363, -1404) $.
\item $ 5(x, y) = (831, -19656) $.
\item $ 6(x, y) = \OO $.
\end{itemize}
This is a cyclic subgroup of order six.

\end{frame}

\begin{frame}{Computation: success}

Choose a trivial solution:
$$ (a, b, c) = (11, 4, -1). $$
Apply the change of variables:
$$ (x, y) = (291, -756). $$
Compute multiples of point:
\begin{itemize}
\item $ 2(x, y) = (\tfrac{22107}{49}, -\tfrac{1506492}{343}) $. Apply the change of variables:
$$ (a, b, c) = (-8784, 5165, 9499). $$
\item $ 3(x, y) = (-\tfrac{2694138}{11881}, -\tfrac{14243306490}{1295029}) $. Apply the change of variables:
$$ (a, b, c) = (679733219, -375326521, 883659076). $$
\item $ 9(x, y) = (\tfrac{3823387580080160076063605209061052603963389916327719142}{13514400292716288512070907945002943352692578000406921}, \dots) $. Apply the change of variables:
$$ (a, b, c) = (\textsc{apple}, \textsc{banana}, \textsc{pineapple}). $$
\end{itemize}

\end{frame}

\begin{frame}{Further facts}

The general equation is
$$ \dfrac{a}{b + c} + \dfrac{b}{a + c} + \dfrac{c}{a + b} = N, \qquad a, b, c \in \N^*, \qquad N \in \Z. $$
\begin{itemize}
\item The elliptic curve is
$$ E \cong \Z^r \oplus
\begin{cases}
\Z / 2\Z \oplus \Z / 6\Z & \text{if} \ N = 2, \\
\Z / 6\Z & \text{otherwise},
\end{cases}
\qquad r \in \N^*. $$
\item The smallest solution for $ N = 4 $ is $ (a, b, c) = (\textsc{apple}, \textsc{banana}, \textsc{pineapple}) $.
\item The smallest solution for $ N = 178 $ has four hundred million digits.
\item There are no solutions for $ N $ is odd.
\item There may also be no solutions if $ N $ is even.
\item There are infinitely many even $ N $ with solutions.
\end{itemize}

\end{frame}

\begin{frame}[c]{Further references}

\begin{itemize}
\item A Amit's 2017 Quora answer on \emph{How do you find the positive integer solutions to}
$$ \dfrac{x}{y + z} + \dfrac{y}{z + x} + \dfrac{z}{x + y} = 4? $$
\item A Bremner and A Macleod's 2014 paper on \emph{An unusual cubic representation problem}
\item J Silverman's 1986 book on \emph{The arithmetic of elliptic curves}
\item R Hartshorne's 1977 book on \emph{Algebraic geometry}
\item N Duif's 2011 implementation on \emph{Transforming a general cubic elliptic curve equation to Weierstrass form}
\item M Laska's 1982 paper on \emph{An algorithm for finding a minimal weierstrass equation for an elliptic curve}
\end{itemize}

\end{frame}

\end{document}