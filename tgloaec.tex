\ifx\type\undefined
  \documentclass[10pt, t]{beamer}
  \setbeamertemplate{footline}[page number]
\else
  \documentclass[10pt]{article}
  \usepackage[margin=1in]{geometry}
\fi

\usepackage{amsmath}
\usepackage{amssymb}
\usepackage{amsthm}
\usepackage{bbm}
\usepackage{cancel}
\usepackage{listings}
\usepackage{mathrsfs}
\usepackage{multirow}
\usepackage{soul}
\usepackage{stmaryrd}
\usepackage{tikz}
\usepackage{tikz-cd}
\usepackage{wrapfig}

\newtheorem*{algorithm}{Algorithm}
\newtheorem*{assumptions}{Assumptions}
\newtheorem*{conjecture}{Conjecture}
\newtheorem*{consequences}{Consequences}
\newtheorem*{exercise}{Exercise}
\newtheorem*{formalisation}{Formalisation}
\newtheorem*{proposition}{Proposition}
\newtheorem*{question}{Question}
\newtheorem*{remark}{Remark}

\ifx\type\undefined\else
  \newtheorem*{definition}{Definition}
  \newtheorem*{example}{Example}
  \newtheorem*{lemma}{Lemma}
  \newtheorem*{theorem}{Theorem}
\fi

\definecolor{keywordcolor}{rgb}{0.7, 0.1, 0.1}
\definecolor{tacticcolor}{rgb}{0.0, 0.1, 0.6}
\definecolor{commentcolor}{rgb}{0.4, 0.4, 0.4}
\definecolor{symbolcolor}{rgb}{0.0, 0.1, 0.6}
\definecolor{sortcolor}{rgb}{0.1, 0.5, 0.1}
\definecolor{attributecolor}{rgb}{0.7, 0.1, 0.1}
\def\lstlanguagefiles{lstlean.tex}
\lstset{language=lean}

\newcommand\A{\mathbb{A}}
\newcommand\C{\mathbb{C}}
\newcommand\F{\mathbb{F}}
\newcommand\G{\mathbb{G}}
\renewcommand\H{\mathbb{H}}
\newcommand\I{\mathbb{I}}
\newcommand\N{\mathbb{N}}
\renewcommand\P{\mathbb{P}}
\newcommand\Q{\mathbb{Q}}
\newcommand\R{\mathbb{R}}
\newcommand\Z{\mathbb{Z}}

\renewcommand\AA{\mathcal{A}}
\newcommand\BB{\mathcal{B}}
\newcommand\CC{\mathcal{C}}
\newcommand\DD{\mathcal{D}}
\newcommand\EE{\mathcal{E}}
\newcommand\FF{\mathcal{F}}
\newcommand\GG{\mathcal{G}}
\newcommand\HH{\mathcal{H}}
\newcommand\II{\mathcal{I}}
\newcommand\LL{\mathcal{L}}
\newcommand\MM{\mathcal{M}}
\newcommand\NN{\mathcal{N}}
\newcommand\OO{\mathcal{O}}
\newcommand\PP{\mathcal{P}}
\newcommand\RR{\mathcal{R}}
\renewcommand\SS{\mathcal{S}}
\newcommand\TT{\mathcal{T}}
\newcommand\XX{\mathcal{X}}

\renewcommand\aa{\mathfrak{a}}
\newcommand\cc{\mathfrak{c}}
\newcommand\dd{\mathfrak{d}}
\newcommand\ff{\mathfrak{f}}
\renewcommand\gg{\mathfrak{g}}
\newcommand\mm{\mathfrak{m}}
\newcommand\pp{\mathfrak{p}}
\newcommand\qq{\mathfrak{q}}
\renewcommand\ss{\mathfrak{s}}

\newcommand\LLL{\mathscr{L}}

\newcommand\ab{\mathrm{ab}}
\newcommand\Ab{\mathbf{Ab}}
\newcommand\Alg{\mathbf{Alg}}
\newcommand\Aff{\mathbf{Aff}}
\newcommand\Aut{\operatorname{Aut}}
\newcommand\Az{\mathrm{Az}}
\newcommand\Br{\operatorname{Br}}
\newcommand\BSD{\operatorname{BSD}}
\newcommand\ch{\operatorname{char}}
\newcommand\Cl{\operatorname{Cl}}
\newcommand\coker{\operatorname{coker}}
\newcommand\cris{\mathrm{cris}}
\renewcommand\d{\mathrm{d}}
\newcommand\Div{\operatorname{Div}}
\newcommand\dR{\mathrm{dR}}
\newcommand\EN{\operatorname{EN}}
\newcommand\End{\operatorname{End}}
\newcommand\ES{\operatorname{ES}}
\newcommand\et{\mathrm{\acute{e}t}}
\newcommand\Et{\mathbf{\acute{E}t}}
\newcommand\Ext{\operatorname{Ext}}
\newcommand\Fr{\operatorname{Fr}}
\newcommand\Frac{\operatorname{Frac}}
\newcommand\Gal{\operatorname{Gal}}
\newcommand\GL{\operatorname{GL}}
\newcommand\Gr{\mathrm{Gr}}
\newcommand\Hom{\operatorname{Hom}}
\newcommand\HT{\mathrm{HT}}
\newcommand\id{\operatorname{id}}
\newcommand\im{\operatorname{im}}
\newcommand\Ind{\operatorname{Ind}}
\renewcommand\inf{\operatorname{inf}}
\newcommand\inv{\operatorname{inv}}
\newcommand\Irr{\operatorname{Irr}}
\newcommand\Jac{\operatorname{Jac}}
\newcommand\lcm{\operatorname{lcm}}
\newcommand\Mat{\operatorname{Mat}}
\newcommand\Mod{\mathbf{Mod}}
\newcommand\Nm{\operatorname{Nm}}
\newcommand\nr{\mathrm{nr}}
\newcommand\NS{\operatorname{NS}}
\newcommand\Ob{\operatorname{Ob}}
\newcommand\ord{\operatorname{ord}}
\newcommand\op{\mathrm{op}}
\newcommand\PGL{\operatorname{PGL}}
\newcommand\Pic{\operatorname{Pic}}
\newcommand\Prob{\operatorname{Prob}}
\newcommand\Proj{\operatorname{Proj}}
\newcommand\PSh{\mathbf{PSh}}
\newcommand\Reg{\operatorname{Reg}}
\newcommand\res{\operatorname{res}}
\newcommand\rk{\operatorname{rk}}
\newcommand\Sch{\mathbf{Sch}}
\newcommand\Sel{\operatorname{Sel}}
\newcommand\Set{\mathbf{Set}}
\newcommand\sgn{\operatorname{sgn}}
\newcommand\Sh{\mathbf{Sh}}
\newcommand\SL{\operatorname{SL}}
\newcommand\Spec{\operatorname{Spec}}
\newcommand\supp{\operatorname{supp}}
\newcommand\Tam{\operatorname{Tam}}
\newcommand\Top{\mathbf{Top}}
\newcommand\tor{\operatorname{tor}}
\newcommand\tr{\operatorname{tr}}
\newcommand\tra{\operatorname{tra}}
\newcommand\WC{\operatorname{WC}}

\DeclareFontFamily{U}{wncyr}{}
\DeclareFontShape{U}{wncyr}{m}{n}{<->wncyr10}{}
\DeclareSymbolFont{cyr}{U}{wncyr}{m}{n}
\DeclareMathSymbol{\Sha}{\mathord}{cyr}{"58}

\newcommand{\function}[5][]{
  \if &#1&
    \begin{array}{rcl}
      #2 & \longrightarrow & #3 \\
      #4 & \longmapsto     & #5
    \end{array}
  \else
    \begin{array}{rcrcl}
      #1 & : & #2 & \longrightarrow & #3 \\
         &   & #4 & \longmapsto     & #5
    \end{array}
  \fi
}

\newcommand{\functions}[7][]{
  \if &#1&
    \begin{array}{rcl}
      #2 & \longrightarrow & #3 \\
      #4 & \longmapsto     & #5 \\
      #6 & \longmapsto     & #7 \\
    \end{array}
  \else
    \begin{array}{rcrcl}
      #1 & : & #2 & \longrightarrow & #3 \\
         &   & #4 & \longmapsto     & #5 \\
         &   & #6 & \longmapsto     & #7
    \end{array}
  \fi
}
\title{The group law on an elliptic curve \footnote{Angdinata, David Kurniadi and Xu, Junyan. \emph{An Elementary Formal Proof of the Group Law on Weierstrass Elliptic Curves in Any Characteristic}. Fourteenth International Conference on Interactive Theorem Proving (ITP 2023)}}
\subtitle{Postgraduate seminar}
\author{David Kurniadi Angdinata}
\institute{University College London}
\date{Thursday, 5 October 2023}

\begin{document}

\frame\maketitle

\begin{frame}{Introduction}

Pedagogical question:
\begin{itemize}
\item Is there an \emph{elementary} proof of the group law on \emph{any} elliptic curve?
\end{itemize}

\bigskip Status quo:
\begin{itemize}
\item Yes. But it depends on what is considered \emph{elementary}.
\end{itemize}

\bigskip Our answer:
\begin{itemize}
\item Yes. And we formalised the argument in the \emph{Lean theorem prover}.
\end{itemize}

\bigskip Talk overview:
\begin{itemize}
\item What is an elliptic curve?
\item Why is it a group?
\item Where is the problem then?
\item How did we do it?
\end{itemize}

\end{frame}

\begin{frame}{Elliptic curves}

An \textbf{elliptic curve} over a field $ F $ is a pair $ (E, 0) $, where
\begin{itemize}
\item $ E $ is a \emph{smooth projective curve} of \emph{genus one} defined over $ F $, and
\item $ 0 $ is a distinguished point on $ E $ defined over $ F $.
\end{itemize}

\bigskip They are the simplest non-trivial objects in arithmetic geometry.
\begin{itemize}
\item Wiles proved \emph{Fermat's last theorem} by drawing a correspondence between certain elliptic curves and certain \emph{modular forms}.
\item The \emph{Birch and Swinnerton-Dyer conjecture} predicts the behaviour of the \emph{L-function} of an elliptic curve based on its arithmetic invariants.
\end{itemize}

\bigskip Outside pure mathematics, they see many computational applications.
\begin{itemize}
\item Intractability of the \emph{discrete logarithm problem} for elliptic curves forms the basis behind many public key cryptographic protocols.
\item The \emph{Atkin--Morain primality test} and \emph{Lenstra's factorisation method} use elliptic curves and are two of the fastest known algorithms.
\end{itemize}

\end{frame}

\begin{frame}{Elliptic curves}

An \textbf{elliptic curve} over a field $ F $ is a pair $ (E, 0) $, where
\begin{itemize}
\item $ E $ is a \emph{smooth projective curve} of \emph{genus one} defined over $ F $, and
\item $ 0 $ is a distinguished point on $ E $ defined over $ F $.
\end{itemize}

\begin{theorem}[long Weierstrass model]
Any elliptic curve $ E $ over $ F $ can be given by $ E(X, Y) = 0 $, where
$$ E(X, Y) := Y^2 + a_1XY + a_3Y - (X^3 + a_2X^2 + a_4X + a_6), $$
for some $ a_i \in F $ such that $ \Delta \ne 0 $, \footnote{\tiny $ \Delta := -(a_1^2 + 4a_2)^2(a_1^2a_6 + 4a_2a_6 - a_1a_3a_4 + a_2a_3^2 - a_4^2) - 8(2a_4 + a_1a_3)^3 - 27(a_3^2 + 4a_6)^2 + 9(a_1^2 + 4a_2)(2a_4 + a_1a_3)(a_3^2 + 4a_6) $} with $ 0 $ being the ``point at infinity''.
\end{theorem}

\begin{proof}
Follows from the \emph{Riemann--Roch theorem} in algebraic geometry.
\end{proof}

If $ \ch(F) \ne 2, 3 $, then $ E $ has a \textbf{short Weierstrass model}, where
$$ E(X, Y) := Y^2 - (X^3 + aX + b), $$
for some $ a, b \in F $ such that $ \Delta = -16(4a^3 + 27b^2) \ne 0 $.

\end{frame}

\begin{frame}{Group law}

\begin{theorem}[the group law]
The points of an elliptic curve form an abelian group, where the identity element is $ 0 $, and the addition law is characterised by
$$ P + Q + R = 0 \qquad \iff \qquad P, Q, R \ \text{are collinear}. $$
\end{theorem}

If $ R = 0 $, then this translates to $ P + Q = 0 $ if and only if the line through $ P $ and $ Q $ is vertical. Thus negation can be given by
$$ -(x, y) := (x, -y - a_1x - a_3). $$
Define an affine involution given by
$$ \sigma(Y) := -Y - a_1X - a_3. $$
Note that in the \textbf{coordinate ring} $ F[E] := F[X, Y] / (E(X, Y)) $,
$$ -(Y \cdot \sigma(Y)) = Y^2 + a_1XY + a_3Y \equiv X^3 + a_2X^2 + a_4X + a_6. $$

\end{frame}

\begin{frame}{Group law}

\begin{theorem}[the group law]
The points of an elliptic curve form an abelian group, where the identity element is $ 0 $, and the addition law is characterised by
$$ P + Q + R = 0 \qquad \iff \qquad P, Q, R \ \text{are collinear}. $$
\end{theorem}

Addition can be given by $ (x_1, y_1) + (x_2, y_2) := -(x_3, y_3) $. Here,
\begin{align*}
\lambda & :=
\begin{cases}
\dfrac{y_1 - y_2}{x_1 - x_2} & \text{if} \ x_1 \ne x_2, \\[0.1cm]
\dfrac{3x_1^2 + 2a_2x_1 + a_4 - a_1y_1}{y_1 - \sigma(y_1)} & \text{if} \ y_1 \ne \sigma(y_1), \\
\infty & \text{otherwise},
\end{cases} \\
x_3 & := \lambda^2 + a_1\lambda - a_2 - x_1 - x_2, \\
y_3 & := \lambda(x_3 - x_1) + y_1.
\end{align*}

\end{frame}

\begin{frame}{Hard problem}

One may attempt to prove the axioms directly.
\begin{itemize}
\item Identity: $ 0 + P = P = P + 0 $ is trivial.
\item Inverses: $ (-P) + P = 0 = P + (-P) $ is easy.
\item Commutativity: $ P + Q = Q + P $ is easy.
\item Associativity: $ (P + Q) + R = P + (Q + R) $ seems impossible?

\bigskip Recall that each addition operation has five cases!

\bigskip In the generic case, \footnote{$ P $, $ Q $, $ R $, $ P + Q $, $ P + R $, and $ Q + R $ are affine and have distinct $ X $-coordinates} checking that their $ X $-coordinates are equal is an equality of polynomials with 26,082 terms.

\bigskip In the short Weierstrass model, this reduces to 2,636 terms.

\bigskip Automation in an interactive theorem prover enables manipulation of multivariate polynomials with at most 5,000 terms.
\end{itemize}

\end{frame}

\begin{frame}{Hard problem}

Associativity is known to be mathematically difficult with many proofs.
\begin{itemize}
\item[Pf 1.] Just do it. \\
Polynomial manipulation, but impossibly slow and many cases.
\item[Pf 2.] Count dimensions. \\
Projective geometry (\emph{Cayley--Bacharach}), but only works generically.
\end{itemize}
One may instead identify the set of points $ E(F) $ with a known group $ G $.
\begin{itemize}
\item[Pf 3.] $ G = \C / \Lambda_E $. \\
Riemann surfaces (\emph{uniformisation}), but only works for $ \ch(F) = 0 $.
\item[Pf 4.] $ G = \Pic_F^0(E) $. \\
Algebraic geometry (\emph{Riemann--Roch}) in general. \\
Ring theory (\emph{Fermat descent}), but only works for $ \ch(F) \ne 2 $.
\end{itemize}
Undergraduate courses typically teach Pf 2 (assuming genericity), Pf 3 (assuming uniformisation), or Pf 4 (assuming Riemann--Roch).

\bigskip Existing interactive theorem provers have used Pf 1 (Th\'ery 2007) or Pf 4 (Bartzia--Strub 2014), both assuming the short Weierstrass model.

\end{frame}

\begin{frame}{Algebraic analogue}

Let us examine the argument for Pf 3 and Pf 4 in more detail.

\bigskip To identify $ E(F) $ with \emph{a subgroup of} $ G $ is to
\begin{itemize}
\item define a function $ \phi : E(F) \to G $,
\item prove that $ \phi $ respects addition, and
\item prove that $ \phi $ is injective.
\end{itemize}

\bigskip Pf 4 sets $ G = \Pic_F^0(E) $, and
$$ \phi \ \text{is injective} \qquad \iff \qquad \text{there is no isomorphism} \ E \xrightarrow{\sim} \P^1, $$
which follows from isomorphism invariance of the \emph{genus}.

\bigskip Our proof sets $ G = \Cl(F[E]) $, and
$$ \phi \ \text{is injective} \qquad \iff \qquad \text{an ideal of} \ F[E] \ \text{is not principal}, $$
which is just a statement in ring theory.

\end{frame}

\begin{frame}{Algebraic analogue}

The group $ \Cl(F[E]) $ is the \emph{ideal class group} of the coordinate ring
$$ F[E] := F[X, Y] / (E(X, Y)). $$
\underline{Exercise (easy)}: $ F[E] $ is an integral domain.

\bigskip For any integral domain $ R $, the \textbf{ideal class group} $ \Cl(R) $ is the quotient group of \emph{invertible fractional ideals} by those that are \emph{principal}.
\begin{itemize}
\item A submodule $ I $ is a \textbf{fractional ideal} if $ \exists r \in R $ such that $ r \cdot I \subseteq R $.
\item $ I $ is \textbf{invertible} if there is a fractional ideal $ J $ such that $ I \cdot J = R $.
\item $ I $ is \textbf{principal} if $ \exists r, s \in R $ such that $ r \cdot I = (s) $.
\end{itemize}
\underline{Exercise (hard)}: $ \Cl(R) $ is an abelian group.

\bigskip

\begin{example}[of invertible fractional ideals]
Any nonzero ideal $ I $ such that $ I \cdot J $ is principal for some ideal $ J $.
\end{example}

\end{frame}

\begin{frame}{Algebraic analogue}

\begin{proof}[Pf 5 (A.--Xu)]
\renewcommand\qedsymbol{}
\begin{itemize}
\item Define a function $ \phi : E(F) \to \Cl(F[E]) $. This will be
$$ \functions[\phi]{E(F)}{\Cl(F[E])}{0}{[(1)]}{(x, y)}{[(X - x, Y - y)]}. $$
Note that $ \phi $ is well-defined since
$$ (X - x, Y - y) \cdot (X - x, Y - \sigma(y)) = (X - x). $$
\item Prove that $ \phi $ respects addition. This holds since
$$ (X - x_1, Y - y_1) \cdot (X - x_2, Y - y_2) \cdot (X - x_3, Y - \sigma(y_3)) \qquad $$
$$ \qquad = ((Y - y_3) - \lambda(X - x_3)). $$
\item Prove that $ \phi $ is injective. $ \square $
\end{itemize}
\vspace{-0.5cm}
\end{proof}

\end{frame}

\begin{frame}{Injectivity proof}

Note that $ F[E] $ is free over $ F[X] $ with basis $ \{1, Y\} $. Thus it has a \textbf{norm}
$$ \function[\Nm]{F[E]}{F[X]}{f}{\det([\cdot f])}. $$

\begin{example}[of norms]
Recall that $ Y \cdot Y \equiv -(a_1X + a_3) \cdot Y + (X^3 + a_2X^2 + a_4X + a_6) $. Then
\begin{align*}
\Nm(Y) & \equiv \det\begin{pmatrix} 0 & 1 \\ X^3 + a_2X^2 + a_4X + a_6 & -(a_1X + a_3) \end{pmatrix} \\
& = X^3 + a_2X^2 + a_4X + a_6.
\end{align*}
\end{example}

In general, if $ f = p + qY \in F[E] $ for some $ p, q \in F[X] $,
$$ \Nm(f) = p^2 - pq(a_1X + a_3) - q^2(X^3 + a_2X^2 + a_4X + a_6). $$
This has degree $ \max(2\deg(p), 2\deg(q) + 3) \ne 1 $.

\end{frame}

\begin{frame}{Injectivity proof}

Note that $ F[E] $ is free over $ F[X] $ with basis $ \{1, Y\} $. Thus it has a \textbf{norm}
$$ \function[\Nm]{F[E]}{F[X]}{f}{\det([\cdot f])}. $$
Now $ [\cdot f] $ has a Smith normal form
$$ [\cdot f] \sim \begin{pmatrix} p & 0 \\ 0 & q \end{pmatrix}, \qquad p, q \in F[X]. $$
On one hand, $ F[E] / (f) \cong F[X] / (p) \oplus F[X] / (q) $. Then
$$ \dim(F[E] / (f)) = \deg(p) + \deg(q). $$
On the other hand, $ \Nm(f) = pq $. Then
$$ \deg(\Nm(f)) = \deg(p) + \deg(q). $$
Combining these with $ \deg(\Nm(f)) \ne 1 $ yields $ \dim(F[E] / (f)) \ne 1 $.

\end{frame}

\begin{frame}{Injectivity proof}

\begin{proof}[Pf 5 (A.--Xu)]
\renewcommand\qedsymbol{}
\begin{itemize}
\item Define a function $ \phi : E(F) \to \Cl(F[E]) $.
\item Prove that $ \phi $ respects addition.
\item Prove that $ \phi $ is injective. It suffices to show that $ (X - x, Y - y) $ is not principal for any $ (x, y) \in E(F) $. Suppose otherwise, that
$$ (X - x, Y - y) = (f), \qquad f \in F[E]. $$
Then
\begin{align*}
F[E] / (f)
& = F[E] / (X - x, Y - y) \\
& \cong F[X, Y] / (E(X, Y), X - x, Y - y) & \text{3\textsuperscript{rd} iso thm} \\
& = F[X, Y] / (X - x, Y - y) & (x, y) \in E(F) \\
& \cong F & \text{1\textsuperscript{st} iso thm}.
\end{align*}
Since $ \dim(F) = 1 $, this contradicts $ \dim(F[E] / (f)) \ne 1 $! $ \square $
\end{itemize}
\vspace{-0.5cm}
\end{proof}

\end{frame}

\begin{frame}[c]{Conclusions}

Some retrospectives:
\begin{itemize}
\item formalisation encouraged proof accessible to undergraduates
\item novel injectivity proof and novel formalisation
\item proof works for \emph{nonsingular} points of \emph{Weierstrass} curves
\item heavy use of linear algebra and ring theory in Lean's \texttt{mathlib}
\item generality of ideal class groups of integral domains
\item plans for many more formalisation projects!
\end{itemize}

\end{frame}

\end{document}